% o vjerojatnosnom prostoru

\chapter{Vjerojatnosni prostor}


Po\v cinjemo s definicijom $\sigma-$algebre, osnovne strukuture na kojoj ima
smisla definirati vjerojatnost.

\begin{defn}
    Neka je $\Omega \neq \emptyset$ skup i $ \F
        \subseteq \partitive{\Omega}$ familija podskupova od $\Omega$.
    Ka\v emo da je \emph{$\F$ $\sigma -$algebra} ako vrijedi:
    \begin{enumerate}[label={(\arabic*)}]
        \item $\emptyset \in \F$
        \item $A \in \F \implies A^{C} \in \F$
        \item $(A_n)_{n \in \N} \subseteq \F \implies \unija{n \in \N}{} A_n \in \F$.
    \end{enumerate}
\end{defn}

\begin{nap}
    Osim termina $\sigma-$algebra koristi se i termin $\sigma-$polje.
\end{nap}

Nastavljamo s osnovnim definicijama.

\begin{defn}
    Neka je $\Omega \neq \emptyset$ i $\F$ $\sigma-$algebra na $\F$, uređeni par
    $(\Omega, \: \F)$ zovemo \emph{izmjerivim prostorom}.
\end{defn}

Izmjeriv prostor je dovoljno bogata struktura za definirati pojam mjere, time
i vjerojatnosti.

\begin{defn}
    Neka je $\izmjerivProstor$ izmjeriv prostor, za funkciju $\mu : \F \to \R$
    ka\v zemo da je \emph{mjera}, ako vrijedi:
    \begin{enumerate}[label={(\arabic*)}]
        \item $\mjera{A} \geq 0$, $\forall A \in \F$
        \item $\mjera{\emptyset} = 0$
        \item $(A_n)_{n \in \N} \subseteq \F$ td. $A_i \cap A_j =
            \emptyset, \: i \neq j, \;$ vrijedi $\; \mjera{ \unija{n \in \N}{} A_n }
            = \suma{n \in \N}{} \mjera{ A_n }.$
    \end{enumerate}
\end{defn}

Mjera za koju vrijedi i uvijet normiranosti, tj. $\mjera{\Omega} = 1$, zovemo
vjerojatnosnom mjerom. Ona se aksiomatski mo\v ze opisati na sljede\' ci na\v cin.

\begin{defn} \label{definicijaVjerojatnosti}
    Neka je $\izmjerivProstor$ izmjeriv prostor, za funkciju
    $ \mathbb{P} : \F \to [ 0, \: 1 ]$, ka\v zemo da je \emph{vjerojatnost}
    (\emph{vjerojatnosna mjera}), ako vrijedi:
    \begin{enumerate}[label={(\arabic*)}]
        \item $\vjeroj{A} \geq 0$, $\forall A \in \F $
        \item $\vjeroj{\Omega} = 1$
        \item $(A_n)_{n \in \N} \subseteq \F $ td. $A_i \cap A_j = \emptyset, \:
            i \neq j, \;$ vrijedi $\; \vjeroj{\unija{n \in \N}{} A_n} =
            \suma{n \in \N}{} \vjeroj{A_n}$. \label{sigmaAditivnost}
    \end{enumerate}
\end{defn}

\begin{nap}
    Svojstvo \eqref{sigmaAditivnost} iz definicije \ref{definicijaVjerojatnosti}
    nazivamo svojstvom \emph{$\sigma-$aditivnosti}.
\end{nap}

\begin{defn}
    Neka je $\izmjerivProstor$ i $\mu$ mjera na njemu. Ure\dj enu trojku
    $\prostorMjere$ nazivamo \emph{prostorom mjere}. Ako je $\mu$
    vjerojatnost, tada $\prostorMjere$ nazivamo
    \emph{vjerojatnosnim prostorom}.
\end{defn}

\begin{prop}
    Neka je $\vjerojatnosniProstor$ vjerojatnosni prostor, vrijedi
    $\vjeroj{\emptyset} = 0$.
\end{prop}

\begin{proof}
    Definirajmo niz $(A_n)_{n \in \N}$ sa $A_1 = \Omega$ i $A_n = \emptyset$,
    $n \geq 2$. Vidimo da su skupovi me\dj usobno disjunktni, tako\dj er vidimo
    da vrijedi $\unija{n \in \N}{} A_n = \Omega$. Sada vidimo:
    \begin{align*}
        \vjeroj{\Omega} &= \vjeroj{\unija{n=1}{\infty} A_n} = \suma{n=1}{\infty}
            \vjeroj{A_n} = \vjeroj{A_1} + \suma{n=2}{\infty} \vjeroj{A_n} = \\
            &= \vjeroj{\Omega} + \suma{n=2}{\infty} \vjeroj{\emptyset}
    \end{align*}
    Odavde vrijedi:
    \begin{equation*}
        1 = 1 + \suma{n = 2}{\infty} \vjeroj{\emptyset}
    \end{equation*}
    Odakle nu\v zno vrijedi $\vjeroj{\emptyset} = 0$.

\end{proof}

\begin{prop} \label{sigAditImplKonAdit}
    Neka je $\prostorMjere$ prostor mjere, tada za me\dj usobno disjunktne skupove
    $A_1, \: A_2, \: \dots A_n$ vrijedi $\mjera{\unija{k=1}{n} A_k} = \suma{k = 1}{n}
    \mjera{A_k}.$
\end{prop}

Gornja propozicija \ref{sigAditImplKonAdit} zapravo tvrdi da $\sigma-$aditivnost
povla\' ci kona\v cnu aditivnost.

\begin{proof}
    Definirajmo niz $(B_n)_{n \in \N}$ na sljede\' ci na\' cin, za $k \leq n$
    $B_k := A_k$, za $k > n$ $B_k := \emptyset$. Primjetimo
    $\unija{n \in \N}{}B_n = \unija{K=1}{n} A_k$, tako\dj er $(B_k)_{k \in \N}$ su
    me\dj usobno disjunktni, tada vrijedi:
    \begin{align*}
        \mjera{\unija{k = 1}{n} A_k} &= \mjera{\unija{k = 1}{\infty} B_k} =
            \suma{k = 1}{n} \mjera{B_k} + \suma{k = n + 1}{\infty} \mjera{B_k}
            \suma{k=1}{n} \mjera{A_k} + \suma{k=n+1}{\infty}
            \underbrace{\mjera{\emptyset}}_{= 0} = \\
        &= \suma{k=1}{n} \mjera{A_k}.
    \end{align*}
\end{proof}

\begin{lm}  \label{beskDeMorgan}
    Neka je $A$ skup i $(B_n)_{n \in \N}$ niz skupova. Tada vrijedi $\unija{n=1}{\infty}
    (A \setminus B_n) = A \setminus \presjek{n=1}{\infty} B_n$.
\end{lm}

\begin{proof}
    \begin{itemize}
        \item[($\subseteq$)] Neka je $x \in \unija{n=1}{\infty} (A \setminus B_n) \implies
            \exists n_o \in \N$ takav da $x \in A \setminus B_{n_0}$, to jest $x \in A \land
            x \in B_{n_0}$, tada $x \notin B_{n_0} \implies x \notin \presjek{n=1}{\infty} B_n$.
            U suprotnom, ako bi vrijedilo $x \in \presjek{n=1}{\infty} B_n$, tada bi nu\v zno
            moralo vrijediti $x \in B_{n_0}$.
        \item[($\subseteq$)] Neka je sada $x \in A \setminus \presjek{n=1}{\infty} B_n$,
            tada $x \in A \land x \notin \presjek{n=1}{\infty} B_n$, ali tada postoji
            $n_0$ takav da $x \notin B_{n_0}$. U suprotnom, kada bi vrijedilo
            $x \in B_n \forall n \in \N$, tada bi $x \in \presjek{n=1}{\infty} B_n$.
            Budu\' ci imamo $x \in A \land x \notin B_{n_0} \implies x \in A \setminus
            B_{n_0} \implies x \in \unija{n=1}{\infty} (A \setminus B_n)$.
    \end{itemize}
\end{proof}

\begin{tm}[O osnovnim svojstvima mjere]
    Neka je $\prostorMjere$ prostor mjere, tada vrijedi:
    \begin{enumerate}[label={(\roman*)}]
        \item \emph{Monotonost}: Ako je $A \subseteq B$, vrijedi \label{tmOsSvVjerojMon}
            $\mjera{A} \leq \mjera{B}$
        \item \emph{Subaditivnost}: Ako je $A \subseteq \unija{n=1}{\infty} A_n$, tada
            vrijedi $\mjera{A} \leq \suma{n=1}{\infty} \mjera{A_n}$
        \item \emph{Neprekidnost odozdo}: Ako $A_n \nearrow A$, \label{tmOsSvVjerojNOdd}
            tj. $(A_n)_{n \in \N}$ td. $A_1 \subseteq A_2 \subseteq \dots$ i vrijedi
            $\unija{n \in \N}{} A_n = A$, tada vrijedi $\mjera{A} =
            \lim\limits_{n} \mjera{A_n}$.
        \item \emph{Neprekidnost odozgo}: Ako $A_n \searrow A$, tj. $(A_n)_{n \in \N}$ td.
            $A_1 \supseteq A_2 \supseteq \dots$, ako postoji $n_0 \in \N$ tako da
            $\mjera{A_{n_0}} < \infty$ i vrijedi $\presjek{n \in \N}{} A_n = A$,
            tada vrijedi $\mjera{A} = \lim\limits_{n} \mjera{A_n}$.
    \end{enumerate}
\end{tm}

\begin{proof}
    \begin{enumerate}[label={(\roman*)}]
        \item Neka je $A \subseteq B$, tada je $B = A \dot{\cup} (B \setminus A)$, a sada
            vrijedi $\mjera{B} = \mjera{A} + \mjera{B \setminus A}$, budu\' ci je
            $\mjera{B \setminus A} \geq 0$, vrijedi da je $\mjera{A} \leq \mjera{B}$.

        \item Definiramo $B_1 := A_1, \: B_k := A_k \setminus (\unija{i=1}{k} A_i),
            \: k \geq 1$. Tvrdim da je $(B_n)_{n \in \N}$ niz disjunktnih skupova.
            Neka su $m, \: n \in \N$, takvi da $m \neq n$, bez smanjenja op\' cenitosti
            mo\v zemo pretpostaviti da je $m < n$. Neka je $x \in B_n \implies x \in A_n 
            \land x \notin A_k, \: k < n \implies x \notin B_m$.
            Dakle $B_m \cap B_n = \emptyset.$ Tako\dj er tvrdimo da je $\unija{k=1}{n} A_k
            = \unija{k=1}{n} B_k, \: \forall n n \in \N$.
            Dokaz vr\v simo matemati\v ckom indukcijom po $n \in \N$.
            \begin{enumerate}
                \item[(B)] $A_1 = B_1$.
                \item[(P)] Neka je $n \in \N$, pretpostavimo da vrijedi $\unija{k=1}{n} A_k
                = \unija{k=1}{n} B_k$.
                \item[(K)] $\unija{k=1}{n+1} B_k = (\unija{k=1}{n} B_k) \cup B_{n+1}
                    = (\unija{k=1}{n} A_k) \cup \underbrace{(A_n \setminus
                    \unija{k=1}{n} A_k)}_{\textnormal{po definiciji}}
                    = \unija{k=1}{n+1} A_k$.
            \end{enumerate}
            Po principu matemati\v cke indukcije tvrdnja vrijedi za svaki $n \in \N$.
            Sada vrijedi $\unija{n \in \N}{} A_n = \unija{n \in \N}{} B_n$, primjetimo
            $x \in \unija{n \in \N}{} A_n \implies \exists n_0 \in \N$, takav da
            $x \in A_{n_0} \implies x \in \unija{k=1}{n_0} A_k = \unija{k=1}{n_0} B_k
            \implies x \in \unija{n \in \N}{} B_n$. Obratna inkluzija se dokazuje identi\v
            cno.
            Po pretpostavci je $A \subset \unija{n \in \N}{} A_n = \unija{n \in \N}{} B_n$,
            sada po \ref{tmOsSvVjerojMon} vrijedi $\mjera{A} \leq
            \mjera{\unija{n \in \N}{} B_n} = \suma{n=1}{\infty} B_n$. Tako\dj er budu\' ci
            vrijedi $B_k \subseteq A_k$, vrijedi i $\mjera{B_k} \leq \mjera{A_k}$, stoga
            imamo $\suma{n=1}{\infty} \mjera{B_n} \leq \suma{n=1}{\infty} \mjera{A_n}$.
            Odavde vidimo da vrijedi:
            \begin{equation*}
                \mjera{A} \leq \suma{n=1}{\infty} A_n.
            \end{equation*}
        \item Definiramo niz $(B_n)_{n \in \N}$ sa $B_1 := A_1, \: B_n = A_n \setminus
            A_{n-1}, \: n > 1$. Tvrdimo da je $(B_n)_{n \in \N}$ niz disjunktnih skupova
            te vrijedi $\unija{k=1}{n} A_k = \unija{k=1}{n} B_k$. Neka $m, \: n \in \N
            \: m \neq n$, bez smanjenja op\' cenitosti mo\v zemo pretpostaviti $m < n$.
            Neka je $x \in B_m \cap B_n, \: x \in B_n \implies x \notin A_{n-1}
            \implies x \in A_k, \: k \leq n-1$, jer je $(A_n)_{n \in \N}$ rastu\' c,
            a kako je $B_m \subseteq A_{n-1}$, \v sto je kontradikcija, dakle vrijedi
            $B_n \cap B_m = \emptyset$. Drugu tvrdnju dokazujemo po indukciji.
            \begin{enumerate}
                \item[(B)] $B_1 = A_1$.
                \item[(P)] Neka je $n \in \N$, tada vrijedi $\unija{k=1}{n} A_k
                    = \unija{k=1}{n} B_k$.
                \item[(K)] $\unija{k=1}{n+1} B_k = (\unija{k=1}{n} A_k) \cup B_{n+1}
                    = (\unija{k=1}{n} A_k) \cap (A_{n+1} \setminus A_n)
                    = \unija{k=1}{n+1} A_k$.
            \end{enumerate}
            Po Principu matemati\v cke indukcije tvrdnja vrijedi za svaki $n \in \N$.
            Vrijedi $A_n = \unija{k=1}{n} A_k$, jer je rije\v c o rastu\' cim nizu,
            dakle vrijedi $\unija{k=1}{n} A_k = \unija{k = 1}{n} B_k = A_n$.
            Sada imamo $\mjera{A} = \mjera{\unija{n \in \N}{} A_n} = \mjera{
                \unija{n \in \N}{} B_n} = \suma{n=1}{\infty} \mjera{B_n}
                = \lim\limits_{n} \suma{k=1}{n} \mjera{B_k} = \lim\limits_{n}
                \mjera{\unija{k=1}{n} B_k} = \lim\limits_{n} \mjera{A_n}$.
        \item Definirajmo niz $(B_n)_{n \in \N}$ sa $B_n := A_{n_0} \setminus
            A_{n_0 + n}$, tvrdimo da vrijedi $B_1 \subseteq B_2 \subseteq \dots$ te
            $\unija{n=1}{\infty} B_n = A_{n_0} \setminus \presjek{n=1}{\infty} A_n$.
            Neka je $x \in B_m = A_{n_0} \setminus A_{n_0+m} \implies x \in A_{n_0}
            \land x \notin A_{n_0+m} \implies x \in A_{n_0} \land x \in (A_{n_0+m})^c$.
            Budu\' ci je $A_{n_0+m} \supseteq A_{n_0+m+k} \implies (A_{n_0+m})^c
            \subseteq (A_{n_0+m+k})^c$.
            Dakle $x \in A_{n_0} \land x \in (A_{n_0+m})^c \implies x \in A_{n_0}
            \land x \in (A_{n_0+m+k})^c \implies x \in A_{n_0} \setminus A_{n_0+(m+k)}
            = B_{m+k}$.
            Posebno za $k=1$ tvrdnja slijedi to jest $B_n \subseteq B_{n+1},
            \forall n \in \N$, pa vrijedi $B_1 \subseteq B_2 \subseteq \dots$.
            Nadalje po \eqref{beskDeMorgan} vrijedi $\unija{n=1}{\infty}
            B_n = A_{n_0} \setminus \presjek{n=1}{\infty} A_{n_0+n}$.
            Budu\' ci je $(A_n)_{n \in \N}$ padaju\' ci niz, onda vrijedi
            $\presjek{n=1}{\infty} A_n = \presjek{n=1}{\infty} A_{n_0+n}$, stoga je
            $\unija{n=1}{\infty} B_n = A_{n_0} \setminus \presjek{n=1}{\infty} A_n$.
            Sada iz tvrdnje \ref{tmOsSvVjerojNOdd} vrijedi $\mjera{A_{n_0} \setminus
            \presjek{n=1}{\infty} A_n} = \mjera{\unija{n=1}{\infty} B_n}
            = \lim\limits_{n} \mjera{B_n} = \lim\limits_{n} \mjera{A_{n_0} \setminus
            A_{n_0+n}}$.
            Sada, zbog $\mjera{A_{n_0}} < \infty$, vrijedi $\mjera{A_{n_0}}
            - \mjera{\presjek{n=1}{\infty} A_n} = \lim\limits_{n}
            (\mjera{A_{n_0}} - \mjera{A_{n_0+n}}) = \mjera{A_{n_0}}
            - \lim\limits_{n} \mjera{A_{n_0+n}} = \mjera{A_{n_0}} - \lim\limits_{n}
            \mjera{A_n}$, dakle vrijedi
            \begin{equation*}
                \mjera{\presjek{n=1}{\infty} A_n} = \lim\limits_{n} \mjera{A_n}.
            \end{equation*}
    \end{enumerate}
\end{proof}

\begin{prop}
    Neka je $\izmjerivProstor$ izmjeriv prostor, neka je $\mu : \F \to \R$ kona\v cno
    aditivna mjera za koju vrijedi da za svaki niz $(A_n)_{n \in \N}$ iz $\F$ tako da
    $A_1 \supseteq A_2 \supseteq \dots \land \presjek{n=1}{\infty} A_n = \emptyset$,
    vrijedi $\lim\limits_{n} \mjera{A_n} = 0$. Tada je $\mu$ mjera.
\end{prop}

\begin{proof}
    Neka je $(A_n)_{n \in \N}$ disjunktan niz u $\F$. Tada za $n \in \N$ vrijedi:
    $\mjera{\unija{k=1}{\infty} A_k} = \mjera{(\unija{k=1}{n} A_k) \cup
    (\unija{k=n+1}{\infty} A_k)} = \mjera{\unija{k=1}{n} A_k}
    + \mjera{\unija{k=n+1}{\infty} A_k} = \suma{k=1}{n} \mjera{A_k}
    + \mjera{\unija{k=n+1}{\infty} A_k}$.
    Definirajmo niz $(B_n)_{n \in \N}$ sa $B_n := \unija{k=n}{\infty} A_k$.
    Tvrdimo da vrijedi $B_1 \supseteq B_2 \subseteq \dots$, te vrijedi
    $\presjek{n=1}{\infty} B_n = \emptyset$.
    Neka je $x \in B_{n+1} \iff x \in \unija{k=n}{\infty} A_k \implies x \in
    (\unija{k=n+1}{\infty} A_k) \cup A_n = \unija{k=n}{\infty} A_n = B_n$, dakle
    $B_n \supseteq B_{n+1}$.
    Pretpostavimo da je $x \in \presjek{n=1}{\infty}$, tada $x \in B_n \forall n \in \N$,
    to jest $x \in \unija{k=n}{\infty} A_k$, budu\' ci su $(A_n)_{n \in \N}$ disjunktni,
    postoji jedinstveni $n_0 \in \N$ takav da $x \in A_{n_0}$ ali tada $x \notin
    \unija{k=n_0+1}{\infty} A_k \implies x \notin B_{n_0+1} \implies x \notin
    \presjek{n=1}{\infty} B_n$, \v sto je kontradikcija, dakle $\presjek{n=1}{\infty}
    B_n = \emptyset$.
    Tada, po pretpostavci, vrijedi $\lim\limits_n \mjera{B_n} = \lim\limits_n
    \mjera{\unija{k=n}{\infty} A_k} = 0$.
    Sada imamo $\mjera{\unija{k=1}{\infty} A_k} = \suma{k=1}{n} \mjera{A_k}
    + \mjera{\unija{k=n+1}{\infty} A_k}$. Napadnemo li izraz limesom po n, dobijemo
    sljede\' ci izraz.
    \begin{equation*}
        \mjera{\unija{k=1}{\infty} A_k} = \lim\limits_n \big( \suma{k=1}{n}
        \mjera{A_k} + \mjera{\unija{k=n+1}{\infty} A_k} \big).
    \end{equation*}
    Malo mentalne akrobatike, limes na lijevoj strani postoji (jer izraz ne ovisi o n),
    stoga mora postojati i na desnoj, me\dj utim jo\v s ne znamo smijemo li limes suma
    razbiti na sumu limesa, da bi to mogli, svaki od sumanada mora imati limes, budu\'
    ci je $\lim\limits_n \mjera{\unija{k=1}{\infty}A_k} = 0$, onda vrijedi da je
    $\lim\limits_n (\suma{k=1}{n} \mjera{A_k}) = \suma{k=1}{\infty} \mjera{A_k}$
    postoji i vrijedi:
    \begin{equation*}
        \mjera{\unija{k=1}{\infty} A_k} = \suma{k=1}{\infty} \mjera{A_k}.
    \end{equation*}
    Primjetimo ovaj \v cinjenicu da zadnji limes postoji mo\v zemo opravdati tako
    da promatramo izraz $\mjera{\unija{k=1}{\infty} A_k}
    - \mjera{\unija{k=n+1}{\infty} A_k} = \suma{k=1}{n} \mjera{A_k}$. Sada na lijevoj strani
    imamo dva izraza koji imaju limese, pa kada napadnemo limesom mo\v zemo ga razbiti.

\end{proof}