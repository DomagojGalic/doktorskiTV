% martingali i vremena zaustavljanja, cjelina 5.5 -> predavanje 25

\chapter{Martingali i vremena zaustavljanja}

Neka je $\uptau$ vrijeme zaustavljanja u odnosu na filtraciju $\indFamilija{\famF_n}{n \in \nat_0}$.
Neka je $\famF_\infty := \indSigAlg{\famF_n}{n \in \nat}$.
Dakle $\indFamilija{\famF_n}{n \in \nat_0 \cup \{+\infty\}}$ je filtracija.
Neka je $\indFamilija{\famF_n}{n \in \nat_0 } \subseteq L^1 (\masP)$ i neka je $\indFamilija{\famF_n}{n \in \nat_0}$-adaptiran.
Ove oznake vrijede u cijelom poglavlju.

Uo\v cimo da je u teoremu \ref{tm:24.1} bilo bitno samo da su slu\v cajne varijable $\varepsilon_k$ $\famF_k$-izmjerive.
Uzemo li $\varepsilon_k := \karaktFja_{\{\uptau \geq k\}}$, direktno iz teorema \ref{tm:24.1} dobivamo

\begin{tm}  \label{tm:25.1}
    Ako je $\niz{X_n}{n \in \nat_0}$ submartingal, tada je i $\niz{X_{\uptau \land n}}{n \in \nat_0}$ submartingal.
    Ako smo krenuli od martingala, dobit \' cemo martingal.
\end{tm}

\begin{kor} \label{kor:25.2}
    Ako je $\niz{X_n}{n \in \nat_0}$ submartingal i $\uptau \leq m$ gotovo sigurno, za svaki $m \in \nat$, tada je
    \begin{equation*}
        \masE X_0 \leq \masE X_\uptau \leq \masE X_m.
    \end{equation*}
\end{kor}

\begin{proof}
    Zbog $\uptau \leq m$ ima smisla
    \begin{equation*}
        X_\uptau (\omega) := X_{\uptau (\omega)} (\omega).
    \end{equation*}
    Po teoremu \ref{tm:25.1} $\niz{X_{\uptau \land n}}{n \in \nat_0}$ je submartingal \v sto daje
    \begin{equation*}
        \masE X_0 = \masE X_{\uptau \land 0} \leq \masE X_{\uptau \land m} = \masE X_\uptau.
    \end{equation*}
    Posebno to daje i $X_\uptau \in L^1 (\masP)$.
    Uzemo li u teoremu \ref{tm:24.1}
    \begin{equation*}
        \varepsilon_k := \karaktFja_{\{ \uptau < k \}} = \karaktFja_{\{ \uptau \leq k - 1 \}}
    \end{equation*}
    dobijemo da je $X_n - X_{\uptau \land n}$ tako\dj er submartingal, \v sto daje
    \begin{equation*}
        \masE X_m - \masE X_\uptau = \masE X_m - \masE X_{\uptau \land m} \geq \masE X_0 - \masE X_{\uptau \land 0} = 0.
    \end{equation*}
\end{proof}

Direktno iz ovoga dobijemo

\begin{kor} \label{kor:25.3}
    Ako je $\niz{X_n}{n \in \nat_0}$ submartingal, $m \in \nat$ te ako su $\uptau_1 \leq \uptau_2 \leq m$ vremena zaustavljanja, tada vrijedi
    \begin{equation*}
        \masE X_0 \leq \masE X_{\uptau_1} \leq \masE X_{\uptau_2} \leq \masE X_m.
    \end{equation*}
\end{kor}

\begin{pr}  \label{pr:25.4}
    Neka je $\niz{S_n}{n \in \nat_0}$ jednostavna simetri\v cna slu\v cajna \v setnja ($p = q = \frac{1}{2}$) to jest, $(S_n)$ je martingal.
    Neka je
    \begin{equation*}
        \uptau := \min \: \indFamilija{n \in \nat}{S_n = 0}.
    \end{equation*}
    Krenimo od $S_0 \equiv 1$.
    Lako se vidi da je $\uptau$ vrijeme zaustavljanja, kona\v cno je, ali nije ome\dj eno.
    Uo\v cimo $S_\uptau = 0$, pa slijedi
    \begin{equation*}
        \masE S_0 = 1 > 0 = \masE S_\uptau.
    \end{equation*}
    O\v cito korolar \ref{kor:25.3} u ovom slu\v caju ne vrijedi.
\end{pr}

Uz koje uvjete vrijedi ja\v ca verzija korolara \ref{kor:25.3}?

\begin{tm}["Optimal sampling theorem"]  \label{tm:25.5}
    Neka je $\niz{X_n}{n \in \nat_0}$ submartingal.
    Neka je $\uptau_0 \leq \uptau_1 \leq \ldots$ niz kona\v cnih vremena zaustavljanja.
    Neka je $Y_n := X_{\uptau_n}$, $n \in \nat_0$.
    Ako vrijedi
    \begin{equation}    \label{jed:25.6}
        \masE [|Y_n|] < +\infty, \quad \forall n \in \nat_0,
    \end{equation}
    te
    \begin{equation}    \label{jed:25.7}
        \liminf\limits_{k \to \infty} \int\limits_{\{ \uptau_n > k \}} |X_k| \: d \masP = 0, \quad \forall n \in \nat_0,
    \end{equation}
    tada je $\niz{Y_n}{n \in \nat}$ submartingal u odnosu na filtraciju $\indFamilija{\famF_{\tau_n}}{n \in \nat_0}$.
    Ako je $(X_n)$ martingal, tada je i $(Y_n)$ martingal.
\end{tm}

\begin{nap} \label{nap:25.5-1}
    Primjetimo definicija $Y_n$ ima smisla, zato \v sto su $\uptau_n$ kona\v cni.
\end{nap}

\begin{proof}
    Za $A \in \famF_{\uptau_n}$ slijedi
    \begin{equation*}
        A \cap \{ \uptau_{n + 1 } \leq k \} = \unija{i = 1}{k} \big( A \cap \{ \uptau_n = i \} \big) \cap \{ \uptau_{n + 1} \leq k\}.
    \end{equation*}
    Budu\' ci je
    \begin{equation*}
        A \cap \{ \uptau_n = i \} \in \famF_i \subseteq \famF_k
    \end{equation*}
    i vrijedi
    \begin{equation*}
        \{ \uptau_{n + 1} \leq k \} \in \famF_k,
    \end{equation*}
    slijedi da je $A \in \famF_{\uptau_{n + 1}}$, dakle $\{\famF_{\uptau_n}\}$ je filtracija.

    Kako je proces adaptiran i u $L^1 (\masP)$ po \eqref{jed:25.6}, treba pokazati jo\v s samo da za svaki $A \in \famF_{\uptau_{n}}$ vrijedi
    \begin{equation*}
        \int\limits_A Y_{n + 1} \: d \masP
        \begin{smallmatrix}
            \geq\\
            (=)
        \end{smallmatrix}
        \int\limits_A Y_n \: d \masP
    \end{equation*}
    Budu\' ci je
    \begin{equation*}
        A = \unija{j \in \nat_0}{} \big( A \cap \{ \uptau_n = j \} \big),
    \end{equation*}
    \v sto je disjunktna unija, dovoljno je pokazati tvrdnju za
    \begin{equation*}
        D_j := A \cap \{ \uptau_n = j \}, \quad D_j \in \famF_j.
    \end{equation*}
    Uo\v cimo, za $k > j$ imamo, zbog
    \begin{equation*}
        \uptau_n = j \quad \implies \quad \uptau_{n + 1} \geq j
    \end{equation*}

    \begin{equation*}
        \int\limits_{D_j} Y_{n + 1} \: d \masP = \suma{i = j}{k} \int\limits_{D_j \cap \{ \uptau_{n + 1} = i \}} Y_{n + 1} \: d \masP + \int\limits_{D_j \cap \{ \uptau_{n + 1} > k\}} Y_{n + 1} \: d \masP,
    \end{equation*}
    a vrijedi
    \begin{equation*}
        \int\limits_{D_j \cap \{ \uptau_{n + 1} > k \}} Y_{n + 1} \: d \masP = \int\limits_{D_j \cap \{ \uptau_{n + 1} > k \}} X_k \: d \masP - \int\limits_{D_j \cap \{ \uptau_{n + 1} > k \}} (X_k - Y_{n + 1}) \: d \masP,
    \end{equation*}
    dok
    \begin{equation*}
        \int\limits_{D_j \cap \{ \uptau_{n + 1} = i \}} Y_{n + 1} \: d \masP = \int\limits_{D_j \cap \{ \uptau_{n + 1} = i \}} X_i \: d \masP.
    \end{equation*}
    Uzmemo li $i = k$ \v clan i $\int X_k \: d \masP$ dobijemo
    \begin{equation*}
        \int\limits_{D_j \cap \{ \uptau_{n + 1} = k \}} X_k \: d \masP + \int\limits_{D_j \cap \{ \uptau_{n + 1} > k \}} X_k \: d \masP = \int\limits_{D_j \cap \{ \uptau_{n + 1} \geq k \}} X_k \: d \masP
        \begin{smallmatrix}
            \geq\\
            (=)
        \end{smallmatrix}
        \int\limits_{D_j \cap \{ \uptau_{n + 1} \geq k \}} X_{k - 1} \: d \masP,
    \end{equation*}
    zbog
    \begin{equation*}
        \{ \uptau_{n + 1} \geq k \} = \{ \uptau_{n + 1} \leq k - 1 \}^c \in \famF_{k - 1} \supseteq \famF_j.
    \end{equation*}
    No,
    \begin{equation*}
        \int\limits_{D_j \cap \{ \uptau_{n + 1} \geq k \}} X_{k - 1} \: d \masP = \int\limits_{D_j \cap \{ \uptau_{n + 1} > k - 1 \}} X_{k - 1} \: d \masP,
    \end{equation*}
    pa ga na isti na\v cin kombiniramo sa $i = k - 1$, te dobijemo
    \begin{equation*}
        \int\limits_{D_j \cap \{ \uptau_{n + 1} > k - 2 \}} X_{k - 2} \: d \masP.
    \end{equation*}
    Induktivno nastavljamo dalje i dobivamo
    \begin{equation*}
        \int\limits_{D_j} Y_{n + 1} \: d \masP
        \begin{smallmatrix}
            \geq\\
            (=)
        \end{smallmatrix}
        \int\limits_{D_j \cap \{ \uptau_{n + 1} \geq j \}} X_j \: d \masP - \int\limits_{D_j \cap \{ \uptau_{n + 1} > k \}} (X_k - Y_{n + 1}) \: d \masP.
    \end{equation*}

    Zbog \eqref{jed:25.7} postoji podniz $(k_s)$ takav da vrijedi
    \begin{equation*}
        \int\limits_{D_j \cap \{ \uptau_{n + 1} > k_s \}} X_{k_s} \: d \masP \xrightarrow[s \to \infty]{} 0.
    \end{equation*}
    Zbog
    \begin{equation*}
        \{ \uptau_{n + 1} > k \} \searrow \varnothing,
    \end{equation*}
    slijedi
    \begin{equation*}
        \int\limits_{D_j \cap \{ \uptau_{n + 1} > k_s \}} Y_{n + 1} \: d \masP \xrightarrow[s \to \infty]{} 0.
    \end{equation*}
    Budu\' ci je $D_j \subseteq \{ \uptau_n = j \}$ slijedi $D_j \cap \{ \uptau_{n + 1} \geq j \} = D_j$, te je $X_j = Y_n$ na $D_j$.
    Pustimo li $s \to +\infty$, dobivamo
    \begin{equation*}
        \int\limits_{D_j} Y_{n + 1} \: d \masP
        \begin{smallmatrix}
            \geq\\
            (=)
        \end{smallmatrix}
        \int\limits_{D_j} Y_n \: d \masP.
    \end{equation*}
\end{proof}

\begin{zad} \label{zad:25.8}
    Ako za svaki $n \in \nat$ postoji $m_n \in \nat$ takav da je $\uptau_n \leq m_n$ gotovo sigurno, tada su \eqref{jed:25.6} i \eqref{jed:25.7} ispunjeni.
\end{zad}

\begin{nap} \label{nap:25.9}
    \begin{enumerate}[label=(\alph*)]
        \item \label{nap:25.9.1}
        Nije te\v sko pokazati da uniformna integrabilnost daje \eqref{jed:25.6} i \eqref{jed:25.7}.
        \item \label{nap:25.9.2}
        Ako je
        \begin{equation*}
            Z := \sup\limits_{n \in \nat} \: |X_n| \in L^1 (\masP),
        \end{equation*}
        tada se direktno vidi (ili pomo\' cu \ref{nap:25.9.1}) da su opet zadovoljeni \eqref{jed:25.6} i \eqref{jed:25.7}.
        \item \label{25.9.3}
        Ako imamo kona\v can $\masT$ u prethodnim slu\v cajevima, to mo\v zemo pro\v siriti sa $\masT = \{0, 1, \ldots, n\}$ na $\masT = \nat_0$, pomo\' cu
        \begin{equation*}
            X_m := X_n, \quad m > n.
        \end{equation*}
        U tom slu\v caju je opet ispunjeno \eqref{jed:25.6} i \eqref{jed:25.7}.
    \end{enumerate}
\end{nap}

Uo\v cimo da su uvjeti sljede\' ceg teorem ispunjeni ako je $(X_n)$ uniformno integrabilan submartingal.

\begin{tm}  \label{tm:25.10}
    Neka je $\niz{X_n}{n \in \nat_0 \cup \{+ \infty\}}$ submartingal (martingal) u odnosu na filtraciju $\indFamilija{\famF_n}{n \in \nat_0 \cup \{+ \infty\}}$.
    Ako je $\uptau_0 \leq \uptau_1 \leq \ldots$ rastu\' ci niz kona\v cnih vremena zaustavljanja, tada je
    \begin{equation*}
        \niz{Y_n := X_{\uptau_n}}{n \in \nat_0}
    \end{equation*}
    submartingal (martingal) u odnosu na $\indFamilija{\famF_{\uptau_n}}{n \in \nat_0}$.
\end{tm}

\begin{proof}
    \begin{enumerate}[label=(\alph*)]
        \item   \label{proof:25.10.1}
        U slu\v caju martingala imamo
        \begin{equation*}
            X_n = \masE [X_\infty | \famF_n]
        \end{equation*}
        i tvrdnja slijedi iz teorema \ref{tm:25.5} i napomene \ref{nap:25.9} \ref{nap:25.9.1}.
        \item   \label{proof:25.10.2}
        $X_n \leq 0$, $n \in \nat_0$, $X_\infty \equiv 0$.
        Za svaki fiksni $n \in \nat_0$ promatramo
        \begin{equation*}
            \sigma_k := \uptau_n \land k, \quad k \in \nat_0.
        \end{equation*}
        O\v cito su $\sigma_k$ rastu\' ca vremena zaustavljanja i
        \begin{equation*}
            X_{\sigma_k} \xrightarrow[k \to \infty]{} Y_n = X_{\uptau_n}.
        \end{equation*}
        Po Fatouovoj lemi
        \begin{equation*}
            \limsup\limits_{k \to \infty} \int\limits_\Omega X_{\sigma_k} \: d \masP \leq \int\limits_\Omega Y_n \: d \masP \leq 0.
        \end{equation*}
        Prema zadatku \ref{zad:25.8} $(X_{\sigma_k})$ je submartingal, pa slijedi
        \begin{equation*}
            \int\limits_\Omega X_{\sigma_k} \: d \masP \geq \int\limits_\Omega X_{\sigma_0} \: d \masP = \int\limits_\Omega X_0 \: d \masP,
        \end{equation*}
        koji je kona\v can.
        To daje $Y_n \in L^1 (\masP)$.
        
        Neka je sada $k$ fiksan.
        Sada po zadatku \ref{zad:25.8} slijedi
        $\niz{X_{\uptau_n \land k}}{n \in \nat_0}$ je submartingal u odnosu na $\{ \famF_{\uptau_n \land k} \}$.
        Dakle za $A \in \famF_{\uptau_n \land k}$ je
        \begin{equation*}
            \int\limits_A X_{\uptau_n \land k} \: d \masP \leq \int\limits_A X_{\uptau_{n + 1} \land k} \: d \masP.
        \end{equation*}
        Uo\v cimo da za $A \in \famF_{\uptau_n}$ vrijedi
        \begin{equation*}
            A \cap \{ \uptau_n \leq k \} \cap \{ \uptau_n \land k \leq i \} =
            \begin{cases}
                A \cap \{ \uptau_n \leq i \} ,& i \leq k\\
                A \cap \{ \uptau_n \leq k \} ,& i > k
            \end{cases},
        \end{equation*}
        \v sto daje
        \begin{equation*}
            A \cap \{ \uptau_n \leq k \} \in \famF_{\uptau_n \land k}.
        \end{equation*}
        Dakle, za $A \in \famF_{\uptau_n}$
        \begin{equation*}
            \int\limits_{A \cap \{ \uptau_n \leq k \}} X_{\uptau_n \land k} \: d \masP \leq \int\limits_{A \cap \{ \uptau_n \leq k \}} X_{\uptau_{n + 1} \land k} \: d \masP.
        \end{equation*}
        Uo\v cimo da je na $\{ \uptau_n \leq k \}$, $\uptau_n \leq k = \uptau_n$, ta da vrijedi
        \begin{equation*}
            \begin{gathered}
                \{ \uptau_{n + 1} \leq k \} \subseteq \{ \uptau_n \leq k \},\\
                X_{\uptau_{n + 1} \land k} \leq 0,
            \end{gathered}
        \end{equation*}
        \v sto daje
        \begin{equation*}
            \int\limits_{A \cap \{ \uptau_n \leq k \}} X_{\uptau_n} \: d \masP \leq \int\limits_{A \cap \{ \uptau_{n + 1} \leq k \}} X_{\uptau_{n + 1}} \: d \masP.
        \end{equation*}
        Uz $k \to \infty$, dobivamo tvrdnju.
        \item   \label{proof:25.10.3}
        Op\' ci slu\v caj.
        Neka je
        \begin{equation*}
            Z_n := \masE [X_\infty | \famF_n]
        \end{equation*}
        i neka je
        \begin{equation*}
            W_n := X_n - Z_n \quad \quad (X_n = W_n + Z_n).
        \end{equation*}
        Tada $(Z_n)$ pada pod \ref{proof:25.10.1}, a $(W_n)$ pod \ref{proof:25.10.2}, te tvrdnja vrijedi.
    \end{enumerate}
\end{proof}

\begin{zad} \label{zad:25.11}
    Neka je $A \in \famF_\infty$.
    Doka\v zite da
    \begin{equation*}
        \masE [\karaktFja_A | \famF_n] \xrightarrow[n \to \infty]{g.s.} \karaktFja_A
    \end{equation*}
    i razmislite o interpretaciji ovog rezultata.
\end{zad}