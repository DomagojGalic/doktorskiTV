% martingali i vremena zaustavljanja, cjelina 5.5 -> predavanje 25

\chapter{Martingali i vremena zaustavljanja}

Neka je $\uptau$ vrijeme zaustavljanja u odnosu na filtraciju $\indFamilija{\famF_n}{n \in \nat_0}$.
Neka je $\famF_\infty := \indSigAlg{\famF_n}{n \in \nat}$.
Dakle $\indFamilija{\famF_n}{n \in \nat_0 \cup \{+\infty\}}$ je filtracija.
Neka je $\indFamilija{\famF_n}{n \in \nat_0 } \subseteq L^1 (\masP)$ i neka je $\indFamilija{\famF_n}{n \in \nat_0}$-adaptiran.
Ove oznake vrijede u cijelom poglavlju.

Uo\v cimo da je u teoremu \ref{tm:24.1} bilo bitno samo da su slu\v cajne varijable $\varepsilon_k$ $\famF_k$-izmjerive.
Uzemo li $\varepsilon_k := \karaktFja_{\{\uptau \geq k\}}$, direktno iz teorema \ref{tm:24.1} dobivamo

\begin{tm}  \label{tm:25.1}
    Ako je $\niz{X_n}{n \in \nat_0}$ submartingal, tada je i $\niz{X_{\uptau \land n}}{n \in \nat_0}$ submartingal.
    Ako smo krenuli od martingala, dobit \' cemo martingal.
\end{tm}

\begin{kor} \label{kor:25.2}
    Ako je $\niz{X_n}{n \in \nat_0}$ submartingal i $\uptau \leq m$ gotovo sigurno, za svaki $m \in \nat$, tada je
    \begin{equation*}
        \masE X_0 \leq \masE X_\uptau \leq \masE X_m.
    \end{equation*}
\end{kor}