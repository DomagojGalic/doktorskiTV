% p12 - slabi zakoni velikih brojeva

\chapter{Slabi zakoni velikih brojeva}

U ovom poglavlju $\niz{X_n}{n \in \nat}$ je niz slu\v cajnih varijabli na vjerojatnosnom prostoru $\vjerojatnosniProstor$ i $S_n := X_1 + \ldots + X_n$, za svaki $n \in \nat$.
Prvo promatramo \v sto se doga\dj a ako postoje drugi momenti (vidi poglavlje \ref{dist_sl_elem}).
Osnovna tehnika u tom slu\v caju je poznata \emph{\v Cebi\v sevljeva nejednakost}:

\begin{prop}    \label{prop:12.1}
    \quad
    \begin{enumerate}[label=(\roman*)]
        \item   \label{prop:12.1.1}
        Ako je $X \geq 0$ slu\v cajna varijabla takva da je $0 < \masE X < +\infty$, tada, za svaki $r > 0$ vrijedi
        \begin{equation*}
            \vjeroj{X > r \: \masE X} \leq \frac{1}{r}.
        \end{equation*}
        \item   \label{prop:12.1.2}
        Ako je $X$ slu\v cajna varijabla i $\masE [X^2] < +\infty$, tada, za svaki $\varepsilon > 0$ vrijedi:
        \begin{equation*}
            \vjeroj{|X - \masE X| > \varepsilon} \leq \frac{\Var X}{\varepsilon^2}.
        \end{equation*}
    \end{enumerate}
\end{prop}

\begin{proof}
    \quad
    \begin{enumerate}[label=(\roman*)]
        \item Bez smanjenja op\' cenitosti mo\v zemo pretpostaviti da je $\masE X = 1$.
        Uo\v cimo da je $r \cdot \karaktFja_{\{X > r\}} \leq X$.
        \item Iz \ref{prop:12.1.1} za $Y:= |X - \masE X|^2$ i za $r:= \frac{\varepsilon^2}{\Var X}$, jer vrijedi:
        \begin{equation*}
            \begin{aligned}
                \vjeroj{|X - \masE X| > \varepsilon} &= \vjeroj{Y > \varepsilon^2} = \vjeroj{Y > r \: \masE Y}\\
                & \leq \frac{1}{r} = \frac{\Var X}{\varepsilon^2}.
            \end{aligned}
        \end{equation*}
    \end{enumerate}
\end{proof}

Podsjetimo se da ako $X$, $Y$ imaju kona\v cne druge momente, tada imamo sljede\' cu defninciju.

\begin{defn}    \label{defn:12.1-1}
    Neka su $X, \; Y \in L^2(\masP)$ slu\v cajne varijable, tada definiramo \emph{kovarijancu} od $X$ i $Y$ kao:
    \begin{equation}    \label{jed:12.2}
        \Cov \urePar{X}{Y} := \frac{\masE [X \cdot Y] - \masE X \cdot \masE Y}{\sqrt{\Var X} \cdot \sqrt{\Var Y}}.
    \end{equation}
    Ako je $\Cov \urePar{X}{Y} = 0$, ka\v zemo da su $X$ i $Y$ \emph{nekorelirane}.
\end{defn}

\begin{zad} \label{zad:12.3}
    Doka\v zi da vrijedi:
    \begin{equation*}
        X, \; Y \in L^2(\masP), \textnormal{ nezavisne} \implies \Cov \urePar{X}{Y} = 0.
    \end{equation*}
    Poka\v zi da obrat op\' cenito ne vrijedi.
\end{zad}

\begin{zad} \label{zad:12.4}
    Neka su $X_1, \ldots, X_n \in L^2(\masP)$ i neka vrijedi $\Cov \urePar{X_i}{X_j} = 0$, za svaki $i \neq j$, tada vrijedi:
    \begin{equation*}
        \Var (X_1 + \ldots + X_n) = \suma{k = 1}{n} \Var X_k.
    \end{equation*}
\end{zad}

\begin{tm}  \label{tm:12.5}
    Ako je $X_n \in L^2 (\masP)$, za svaki $n \in \nat$, i ako vrijedi $\Cov \urePar{X_n}{X_m} = 0$, za svaki $m, \; n \in \nat$, te ako postoji $0 < c < + \infty$ takva da je $\Var (X_n) \leq c$, za svaki $n \in \nat$, tada
    \begin{equation*}
        \frac{S_n - \masE S_n}{n} \xrightarrow[n \to \infty]{\masP} 0.
    \end{equation*}
\end{tm}

\begin{proof}
    Neka je $Z_n := \frac{S_n}{n}$.
    Tada je $\frac{S_n - \masE S_n}{n} = Z_n - \masE Z_n$ i vrijedi $\Var Z_n = \frac{1}{n^2} \suma{k = 1}{n} \Var X_k \leq \frac{n \: c}{n^2} = \frac{C}{n}$.
    Prema propoziciji \ref{prop:12.1} \ref{prop:12.1.2} imamo
    \begin{equation*}
        \vjeroj{|Z_n - \masE Z_n| > \varepsilon} \leq \frac{\Var Z_n}{\varepsilon^2} \leq \frac{c}{\varepsilon^2} \cdot \frac{1}{n} \xrightarrow[n \to \infty]{} 0.
    \end{equation*}
\end{proof}

\begin{zad} \label{zad:12.6}
    Neka je $\niz{X_n}{n \in \nat}$ niz nezavisnih slu\v cajnih varijabli i $X_n \in L^2(\masP)$, za svaki $n \in \nat$.
    \begin{enumerate}[label=(\alph*)]
        \item Ako je $\lim\limits_{n \to \infty} \suma{k = 1}{n} \Var X_k = 0$, tada vrijedi
        \begin{equation*}
            \frac{S_n - \masE S_n}{n} \xrightarrow[n \to \infty]{\masP} 0.
        \end{equation*}
        \item Ako je $\masE X_n = \mu$ i ako je $\Var X_n = \sigma^2$, za svaki $n \in \nat$, tada je
        \begin{equation*}
            \frac{S_n}{n} \xrightarrow[n \to \infty]{\masP} \mu.
        \end{equation*}
        \item Ako je $Y_n \sim B \urePar{n}{p}$, za svaki $n \in \nat$, tada je
        \begin{equation*}
            \frac{Y_n}{n} \xrightarrow[n \to \infty]{\masP} p.
        \end{equation*}
    \end{enumerate}
\end{zad}

Vidimo da iz teorema \ref{tm:12.5} slijedi:

\begin{kor}[\v Cebi\v sevljev slabi zakon] \label{kor:12.6}
    Ako su $\niz{X_n}{n \in \nat}$ nezavisne slu\v cajne varijable u $L^2(\masP)$ i ako postoji $0 < c < +\infty$, takav da je $\Var X_n \leq c$, za svaki $n qin \nat$, tada je
    \begin{equation*}
        \frac{1}{n} (S_n - \masE S_n) \xrightarrow[n \to \infty]{\masP} 0.
    \end{equation*}
\end{kor}

Mo\v zemo li u nezavisnom slu\v caju ispustiti pretpostavku o postojanju 2. momenta?
Pogledat \' cemo $\iid$ slu\v caj, ali nam prvo trebaju neki pomo\' cni rezultati.
Biti \' ce nam korisna sljede\' ca definicija.

\begin{defn}    \label{defn:12.6-1}
    Neka su $(a_n)$, $(b_n)$ nizovi realnih brojeva, ka\v zemo da je niz $a_n$ \emph{malo o} od niza $b_n$, u oznaci $a_n = o(b_n)$, ako vrijedi
    \begin{equation}    \label{jed:12.7}
        \lim\limits_{n \to \infty} \frac{a_n}{b_n} = 0.
    \end{equation}
\end{defn}

Posebno, vidimo da je
\begin{equation*}
    a_n = o(1) \iff \lim\limits_{n \to \infty} a_n = 0.
\end{equation*}

\begin{defn}    \label{defn:12.7-1}
    Za slu\v cajnu varijablu $X$ realni broj $m(X)$ zovemo \emph{medijanom} ako vrijedi
    \begin{equation*}
        \vjeroj{X \leq m(X)} \geq \frac{1}{2} \quad \textnormal{i} \quad \vjeroj{X \geq m(X)} \geq \frac{1}{2}.
    \end{equation*}
\end{defn}

\begin{lm}[L\' evyjeva nejednakost]  \label{lm:12.8}
    \quad \\
    Ako su $\niz{X_n}{n \in \nat}$ nezavisne slu\v cajne varijable, tada, za svaki $\varepsilon > 0$ vrijedi:
    \begin{equation*}
        \begin{gathered}
            \vjeroj{\max\limits_{1 \leq j \leq n} [S_j - m(S_j - S_n)] \geq \varepsilon } \leq 2 \: \vjeroj{S_n \geq \varepsilon}\\
            \vjeroj{\max\limits_{1 \leq j \leq n} |S_j - m(S_j - S_n)| \geq \varepsilon } \leq 2 \: \vjeroj{|S_n| \geq \varepsilon}.
        \end{gathered}
    \end{equation*}
\end{lm}

\begin{proof}
    Koriste\' ci prvu nejednakost na $(X_n)$ i na $(-X_n)$, te koriste\' ci
    \begin{equation*}
        m(-X) = -m(X),    
    \end{equation*}
    dobivamo drugu nejednakost.
    Doka\v zimo prvu nejednakost.
    Stavimo
    \begin{equation*}
        T:= \min ( \skup{j \in \{1, \ldots, n\}}{S_j - m(S_j - S_n) \geq \varepsilon} ),
    \end{equation*}
    ako takav minimum postoji, odnosno $T:= n + 1$, ako ne postoji.
    Nadalje, stavimo
    \begin{equation*}
        B_j := \{ m(S_j - S_n) \geq S_j - S_n \}, \quad 1 \leq j \leq n.
    \end{equation*}
    Stoga je $\vjeroj{B_j} \geq \frac{1}{2}$, za svaki $1 \leq j \leq n$.
    Uo\v cimo da je $\{ T = j \} \in \sigAlg{X_1, \ldots X_j}$, te $B_j \in \sigAlg{X_{j + 1}, \ldots, X_n}$, \v sto daje da su $\{T = j\}$ i $B_j$ nezavisni.
    Budu\' ci da su $\{T = j\}$ disjunktni, iz $\{ S_n \geq \varepsilon \} \supseteq \unija{j = 1}{n} (B_j \cap \{ T = j \})$, dobivamo
    \begin{equation*}
        \begin{aligned}
            \vjeroj{S_n \geq \varepsilon} &\geq \suma{j = 1}{n} \vjeroj{B_j \cap \{ T = j \}} = \suma{j = 1}{n} \vjeroj{B_j} \cdot \vjeroj{T = j}\\
            &\geq \frac{1}{2} \suma{j = 1}{n} \vjeroj{T = j} = \frac{1}{2} \vjeroj{1 \leq T \leq n}.
        \end{aligned}
    \end{equation*}
\end{proof}

Sljede\' ci teorem daju potpuni odgovor u slu\v caju $\iid$ slu\v cajnog niza.

\begin{tm}[W. Feller]   \label{tm:12.9}
    Neka je $\niz{X_n}{n \in \nat}$ niz nezavisnih, jednako distribuiranih slu\v cajnih varijabli.
    Tada postoji niz brojeva $\niz{b_n}{n \in \nat}$, takav da vrijedi
    \begin{equation*}
        \frac{S_n}{n} - b_n \xrightarrow[n \to \infty]{\masP} 0 \iff \lim\limits_{n \to \infty} n \: \vjeroj{|X_1| > n} = 0.
    \end{equation*}
    U tom slu\v caju je
    \begin{equation*}
        \lim\limits_{n \to \infty} (b_n - \masE [ X_1 \cdot \karaktFja_{\{ |X_1| \leq n \}}]) = 0.
    \end{equation*}
\end{tm}

\begin{zad} \label{zad:12.10}
    Ako za slu\v cajne varijable $\niz{Z_n}{n \in \nat}$ postoji niz brojeva $0 < b_n \nearrow + \infty$ takav da vrijedi
    \begin{equation*}
        \frac{Z_n}{b_n} \xrightarrow[n \to \infty]{\masP} 0,
    \end{equation*}
    tada vrijedi
    \begin{equation*}
        \max\limits_{1 \leq j \leq n} |m(Z_j - Z_n)| = o(b_n).
    \end{equation*}
\end{zad}

\begin{proof}{(teorema \ref{tm:12.9})}
    \begin{itemize}
        \item[$\implies$] Pretpostavimo da postoji niz $(c_n)$ takav da vrijedi $\frac{S_n - c_n}{n} \xrightarrow[n \to \infty]{\masP} 0$.
        Stavimo $c_0 = 0$ i stavimo $d_n := c_n - c_{n - 1}, \; n \in \nat$.
        Tada je
        \begin{equation*}
            \frac{X_n -d_n}{n} = \frac{S_n - c_n}{n} - \frac{n - 1}{n} \cdot \Big( \frac{S_{n - 1} - c_{n - 1}}{n - 1} \Big) \xrightarrow[n \to \infty]{\masP} 0,
        \end{equation*}
        pa je zbog $\iid$
        \begin{equation*}
            \frac{X_1 - d_n}{n} \xrightarrow[n \to \infty]{\masP} 0 \implies d_n = o(n).
        \end{equation*}
        Po L\' evyjevoj nejednakosti (lema \ref{lm:12.8}) slijedi da za svaki $\varepsilon > 0$
        \begin{equation*}
            \masP \Big( \max\limits_{1 \leq j \leq n} | S_j - c_j -m( S_j - c_j - S_n + c_n) | \geq \frac{n \: \varepsilon}{2}  \Big)\leq 2 \: \masP \Big( |S_n - c_n| \geq \frac{n \: \varepsilon}{2} \Big) \xrightarrow[n \to \infty]{} 0.
        \end{equation*}
        U zadatku \ref{zad:12.10} stavimo $Z_j := S_j - c_j$, odakle slijedi
        \begin{equation*}
            \begin{gathered}
                \max\limits_{1 \leq j \leq n} | m(S_j - c_j - S_n + c_n) | = o(n) \implies\\
                (\forall \varepsilon > 0) \quad \lim\limits_{n \to \infty} \masP \Big( \max\limits_{1 \leq j \leq n} |S_j - c_j | < n \varepsilon \Big) = 1.
            \end{gathered}
        \end{equation*}
        Kako je $d_n = o(n)$, za dovoljno velike $n$ je $\max\limits_{1 \leq j \leq n} |d_n| < n \: \varepsilon$, pa imamo:
        \begin{equation*}
            \begin{gathered}
                \begin{aligned}
                    \masP \Big( \max\limits_{1 \leq j \leq n}  |S_j - c_j| < n \: \varepsilon \Big) &\leq \masP \Big( \max\limits_{1 \leq j \leq n} |X_j - c_j| < 2 n \varepsilon \Big)\\
                    &\leq \masP \Big( \max\limits_{1 \leq j \leq n} |X_j| < 3 n \varepsilon \Big)\\
                    &= \vjeroj{|X_1| < 3 n \varepsilon, \; |X_2| < 3 n \varepsilon, \ldots, |X_n| < 3 n \varepsilon}\\
                    &= (\iid) = \big[ \vjeroj{|X_1| < 3 n \varepsilon} \big]^n \implies                    
                \end{aligned}\\
                    (\forall \varepsilon > 0) \quad \lim\limits_{n \to \infty} \big[ \vjeroj{|X_1| < 3 n \varepsilon} \big]^n = 1 \implies\\
                    \lim\limits_{n \to \infty} n \: \ln ( 1 - \vjeroj{|X_1| \geq 3 n \varepsilon} ) = 0.
            \end{gathered}
        \end{equation*}
        Za mali $x$ je $\ln (1 - x) \approx x$, pa za svaki $\varepsilon > 0$ vrijedi
        \begin{equation*}
            \lim\limits_{n \to \infty} n \: \vjeroj{|X_1| \geq 3 n \varepsilon} = 0.
        \end{equation*}
        \item[$\impliedby$] Pretpostavimo da je $\lim\limits_{n \to \infty} n \: \vjeroj{|X_1| > n} = 0$.
        Rezanjem dobivamo
        \begin{equation*}
            \begin{gathered}
                X_{n, k} := X_k \karaktFja_{\{ |X_k| \leq n \}}\\
                S_n' := X_{n, 1} + \ldots + X_{n, n}.
            \end{gathered}
        \end{equation*}
        Sada imamo
        \begin{equation*}
            \masP \Big( \Big| \frac{S_n}{n} - \equalto{b_n}{\frac{c_n}{n}} \Big| > \varepsilon \Big) \leq \masP \Big( \Big| \frac{S_n'}{n} - \frac{c_n}{n} \Big| > \varepsilon \Big) + \vjeroj{S_n \neq S_n'}.
        \end{equation*}
        Uo\v cimo nadalje da za $t \in \real$ vrijedi $t \: \vjeroj{|X_1| > t} \xrightarrow[t \to \infty]{} 0$, stoga imamo:
        \begin{equation*}
            \begin{aligned}
                \vjeroj{S_n \neq S_n'} &\leq \masP \Big( \unija{k = 1}{n} \{ X_k \neq X_{n, k} \} \Big) \leq \suma{k = 1}{n} \vjeroj{X_k \neq X_{n, k}}\\
                &= (\iid) = n \: \vjeroj{X_1 \neq X_{n, 1}}\\
                &= n \: \vjeroj{|X_1| > n} \xrightarrow[n \to \infty]{} 0.
            \end{aligned}
        \end{equation*}
        Uzmemo li sada $b_n := \masE [X_1 - \karaktFja_{\{ |X_1| \leq n \}}] $.
        \v Cebi\v sevljev zakon primjenjen na odrezane varijable, uz $b_n = \masE \Big[ \frac{S_n'}{n} \Big]$, daje
        \begin{equation*}
            \begin{aligned}
                \masP \Big( \Big| \frac{S_n'}{n} - b_n \Big| > \varepsilon \Big) &\leq \frac{1}{\varepsilon^2} \: \frac{1}{n^2} \Var (S_n') = (\textnormal{nezavisnost})\\
                &= \frac{1}{n^2 \: \varepsilon^2} n \Var (X_{n, 1})\\
                &\leq \frac{\masE [ X_{n, 1}^2 ]}{n \: \varepsilon^2} \int\limits_0^{\infty} 2 t \: \vjeroj{|X_{n, 1}| > t} \: dt\\
                &= \frac{2}{n \: \varepsilon^2} \int\limits_0^n t \: \vjeroj{|X_1| > t} \: dt.
            \end{aligned}
        \end{equation*}

        %ovdje nisam siguran za ove t_0
        Za $\delta > 0$ postoji $t_0$ takav da $\vjeroj{|X_1| > t_0} \leq \delta$, pa za $t > t_0$ i za dovoljno veliki $n$ imamo
        \begin{equation*}
            \begin{gathered}
                \begin{aligned}
                    \frac{1}{n} \int\limits_0^n t \: \vjeroj{|X_1| > t} \: dt &= \frac{1}{n} \int\limits_0^{t_0} t \: \vjeroj{|X_1| > t} \: dt + \frac{1}{n} \int\limits_{t_0}^n t \: \vjeroj{|X_1| > t} \: dt\\
                    &\leq \frac{1}{n} \cdot \textnormal{ fiksni broj } + \frac{1}{n} \cdot \delta \implies
                    \begin{smallmatrix}
                        \textnormal{za dovoljno veliki } n\\
                        \textnormal{je to manje od } 2 \delta
                    \end{smallmatrix}
                \end{aligned}\\
                \implies \quad \lim\limits_{n \to \infty} \frac{1}{n} \int\limits_0^n t \: \vjeroj{|X_1| > t} \: dt = 0\\
                \implies \quad \lim\limits_{n \to \infty} \masP \Big( \Big| \frac{S_n'}{n} -b_n \Big| > \varepsilon \Big) = 0, \quad \forall \varepsilon > 0.
            \end{gathered}
        \end{equation*}
        Sada vidimo,
        \begin{equation*}
            \begin{aligned}
                \frac{S_n}{n} - \widetilde{b_n} \xrightarrow[n \to \infty]{\masP} 0 &\implies n \: \vjeroj{|X_1| > n} \xrightarrow[n \to \infty]{} 0\\
                &\implies \frac{S_n}{n} - b_n \xrightarrow[n \to \infty]{\masP} 0\\
                &\implies b_n - \widetilde{b_n} \xrightarrow[n \to \infty]{} 0.
            \end{aligned}
        \end{equation*}
    \end{itemize}
\end{proof}

\begin{pr}  \label{pr:12.11}
    Cauchyjeva razdioba ne zadovoljava uvijete teorema \ref{tm:12.9}.
    Neka je $f_X (t) = \frac{1}{\pi \: (1 + t^2)}$.
    Tada je
    \begin{equation*}
        \begin{gathered}
            \begin{aligned}
                \vjeroj{|X| > x} &= 2 \int\limits_x^{+\infty} \frac{dt}{\pi \: (1 + t^2)} \sim \frac{2}{\pi} \int\limits_x^{+\infty} \frac{dt}{t^2}\\
                &= \frac{2}{\pi \: x}
            \end{aligned}\\
            \implies \quad \lim\limits_{x \to \infty} x \: \vjeroj{|X| > x} = \frac{2}{\pi} > 0.
        \end{gathered}
    \end{equation*}
\end{pr}

\begin{tm}  \label{tm:12.11-1}
    Neka su $(\real, \borel{\real}, \nu)$, te $ (\Omega, \famF, \mu) $ prostori sa $\sigma$-kona\v cnim mjerama i neka je $f : \Omega \to \real$ nenegativna, Borelova funkcija, tada vrijedi
    \begin{equation*}
        \int\limits_\Omega \nu (\desInt{0}{f(\omega)}) \: d \mu (\omega) = \int\limits_{0}^{+\infty} \mu (f>t) \: d \nu (t).
    \end{equation*}
    Ako uzmemo $\nu$ kao apsolutno neprekidnu s obzirom na Lebesgueovu mjeru $\lambda$ s gusto\' com $t \mapsto p \: t^{p-1}$, za $p>0$, tada imamo
    \begin{equation*}
        \int\limits_\Omega f(\omega)^p \: d \mu (\omega) = p \int\limits_{0}^{+\infty} t^{p-1} \mu (f > t) \: d \lambda (t).
    \end{equation*}
\end{tm}

\begin{proof}
    Doka\v zimo najprije da je funkcija
    \begin{equation*}
        (\omega, t) \mapsto \karaktFja_{\{ f > t \}} (\omega)
    \end{equation*}
    izmjeriva na $\famF \otimes \borel{\real}$.
    Prema napomeni \ref{nap:3.20-1} dovoljno je pokazati da je za $s \in \real$
    \begin{equation*}
        \Big\{\karaktFja_{\{f > t\}} > s \Big\} \in \famF \otimes \borel{\real}.
    \end{equation*}
    Za $s \geq 1$ gornji skup je prazan, \v sto je izmjerivo, dok za $s < 0$ gornji skup je jednak $\Omega \times \real$, jer je $\karaktFja_{\{ f > t \}} \geq 0$, \v sto je ponovo izmjerivo.
    Preostaje promotriti slu\v caj $s \in \desInt{0}{1}$.
    Primjetimo za takve vrijednosti parametra $s$ gornji skup ne ovisi o njemu, dakle
    \begin{equation*}
        \begin{aligned}
            \Big\{ \karaktFja_{\{f > t\}} > s \Big\} &= \bigSkup{(\omega, t) \in \Omega \times \real}{\karaktFja_{\{ f > t \}} (\omega) = 1}\\
            &= \bigSkup{(\omega, t) \in \Omega \times \real}{\omega \in \{ f > t \}}\\
            &= \skup{(\omega, t) \in \Omega \times \real}{f(\omega) > t}\\
            &= \skup{(\omega, t) \in \Omega \times \real}{f(\omega) - t > 0}.
        \end{aligned}
    \end{equation*}
    Primjetimo
    \begin{equation*}
        (\omega, t) \overset{g}{\mapsto} f(\omega) - t
    \end{equation*}
    je izmjeriva funkcija po propoziciji \ref{prop:4.9}.
    Tako\dj er primjetimo da je
    \begin{equation*}
        \skup{(\omega, t) \in \Omega \times \real}{f(\omega) - t > 0} = \{ g > 0 \},
    \end{equation*}
    \v sto je izmjerivo.
    Dakle funkcija
    \begin{equation*}
        (\omega, t) \mapsto \karaktFja_{\{ f > t \}} (\omega)
    \end{equation*}
    je izmjeriva.

    Primjetimo, po Fubinijevom teoremu imamo
    \begin{equation*}
        \begin{aligned}
            \int\limits_{0}^{+\infty} \int\limits_\Omega \karaktFja_{\{  f > t \}} (\omega)\: d \mu (\omega) \: d \nu (t) &= \int\limits_{0}^{+\infty} \Big( \int\limits_\Omega \karaktFja_{\{  f > t \}} (\omega) \: d \mu (\omega) \Big) \: d \nu (t) = \int\limits_{0}^{+\infty} \mu (f > t) \: d \nu (t)\\
            &= \int\limits_\Omega \Big( \int\limits_{0}^{+\infty} \karaktFja_{\{ f > t \}} (\omega) \: d \nu (t) \Big) \: d \mu (t) = \int\limits_\Omega \Big( \int\limits_{0}^{+\infty} \karaktFja_{\{ t < f \}} \Big) \: d \nu (t)\\
            &= \int\limits_\Omega \Big( \int\limits_{0}^{+\infty} \karaktFja_{\desInt{0}{f(\omega)}} (t) \Big) \: d \nu (t) = \int\limits_\Omega \nu (\desInt{0}{f (\omega)}) \: d \mu (\omega).
        \end{aligned}
    \end{equation*}
    Dakle vrijedi
    \begin{equation*}
        \int\limits_\Omega \nu (\desInt{0}{f(\omega)}) \: d \mu (\omega) = \int\limits_{0}^{+\infty} \mu (f>t) \: d \nu (t).
    \end{equation*}

    Uzmimo da je
    \begin{equation*}
            \nu (A) = \int\limits_A p t^{p-1} \: d \lambda (t) ,
    \end{equation*}
    sada imamo
    \begin{equation*}
        \nu (\desInt{0}{f(\omega)}) = \int\limits_{0}^{f(\omega)} p t^{p-1} \: d \lambda (t) = f^p (\omega).
    \end{equation*}
    Odakle vidimo da slijedi
    \begin{equation*}
        \int\limits_\Omega f(\omega)^p \: d \mu (\omega) = p \int\limits_{0}^{+\infty} t^{p-1} \mu (f > t) \: d \lambda (t).
    \end{equation*}
\end{proof}

\begin{nap} \label{nap:12.11-2}
    Za nenegativnu slu\v cajnu varijablu $X$ iz teorema \ref{tm:12.11-1} slijedi
    \begin{equation*}
        \masE [X^\alpha] = \alpha \int\limits_{0}^{+\infty} x^{\alpha - 1} \masP (X > x) \: d \lambda (x), \quad \alpha > 0,
    \end{equation*}
    te posebno za $\alpha = 1$
    \begin{equation*}
        \masE [X] = \int\limits_{0}^{+\infty} \masP (X > x) \: d \lambda (x).
    \end{equation*}
\end{nap}

\begin{zad} \label{zad:12.12}
    Ako je $X \in L^1(\masP)$, tada je
    \begin{equation*}
        \lim\limits_{x \to \infty} x \: \vjeroj{|X| > x} = 0.
    \end{equation*}
\end{zad}

\begin{rj}[\ref{zad:12.12}]
    Pretpostavimo suprotno, to jest, da vrijedi
    \begin{equation*}
        (\exists \varepsilon > 0)(\forall M \geq 0) \quad  (x \geq M ) \land (x \: \masP (|X| > x)).
    \end{equation*}

    Prema teoremu \ref{tm:12.11-1} imamo
    \begin{equation*}
        \begin{aligned}
            \masE [|X|] &= \int\limits_{0}^{+\infty} \masP (|X| > x) \: d \lambda (x)\\
            &= \int\limits_{0}^{1} \masP (|X| > x) \: d \lambda (x) + \int\limits_{1}^{+\infty} \masP (|X| > x) \: d \lambda (x),
        \end{aligned}
    \end{equation*}
    odakle vidimo da vrijedi
    \begin{equation*}
        \masE [|X|] \geq \int\limits_{1}^{+\infty} \masP (|X| > x) \: d \lambda (x).
    \end{equation*}
    Uzmimo $M = 1$, sada za $x \geq M$ imamo
    \begin{equation*}
        \masP (|X| > x) \geq \frac{\varepsilon}{x}.
    \end{equation*}
    Odavde slijedi
    \begin{equation*}
        \int\limits_{1}^{+\infty} \masP (|X| > x) \: d \lambda (x) \geq \varepsilon \int\limits_{1}^{+\infty} \frac{1}{x} \: d \lambda (x) = \varepsilon \ln x \Big|_{1}^{+\infty} = +\infty.
    \end{equation*}
    Sada imamo
    \begin{equation*}
        \masE [|X|] \geq \int\limits_{1}^{+\infty} \masP (|X| > x) \: d \lambda (x) \geq +\infty,
    \end{equation*}
    \v sto je kontradikcija s pretpostavkom $X \in L^1(\masP)$, dakle mora biti
    \begin{equation*}
        x \: \masP (|X| > x) \xrightarrow[n \to \infty]{} 0
    \end{equation*}
\end{rj}

\begin{zad} \label{zad:12.13}
    Ako je $X$ slu\v cajna varijabla za koju vrijedi
    \begin{equation*}
        \lim\limits_{x \to \infty} x \: \vjeroj{|X| > x} = 0,
    \end{equation*}
    tada je, za svaki $0 < \varepsilon < 1$,
    \begin{equation*}
        \masE [ |X|^{1 - \varepsilon} ] < +\infty.
    \end{equation*}
\end{zad}

\begin{rj}[\ref{zad:12.13}]
    Iz teorema \ref{tm:12.11-1} vidimo
    \begin{equation*}
            \masE [|X|^{1-\varepsilon}] = (1 - \varepsilon) \int\limits_{0}^{+\infty} x^{-\varepsilon} \: \masP (|X| > x) \: d \lambda (x)
    \end{equation*}
    Iz pretpostavke vidimo
    \begin{equation*}
        (\forall \delta > 0) (\exists M > 0) \quad x \geq M \implies x \: \masP (|X| > x) < \delta.
    \end{equation*}
    Fiskirajmo $\varepsilon$ i $\delta$, sada za $x \geq M$ imamo
    \begin{equation*}
        x^{-\varepsilon} \: \masP (|X| > x) < \frac{\delta}{x^{1 + \varepsilon}}.
    \end{equation*}
    Odavde slijedi
    \begin{equation*}
        \begin{aligned}
            \int\limits_{M}^{+\infty} x^{-\varepsilon} \: \masP (|X| > x) \: d \lambda (x) &< \delta \int\limits_{M}^{+\infty} x^{-(1 + \varepsilon)} \: d \lambda (x)\\
            &= -\frac{\delta}{\varepsilon} \Big( \frac{1}{x} \Big)^{\varepsilon} \Big|_{M}^{+\infty} = - \frac{\delta}{\varepsilon} \Big(0 - \Big( \frac{1}{M} \Big)^\varepsilon \Big)\\
            &= \frac{\delta}{\varepsilon} \cdot \frac{1}{M^\varepsilon} < +\infty.
        \end{aligned} 
    \end{equation*}
    S druge strane, imamo
    \begin{equation*}
        \begin{aligned}
            \int\limits_{0}^{M} x^{-\varepsilon} \: \masP (|X| > x) \: d \lambda (x) &\leq \int\limits_{0}^{M} x^{-\varepsilon} \: d \lambda (x) = \frac{1}{1 - \varepsilon} \cdot x^{1 - \varepsilon} \Big|_0^M\\
            &= \frac{M^{1 - \varepsilon}}{1 - \varepsilon} < +\infty.
        \end{aligned}
    \end{equation*}
    Iz \v cega slijedi
    \begin{equation*}
        \masE [|X|^{1 - \varepsilon}] < +\infty.
    \end{equation*}
\end{rj}

\begin{nap} \label{nap:12.13-1}
    Vrijede i generalizacije zadataka \ref{zad:12.12} i \ref{zad:12.13}.

    Za $p > 0$ vrijedi
    \begin{equation*}
        X  \in L^p (\masP) \quad \implies \quad \lim\limits_{x \to \infty} x^p \: \masP (|X| > x) = 0,
    \end{equation*}
    tako\dj er,
    \begin{equation*}
        \lim\limits_{x \to \infty} x^p \: \masP (|X| > x) = 0 \quad \implies \quad \masE [ |X|^{p-\varepsilon} ] < +\infty, \quad 0 < \varepsilon < p.
    \end{equation*}
\end{nap}

\begin{kor}[A. J. Hin\v cin] \label{kor:12.14}
    Ako je $\niz{X_n}{n \in \nat}$ $\iid$ niz takav da je $X_1 \in L^1(\masP)$ i $\masE X_1 = \mu$, tada vrijedi
    \begin{equation*}
        \frac{S_n}{n} \xrightarrow[n \to \infty]{\masP} \mu.
    \end{equation*}
\end{kor}

\begin{proof}
    Budu\' ci je $X_1 \in L^1(\masP)$, po teoremu o dominiranoj konvergenciji slijedi
    \begin{equation*}
        \masE \big[ X_1 \cdot \karaktFja_{\{ |X_1| \leq n \}} \big] \xrightarrow[n \to \infty]{} \masE X_1 = \mu.
    \end{equation*}
    Tvrdnja sada slijedi iz zadatka \ref{zad:12.12} primjenom teorema \ref{tm:12.9}.
\end{proof}

"Rezanjem" iz navedenih tehnika lako dobijemo i druge rezultate.

\begin{tm}  \label{tm:12.15}
    Neka je $\niz{X_{n, k}}{n \in \nat, \; 1 \leq k \leq n}$ "trokutasta" familija slu\v cajnih varijabli na $\vjerojatnosniProstor$ i $S_n := X_{n, 1} + \ldots + X_{n, n}$, za svaki $n \in \nat$.
    Neka je $X_{n, k} \sim F_{n, k}$.
    Ako su, za svaki $n \in \nat$ slu\v cajne varijable $X_{n, 1}, \ldots, X_{n, n}$ nezavisne i postoji niz $(b_n)$, takav da $0 < b_n \nearrow +\infty$, te vrijedi
    \begin{enumerate}[label=(\roman*)]
        \item \label{tm:12.15.1}
        \begin{equation*}
            \suma{k = 1}{n} \vjeroj{|X_{n, k}| > b_n} \xrightarrow[n \to \infty]{} 0,
        \end{equation*}
        \item \label{tm:12.15.2}
        \begin{equation*}
            \frac{1}{b_n^2} \suma{k = 1}{n} \int\limits_{|t| < b_n} t^2 \: dF_{n, k}(t) \xrightarrow[n \to \infty]{} 0.
        \end{equation*}
    \end{enumerate}
    Ozna\v cimo li sa
    \begin{equation*}
        a_n = \suma{k = 1}{n} \int\limits_{|t| \leq b_n} t \: dF_{n, k} (t).
    \end{equation*}
    tada vrijedi:
    \begin{equation*}
        \frac{S_n - a_n}{b_n} \xrightarrow[n \to \infty]{\masP} 0.
    \end{equation*}    
\end{tm}

\begin{proof}
    "Rezanje"
    \begin{equation*}
        \begin{aligned}
            X_{n, k}' &:= X_{n, k} \cdot \karaktFja_{\{ |X_{n, k}| \leq b_n \}}\\
            S_n' & := X_{n, 1}' + \ldots + X_{n, n}' 
        \end{aligned}
    \end{equation*}
    Uo\v cimo da je
    \begin{equation*}
        \vjeroj{S_n \neq S_n'} \leq \suma{k = 1}{n} \vjeroj{| X_{n, k} | > b_n} \xrightarrow[n \to \infty] 0,
    \end{equation*}
    po \ref{tm:12.15.1}.
    Budu\' ci da vrijedi
    \begin{equation*}
        \masP \Big( \Big| \frac{S_n - a_n}{b_n} \Big| > \varepsilon \Big) \leq \masP \Big( \Big| \frac{S_n' - a_n}{b_n} \Big| Y \varepsilon \Big) + \vjeroj{S_n \neq S_n'},
    \end{equation*}
    dovoljno je dokazati da za svaki $\varepsilon > 0$ vrijedi
    \begin{equation*}
        \begin{aligned}
            \masP \Big( \Big| \frac{S_n' - a_n}{b_n} \Big| > \varepsilon \Big) &= \masP \Big( \Big| \suma{k = 1}{n} X_{n, k} \cdot \karaktFja_{\{ |X_{n, k}| \leq b_n \}} - \masE \Big[ \suma{k = 1}{n} X_{n, k} \cdot \karaktFja_{\{ |X_{n, k}| \leq b_n \}} \Big] \Big| > \varepsilon \: b_n \Big)\\
            &\leq (\textnormal{\v Cebi\v sevljev}) \leq \frac{1}{\varepsilon^2 \: b_n^2} \Var \Big( \suma{k = 1}{n} X_{n, k}' \Big) = (\textnormal{nezavisnost})\\
            &= \frac{1}{\varepsilon^2 \: b_n^2} \suma{k = 1}{n} \Var (X_{n, k}') \leq \frac{1}{\varepsilon^2 \: b_n^2} \suma{k = 1}{n} \int\limits_{|t| \leq b_n} t^2 \: dF_{n, k} (t) \xrightarrow[n \to \infty]{} 0,
        \end{aligned}
    \end{equation*}
    \v sto vrijedi po \ref{tm:12.15.2}.
\end{proof}