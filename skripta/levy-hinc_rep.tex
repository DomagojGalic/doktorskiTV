% poglavlje 4.5 -> predavanje 20 - Levy-Hinčinova reprezentacija

\chapter{L\' evy-Hin\v cinova reprezentacija}

Tre\' ce fundamentalno pitanje glasi: ako je $\niz{X_n}{n \in \nat}$ niz jednako distribuiranoh slu\v cajnih varijabli i vrijedi
\begin{equation*}
    \frac{S_n - b_n}{a_n} \xrightarrow[n \to \infty]{d} X,
\end{equation*}
\v sto sve mo\v ze biti limes $X$.
O\v cito $X$ mo\v ze biti degenerirana slu\v cajna varijabla i mo\v ze vrijediti $X \sim N(\mu, \sigma^2)$.
Postoje li druge mogu\' cnosti?
Promotrimo slu\v caj $a_n > 0$ te $b_n \in \real$, $\forall n \in \nat$.
Neka je $\niz{X_n}{n \in \nat}$ niz nezavisnih slu\v cajnih varijabli na vjerojatnosnom prostoru $\vjerojatnosniProstor$ i neka vrijedi
\begin{equation*}
    \frac{1}{a_n} \big( S_n - b_n \big) \xrightarrow{d} X.
\end{equation*} 
Pretpostavimo da $X$ nije degenerirana.
Prema teoremu \ref{tm:19.16} mora vrijediti $a_n \to \infty$, te tako\dj er $\frac{a_n}{a_{n - 1}} \to 1$.
Neka su $0 < c_1 < c_2 < +\infty$ konstante.
Neka je
\begin{equation*}
    m_n := \min \bigSkup{m > n}{\frac{a_m}{a_n} > \frac{c_2}{c_1}},
\end{equation*}
on postoji jer $a_n \xrightarrow[n \to \infty]{} \infty$.
Tada slijedi
\begin{equation*}
    \lim\limits_{n \to \infty} \frac{a_{m_n}}{a_n} = \frac{c_2}{c_1}.
\end{equation*}
Neka su $\widetilde{b}_1$, $\widetilde{b}_2 \in \real$.
Postoje $B_{m, n}$ takvi da
\begin{equation*}
    \Big[c_1 \Big( \frac{1}{a_n} \suma{i = 1}{n} X_i - \frac{b_n}{a_n} \Big) - \widetilde{b}_1 c_1 \Big]
    + \Big[ c_1 \frac{a_{m_n}}{a_n} \Big( \frac{1}{a_{m_n}} \suma{i = n + 1}{n + m_n} X_i - \frac{b_{m_n}}{a_{m_n}} \Big) - \widetilde{b}_2 c_2 \Big]
    = \frac{c_1}{a_n} \suma{i = 1}{n + m_n} X_i - B_{m, n}.
\end{equation*}
Tada lijeva, pa onda i desna strana konvergiraju po distribuciji prema $G$ takvoj da je
\begin{equation*}
    \varphi_G = \varphi_{H_1} \cdot \varphi_{H_2},
\end{equation*}
gdje su
\begin{equation*}
    \begin{aligned}
        H_1 (x) &:= F_X \Big( \frac{x}{c_1} + \widetilde{b}_1 \Big)\\
        H_2 (x) &:= F_X \Big( \frac{x}{c_2} + \widetilde{b}_2 \Big).
    \end{aligned}
\end{equation*}
S druge strane
\begin{equation*}
    \frac{1}{a_{m_n + n}} (S_{m_n + n} - b_{m_n + n}) \xrightarrow[n \to \infty]{d} X.
\end{equation*}
Prema zadatku \ref{zad:19.14} postoje $a > 0$, $b \in \real$, takvi da je $G(x) = F_X (ax + b)$.
Dakle gornji limes postoji, razdioba od $F_X$ pripada sljede\' coj klasi.

\begin{defn}    \label{defn:20.1}
    Slu\v cajna varijabla $X$ je \emph{stabilna} ako pripadna karakteristi\v cna funkcija $\varphi$ zadovoljava sljede\' ca svojstva:
    \begin{equation}    \label{jed:20.2}
        \begin{gathered}
            (\forall a_1, a_2 > 0) (\exists a > 0, b \in \real)\\
            \varphi (a_1 \: t) \cdot \varphi (a_2 \: t) = e^{i b t} \varphi (a \: t), \quad t \in \real.
        \end{gathered} 
    \end{equation}
\end{defn}

\begin{pr}  \label{pr:20.3}
    Neka je $\alpha > 0$ i $\varphi (t) = e^{-|t|^\alpha}$.
    Budu\' ci da za svaki $a_1, a_2 > 0$ postoji (jedinstveno odre\dj en) $a > 0$ takav da je $a^\alpha = a_1^\alpha + a_2^\alpha$, funkcija $\varphi$ zadovoljava \eqref{jed:20.2}.
    Je li takva $\varphi$ karakteristi\v cna funkcija?
    Ne uvijek.
    Mo\v ze se pokazati da je $\varphi$ karakteristi\v cna funkcija samo za $0< \alpha \leq 2$.
    Budu\' ci da moment razdiobe implicira diferencijabilnost, jedina razdioba koja ima drugi moment je za slu\v caj $\alpha = 2$, \v sto daje normalnu razdiobu. 
\end{pr}

\begin{zad}[*]  \label{zad:20.4}
    Neka je $\niz{X_n}{n \in \nat} \subseteq L^2 (\masP)$, niz nezavisnih jednako distribuiranih slu\v cajnih varijabli, te neka su $(a_n) \subseteq \obInt{0}{+\infty}$, $(b_n) \subseteq \real$, takvi da
    \begin{equation*}
        \frac{S_n - b_n}{a_n} \xrightarrow[n \to \infty]{d} X.
    \end{equation*}
    Tada je $X$ ili degenerirana ili ima normalnu razdiobu.
\end{zad}

Mogu\' ca je i ne\v sto druga\v cija karakterizacija stabilnih distribucija.

\begin{zad}[*]  \label{zad:20.5}
    Karakteristi\v cna funkcija $\varphi$ pripada stabilnoj distribuciji ako i samo ako vrijedi
    \begin{equation*}
        \begin{gathered}
            (\forall n \in \nat) (\exists a_n > 0, b_n \in \real)\\
            [\varphi (t)]^n = e^{ib_n t} \varphi (a_n \: t).
        \end{gathered}
    \end{equation*}
\end{zad}

\begin{tm}  \label{tm:20.6}
    Slu\v cajna varijabla $X$ je stabilna ako i samo ako postoji niz nezavisnih, jednako distribuiranih slu\v cajnih varijabli $\niz{X_n}{n \in \nat}$ i nizovi $(a_n) \subseteq \obInt{0}{+\infty}$, $(b_n) \subseteq \real$ takvi da
    \begin{equation*}
        \frac{S_n - b_n}{a_n} \xrightarrow[n \to \infty]{d} X.
    \end{equation*}
\end{tm}

\begin{proof}
    \quad
    \begin{enumerate}
        \item[$\implies$]
        ovaj smjer je dokazan na po\v cetku paragrafa.
        \item[$\impliedby$]
        Neka je $(X_n)$ niz nezavisnih kopija varijable $X$ i neka su $(a_n)$, $(b_n)$ iz zadatka \ref{zad:20.5}.
        Tada vrijedi
        \begin{equation*}
            \begin{aligned}
                \varphi_{\frac{S_n - b_n}{a_n}} (t) &= e^{-i\frac{t}{a_n} b_n} \varphi_{S_n} \Big( \frac{t}{a_n} \Big) \overset{\iid}{=} e^{-i \frac{t}{a_n} b_n} \Big[ \varphi_X \Big( \frac{t}{a_n} \Big)  \Big]^n = \varphi \Big( a_n \cdot \frac{t}{a_n} \Big)\\
                &= \varphi_X (t).
            \end{aligned}
        \end{equation*}
        Dakle
        \begin{equation*}
            \frac{S_n - b_n}{a_n} \distJed X, \quad \forall n \in \nat,    
        \end{equation*}
        stoga vrijedi
        \begin{equation*}
            \frac{S_n - b_n}{a_n} \xrightarrow[n \to \infty]{d} X.
        \end{equation*}
    \end{enumerate}
\end{proof}

\begin{kor} \label{kor:20.7}
    Ako je $\varphi$ karakteristi\v cna funkcija stabilne distribucije, tada je $\varphi$ beskona\v cno djeljiva.
\end{kor}

\begin{proof}
    U dokazu teorema \ref{tm:20.6} stavimo
    \begin{equation*}
        Y_k^n := \frac{X_k - \frac{b_n}{n}}{a_n}.
    \end{equation*}
    Slu\v cajne varijable $(Y_k^n)$ su nezavisne i jednako distribuirane i vrijedi
    \begin{equation*}
        \suma{k = 1}{n} Y_k^n = \frac{S_n - b_n}{a_n},
    \end{equation*}
    stoga je
    \begin{equation*}
        [ \varphi_{Y_1^n} ]^n = \varphi.
    \end{equation*}
\end{proof}

\begin{nap} \label{nap:20.8}
    Karakterizacija karakteristi\v cnih funkcija stabilnih distribucija je poznata.
    Ne\' cemo ovdje dokazivati tu tvrdnju.
    Uo\v cimo da korolar \ref{kor:20.7} pokazuje da uvijek postoji $\ln \varphi$ za stabilne distribucije.
    Mo\v ze se pokazati da je distribucija stabilna ako i samo ako $\ln \varphi$ ima oblik:
    \begin{equation}    \label{jed:20.9}
        \ln \varphi (t) = itb - \beta |t|^\alpha \Big[ 1 + i \theta \frac{t}{|t|} \tg \frac{\pi}{2} \alpha \Big],
    \end{equation}
    pri \v cemu je $b \in \real$, $\beta \geq 0$, $|\theta| \leq 1$, $\alpha \in \obInt{0}{1} \cup \lijInt{1}{2}$, ili
    \begin{equation}    \label{jed:20.10}
        \ln \varphi (t) = i t b - \beta |t| \Big[ 1 + i \theta \frac{t}{|t|} \ln |t| \Big],
    \end{equation}
    gdje je $b \in \real$, $\beta \geq 0$, $|\theta| \leq 1$.
    Uzimamo $\frac{t}{|t|} = 0$, za $t = 0$.

    Razdioba je simetri\v cna ako i samo ako je $\varphi$ realna funkcija.
    Stoga slijedi da je razdioba \emph{simetri\v cna stabilna} ako i samo ako je
    \begin{equation}    \label{jed:20.11}
        \varphi (t) = e^{-\beta |t|^\alpha}, \quad \beta \geq 0, \; 0 < \alpha \leq 2.
    \end{equation}

    O\v cito za $\beta = 0$ je pripadna razdioba degenerirana u nuli.
    Za $\beta = \frac{1}{2}$ i $\alpha = 2$ to je $N(0,1)$, za $\beta > 0$ i $\alpha = 1$, to je Cauchyjeva s parametrom $\beta$.
\end{nap}

Preostaje vidjeti \v sto se mo\v ze dogoditi u op\' cem slu\v caju.

\v Cetvrto fundamentalno pitanje glasi: "\v Sto sve mo\v ze biti limes po distribuciji "velikog broja" "malih" nezavisnih doprinosa?"
Precizirajmo to.
Neka je $\niz{X_{n, k}}{n \in \nat, \; 1 \leq k \leq k_n}$ dvostrani niz slu\v cajnih varijabli za koji vrijedi: $k_n \nearrow +\infty$ i
\begin{equation*}
    X_{n, 1}, \ldots, X_{n, k_n},
\end{equation*}
su nezavisne slu\v cajne varijable.
Pretpostavljamo nadalje da je $(X_{n, k})$ \emph{infinitezimalni sustav} to jest da za svaki $\varepsilon > 0$ vrijedi
\begin{equation}    \label{jed:20.12}
    \lim\limits_{n \to \infty} \max\limits_{1 \leq k \leq k_n} \masP (|X_{n, k}| \geq \varepsilon) = 0.
\end{equation}
Pitamo se, dakle, \v sto sve mogu biti limesi $X$ centriranog infinitezimalnog sustava, to jest, za razdiobe od $X$ vrijedi
\begin{equation}    \label{jed:20.13}
    \suma{k = 1}{k_n} X_{n, k} - c_n \xrightarrow[n \to \infty]{d} X?
\end{equation}
Lako je uo\v citi da su kandidati za odgovor na ovo pitanje beskona\v cno djeljive distribucije.
Zaista, neka je $X$ (dakle $\varphi_X$) beskona\v cno djeljiva.
Tada, za svaki $n \in \nat$, postoji karakteristi\v cna funkcija $\varphi_n$ takva da je $[\varphi_n]^n = \varphi_X$.
Po velikom Kolmogorovljevom teoremu je konstantni vjerojatnosni prostor $\vjerojatnosniProstor$ i familiju nezavisnih slu\v cajnih varijabli $\niz{X_{n, k}}{n \in \nat, \; 1 \leq k \leq n}$ takvih da je za svaki $n \in \nat$, $F_{X_{n, k}} = F_{\varphi_n}$, $1 \leq k \leq n$.
O\v cito je takav sustav onda i nezavisan po recima.
Po zadatku \ref{zad:18.12} \ref{zad:18.12.2} (vidi dokaz propozicije \ref{prop:18.8}) slijedi da
\begin{equation*}
    |\varphi_n (t) - 1| \xrightarrow[n \to \infty]{} 0,
\end{equation*}
uniformno po $t$ na ograni\v cenim intervalima.

\begin{zad} \label{zad:20.14}
    Ako za sustav $(X_{n, k})$ vrijedi
    \begin{equation}
        \lim\limits_{n \to \infty} \max\limits_{1 \leq k \leq k_n} |\varphi_{X_{n, k}}(t) - 1| = 0,
    \end{equation}
    uniformno po $t$ na svakom ograni\v cenom intervalu, tada je taj sustav infinitezimalan.
\end{zad}

Dakle, sustav koji smo gore konstruirali je infinitezimalan.
Uzmimo $c_n = 0$ i onda slijedi
\begin{equation*}
    \varphi_{\suma{k = 1}{n} X_{n, k}} = \produkt{k = 1}{n} \varphi_{X_{n, k}} = [\varphi_n]^n = \varphi,
\end{equation*}
\v sto po teoremu neprekidnosti daje
\begin{equation}    \label{jed:20.15}
    \suma{k = 1}{n} X_{n, k} \xrightarrow[n \to \infty]{d} X.
\end{equation}

Vrijedi li i neka vrsta obrata?
Za to je prvo potrebno to\v cno opisati beskona\v cno djeljive distribucije.

\begin{tm}[L\' evy-Hin\v cin]   \label{tm:20.16}
    \quad \\
    Funkcija $\varphi$ je beskona\v cno djeljiva karakteristi\v cna funkcija ako i samo ako postoje $\gamma \in \real$ i $G \in \inc$, $G^R = G$, $\Delta G < +\infty$, takvi da je
    \begin{equation*}
        \varphi (t) = e^{  i \gamma t + \int\limits_{-\infty}^{+\infty} \big(e^{itx} - 1 - \frac{itx}{1 + t^2} \big) \frac{1 + x^2}{x^2}  \: d G(x)}, \quad t \in \real.
    \end{equation*}
\end{tm}

\begin{proof}{(skica)}
    \begin{enumerate}
        \item[$\implies$]
        Neka je $\varphi_n$ karakteristi\v cna funkcija, takva da je $[\varphi_n]^n = \varphi$.
        Po zadatku \ref{zad:18.12} \ref{zad:18.12.1}
        \begin{equation*}
            \ln \varphi (t) = \lim\limits_{n \to \infty} n \int\limits_{-\infty}^{+\infty} (e^{itx} - 1) \: d F_{\varphi_n} (x),
        \end{equation*}
        uniformno po $t$ na ograni\v cenim intervalima.
        Definirajmo
        \begin{equation*}
            G_n (x) := n \int\limits_{-\infty}^x \frac{y^2}{1 + y^2} \: d F_{\varphi_n} (y) \implies G(-\infty) = 0,
        \end{equation*}
        odakle slijedi $G_n(-\infty) = 0$, $G_n \in \inc$, $G_n^R = G_n$, $\Delta G_n = G_n (+\infty)$ je uniformno ograni\v cena.
        Po teoremu \ref{tm:16.16} (isti dokaz kao i za vjerojatnosne d.f.) postoji $G \in \inc$, $G^R = G$ i $\Delta G < +\infty$, takva da
        \begin{equation*}
            G_{n_k} \xrightarrow[n \to \infty]{w} G.
        \end{equation*}
        Stavimo
        \begin{equation*}
            L(x, t) :=
            \begin{cases}
                \big( e^{itx} - 1 - \frac{itx}{1 + x^2} \big) \frac{1 + x^2}{x^2}, &x \neq 0\\
                -\frac{t^2}{2}, &x = 0
            \end{cases}.
        \end{equation*}
        Vidimo da je $x \mapsto F(x, t)$ neprekidna i ograni\v cena za svaki $t$.
        Stoga po Helly-Bray teoremu (teorem \ref{tm:18.1}) vrijedi
        \begin{equation*}
            \lim\limits_{k \to \infty} \int\limits_{-\infty}^{+\infty} L(x, t) \: d G_{n_k} (x) = \int\limits_{-\infty}^{+\infty} L(x, t) \: d G(x).
        \end{equation*}
        Neka je
        \begin{equation*}
            I_n(t) := \int\limits_{-\infty}^{+\infty} (e^{itx} - 1) \frac{1 + x^2}{x^2} \: d G_n (x).
        \end{equation*}
        Sada vrijedi
        \begin{equation*}
            I_n (t) = n \int\limits_{-\infty}^{+\infty} (e^{itx} - 1) \: d F_{\varphi_n} (x),
        \end{equation*}
        odakle slijedi
        \begin{equation*}
            I_n (t) \xrightarrow[n \to \infty]{} \ln \varphi (t).
        \end{equation*}

        Stavimo
        \begin{equation*}
            \gamma_n := n \int\limits_{-\infty}^{+ \infty} \frac{x}{1 + x^2} \: d F_{\varphi_n} (x),
        \end{equation*}
        pa vrijedi
        \begin{equation*}
            I_{n_k} (t) = i \gamma_{n_k} t + \int\limits_{-\infty}^{+\infty} L(x, t) \: d G_{n_k} (x)
        \end{equation*}
        odakle vidimo
        \begin{equation*}
            \gamma_{n_k} \xrightarrow[k \to \infty]{} \gamma,
        \end{equation*}
        te vrijedi
        \begin{equation*}
            \ln \varphi (t) = i \gamma t + \int\limits_{-\infty}^{+\infty} L(x, t) \: d G (x).
        \end{equation*}
    \end{enumerate}
\end{proof}

Postoji i drugi zapis
\begin{equation}    \label{jed:20.17}
    \varphi (t) = e^{ i \gamma t - \sigma^2 \frac{t^2}{2} + \int\limits_{ \real \setminus \{0\}} \big( e^{itx} - 1 - \frac{itx}{1 + x^2} \big) \: d \nu (x) },
\end{equation}
pri \v cemu je $\gamma \in \real$, $\sigma^2 \geq 0$ i $\nu$ (L\' evyjeva) mjera na $\real \setminus \{0\}$ takva da je
\begin{equation*}
    \int\limits_{\obInt{-1}{1} \setminus \{0\}} x^2 \: d \nu (x) < +\infty.
\end{equation*}

\begin{tm}[Op\' ci centralni grani\v cni teorem]    \label{tm:20.18}
    \quad\\
    Vrijedi
    \begin{equation*}
        \suma{k = 1}{k_n} \big( X_{n, k} - c_n \big) \xrightarrow[n \to \infty]{d} X \quad \iff \quad \varphi_x \; \textnormal{ beskona\v cno djeljiva}.
    \end{equation*}
    To je ispunjeno ako i samo ako $\gamma_n \xrightarrow[n \to \infty]{} \gamma$ te $G_n \xrightarrow[n \to \infty]{c} G$, pri \v cemu su
    \begin{equation*}   %interval po kojem području?
        \begin{aligned}
            \gamma_n &:= -c_n + \suma{k = 1}{k_n} \Big( c_{n, k} + \int \frac{x}{1 + x^2} \: d \widetilde{F}_{n, k} (x) \Big),\\
            G_n (x) &:= \suma{k = 1}{k_n} \int\limits_{-\infty}^x \frac{y^2}{1 + y^2} \: d \widetilde{F}_{n, k}(y),\\
            c_{n, k} &:= \int\limits_{|x| < 1} x \: d F_{x_{n, k}}(x),\\
            \widetilde{F}_{n, k} (x) &:= F_{X_{n, k}} (x + c_{n, k}).
        \end{aligned}
    \end{equation*}
\end{tm}