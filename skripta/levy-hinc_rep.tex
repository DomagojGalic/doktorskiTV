% poglavlje 4.5 -> predavanje 20 - Levy-Hinčinova reprezentacija

\chapter{L\' evy-Hin\v cinova reprezentacija}

Tre\' ce fundamentalno pitanje glasi: ako je $\niz{X_n}{n \in \nat}$ niz jednako distribuiranoh slu\v cajnih varijabli i vrijedi
\begin{equation*}
    \frac{S_n - b_n}{a_n} \xrightarrow[n \to \infty]{d} X,
\end{equation*}
\v sto sve mo\v ze biti limes $X$.
O\v cito $X$ mo\v ze biti degenerirana slu\v cajna varijabla i mo\v ze vrijediti $X \sim N(\mu, \sigma^2)$.
Postoje li druge mogu\' cnosti?
Promotrimo slu\v caj $a_n > 0$ te $b_n \in \real$, $\forall n \in \nat$.
Neka je $\niz{X_n}{n \in \nat}$ niz nezavisnih slu\v cajnih varijabli na vjerojatnosnom prostoru $\vjerojatnosniProstor$ i neka vrijedi
\begin{equation*}
    \frac{1}{a_n} \big( S_n - b_n \big) \xrightarrow{d} X.
\end{equation*} 
Pretpostavimo da $X$ nije degenerirana.
Prema teoremu \ref{tm:19.16} mora vrijediti $a_n \to \infty$, te tako\dj er $\frac{a_n}{a_{n - 1}} \to 1$.
Neka su $0 < c_1 < c_2 < +\infty$ konstante.
Neka je
\begin{equation*}
    m_n := \min \bigSkup{m > n}{\frac{a_m}{a_n} > \frac{c_2}{c_1}},
\end{equation*}
on postoji jer $a_n \xrightarrow[n \to \infty]{} \infty$.
Tada slijedi
\begin{equation*}
    \lim\limits_{n \to \infty} \frac{a_{m_n}}{a_n} = \frac{c_2}{c_1}.
\end{equation*}
Neka su $\widetilde{b_1}$, $\widetilde{b_2} \in \real$.
Postoje $B_{m, n}$ takvi da
\begin{equation*}
    \Big[c_1 \Big( \frac{1}{a_n} \suma{i = 1}{n} X_i - \frac{b_n}{a_n} \Big) - \widetilde{b_1} c_1 \Big]
    + \Big[ c_1 \frac{a_{m_n}}{a_n} \Big( \frac{1}{a_{m_n}} \suma{i = n + 1}{n + m_n} X_i - \frac{b_{m_n}}{a_{m_n}} \Big) - \widetilde{b_2} c_2 \Big]
    = \frac{c_1}{a_n} \suma{i = 1}{n + m_n} X_i - B_{m, n}.
\end{equation*}
Tada lijeva, pa onda i desna strana konvergiraju po distribuciji prema $G$ takvoj da je
\begin{equation*}
    \varphi_G = \varphi_{H_1} \cdot \varphi_{H_2},
\end{equation*}
gdje su
\begin{equation*}
    \begin{aligned}
        H_1 (x) &:= F_X \Big( \frac{x}{c_1} + \widetilde{b_1} \Big)\\
        H_2 (x) &:= F_X \Big( \frac{x}{c_2} + \widetilde{b_2} \Big).
    \end{aligned}
\end{equation*}
S druge strane
\begin{equation*}
    \frac{1}{a_{m_n + n}} (S_{m_n + n} - b_{m_n + n}) \xrightarrow[n \to \infty]{d} X.
\end{equation*}
Prema zadatku \ref{zad:19.14} postoje $a > 0$, $b \in \real$, takvi da je $G(x) = F_X (ax + b)$.
Dakle gornji limes postoji, razdioba od $F_X$ pripada sljede\' coj klasi.

\begin{defn}    \label{defn:20.1}
    Slu\v cajna varijabla $X$ je \emph{stabilna} ako pripadna karakteristi\v cna funkcija $\varphi$ zadovoljava sljede\' ca svojstva:
    \begin{equation}    \label{jed:20.2}
        \begin{gathered}
            (\forall a_1, a_2 > 0) (\exists a > 0, b \in \real)\\
            \varphi (a_1 \: t) \cdot \varphi (a_2 \: t) = e^{i b t} \varphi (a \: t), \quad t \in \real.
        \end{gathered} 
    \end{equation}
\end{defn}

\begin{pr}  \label{pr:20.3}
    Neka je $\alpha > 0$ i $\varphi (t) = e^{-|x|^\alpha}$.
    Budu\' ci da za svaki $a_1, a_2 > 0$ postoji (jedinstveno odre\dj en) $a > 0$ takav da je $a^\alpha = a_1^\alpha + a_2^\alpha$, funkcija $\varphi$ zadovoljava \eqref{jed:20.2}.
    Da li je takva $\varphi$ karakteristi\v cna funkcija?
    Ne uvijek.
    Mo\v ze se pokazati da je $\varphi$ karakteristi\v cna funkcija samo za $0< \alpha \leq 2$.
    Budu\' ci da moment razdiobe implicira diferencijabilnost, jedina razdioba koja ima drugi moment je za slu\v caj $\alpha = 2$, \v sto daje normalnu razdiobu. 
\end{pr}

\begin{zad}[*]  \label{zad:20.4}
    Neka je $\niz{X_n}{n \in \nat} \subseteq L^2 (\masP)$, niz nezavisnih jednako distribuiranih slu\v cajnih varijabli, te neka su $(a_n) \subseteq \obInt{0}{+\infty}$, $(b_n) \subseteq \real$, takvi da
    \begin{equation*}
        \frac{S_n - b_n}{a_n} \xrightarrow[n \to \infty]{d} X.
    \end{equation*}
    Tada je $X$ ili degenerirana ili ima normalnu razdiobu.
\end{zad}

Mogu\' ca je i ne\v sto druga\v cija karakterizacija stabilnih distribucija.

\begin{zad}[*]  \label{zad:20.5}
    Karakteristi\v cna funkcija $\varphi$ pripada stabilnoj distribuciji ako i samo ako vrijedi
    \begin{equation*}
        \begin{gathered}
            (\forall n \in \nat) (\exists a_n > 0, b_n \in \real)\\
            [\varphi (t)]^n = e^{ib_n t} \varphi (a_n \: t).
        \end{gathered}
    \end{equation*}
\end{zad}

\begin{tm}  \label{tm:20.6}
    Za slu\v cajnu varijabu $X$ postoji niz nezavisnih, jednako distribuiranih slu\v cajnih varijabli $\niz{X_n}{n \in \nat}$ i nizovi $(a_n) \subseteq \obInt{0}{+\infty}$, $(b_n) \subseteq \real$ takvi da
    \begin{equation*}
        \frac{S_n - b_n}{a_n} \xrightarrow[n \to \infty]{d} X
    \end{equation*}
    ako i samo ako je $X$ stabilna.
\end{tm}

\begin{proof}
    \begin{enumerate}
        \item[$\implies$]
        je dokazano na po\v cetku paragrafa.
        \item[$\impliedby$]
        Neka je $(X_n)$ niz nezavisnih kopija varijable $X$ i neka su $(a_n)$, $(b_n)$ iz zadatka \ref{zad:20.5}.
        Tada vrijedi
        \begin{equation*}
            \begin{aligned}
                \varphi_{\frac{S_n - b_n}{a_n}} (t) &= e^{-i\frac{t}{a_n} b_n} \varphi_{S_n} \Big( \frac{t}{a_n} \Big) \overset{\iid}{=} e^{-i \frac{t}{a_n} b_n} \Big[ \varphi_X \Big( \frac{t}{a_n} \Big)  \Big]^n = \varphi \Big( a_n \cdot \frac{x}{a_n} \Big)\\
                &= \varphi_X (t).
            \end{aligned}
        \end{equation*}
        Dakle
        \begin{equation*}
            \frac{S_n - b_n}{a_n} \distJed X, \quad \forall n \in \nat,    
        \end{equation*}
        stoga vrijedi
        \begin{equation*}
            \frac{S_n - b_n}{a_n} \xrightarrow[n \to \infty]{d} X.
        \end{equation*}
    \end{enumerate}
\end{proof}