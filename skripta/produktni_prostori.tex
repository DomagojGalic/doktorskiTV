% produktni prostori

\chapter{Produktni prostori}

Zapo\v cinjemo sa definicijom kartezijevog produkta proizvoljne indeksirane familije skupova.
\begin{defn}
    Neka je $T \neq \varnothing$ skup indeksa i $\indFamilija{\Omega_t}{t \in T}$ familija nepraznih skupova.
    Skup svih funkcija
    \begin{equation*}    \label{defn:4.1-1}
        f: T \to \unija{t \in T}{} \Omega_t,
    \end{equation*}
    takvih da je $f(t) \in \Omega_t, \; \forall t \in T$, ozna\v cavamo sa $\produkt{t \in T}{} \Omega_t$ i nazivamo \emph{kartezijevim produktom} indeksirane familije skupova.

    Funkcije
    \begin{equation*}
        \pi_{t_0}: \produkt{t \in T}{} \Omega_t \to \Omega_{t_0}
    \end{equation*}
    definiramo sa
    \begin{equation*}
        \pi_{t_0} (f) := f(t_0)
    \end{equation*}
    i nazivamo \emph{koordinatnim projekcijama}.
\end{defn}


Aksiom izbora garantira da je $\produkt{t \in T}{} \Omega_t$ neprazan (jer su svi $\Omega_t \neq \varnothing$).
Definicija se mo\v ze pro\v siriti na proizvoljne $\Omega_t$ i tada dobivamo da je $\produkt{t \in T}{} \Omega_t$ prazan  \v cim je barem jedan $\Omega_t$ prazan.

\begin{defn}    \label{defn:4.0-1}
    Neka su $(X_t, \: \famU_t), \; t \in T$, topolo\v ski prostori, tada se na $\produkt{t \in T}{} X_t$ definira \emph{produktna (ili Tihonovljeva) topologija}, kao najmanja topologija u odnosu na koju su sve $\pi_t$ neprekidne.
    \begin{equation*}
        \dirProd{t \in T}{} \famU_t := \indSigAlg{\pi_t}{t \in T}
    \end{equation*}
\end{defn}

\begin{tm}[Karakterizacija produktne topologije]  \label{tm:4.0-2}
    Neka su $\indFamilija{\urePar{X_t}{\famU_t}}{t \in T}$ topolo\v ski prostori, tada je $U \subseteq \produkt{t \in T}{} X_t$ otvoren u produktnoj topologiji $\dirProd{t \in T}{} \famU_t$ ako i samo ako postoje skupovi $U_t \in \famU_t$ takvi da
    \begin{equation*}
        U = \produkt{t \in T}{} U_t,
    \end{equation*}
    te je skup
    \begin{equation*}
        \skup{t \in T}{U_t \neq X_t}
    \end{equation*}
    kona\v can.
\end{tm}

\begin{proof}
    Ozna\v cimo gore opisanu familiju sa $\famU$, tvrdimo
    \begin{equation*}
        \famU = \dirProd{t \in T}{} \famU_t.
    \end{equation*}
    Najprije doka\v zimo da je $\famU$ topologija.
    \begin{enumerate}[label=(\roman*)]
        \item Uzmemo $t_0 \in T$, sada je nu\v zno $\varnothing \in \famU_{t_0}$, pa vrijedi $\produkt{t \in T \setminus \{t_0\}}{} X_t \times \varnothing = \varnothing$ pa slijedi $\varnothing \in \famU$.
        Tako\dj er vidimo da je $\produkt{t \in T}{} X_t \in \famU$, jer je skup indeksa prazan, time trivijalno kona\v can.
        \item Neka su $\niz{U_i}{i \in I} \subseteq \famU$, tada za svaki $i \in I$, postoji kona\v can skup $J_i$ takav da vrijedi
        \begin{equation*}
            \pi_t (U_i) = X_t, \quad t \in T \setminus J_i.
        \end{equation*}
        Sada je vrijedi
        \begin{equation*}
            \pi_t \Big(\unija{i \in I}{} U_i \Big) = X_t, \quad t \in T \setminus \presjek{i \in I}{} J_i.
        \end{equation*}
        A kako je $\presjek{i \in I}{} J_i$ presjek kona\v cnih skupova, vrijedi $\unija{i \in I}{} U_i \in \famU$
        \item Neka su $\niz{U_k}{k = 1, \ldots, n} \subseteq \famU$, sada je
        \begin{equation*}
            \bigSkup{t \in T}{\pi_t \Big( \presjek{k = 1}{n} U_k \Big) \neq X_t} = \unija{k = 1}{n} J_k,
        \end{equation*}
        \v sto je kona\v cna unija kona\v cnih skupova, pa je $\presjek{k = 1}{n} U_k \in \famU$.
    \end{enumerate}
    Dakle $\famU$ je topologija.

    Doka\v zimo sada da su te topologije jednake
    \begin{enumerate}
        \item[$\subseteq$] Neka je $U \in \famU$, tada postoji $n \in \nat$ i skup $J \subseteq T$ takav da je $\card{J} = n$ te vrijedi
        \begin{equation*}
            \skup{t \in T}{\pi_t(U) \neq X_t} = J.
        \end{equation*}
        Sada primjetimo da vrijedi
        \begin{equation*}
            U = \presjek{t \in J}{} \praslika{\pi_t} (\pi_t (U)),
        \end{equation*}
        pa je $U \in \famU$.
        \item[$\supseteq$]
        Neka je $t_0 \in T$, promatrajmo $\pi_t : \Big( \produkt{t \in T}{} X_t, \famU \Big) \to \urePar{X_{t_0}}{\famU_{t_0}}$.
        Neka je $U_{t_0} \in \famU_{t_0}$, sada primjetimo da je
        \begin{equation*}
            \praslika{\pi_{t_0}} (U_{t_0}) \in \famU.
        \end{equation*}
        Dakle $\famU$ je topologija i izmjeriva sve projekcije su neprekidne u njoj, dakle ona mora sadr\v zavati produktnu topologiju kao najmanju topologiju u kojoj su sve projekcije neprekidne.

        Dakle $\dirProd{t \in T}{} \famU_t \subseteq \famU$.
    \end{enumerate}
\end{proof}

\begin{tm}  \label{tm:4.0-3}
    Neka je $\urePar{X}{\famU}$ topolo\v ski prostor i neka je $\famV$ neka njegova prebrojiva baza.
    Tada vrijedi:
    \begin{equation*}
        \sigAlg{\famV} = \sigAlg{\famU}.
    \end{equation*}
\end{tm}

\begin{proof}
    \quad
    \begin{enumerate}
        \item[$\subseteq$] Budu\' ci je $\famV$ baza vrijedi da je $\famV \subseteq \famU$, stoga je i $\sigAlg{\famV} \subseteq \sigAlg{\famU}$.
        \item[$\supseteq$] Neka je $U \in \famU$, tada postoji niz $\niz{U_n}{n \in \nat} \subseteq \famV$ takav da je $U = \unija{n \in \nat}{} U_n$, ali tada vrijedi i $\niz{U_n}{n \in \nat} \subseteq \sigAlg{\famV}$, pa onda i $U = \unija{n \in \nat}{} U_n \in \sigAlg{\famU}$ pa je $\famU \subseteq \sigAlg{\famV}$ pa je i $\sigAlg{\famU} \subseteq \sigAlg{\famV}$.  
    \end{enumerate}
\end{proof}

Ako su svi $X_t$ kompaktni, tada je i $\produkt{t \in T}{} X_t$ kompaktan u produktnoj topologiji.
Analogno postupamo u izmjerivoj situaciji.

\begin{defn}    \label{defn:4.0-4}
    Neka su $(\Omega_t, \: \famF_t), \; t \in T$ izmjerivi prostori.
    Produktna $\sigma$-algebra $\dirProd{t \in T}{} \famF_t$ na $\produkt{t \in T}{} \Omega_t$, definirana je sa:
    \begin{equation}    \label{jed:4.1}
        \dirProd{t \in T}{} \famF_t := \indSigAlg{\pi_t}{t \in T}.
    \end{equation}
\end{defn}

\begin{defn}    \label{defn:4.0-5}
    Element $\produkt{t \in T}{} A_t \in \partitive{\produkt{t \in T}{} \Omega_t}$, takav da postoji kona\v can $J \subseteq T$ sa svojstvima:
    \begin{itemize}[label=]
        \item $t \in J \implies A_t \in \famF_t$
        \item $ t \in T \setminus J \implies A_t = \Omega_t $
    \end{itemize}
    nazivamo \emph{izmjerivi cilindri\v cni pravokutnik}.
\end{defn}

Lako se vidi da svi takvi skupovi tvore poluprsten skupova $\prsten{\produkt{t \in T}{} \Omega_t}$ i da vrijedi:
\begin{equation}    \label{jed:4.2}
    \sigAlg{\prsten{\produkt{t \in T}{} \Omega_t}} = \dirProd{t \in T}{} \famF_t.
\end{equation}
Uo\v cimo da se za topolo\v ske prostore javlja pitanje veze $\borel{\produkt{t \in T}{} X_t}$ ($= \sigAlg{\dirProd{t \in T}{} \famU_t}$) i $\dirProd{t \in T}{} \borel{X_t}$.
Lako se vidi da za topolo\v ske prostore $(X_t, \: \famU_t), \; t \in T$, vrijedi:
\begin{equation}    \label{jed:4.3}
    \dirProd{t \in T}{} \borel{X_t} \subseteq \borel{\produkt{t \in T}{} X_t},
\end{equation}
a nije osobito te\v sko konstruirati kontraprimjere koji pokazuju da u \eqref{jed:4.3} op\' cenito ne vrijedi jednakost.

\begin{zad} \label{zad:4.4}
    Ako je $T$ najvi\v se prebrojiv i svaki $X_t$ je separabilan metri\v cki prostor, tada u \eqref{jed:4.3} vrijedi jednakost.
    Posebno, za $d \in \nat$ vrijedi:
    \begin{equation}    \label{jed:4.5}
        \borel{\real^d} = \underbrace{\borel{\real} \otimes \ldots \otimes \borel{\real}}_{d \textnormal{ puta}}.
    \end{equation}
\end{zad}

\begin{rj}[\ref{zad:4.4}]
    Pokazujemo da vrijedi
    \begin{equation*}
        \dirProd{t \in T}{} \borel{X_t} = \borel{\produkt{t \in T}{} X_t}.
    \end{equation*}
    Primjetimo, jedna inkluzija vrijedi bez ograni\v cenja na $X_t$, poka\v zimo prvo nju.
    \begin{enumerate}
        \item[$\subseteq$]
        Sjetimo se, po definiciji, $\dirProd{t \in T}{} \borel{X_t}$ je najmanja $\sigma$-algebra u kojoj su sve projekcije $\pi_t$, $t \in T$ izmjerive.
        Promotrimo sljede\' ci dijagram.
        \begin{figure}[H]
            \centering
            \begin{tikzpicture}
                \matrix (m) [matrix of math nodes,row sep=5em,column sep=8em,minimum width=3em]
                {
                \urePar{\produkt{t \in T}{} X_t}{\borel{\produkt{t \in T}{} X_t}} & \urePar{\produkt{t \in T}{} X_t}{\dirProd{t \in T}{} \borel{X_t}} \\
                & \urePar{X_{t_0}}{\borel{X_{t_0}}} \\};
                \path[-stealth]
                (m-1-1) edge node [above] {$\id$} (m-1-2)
                (m-1-2) edge node [right] {$\pi_{t_0}$} (m-2-2)
                (m-1-1) edge node [below, xshift=-9pt] {$\pi_{t_0} \circ \id$} (m-2-2);
            \end{tikzpicture}
        \end{figure}
        Po propoziciji \ref{prop:4.9} znamo da je identita
        \begin{equation*}
            \id : \Big( \produkt{t \in T}{} X_t, \borel{\produkt{t \in T}{} X_t} \Big) \to \Big( \produkt{t \in T}{} X_t,  \dirProd{t \in T}{} \borel{X_t} \Big)
        \end{equation*}
        neprekidan ako i samo ako joj je svaka od koordinatnih projekcija naprekidna.
        Primjetimo da je $\borel{\produkt{t \in T}{} X_t} = \sigAlg{\dirProd{t \in T}{} U_t}$, stoga nas zanima neprekidnost u odnosu na topologiju $\dirProd{t \in T}{} U_t$, a ona je definirana kao najmanja topologija u kojoj su sve koordinatne projekcije neprekidne, stoga je neprekidna po definiciji, stoga je i identiteta neprekidna, a posljedi\v cno i borelova preka korolaru \ref{kor:3.5}.
        \item[$\supseteq$]
        Budu\' ci su $X_t$ separabilni metri\v cki prostori i $T$ je prebrojiv skup, slijedi da je i $\produkt{t \in T}{} X_t$ separabilan metri\v cki prostor pa ima prebrojivu bazu topologije. Ozna\v cimo sa
        \begin{equation*}
            \famU := \dirProd{t \in T}{} \famU_t,
        \end{equation*}
        produktnu topologiju.
        Ozna\v cimo li s $\famV$ bazu topologije, koristimo tri tvrdnje:
        \begin{enumerate}[label=(\arabic*)]
            \item Skup svih otvorenih cilindri\v cnih pravokutnika je baza topologije $\famU$.
            Vrijedi prema teoremu \ref{tm:4.0-2}.
            \item Ako je baza $\famV$ prebrojiva, tada je $\sigAlg{\famV} = \sigAlg{\famU}$.
            Vrijedi prema teoremu \ref{tm:4.0-3}.
            \item Otvoreni cilindri\v cni pravokutnici se nalaze u produktnoj $\sigma$-algebri.
            Vrijedi prema \eqref{jed:4.2}.
            \begin{equation*}
                \famR \Big( \produkt{t \in T}{} \Omega_t \Big) \subseteq \dirProd{t \in T}{} \borel{X_t}.
            \end{equation*}
        \end{enumerate}
    \end{enumerate}
\end{rj}

\begin{zad} \label{zad:4.6}
    Za svaki skup $A \in \dirProd{t \in T}{} \famF_t$, postoji prebrojiv skup indeksa $J = J(A) \subseteq T$, takav da je $A \in \indSigAlg{\pi_t}{t \in J}$. 
\end{zad}

\begin{rj}
    Promatrajmo skup
    \begin{equation*}
        \famG := \unija{
            \begin{smallmatrix}
                J \subseteq T\\
                \card{J} \leq \aleph_0
            \end{smallmatrix}
        }{} \indSigAlg{\pi_t}{t \in J},
    \end{equation*}
    i ozna\v cimo
    \begin{equation*}
        \famF := \dirProd{t \in T}{} \famF_t.
    \end{equation*}
    Primjetimo da je $\sigma$-algebra $\famF$ generirana izmjerivim cilindri\v cnim pravokutnicima (prema \eqref{jed:4.2}), stoga vrijedi
    \begin{equation*}
        \famR \Big( \produkt{t \in T}{} \Omega_t \Big) \subseteq \famG \subseteq \famF.
    \end{equation*}
    Ako doka\v zemo da je $\famG$ $\sigma$-algebra, tada \' ce vrijediti i $\famF \subseteq \famG$.
    \begin{enumerate}[label=(\roman*)]
        \item Uzmemo neki $t_0 \in T$, tada nu\v zno vrijedi
        \begin{equation*}
            \varnothing \in \sigAlg{\pi_{t_0}} \implies \varnothing \in \unija{
                \begin{smallmatrix}
                    J \subseteq T\\
                    \card{J} \leq \aleph_0
                \end{smallmatrix}
            }{} \indSigAlg{\pi_t}{t \in J} = \famG.
        \end{equation*}
        \item Neka je $A \in \famG$, dakle postoji $J \subseteq T$ najvi\v se prebrojiv, tada vrijedi
        \begin{equation*}
            A \in \indSigAlg{\pi_t}{t \in J} \implies A^c \in \indSigAlg{\pi_t}{t \in J}.
        \end{equation*}
        Dakle vrijedi $A^c \in \famG$.
        \item Neka je niz $\niz{A_n}{n \in \nat} \subseteq \famG$.
        Tada za svaki $n \in \nat$ postoji najvi\v se prebrojiv skup $J_n \subseteq T$, takav da je $A_n \in \indSigAlg{\pi_t}{t \in J_n}$.
        Primjetimo skup $J = \unija{n \in \nat}{} J_n$ je, kao prebrojiva unija najvi\v se prebrojivih skupova, ponovo najvi\v se prebrojiv skup.
        Sada za svaki $n \in \nat$ vrijedi $A_n \in \indSigAlg{\pi_t}{t \in J}$.
        Sada nu\v zno vrijedi
        \begin{equation*}
            \unija{n \in \nat}{} A_n \in \indSigAlg{\pi_t}{t \in J},
        \end{equation*}
        pa samim time vrijedi i $\unija{n \in \nat}{} A_n \in \famG$.
    \end{enumerate}
    Pa je $\famF \subseteq \famG$, dakle vrijedi
    \begin{equation*}
        \famF = \famG.
    \end{equation*}
\end{rj}

\begin{defn}    \label{defn:4.6-1}
    Ako je $I \subseteq T, \; I \neq \varnothing$, onda za svaki $A \subseteq \produkt{t \in T}{} \Omega_t$ i $x \in \produkt{t \in I}{} \Omega_t$ mogu\' ce promarati \emph{prerez od $A$ po $x$}; to jest skup
    \begin{equation*}
        A_x := \bigSkup{y \in \produkt{t \in T \setminus I}{} \Omega_t}{(x, \: y) \in A \; \Big(\subseteq \produkt{t \in T}{} \Omega_t \Big)}.
    \end{equation*}
\end{defn}
Uo\v cimo da je prerez (po $x$) svakog izmjerivog cilindri\v cnog pravokutnika ponovo ili $\varnothing$ ili izmjerivi cilindri\v cni pravokutnik u $\famR \Big( \produkt{t \in T \setminus I}{} \Omega_t \Big)$, pa slijedi:
\begin{equation}    \label{jed:4.7}
    A \in \dirProd{t \in T}{} \famF_t \implies A_x \in \dirProd{t \in T \setminus I}{} \famF_t.
\end{equation}

\begin{nap} \label{nap:4.7-1}
    Neka je $x \in \produkt{t \in I}{} \Omega_t$, defniramo funkciju
    \begin{equation*}
        f_x : \produkt{t \in T \setminus I}{} \Omega_t \to \produkt{t \in T}{} \Omega_t
    \end{equation*}
    tako da vrijedi
    \begin{equation*}
        (\pi_t \circ f_x) (y)
        = \begin{cases}
            \pi_t(x), &t \in I\\
            \pi_t(y), &t \in T \setminus I
        \end{cases}, \quad t \in T.
    \end{equation*}
    \begin{figure}[H]
        \centering
        \begin{tikzpicture}
            \matrix (m) [matrix of math nodes,row sep=5em,column sep=8em,minimum width=3em]
            {
            \produkt{t \in T \setminus I}{} \Omega_t & \produkt{t \in T}{} \Omega_t \\
            & \Omega_{t_0} \\};
            \path[-stealth]
            (m-1-1) edge node [above] {$f_x$} (m-1-2)
            (m-1-2) edge node [right] {$\pi_{t_0}$} (m-2-2)
            (m-1-1) edge node [below, yshift=-9pt] {$\pi_{t_0} \circ f_x$} (m-2-2);
        \end{tikzpicture}
    \end{figure}
    Ona je izmjeriva po koordinatama i vrijedi
    \begin{equation*}
        A_x = \praslika{f_x} (A).
    \end{equation*}
    Stoga je prerez izmjeriv skup.
\end{nap}

\begin{zad} \label{zad:4.8}
    Opi\v site detaljno svojstva prereza u slu\v caju kada je $T = \{1, \: 2\}$.
\end{zad}

Neka je $\izmjerivProstor$ izmjeriv prostor, $T \neq \varnothing$ skup indeksa i $\indFamilija{(E_t, \: \famE_t)}{t \in T}$ familija izmjerivih prostora.
Iz \eqref{jed:4.1} i \eqref{jed:4.2} i napomnene \ref{nap:3.11} direktno slijedi:

\begin{prop} \label{prop:4.9}
    Preslikavanje $X : \Omega \to \produkt{t \in T}{} E_t$ je $(\famF, \: \dirProd{t \in T}{} \famE_t)$-izmjerivo ako i samo ako je za svaki $t \in T$ preslikavanje $\pi_t \circ X$, $(\famF, \: \famE_t)$-izmjerivo.
\end{prop}

Primjetimo, zapravo tvrdimo da sljede\' ci dijagram komutira, te da su sve rezultiraju\' ce funkcije izmjerive.
\begin{figure}[H]
    \centering
    \begin{tikzpicture}
        \matrix (m) [matrix of math nodes,row sep=5em,column sep=8em,minimum width=3em]
        {
        \urePar{\Omega}{\famF} & \\
        \urePar{\produkt{t \in T}{} E_t}{\dirProd{t \in T}{} \famE_t} & \urePar{E_{t_0}}{\famE_{t_0}} \\};
        \path[-stealth]
        (m-1-1) edge node [left] {$X$} (m-2-1)
        (m-2-1) edge node [below] {$\pi_{t_0}$} (m-2-2)
        (m-1-1) edge node [above] {$X_{t_0}$} (m-2-2);
    \end{tikzpicture}
\end{figure}

\begin{nap} \label{nap:4.10}
    \begin{enumerate}[label=(\alph*)]
        \item U slu\v caju kada je $(E_t, \: \famE_t) = (E, \: \famE)$, za svaki $t \in T$, koristimo oznake $E^T := \produkt{t \in T}{} E_t$, $\famE^T := \dirProd{t \in T}{} \famE_t$, ako je uz to jo\v s i $\card{T} = \aleph_0$, koristimo oznake $E^\infty, \; \famE^\infty$.
        \item Ako je $T = \{1, \: 2, \dots, \: d\}$ i $(E, \: \famE) = \urePar{\extReal}{\borel{\extReal}}$ onda je $E^d = \extReal^d$, $(\borel{\extReal})^d = \borel{\extReal^d}$.
        Slu\v cajni element s vrijednostima u $\extReal^d$ nazivamo \emph{pro\v sirenim slu\v cajnim vektorom}.
        Na $\extReal^d$ imamo $d$ projekcija $\pi_1, \dots, \pi_d$ i uvodimo oznaku, za $X : \Omega \to \extReal^d$, $X_i := \pi_i \circ X, \; i = 1, \dots, d$.
        Re\' ci cemo da je pro\v sireni slu\v cajni vektor $X$ slu\v cajni vektor ako je $X \in \real \; (g.s.)$.
        Iz propozicije \ref{prop:4.9} slijedi da je $X$ (pro\v sireni) slu\v cajni vektor ako i samo ako su $X_1, \dots X_d$ (pro\v sirene) slu\v cajne varijable.
    \end{enumerate}
\end{nap}

Neka je $d \in \nat$ i $(\Omega_i, \: \famF_i, \: \mu_i), \; i = 1, \dots, d$ prostori $\sigma$-kona\v cnih mjera.
Za $A = A_1 \times \ldots \times A_d \in \prsten{\Omega_1 \times \ldots \times \Omega_d}$ definiramo
\begin{equation}    \label{jed:4.11}
    \nu (A) := \mu_1 (A_1) \cdot \ldots \cdot \mu_d (A_d),
\end{equation}
te se mo\v ze pokazati da je $\nu$ $\sigma$-kona\v cna funkcija.
Po zadatku \ref{zad:2.9} i teoremu \ref{tm:2.11} slijedi:

\begin{tm}  \label{tm:4.12}
    Postoji i jedinstveno je pro\v sirenje funkcije $\nu$ na mjeru na $\famF_1 \otimes \ldots \otimes \famF_d$.
    To pro\v sirenje je $\sigma$-kona\v cna mjera koju nazivamo \emph{produktnom mjerom} i ozna\v cavamo sa $\mu_1 \otimes \ldots \otimes \mu_d$.
    Bez smanjenja op\' cenitosti niti jedna od mjera $\mu_i$ nije trivijalna (to jest vrijedi $\mu_i (\Omega_i) \neq 0 \; i=1, \dots, d$).
    Tada je $\mu_1 \otimes \ldots \otimes \mu_d$ kona\v cana ako i samo ako su $\mu_1, \ldots, \mu_d$ kona\v cne.
    Ako su $\mu_1, \ldots, \mu_d$ vjerojatnosne, tada je i $\mu_1 \otimes \ldots \otimes \mu_d$ vjerojatnost. 
\end{tm}

\begin{pr}  \label{pr:4.13}
    Neka je $\mu_1 = \ldots = \mu_d = \lambda$ (Lebesgueova mjera).
    Tada je $\mu_1 \otimes \ldots \otimes \mu_d$ $\sigma$-kona\v cna mjera na $\urePar{\real^d}{\borel{\real^d}}$, koju ozna\v cavamo $\lambda^d$ i zovemo \emph{$d$-dimenzionalana Lebesgueova mjera}.
\end{pr}

\begin{zad} \label{zad:4.15}
    Neka su $(\Omega_i, \: \famF_i, \: \mu_i), \; i = 1, \; 2$, prostori $\sigma$-kona\v cne mjere.
    Tada vrijedi:
    \begin{enumerate}[label=(\roman*)]
        \item Za svaki $A \in \famF_1 \otimes \famF_2$, preslikavanje
        \begin{equation*}
            \Omega_1 \ni x \mapsto \mu_2 (A_x)
        \end{equation*}
        je $\famF_1$-izmjerivo, a preslikavanje
        \begin{equation*}
            \Omega_2 \ni y \mapsto \mu_1 (A_y),
        \end{equation*}
        je $\famF_2$-izmjerivo.
        \item Za svaki $A \in \famF_1 \otimes \famF_2$ vrijedi:
        \begin{equation*}
            (\mu_1 \otimes \mu_2) (A) = \int_{\Omega_1} \mu_2 (A_x) \: d \mu_1(x) = \int_{\Omega_2} \mu_1 (A_y) \: d \mu_2 (y).
        \end{equation*}
        \item Za svaki $f: \Omega_1 \times \Omega_2 \to \segment{0}{+\infty}$ koji je $\urePar{\famF_1 \otimes \famF_2}{\borel{\segment{0}{+\infty}}}$-izmjeriv
        \begin{align*}
            \int_{\Omega_1 \times \Omega_2} f(x, \: y) \: d (\mu_1 \otimes \mu_2)(x, \:y) &= \int_{\Omega_2} \Big( \int_{\Omega_1} f(x, \: y) \: d \mu_1 (x) \Big) \: d \mu_2 (y)\\
            &= \int_{\Omega_1} \Big( \int_{\Omega_2} f(x, \: y) \: d \mu_2 (y) \Big) \: d \mu_1 (x)
        \end{align*}
    \end{enumerate} 
\end{zad}

\begin{zad} \label{zad:4.16}
    Neka su $(\Omega_i, \: \famF_i, \: \mu_i), \; i = 1, \: 2$, prostori $\sigma$-kona\v cne mjere i $f : \Omega_1 \times \Omega_2 \to \extReal$ $\urePar{\famF_1 \otimes \famF_2}{\borel{\extReal}}$-izmjeriva, takva da je $|f|$ $\mu_1 \otimes \mu_2$-integrabilna.
    Tada vrijedi (\emph{Fubinijev teorem}):
    \begin{enumerate}[label=(\roman*)]
        \item Funkcija $y \mapsto f(x, \: y)$ je $\mu_2$-integrabilna za gotovo svaki $x \in \Omega_1$, a funkcija $x \mapsto f(x, y)$ $\mu_1$-integrabilna za gotovo svaki $y \in \Omega_2$.
        \item Funkcija
        \begin{equation*}
            x \mapsto
            \begin{cases}
                \int_{\Omega_2} f(x, \: y) \: d \mu_2 (y), &\textnormal{ako je } y \mapsto f(x, \: y) \textnormal{ integrabilna}\\
                0, &\textnormal{ina\' ce} 
            \end{cases}
        \end{equation*}
        je $\mu_1$-integrabilna, a funkcija
        \begin{equation*}
            y \mapsto
            \begin{cases}
                \int_{\Omega_1} f(x, \: y) \: d \mu_1 (x), &\textnormal{ako je } x \mapsto f(x, \: y) \textnormal{ integrabilna}\\
                0, &\textnormal{ina\' ce}
            \end{cases}
        \end{equation*}
        je $\mu_2$-integrabilna.
        \item
        \begin{align*}
            \int_{\Omega_1 \times \Omega_2}  f(x, \: y) \: d (\mu_1 \otimes \mu_2)(x, \: y)  &= \int_{\Omega_2} \Big( \int_{\Omega_1} f(x, \: y) \: d \mu_1 (x) \Big) \: d \mu_2(y)\\
            &= \int_{\Omega_1} \Big( \int_{\Omega_2} f(x, \: y) \: d \mu_2 (y) \Big) \: d \mu_1(x).
        \end{align*}
    \end{enumerate}
\end{zad}

\begin{defn}    \label{defn:4.17}
    Neka su $\urePar{\Omega_1}{\famF_1}$ i $\urePar{\Omega_2}{\famF_2}$ izmerivi prostori. Preslikavanje $k : \Omega_1 \times \famF_2 \to \segment{0}{+\infty}$ je \emph{uniformno $\sigma$-kon\v cna jezgra} ako vrijedi:
    \begin{enumerate}[label=(\alph*)]
        \item $(\forall \omega \in \Omega_1 ) \; \famF_2 \ni A \mapsto k(\omega, \: A)$ je mjera,
        \item $(\forall A \in \famF_2 ) \; \Omega_1 \ni \omega \mapsto k(\omega, \: A)$ je $\famF_1$-izmjeriva,
        \item $(\exists B_n \in \famF)(\exists c_n > 0) $ td. $ \Omega_2 = \unija{n = 1}{\infty} B_n$ i $(\forall n \in \nat)(\forall \omega \in \Omega_1)$ $k(\omega, \: B_n) \leq c_n$.\\
        Ako je $c_n = c$, $\forall n \in \nat$, $B_n = \Omega_2$, $\forall n \in \nat$, ka\v zemo da je $k$ \emph{uniformno kona\v cna jezgra}. Ako je $k(\omega, \: B) \leq 1$, $\forall \omega \in \Omega_1$, $\forall B \in \famF_2$ ka\v zemo da je $k$ \emph{subvjerojatnosna jezgra}. Ako je $k(\omega, \: \Omega_1) = 1$, $\forall \omega \in \Omega_1$, ka\v zemo da je $k$ \emph{vjerojatnosna jezgra}.
    \end{enumerate}
\end{defn}

\begin{zad} \label{zad:4.18}
    Ako je $\mu$ $\sigma$-kona\v cna mjera na $\urePar{\Omega_1}{\famF_1}$ i $k$ uniformno kona\v cna jezgra na skupu $\Omega_1 \times \famF_2$.
    Tada postoji i jedinstvena je $\sigma$-kona\v cna mjera $\nu$ na $\urePar{\Omega_1 \times \Omega_2}{\famF_1 \otimes \famF_2}$ takva da je
    \begin{equation*}
        \nu (A \times B) = \int\limits_A k(\omega, \: B) \: d \mu (\omega),
    \end{equation*}
    za svaki $A \times B \in \prsten{\Omega_1 \times \Omega_2}$.
    Ako je funkcija
    \begin{equation*}
        f:\Omega_1 \times \Omega_2 \to \segment{0}{+\infty}
    \end{equation*}
    $\urePar{\famF_1 \otimes \famF_2}{\borel{\segment{0}{+\infty}}}$-izmjeriva, tada je
    \begin{equation*}
        \int\limits_{\Omega_1 \times \Omega_2} f(\omega_1, \: \omega_2) \: d \nu (\omega_1, \: \omega_2) = \int\limits_{\Omega_1} \Big( \int\limits_{\Omega_2} f(\omega_1, \: \omega_2) \: k(\omega_1, \: d \omega_2) \Big) \: d \mu (\omega_1).
    \end{equation*}
\end{zad}

\begin{zad} \label{zad:4.19}
    Iska\v zite i doka\v zite tvrdnje u zadacima \ref{zad:4.15}, \ref{zad:4.16} i \ref{zad:4.18} za slu\v caj vjerojatnosnih mjera i vjerojatnosnih jezgara.
\end{zad}

\v Sto mo\v zemo re\' ci o beskona\v cnom produktu?
Prvo gledamo slu\v caj $T = \nat$. Neka su $\urePar{\Omega_n}{\famF_n}$, $n \in \nat$, izmjerivi prostori. Po \eqref{jed:4.2}, znamo da je $\prsten{\produkt{n \in \nat}{} \Omega_n}$ generiraju\' ci poluprsten za $\dirProd{n \in \nat}{} \famF$, pa bi bilo dovoljno mjere zadati na $\prsten{\produkt{n \in \nat}{} \Omega_n}$.
Ako na svakom $\urePar{\Omega_n}{\famF_n}$ imamo $\sigma$-kona\v cnu mjeru $\mu_n$, prirodno bi se moglo o\v cekivati da na $A_1 \times \ldots \times A_n \times \Omega_{n + 1} \times \ldots$ zadamo mjeru pomo\' cu $\mu_1(A_1) \cdot \ldots \cdot \mu_n(A_n)$.
Tada se postavlja pitanje \emph{konzistentnosti}, na primjer skup $A \times \Omega_2 \times \Omega_3 \times \ldots$ mo\v ze imati pridru\v zene mjere $\mu_1(A), \; \mu_1(A) \cdot \mu_2(\Omega_2), \: \mu_1(A) \cdot \mu_2(\Omega_2) \cdot \mu_3 (\Omega_3)$, i tako dalje.
U pravilu one \' ce biti jednake, jedino ako su to vjerojatnosne mjere.
No mo\v zemo postupati druga\' cije.
Neka je, za svaki $n \in \nat$, $\famG_n := \sigAlg{\pi_1, \ldots, \pi_n}$, pa je $\famG_1 \subseteq \famG_2 \subseteq \ldots$, tvore uzlazni niz $\sigma$-algebri takav da je $\famG_{\infty} := \unija{n = 1}{\infty} \famG_n$ algebra, ali ne nu\v zno i $\sigma$-algebra, i $\sigAlg{\famG_{\infty}} = \dirProd{n \in \nat}{} \famF_n$.
% nisam siguran što je ovaj G točno :/
% \famG =? \sigAlg{\famG_\infty}
Sada bi se na $\famG$ mogli promatrati nizovi $\niz{(\mu_1 \otimes \ldots \otimes \mu_n)(G)}{n \in \nat}$ i u slu\v caju da postoje limesi $\lim\limits_{n \to \infty} (\mu_1 \otimes \ldots \otimes \mu_n)(G)$, taj bi se limes mogao uzeti za $\nu (G)$.
Ali i taj put vodi na probleme, ilustrirajmo jedan od njih.

\begin{pr}  \label{pr:4.20}
    Neka je $(\Omega_n, \: \famF_n, \: \mu_n) = (\{0, \: 1\}, \: \partitive{\{0, \: 1\}}, \: \mu_n(\{0\}) = \mu_n (\{1\}) = 1)$.
    Uzmemo li bilo koju to\v cku $a \in \{ 0, \: 1 \}^{\nat}$, postoji $\lim\limits_{n \to \infty} (\mu_1 (\{ a_1 \}) \cdot \mu_2 (\{ a_2 \}) \cdot \dots \cdot \mu_n (\{ a_n \}) ) = 1$ pa bi "mjera" na $\{ 0, \: 1 \}^{\infty}$ koja bi bila kandidat za $\dirProd{n \in \nat}{} \mu_n$ imala neprebrojivo jedno\v clanih skupova mjere $1$, to jest nikako ni bi mogla biti $\sigma$-kona\v cna. 
\end{pr}

U vjerojatnosnom slu\v caju sve je u redu.

\begin{tm}  \label{tm:4.21}
    Neka je $(\Omega_n, \: \famF_n, \: \Pp_n), \; n \in \nat$, niz vjerojatnosnih prostora.
    Postoji i jedinstvena je vjerojatnosna mjera $\masP$ na $\Big( \produkt{n \in \nat}{} \Omega_n, \dirProd{n \in \nat}{} \famF_n \Big)$ takva da je za svaki $n \in \nat$, za svaki izbor $A_1 \in \famF_1, \dots, A_n \in \famF_n$,
    \begin{equation*}
        %\masP (\praslika{\pi_1}(A_1) \cap \dots \cap \praslika{\pi_n}(A_n))
        \masP(\pi_1 \in A_1, \: \dots, \pi_n \in A_n)
        = \produkt{k = 1}{n} \masP_k (A_k).
    \end{equation*}
    Tu vjerojatnost nazivamo \emph{beskona\v cnim produktom} i ozna\v cavamo sa $\dirProd{n \in \nat}{} \masP_n$.
\end{tm}

\begin{zad} \label{zad:4.22}
    Formulirajte op\' ci teorem o produktu vjerojatnosnih mjera i poka\v zite da se dokazuje u su\v stini potpuno isto kao kao i teorem \ref{tm:4.21}.
    % uputa: vidi zadatak 6, a u dokazu koristi zadatak 1.12 b i Cantorov diagonalni postupak.
\end{zad}