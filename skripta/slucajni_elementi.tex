% poglavlje o slučajnim elementima
\usetikzlibrary{arrows,chains,matrix,positioning,scopes}

\chapter{Slu\v cajni elementi} \label{poglavlje3}

Neka su $D$, $K$ neprazni podskupovi i $f: D \to K$ preslikavanje.
Postoji bitna razlika izme\dj u slika i inverznih slika kada govorimo o skupovnim operacijama. Ako je $*$ bilo koja od operacija $\cup, \: \cap, \: \setminus \:$ te ako su $B_1, \: B_2 \subseteq K$, tada je
\begin{equation} \label{jed:3.1}
    \praslika{f} (B_1 * B_2) = \praslika{f}(B_1) * \praslika{f}(B_2),
\end{equation}
ali ako su $A_1, \: A_2 \subseteq D$, tada je
\begin{align*}
    f(A_1 \cap A_2) \subsetneqq& f(A_1) \cap f(A_2) \\
    f(A_1 \setminus A_2) \supsetneqq& f(A_1) \setminus f(A_2). 
\end{align*}
Posebno to zna\v ci da ako je $\famD \subseteq \partitive{D}$ $\sigma$-algebra na $D$, $f(\famD) := \skup{f(A)}{A \in \famD}$ nije nu\v zno $\sigma$-algebra na $K$.
S druge strane iz \eqref{jed:3.1} (i sli\v cnih relacija za prebrojive operacije) direktno slijedi:

Ako je $\famK \subseteq \partitive{K}$ "struktura", tada je i
\begin{equation} \label{jed:3.2}
    \praslika{f}(\famK) := \skup{\praslika{f}(B)}{B \in
        \famK}
\end{equation}
tako\dj er "struktura", kao i ako je $\famD \subseteq \partitive{D}$, tada je i
\begin{equation} \label{jed:3.3}
    \skup{B \in \partitive{K}}{\praslika{f}(B) \in \famD},
\end{equation}
tako\dj er "struktura"; pri \v cemu pod "struktura" mislimo $\pi$-sistem, poluprsten, prsten, Dynkinova klasa, algebra, $\sigma$-prsten ili $\sigma$-algebra.

Od interesa su preslikavanja koja po\v stuju odre\dj en
izbor $\sigma$-algebri.
\begin{defn}    \label{defn:3.3-1}
    Ako su $(D, \: \famD)$ i $(K, \: \famK)$ izmjerivi prostori i $f: D \to K$ preslikavanje, tada ka\v zemo da je $f$ \emph{izmjerivo} (ili \emph{$(\famD, \: \famK)$-izmjerivo}) ako je
    \begin{equation*}
        \praslika{f}(\famK) \subseteq \famD.
    \end{equation*}
\end{defn}

Uo\v cimo da iz \eqref{jed:3.2} slijedi da je $\praslika{f}(\famK)$ uvijek $\sigma$-algebra i da je to najmanja $\sigma$-algebra na $D$ u odnosu na koju je $f$ izmjeriva (u odnosu na $\famK$).
Zato ka\v zemo da je $\praslika{f}(\famK)$ \emph{$\sigma$-algebra generirana sa $f$} i \v cesto je ozna\v cavamo sa \emph{$\sigAlg{f}$}.
Dakle $f$ je izmjeriva ako i samo ako vrijedi:
\begin{equation*}
    \sigAlg{f} \subseteq \famD.
\end{equation*}
Ovaj pojam se lako poop\' ci.
Neka je $\indFamilija{(K_t, \: \famK_t)} {t \in T}$ proizvoljna indeksirana familija izmjerivih prostora i neka za svaki $t \in T$ imamo preslikavanje $f_t : D \to K_t$.
Tada je  $\indSigAlg{f_t}{t \in T} := \sigAlg{\unija{t \in T}{} \sigAlg{f_t}}$ \emph{$\sigma$-algebra generirana familijom $\indFamilija{f_t}{t \in T}$} i to je najmanja $\sigma$-algebra na $D$ u odnosu na koju su sva preslikavanja $f_t$ izmjeriva.

Ako su $(X, \: \famU)$ i $(Y, \: \famV)$ topolo\v ski prostori, onda $(\borel{X}, \: \borel{Y})$-izmjerivo preslikavanje $f: X \to Y$ nazivamo \emph{Borelovim preslikavanjem}.
Podsjetimo se da je $f: X \to Y$ neprekidno ako je $\praslika{f}(\famV) \subseteq \famU$.
Kakva je veza izme\dj u ovih pojmova?

\begin{lm}  \label{lm:3.4}
    Ako je $f: D \to K$ preslikavanje i $\famC \subseteq \partitive{K}$, tada je $\sigAlg{\praslika{f}(\famC)} = \praslika{f} (\sigAlg{\famC})$.
\end{lm}

\begin{proof}
    Po \eqref{jed:3.2} $\praslika{f}(\sigAlg{\famC})$ je $\sigma$-algebra, koja o\v cito sadr\v zi $\praslika{f}(\famC)$, pa je
    \begin{equation*}
        \sigAlg{\praslika{f}(\famC)} \subseteq \praslika{f}(\sigAlg{\famC}).
    \end{equation*}
    Sada po \eqref{jed:3.3} slijedi da je $\skup{B \in \partitive{K}} {\praslika{f}(B) \in \sigAlg{\praslika{f}(\famC)}}$ je $\sigma$-algebra, koja o\v cito sadr\v zi $\famC$, pa onda nu\v zno mora sadr\v zavati i $\sigAlg{\famC}$.
    Dakle,
    \begin{equation*}
        \praslika{f}(\sigAlg{\famC}) \subseteq \sigAlg{\praslika{f}(\famC)}.
    \end{equation*}
\end{proof}

Zbog $\praslika{f}(\borel{Y}) = \praslika{f}(\sigAlg{\famV}) = \sigAlg{\praslika{f}(\famV)} \subseteq \sigAlg{\famU} = \borel{X}$, slijedi:

\begin{kor} \label{kor:3.5}
    Ako je $f: (X, \: \famU) \to (Y, \: \famV)$ neprekidna, tada je $f$ Borelova.
\end{kor}

Uo\v cimo nadalje da direktno iz definicije slijedi:

\begin{lm}  \label{lm:3.6}
    Neka su $(A, \: \famA)$, $(B, \: \famB)$ i $(C, \: \famC)$ izmjerivi prostori. Ako su $f: A \to B$ i $g:B \to C$ izmjeriva preslikavanja, tada je izmjeriva i $g \circ f$.
\end{lm}

\begin{defn}    \label{defn:3.7}
    Neka je $\vjerojatnosniProstor$ vjerojatnosni prostor i $(E, \: \E)$ izmjeriv prostor. Preslikavanje $f: \Omega \to E$ je \emph{slu\v cajni element} (\emph{s vrijednostima u $E$}) ako je $f$
    $(\F, \: \E)$-izmjerivo.
\end{defn}

\begin{nap} \label{nap:3.7-1}
    Sjetimo se, skup
    \begin{equation*}
        \extReal := \real \cup \{ -\infty, \; +\infty \}
    \end{equation*}
    nazivamo \emph{pro\v sirenim skupom realnih brojeva}.
    U topologiji na $\extReal$ skupovi oblika $\desInt{-\infty}{a}$, $\lijInt{b}{+\infty}$ su otvoreni.
\end{nap}

\begin{defn}  \label{defn:3.8}
    Neka je $\vjerojatnosniProstor$ vjerojatnosti prostor i neka je $\urePar{E}{\famE} = \urePar{\extReal}{\borel{\extReal}}$ izmjeriv prostor.
    Slu\v cajne elemente s vrijednostima u $\extReal$ nazivamo \emph{pro\v sirenim slu\v cajnim varijablama}.

    Ako je $X$ pro\v sirena slu\v cajana varijabla i $\vjeroj{|X| = \infty} = 0$, tada ka\v zemo da je $X$ \emph{slu\v cajna varijabla}.
\end{defn}

\begin{nap} \label{nap:3.8-1}
    Uo\v cimo da su $\{ -\infty \}, \; \{+\infty\} \in \borel{\extReal}$.
    Na primjer
    \begin{equation*}
        \{+\infty\} = \presjek{n \in \N}{} \lijInt{n}{+\infty}.
    \end{equation*}
    
    Posebno, slu\v cajni element s vrijednostima u $\R$ je slu\v cajna varijabla.
    Na sli\v can na\v cin mo\v zemo promatraiti i slu\v cajne elemente s vrijednostima u $\R_+ := \desInt{0}{+\infty}$, ili $\extReal_+ := \segment{0}{+\infty}$. 
\end{nap}

\begin{nap} \label{nap:3.8.1}
    $\borel{\extReal}$ mo\v zemo promatrati i kao najmanju $\sigma$-algebru koja sadr\v zi sve elemente iz $\borel{\R}$, kao i skupove $\{-\infty\}$, $\{+\infty\}$.
    Odnosno vrijedi:
    \begin{equation*}
        \borel{\extReal} = \sigAlg{\borel{\R} \cup \{\{-\infty\}, \: \{+\infty\}\}}
    \end{equation*}
\end{nap}

\begin{nap} \label{nap:3.9}
    Neka je $\vjerojatnosniProstor$ vjerojatnosni prostor i $S$ neka je neko svojstvo, takvo da za svaki $\omega \in \Omega$ mo\v zemo utvrditi (u principu) da "$\omega$ zadovoljava $S$" ili da "$\omega$ ne zadovoljava $S$".
    Ka\v zemo da je $S$ zadovoljeno \emph{gotovo sigurno} ($g.s.$) ako postoji skup $A \in \F$ takvo da je $\vjeroj{A} = 1$ i $A \subseteq \skup{\omega \in \Omega}{\omega \; \textnormal{zadovoljava} \; S}$.
    Uo\v cimo da skup $\vjeroj{A} = 1$ i $A \subseteq \skup{\omega \in \Omega}{\omega \; \textnormal{zadovoljava} \; S}$ ne mora biti doga\dj aj.
    Sjetimo se da je vjerojatnosni prostor \emph{potpun} ako vrijedi:
    \begin{equation*}
        A \subseteq E, \; E \in \F, \; \vjeroj{E} = 0 \implies A \in \F.
    \end{equation*}
    O\v cito, ako je $\vjerojatnosniProstor$ potpun i $S$ je zadovoljeno gotovo sigurno, tada je $\vjeroj{A} = 1$ i $A \subseteq \skup{\omega \in \Omega}{\omega \; \textnormal{zadovoljava} \; S} \in \F$.

    Uz ove pojmove mo\v zemo re\' ci da je pro\v sirena slu\v cajna varijabla $X$ slu\v cajna varijabla ako i samo ako je $X \in \R \; (g.s.)$
\end{nap}

\begin{nap} \label{nap:3.9-1}
    Neka je $\vjerojatnosniProstor$ vjerojatnosni prostor, \emph{upotpunjenje vjerojatnosnog prostora} je potpun vjerojatnosni prostor $(\Omega, \: \overline{\famF}, \overline{\masP})$, gdje je $\famF \subseteq \overline{\famF}$ i gdje je $\overline{\masP}$ pro\v sireneje od $\masP$, odnosno vrijedi:
    \begin{equation*}
        \forall A \in \famF \quad \vjeroj{A} = \overline{\masP}(A).
    \end{equation*}
\end{nap}

\begin{zad} \label{zad:3.10}
    Doka\v zite da se svaki vjerojatnosni prostor mo\v ze upotpuniti.
    Nadalje, ako je $f$ slu\v cajni element na polaznom vjerojatnosnom
    prostoru, tada je $f$ slu\v cajni element i na upotpunjenu.
\end{zad}


\begin{rj}[\ref{zad:3.10}]  \label{rj:3.10}
    Doka\v zimo prvo da je svaki vjerojatnosni prostor mogu\' ce\\ upotpuniti.
    Neka je $\vjerojatnosniProstor$ vjerojatnosni prostor, tada postoji (minimalano) upotpunjenje $(\Omega, \; \overline{\famF}, \; \overline{\masP})$.

    Definirajmo $\famN := \skup{E \subset F}{F \in \famF \; \land \; \vjeroj{F} = 0}$ te $\overline{\famF} = \skup{A \cup E}{A \in \famF, \; E \in \famN}$ te tako\dj er $\overline{\masP} : \overline{\famF} \to \segment{0}{+\infty}$ na elementima $\overline{\famF}$ sa $\overline{\masP}(A \cup E) = \vjeroj{A}$.
    Dokazujemo sljede\' ce tvrdnje.
    \begin{enumerate}[label=(\arabic*)]
        \item   \label{rj:3.10.1}
        $\overline{\famF}$ je $\sigma$-algebra
        \item   \label{rj:3.10.2}
        $\overline{\masP}$ je vjerojatnosti
        \item   \label{rj:3.10.3}
        $(\Omega, \: \overline{\famF}, \overline{\masP})$ je potpun vjerojatnosni prostor
        \item   \label{rj:3.10.4}
        $(\Omega, \: \overline{\famF}, \overline{\masP})$ je minimalno potpuno pro\v sirenje od $\vjerojatnosniProstor$.
    \end{enumerate}

    Krenimo redom, poka\v zimo \ref{rj:3.15.1}.
    Primjetimo na po\v cetku da je $\forall A \in \famF, \; A = A \cup \varnothing$, stoga vrijedi $\famF \subseteq \overline{\famF}$
    \begin{enumerate}[label=(\roman*)]
        \item O\v cito je $\varnothing \in \overline{\famF}$
        \item Neka je $A \cap E \in \overline{\famF}$, sada postoji $E \in \famF$ takav da $\vjeroj{F} = 0$ i $F \subseteq E$.
        Vrijedi:
        \begin{equation*}
            \begin{aligned}
                (A \cup E)^c &= (A \cup (E \cap F))^c = (A \cup (F \setminus (F \setminus E)))^c = (A \cup (F \cap (F \setminus E)^c))^c\\
                &= A^c \cap (F \cap (F \setminus E)^c)^c = A^c \cap (F^c \cup (F \setminus E))\\
                &= (A^c \cap F^c) \cup (A^c \cap (F \setminus E))
            \end{aligned}
        \end{equation*}
        Budu\' ci su $A, \; F \in \famF$, i da je $A^c \cap (F \setminus E) \subseteq F$, vrijedi da je $\vjeroj{F} = 0$, vidimo da je $(A \cup E)^c \in \overline{\famF}$.
        \item Neka je $\niz{A_n \cup E_n}{n \in \nat}$ niz u $\overline{\famF}$, gdje je $A_n \in \famF$, a $E_n \subseteq F_n$ za neke $F_n \in \famF$, za koje vrijedi $\vjeroj{F_n} = 0$.
        Sada vidimo:
        \begin{equation*}
            \unija{n \in \nat}{} (A_n \cup E_n) = \unija{n \in \nat}{} A_n \cup \unija{n \in \nat}{} E_n.
        \end{equation*}
        Vrijedi da je $\unija{n \in \nat}{} A_n \in \famF$ i tako\dj er vrijedi $\unija{n \in \nat}{} E_n \subseteq \unija{n \in \nat}{} F_n$, gdje je $\unija{n \in \nat}{} F_n \in \famF$ i uz to vrijedi $\vjeroj{\unija{n \in \nat}{} F_n} \leq \suma{n \in \nat}{} \vjeroj{F_n} = \suma{n \in \nat}{} 0 = 0$.
        Dakle vidimo da je $\unija{n \in \nat}{} (A_n \cup E_n) \in \overline{\famF}$
    \end{enumerate}
    pa je $\overline{\famF}$ $\sigma$-algebra na $\Omega$.

    Poka\v zimo sada \ref{rj:3.15.2}.
    Prvo poka\v zimo da je $\overline{\masP}$ dobro definirana, odnosno da ne ovisiti o reprezentaciji, to jest ako imamo skupove $A \cup E$, $A' \cup E'$ takve da $A, \; A' \in \famF$, $E \subseteq F, \; E' \subseteq F'$ takvi da $F, \; F' \in \famF$, sa $\vjeroj{F} = \vjeroj{F'} = 0$ te uz to jo\v s i vrijedi $A \cup E = A' \cup E'$, tada mora biti $\overline{\masP} (A \cup E) = \overline{\masP} (A' \cup E')$.
    Primjetimo vrijedi:
    \begin{equation*}
        (A \cup E) \cup (F \cup F') = (A' \cup E') \cup (F \cup F') \iff A \cup F \cup F' = A' \cup F \cup F',
    \end{equation*}
    dakle,
    \begin{equation*}
        \begin{aligned}
            \vjeroj{A} \leq \vjeroj{A \cup F \cup F'} = \vjeroj{A' \cup F \cup F'} \leq \vjeroj{A'} + \vjeroj{F \cup F'} = \vjeroj{A'} + 0 = \vjeroj{A'}\\
            \vjeroj{A'} \leq \vjeroj{A' \cup F \cup F'} = \vjeroj{A \cup F \cup F'} \leq \vjeroj{A} + \vjeroj{F \cup F'} = \vjeroj{A} + 0 = \vjeroj{A}.
        \end{aligned}
    \end{equation*}
    Odakle vidimo da nu\v zno vrijedi $\vjeroj{A} = \vjeroj{A'}$, samim time mora vrijediti i $\overline{\masP} (A \cup E) = \overline{\masP} (A' \cup E')$, pa je $\overline{\masP}$ dobro definirana.
    Da je $\overline{\masP}$ ekstenzija od $\masP$ vidimo tako da za svaki $A \in \famF$ vrijedi $A = A \cup \varnothing$, pa prema tome $\vjeroj{A} = \overline{\masP} (A)$.
    Poka\v zimo da je $\overline{\masP}$ vjerojatnost.
    \begin{enumerate}[label=(\roman*)]
        \item $\overline{\masP} (A \cup E) = \vjeroj{A} \geq 0, \quad \forall (A \cup E) \in \overline{\famF}$
        \item $\overline{\masP} (\Omega) = \vjeroj{\Omega} = 1$
        \item Neka je $\niz{A_n \cup E_n}{n \in \nat} \subseteq \overline{\famF}$ tada vrijedi $A_n \in \famF$ i za svaki $E_n$, postoji $F_n \in \famF$ takav da vrijedi $E_n \subseteq F_n, \; \vjeroj{F_n} = 0$. Sada imamo
        \begin{equation}    \label{jed:3.10-1}
            \begin{aligned}
                \unija{n \in \nat}{} E_n &\subseteq \unija{n \in \nat}{} F_n\\
                \vjeroj{\unija{n \in \nat}{} F_n} &\leq \suma{n \in \nat}{} \vjeroj{F_n} = 0.
            \end{aligned}
        \end{equation}
        Odakle dobijemo:
        \begin{equation*}
            \begin{aligned}
                \overline{\masP} (\unija{n \in \nat}{} (A_n \cup E_n)) &= \overline{\masP} (\unija{n \in \nat}{} A_n \cup \unija{n \in \nat}{} E_n) \overset{\eqref{jed:3.10-1}}{=} \vjeroj{\unija{n \in \nat}{} A_n} = \suma{n \in \nat}{} \vjeroj{A_n}\\ &= \suma{n \in \nat}{} \overline{\masP} (A_n \cup E_n).
            \end{aligned}
        \end{equation*}
    \end{enumerate}
    Pa je $\overline{\masP}$ mjera na $\overline{\famF}$.

    Poka\v zimo sada \ref{rj:3.10.3}.
    Neka je $F \in \overline{\famF}$ takav da je $\overline{\masP} (F) = 0$ i neka je $E \subseteq F$.
    \v Zelimo pokazati da je $E \in \overline{\famF}$.
    Kako je $F \in \overline{\famF}$, postoje $A \in \famF, \; E' \in \famN$ takvi da vrijedi $F = A \cup E'$.
    Po definicijskom uvijetu je $\overline{\masP} (F) = \vjeroj{A} = 0$.
    Kako je $E' \in \famN$, postoji $F' \in \famF$ takav da je $E' \subseteq F'$, $\vjeroj{F'} = 0$.
    Sada imamo $\vjeroj{A \cup F'} \leq \vjeroj{A} + \vjeroj{F'} = 0 + 0$, a kako imamo:
    \begin{equation*}
        E \subseteq F = A \cup E' \subseteq A \cup F',
    \end{equation*}
    te vrijedi $\vjeroj{A \cup F'} = 0$, pa slijedi da je $E \in \famN$.
    Kako je $\famE \subseteq \overline{\famF}$, slijedi $E \in \overline{\famF}$, pa je po monotonosti vjerojatnosti $\overline{\masP} (E) = 0$.
    Pa je $(\Omega, \: \overline{\famF}, \: \overline{\masP})$ potpun vjerojatnosni prostor.

    Na poslijetku poka\v zimo da vrijedi \ref{rj:3.10.4}.
    Neka je $(\Omega, \: \famG, \: \masO)$ potpun vjerojatnosni prostor, takav da je $\famF \subseteq \famG$ te $\restr{\masO}{\famF} = \masP$.
    Neka je $A \cup E \in \overline{\famF}$, takav da je $A \in \famF, \; E \in \famN$, tada postoji $F \in \famF$ takav da je $E \subseteq F$, $\vjeroj{F} = 0$.
    Kako je $\masO$ pro\v sirenje od $\masP$ nu\v zno vrijedi $\masO (F) = 0$.
    Kako je $(\Omega, \: \famG, \: \masO)$ potpun, tada je $E \in \famG$, pa posljedi\v cno $A \cup E \in \famG$, stoga je $\overline{\famF} \subseteq \famG$.
    Dakle $(\Omega, \; \overline{\famF}, \; \overline{\masP})$ je minimalno upotpunjenje.
    
    Komentirajmo jo\v s drugi dio zadatka.
    Neka je $\vjerojatnosniProstor$ i neka je $\urePar{E}{\famE}$ izmjeriv prostor te $f: \Omega \to E$ slu\v cajni element.
    Ako je $(\Omega, \; \overline{\famF}, \; \overline{\masP})$ upotpunjenje prostora $\vjerojatnosniProstor$, tada je $f$, tako\dj er slu\v cajni element na upotpunjenju.
    Promotrimo sljede\' ci dijagram:
    \begin{figure}[H]
        \centering
        \begin{tikzpicture}
            \matrix (m) [matrix of math nodes,row sep=5em,column sep=8em,minimum width=3em]
            {
            (\Omega, \; \overline{\famF}, \; \overline{\masP}) & (\Omega, \; \famF, \; \masP) \\
            & \urePar{E}{\famE} \\};
            \path[-stealth]
            (m-1-1) edge node [above] {$\id$} (m-1-2)
            (m-1-2) edge node [right] {$f$} (m-2-2)
            (m-1-1) edge node [below] {$f$} (m-2-2);
        \end{tikzpicture}
    \end{figure}
    Vrijedi:
    \begin{equation*}
        \sigAlg{f} \subseteq \famF \subseteq \overline{\famF}.
    \end{equation*}

    S druge strane, ako su $\vjerojatnosniProstor$ i $(E, \; \famE, \; \masO)$ vjerojatnosni prostori i $f: \Omega \to E$ slu\v cajni element, tada $f$ nije nu\v zno slu\v cajni element na upotpunjenju $(E, \; \overline{\famE}, \; \overline{\masO})$, vjerojatnosnog prostora $(E, \; \famE, \; \masO)$.
    Primjetimo, ako uzmemo proizvoljan vjerojatnosni prostor $\vjerojatnosniProstor$ koji nije potpun i identitetu na njemu, ona je slu\v cajni element, ali nije $\urePar{\famF}{\overline{\famF}}$-izmjeriva.
    Naime, promotrimo:
    \begin{figure}[H]
        \centering
        \begin{tikzpicture}
            \matrix (m) [matrix of math nodes,row sep=5em,column sep=8em,minimum width=3em]
            {
            (\Omega, \; \famF, \; \masP) & (E, \; \famE, \; \masO) \\
            & (E, \; \overline{\famE}, \; \overline{\masO}) \\};
            \path[-stealth]
            (m-1-1) edge node [above] {$f$} (m-1-2)
            (m-1-2) edge node [right] {$\id$} (m-2-2)
            (m-1-1) edge node [below] {$f$} (m-2-2);
        \end{tikzpicture}
    \end{figure}
    %Takav prostor mo\v zemo na\' ci, uzmemo $\Omega = \{ 1, \: 2, \:, 3, \: 4\}$, $\famF = \{\varnothing, \: \Omega,\: \{1, \: 2 \}, \: \{3, \: 4 \} \}$ te stavimo $\masP (\{1, \: 2 \}) = 0$, $\masP (\{3, \: 4\}) = 1$, upotpunjenje je $\overline{\famF} = \{ \varnothing, \: \Omega, \:  \{1\}, \: \{2\}, \: \{1, \: 2\}, \: \{3, \: 4\}, \: \{1, \: 3, \: 4 \}, \: \{ 2, \: 3, \: 4 \} \}$
\end{rj}

\begin{nap} \label{nap:3.11}
    Kod slu\v cajnih elemenata treba provjeriti da vrijedi $\praslika{f}(\famE) \subseteq \famF$. Zbog leme \ref{lm:3.4} dovljno je provjeriti da je $\praslika{f}(\famC) \subseteq \famF$, ako je $\famC$ generiraju\' ca familija za $\famE$, to jest ako je $\famE = \sigAlg{\famC}$.
\end{nap}

Koriste\' ci napomenu \ref{nap:3.11} i primjer \ref{pr:2.3} dobivamo:

\begin{kor} \label{kor:3.12}
    Neka je $X : \Omega \to \extReal$ preslikavanje i $\famC$ bilo koja klasa iz primjera \ref{pr:2.3} (za slu\v caj $\real$).
    Tada je $X$ pro\v sirena slu\v cajna varijabla ako i samo ako je $\praslika{X}(\famC) \subseteq \famF$.
\end{kor}

Podsjetimo se da je za $x, \; y \in \extReal$
\begin{equation*}
    \begin{aligned}
        x \lor y &:= \max \{x, \; y\}\\
        x \land y &:= \min \{x, \: y\}.
    \end{aligned}
\end{equation*}
Podsjetimo se i da koristimo konvencije $0 \cdot \pm \infty = 0$, te $c \cdot (\pm \infty) = \pm \infty$ za $c > 0$ i $(-1) \cdot (\pm \infty) = \mp \infty$. Koriste\' ci korolar \ref{kor:3.5} i lemu \ref{lm:3.6} dobijemo:

\begin{kor} \label{kor:3.13}
    Ako je $X$ (pro\v sirena) slu\v cajan varijabla, tada su i
    \begin{itemize}[label=]
        \item $c \cdot X, \; c \in \real$,
        \item $|X|$,
        \item $X^+ := X \lor 0$,
        \item $X^- := (-X) \land 0$
    \end{itemize}
    pro\v sirene slu\v cajne varijable.
\end{kor}

\begin{kor} \label{kor:3.14}
    Ako je $\niz{X_n}{n \in \nat}$ niz pro\v sirenih slu\v cajnih varijabli, tada su:
    \begin{itemize}[label=]
        \item $\sup\limits_{n} X_n$,
        \item $\inf\limits_{n} X_n$,
        \item $\limsup\limits_{n} X_n$,
        \item $\liminf\limits_{n} X_n$,
    \end{itemize}
    pro\v sirena slu\v cajne varijable.
\end{kor}

\begin{proof}
    Po korolaru \ref{kor:3.12} i $\famC = \famK$ iz primjera \ref{pr:2.3} slijedi da je $\sup\limits_n X_n$ pro\v sirena slu\v cajna varijabla, jer
    \begin{equation*}
        \{ \sup\limits_n X_n \leq b \} = \presjek{n \in \nat}{} \{ X_n \leq b \}.
    \end{equation*}
    Iz korolara \ref{kor:3.13} slijedi tvrdnja za infimum, zbog
    \begin{equation*}
        \inf\limits_n X_n = - (\sup\limits_n (- X_n)).
    \end{equation*}
    Nadalje,
    \begin{equation*}
        \begin{gathered}
            \liminf\limits_n X_n = \sup\limits_k (\inf\limits_{n \geq k}) X_n,\\
            \limsup\limits_n X_n = \inf\limits_k (\sup\limits_{n\geq k} X_n).
        \end{gathered}
    \end{equation*}
\end{proof}

Mo\v ze se dogoditi da slu\v cajni element $f:\Omega \to E$ ima vrijednosti u nekom podskupu $A \subseteq E$ koji mo\v ze i ne mora biti u $\famE$. Koriste\' ci \eqref{jed:3.2},\eqref{jed:3.3} i lemu \ref{lm:3.4} te inkluziju $i: A \to E$, mo\v zemo jednostavno opisati takve situacije.

\begin{zad} \label{zad:3.15}
    Neka je $(E, \: \famE)$ izmjeriv prostor i $A \subseteq E$.
    Poka\v zi da je
    \begin{equation*}
        A \cap \famE := \skup{A \cap B}{B \in \famE}
    \end{equation*}
    $\sigma$-algebra na $A$.
    Pove\v zi s pojmom relativne topologije ako je $E$ topolo\v ski prostor i $\famE = \borel{E}$.
    Ako je $f$ slu\v cajni element u $(A, \: A \cap \famE)$ mo\v zete li ga interpretirati kao slu\v cajni element u $E$?
    S druge strane, ako je $f$ slu\v cajni element s vrijednostima u $E$ i $f \in A \; (g.s.)$, mo\v zete li interpretirati $f$ kao slu\v cajni element u $\urePar{A}{A \cap \famE}$?
\end{zad}

\begin{nap} \label{nap:3.15-1}
    Sjetimo se, ako je $\urePar{X}{\famT}$ topolo\v ski prostor, te ako je $Y \subseteq X$, ne nu\v zno otvoren, tada familiju
    \begin{equation*}
        \skup{Y \cap U}{U \in \famT}
    \end{equation*}
    nazivamo \emph{relativnom topologijom na $Y$} (induciranom topologijom $\famT$).
\end{nap}

\begin{rj}[\ref{zad:3.15}]  \label{rj:3.15}
    Neka je $\urePar{E}{\famT}$ topolo\v ski prostor i $\borel{E} = \sigAlg{\famT}$.
    Ovaj zadatak se sastoji od nekoliko tvrdnji.
    \begin{enumerate}[label=(\arabic*)]
        \item \label{rj:3.15.1}
        Treba pokazati da je $ A \cap \famE$ $\sigma$-algebra.
        \item \label{rj:3.15.2}
        Vrijedi li $A \cap \famE = \sigAlg{A \cap \famT}$?
        \item \label{rj:3.15.3}
        Ako je $f$ slu\v cajni element u $\urePar{A}{A \cap \famE}$, je li onda i slu\v cajni element u $E$?
        \item \label{rj:3.15.4}
        Ako je $f$ slu\v cajni element u $E$ i $f \in A \; (g.s.)$, je li $f$ i slu\v cajni element u $\urePar{A}{A \cap \famE}$?
    \end{enumerate}
    Idemo redom.
    Poka\v zimo prvo \ref{rj:3.15.1}.
    \begin{enumerate}[label=(\roman*)]
        \item Primjetimo $\varnothing \in \famE \implies \varnothing \in A \cap \famE$.
        \item U ovom slu\v caju komplement se podrazumjeva u odnosu na $A$.
        Neka je $B \in A \cap \famE$, dakle postoji $C \in \famE$ takav da $B = A \cap C$, a kako je $C \in \famE$, tada slijedi $C^c \in \famE$ odakle imamo $A \cap C^c = B^c \in A \cap \famE$.
        \item Neka su $\niz{B_n}{n \in \nat} \subseteq A \cap \famE$, tada za svaki $n \in \nat$ postoji $C_n \in \famE$, takav da $B_n = A \cap C_n$.
        Tada imamo $\unija{n \in \nat}{} B_n = \unija{n \in \nat}{} (A \cap C_n) = A \cap \unija{n \in \nat}{} C_n$, a kako je $\unija{n \in \nat}{} C_n \in \famE$, tada je nu\v zno $\unija{n \in \nat}{} B_n = A \cap \unija{n \in \nat}{} C_n \in A \cap \famE$.
    \end{enumerate}

    Poka\v zimo sada \ref{rj:3.15.2}.
    \begin{itemize}
        \item[$\subseteq$] Neka je $i_A : A \to E$ inkluzija.
        To je neprekidna funckija, pa je i izmjeriva u paru Borelovih $\sigma$-algebri.
        Dakle za $B \in \famT$ je $\praslika{i_A}(B)$ Borelov u $\sigma$-algebri generiranoj relativnom topologijom i vrijedi $\praslika{i_A}(B) = A \cap B$, pa je\\
        $\skup{A \cap B}{A \in \sigAlg{\famT}} = A \cap \famE \subseteq \sigAlg{A \cap \famT}$.
        \item[$\supseteq$] Primjetimo $\famT \subseteq \famE$, dakle vrijedi i $A \cap \famT \subseteq \famE$ pa $A \cap \famE$ sadr\v zi sve otvorene podskupove od $A$ i prema \ref{rj:3.15.1} je $\sigma$-algebra, stoga vrijedi:\\
        $\sigAlg{A \cap \famT} \subseteq \sigAlg{A \cap \famE} = A \cap \famE$.
    \end{itemize}
    
    Promotrimo \ref{rj:3.15.3}.
    Neka je $f: \Omega \to A \cap E$ slu\v cajni element u $\urePar{A}{A \cap \famE}$ te neka je $i_A : A \cap E \to E$ inkluzija.
    \begin{figure}[H]
        \centering
        \begin{tikzpicture}
            \matrix (m) [matrix of math nodes,row sep=5em,column sep=8em,minimum width=3em]
            {
            (\Omega, \; \famF, \; \masP) & \urePar{A}{A \cap \famE} \\
            & \urePar{E}{\famE} \\};
            \path[-stealth]
            (m-1-1) edge node [above] {$f$} (m-1-2)
            (m-1-2) edge node [right] {$i_A$} (m-2-2)
            (m-1-1) edge node [below, xshift=-9pt] {$i_A \circ f$} (m-2-2);
        \end{tikzpicture}
    \end{figure}
    Funkcija $i_A$ je neprekidna, pa samim time i izmjeriva, pa je
    \begin{equation*}
        X := i_A \circ f : \Omega \to E
    \end{equation*}
    slu\v cajni element na $\urePar{E}{\famE}$ koji mo\v zemo promatrati kao pro\v sirenje slu\v cajnog elementa $f$ na $E$.
    Na taj na\v cin mo\v zemo $f$ promatrati kao slu\v cajni element na $E$.

    Te na kraju promotrimo \ref{rj:3.15.4}.
    Ako je $f \in A \; (g.s.)$, onda zna\v ci da postoji $N \in \famF$ takav da je $\{f \notin A\} \subseteq N$, $\vjeroj{N} = 0$, me\dj utim, ako je vjerojatnosni prostor potpun, slijedi da je $\{f \notin A\} \in \famF$, posljedi\v cno $\{f \notin A\}^c = \{f \in A\} \in \famF$.
    Ako je $B \in \famE$, promatramo $\{f \in B \cap A\} = \{ f \in B \} \cap \{f \in A\}$.
    Ako je prostor potpun, onda je $\{f \in A\} \in \famF$, pa je $\{f \in B \cap A\} = \{ f \in B \} \cap \{f \in A\} \in \famF$, u suprotnom $\{f \in A\}$ ne mora nu\v zno biti u $\famF$.
\end{rj}

% Ovo revidiraj, primjer nije točan

\begin{defn}    \label{defn:3.15-1}
    Neka su $\urePar{A}{\famA}$, $\urePar{B}{\famB}$ izmjerivi prostori i neka je
    \begin{equation*}
        f: A \to B,    
    \end{equation*}
    $\urePar{\famA}{\famB}$-izmjerivo preslikavanje.
    Ka\v zemo da je $f$ \emph{jednostavno} ako postoje $n \in \nat$, $\{ b_1, \: \dots, \: b_n \} \subseteq B$, particija $\{ A_1, \: \dots, \: A_n \} \subseteq \famA$ skupa $A$, takvi da je $\restr{f}{A_j} = b_j$, za $j = 1, \dots, n$.
\end{defn}
    

\begin{pr}  \label{pr:3.16}    
    U slu\v caju da je $(B, \: \famB) = (\extReal, \: \borel{\extReal})$ vrijedi:
    \begin{equation*}
        f = \suma{j = 1}{n} b_j \mathbb{1}_{A_j}.
    \end{equation*}
    Posebno, jednostave slu\v cajne varijable tvore vektorski prostor nad $\extReal$.
\end{pr}

\begin{lm}  \label{lm:3.17}
    Ako je $X$ nenegativan pro\v sirena slu\v cajna varijabla, tada postoji niz jednostavnih slu\v cajnih varijabli $\niz{X_n}{n \in \nat}$ takav da, za svaki $\omega \in \Omega$ vrijedi:
    \begin{equation*}
        0 \leq X_1(\omega) \leq X_1(\omega) \leq  \dots \nearrow X(\omega).
    \end{equation*} 
\end{lm}

\begin{proof}
    Za $\omega \in \Omega$ i $n \in \nat$ definiramo
    \begin{equation*}
        X_n(\omega) := \frac{1}{2^n} \floor*{2^n \cdot X(\omega)} \land n.
    \end{equation*}
\end{proof}

\begin{nap} \label{nap:3.17-1}
    Elemente niza iz leme \ref{lm:3.17} mo\v zemo definirati i sa:
    \begin{equation*}
        X_n := \suma{k = 0}{n \: 2^n - 1} \frac{k}{2^n} \karaktFja_{\{ X \in \desInt{\frac{k}{2^n}}{\frac{k + 1}{2^n}} \}} + n \: \karaktFja_{\{X \geq n\}}.
    \end{equation*}
\end{nap}

Iz leme \ref{lm:3.17} i iz korolara \ref{kor:3.14} slijedi:

\begin{tm}  \label{tm:3.18}
    Neka je $\vjerojatnosniProstor$ vjerojatnosni prostor i $X: \Omega \to \real$. Tada je $X$ slu\v cajna varijabla ako i samo ako je $X$ limes niza jednostavnih slu\v cajnih varijabli.
\end{tm}

\begin{zad} \label{zad:3.19}
    Neka je $\vjerojatnosniProstor$ vjerojatnosni prostor, $X, \; Y$ pro\v sirene slu\v cajne varijable na $\Omega$ i $\alpha, \; \beta \in \real$.
    \begin{enumerate}[label=(\alph*)]
        \item Ako su $X \geq 0$, $Y \geq 0$, $\alpha \geq 0$, $\beta    \geq 0$, tada je $\alpha X + \beta Y$ pro\v sirena nenegativan slu\v cajna varijabla.
        \item Ako su $X$ i $Y$ slu\v cajne  varijable, tada je $\alpha X + \beta Y$ gotovo sigurno dobro definirana i mo\v ze se interpretirati kao slu\v cajna varijabla.
    \end{enumerate}
\end{zad}

\begin{rj}[\ref{zad:3.19}]
    Koristimo neke rezultate iz idu\' ceg poglavlja.
    \begin{figure}[H]
        \centering
        \begin{tikzpicture}
            \matrix (m) [matrix of math nodes,row sep=5em,column sep=8em,minimum width=3em]
            {
            & \Omega & \\
            E_1 & E_1 \times E_2 & E_2 \\};
            \path[-stealth]
            (m-1-2) edge node [left, xshift=-9pt] {$\pi_2 \circ f$} (m-2-1)
            (m-1-2) edge node [right, xshift=9pt] {$\pi_1 \circ f$} (m-2-3)
            (m-1-2) edge node [right] {$f$} (m-2-2)
            (m-2-2) edge node [below] {$\pi_1$} (m-2-1)
            (m-2-2) edge node [below] {$\pi_2$} (m-2-3);
        \end{tikzpicture}
    \end{figure}
    \begin{enumerate}[label=(\roman*)]
        \item O\v cito vrijedi da je $\alpha X + \beta Y \geq 0$.
        Primjetimo da su funkcije $+ : \extReal \times \extReal \to \extReal$ i $\alpha \: \cdot : \real \to \real$ neprekidna preslikavanja, pa time i Borelova.
        Definiramo preslikavanje $Z := \urePar{X}{Y}$, primjetimo $X = \pi_1 \circ F, \; Y = \pi_2 \circ F$, budu\' ci su $X$ i $Y$ izmjerive i $F$ je izmjeriva, po propoziciji \ref{prop:4.9}.
        Kako je kompozicija izmjerivih preslikavanja izmjerivo preslikavanje, slijedi da su $\alpha X$, $\beta Y$ i $X + Y$ izmjeriva preslikavanja, odakle slijedi tvrdnja.
        \item Primjetimo $\{ | \alpha X + \beta Y | = \infty \} \subseteq \{ |\alpha X| + |\beta Y| = \infty \} \subseteq \{|X| = \infty\} \cup \{ |Y| = \infty \}$ stoga vrijedi $\vjeroj{ \{ | \alpha X + \beta Y | = \infty \} } \leq \vjeroj{\{|X| = \infty\}} + \vjeroj{\{|Y| = \infty\}} = 0$.
        Dakle $\alpha X + \beta Y$ je slu\v cajna varijabla.
    \end{enumerate}
\end{rj}

\begin{zad} \label{zad:3.20}
    Neka je $X$ slu\v cajna varijabla i $Y$ slu\v cajni element s vrijednostima u $(E, \: \famE)$.
    Tada je $X$ $(\sigAlg{Y}, \: \borel{\real})$-izmjeriva ako i samo ako postoji funckija $f: E \to \real$ $(\famE, \: \borel{\real})$-izmjeriva takva da je $X = f \circ Y$.
\end{zad}

\begin{rj}[\ref{zad:3.20}]
    Dokazujemo obje implikacije.
    \begin{itemize}
        \item[$\implies$]
        Dokazujemo da postoji $f$ takva da sljede\' ci dijagram komutira, te da su sve rezultiraju\' ce funkcije izmjerive.
        \begin{figure}[H]
            \centering
            \begin{tikzpicture}
                \matrix (m) [matrix of math nodes,row sep=5em,column sep=8em,minimum width=3em]
                {
                \urePar{\Omega}{\sigAlg{Y}} & \urePar{E}{\famE} \\
                & \urePar{\real}{\borel{\real}} \\};
                \path[-stealth]
                (m-1-1) edge node [above] {$Y$} (m-1-2)
                (m-1-2) edge node [right] {$f$} (m-2-2)
                (m-1-1) edge node [below] {$X$} (m-2-2);
            \end{tikzpicture}
        \end{figure}
        Neka je $X$ $\urePar{\sigAlg{Y}}{\borel{\real}}$-izmjeriva.
        Dokaz provodimo Lebesgueovom indukcijom po $X$.
        \begin{enumerate}[label=(\arabic*. korak)]
            \item Neka je $A \in \famF$ $X = \karaktFja_A$.
            Primjetimo $\{ \karaktFja_A = 1 \} = A$, $\{ \karaktFja_A \neq 1 \} = A^c$.
            Budu\' ci je $X$ $\sigAlg{Y}$-izmjeriva slijedi da postoji $F \in \famE$ takava da je $\{Y \in F\} = A$.
            Stavimo $f = \karaktFja_F$, primjetimo:
            \begin{equation*}
                \begin{aligned}
                    (\karaktFja_F \circ Y)(\omega)
                    &= \karaktFja_F (Y(\omega))
                    =
                    \begin{cases}
                        1, &Y(\omega) \in F\\
                        0, &Y(\omega) \notin F 
                    \end{cases}\\
                    &=
                    \begin{cases}
                        1, &\omega \in \{Y \in F\}\\
                        0, &\omega \notin \{Y \in F\}
                    \end{cases}
                    = \karaktFja_{\{Y \in F\}} (\omega)\\
                    &=
                    \begin{cases}
                        1, &\omega \in A\\
                        0, &\omega \notin A
                    \end{cases}\\
                    &= \karaktFja_A (\omega)
                \end{aligned}
            \end{equation*}
            Pa uzmemo $f = \karaktFja_F$ i vrijedi $X = f \circ Y$.
            \item Neka je $n \in \nat$ i neka su $\niz{A_k}{k = 1, \ldots, n} \subseteq \famF$, $\niz{\alpha_k}{k = 1, \ldots, n}$, $\alpha_k \geq 0$.
            Neka je $X = \suma{k = 1}{n} \alpha_k \karaktFja_{A_k}$, bez smanjenja op\' cenitosti mo\v zemo pretpostaviti da $A_k$ particioniraju $\Omega$.
            %Poka\v ze se da je $\sigAlg{X} = \skup{\unija{i \in I}{} A_i}{I \subseteq \{ 1,\ldots, n \}}$
            Budu\' ci je $X$ $\sigAlg{Y}$-izmjeriva, vrijedi $A_k \in \sigAlg{Y}$, stoga postoje $F_k \in \famE$, takvi da je $\{Y \in F_k\} = A_k$.
            Definirajmo $f = \suma{k = 1}{n} \alpha_k \karaktFja_{F_k}$.
            Primjetimo, sada vrijedi:
            \begin{equation*}
                \begin{aligned}
                    (f \circ Y)(\omega) &= (\suma{k = 1}{n} \alpha_k \karaktFja_{F_k})(Y(\omega)) = \suma{k = 1}{n} \alpha_k \karaktFja_{F_k} (Y(\omega))\\
                    &= \suma{k = 1}{n} \alpha_k \karaktFja_{\{Y \in F_k\}} (\omega) = \suma{k = 1}{n} \alpha_k \karaktFja_{A_k} (\omega)\\
                    &= X (\omega).
                \end{aligned}
            \end{equation*}
            Dakle slijedi $X = f \circ Y$.
            \item Neka je $X$ nenegativna izmjeriva funkcija, $X \geq 0$, tada postoji rastu\' ci niz nenegativnih jednostavnih funkcija $\niz{X_n}{n \in \nat}$, takvih da vrijedi $X_n \nearrow X \; (g.s.)$.
            Po prethodnom koraku znamo da sada postoji niz funkcija $\niz{f_n}{n \in \nat}$ takav da vrijedi $f_n \circ Y = X_n$.
            Primjetimo, da sada vrijedi:
            \begin{equation*}
                \lim\limits_{n \to \infty} (f_n \circ Y) = \lim\limits_{n \to \infty} X_n = X \quad (g.s.).
            \end{equation*}
            Treba pokazati da postoji $f = \lim\limits_{n \to \infty} f_n$, te da vrijedi $X = f \circ Y$.
            %Budu\' ci je $X_n$ rastu\' c niz, mo\v zemo zaklju\v citi da je i $f_n$ rasu\' c niz funkcija.

            Ozna\v cimo sa $R_Y$ sliku od $Y$.
            Definirajmo $\tilde{f} : R_Y \subseteq E \to \real$, neka je $\eta \in R_Y$, tada postoji $\omega \in \Omega$ takav da je $Y(\omega) = \eta$. 
            Stavimo
            \begin{equation*}
                \tilde{f} (\eta) := \lim\limits_{n \to \infty} f_n (Y(\omega)) = \lim\limits_{n \to \infty} X_n (\omega).
            \end{equation*}
            Poka\v zimo da je $f$ dobro definirana.

            Neka su $\omega, \; \omega' \in \Omega$ takve da
            \begin{equation*}
                \begin{aligned}
                    Y(\omega) &= Y(\omega')\\
                    X(\omega) &\neq X(\omega').
                \end{aligned}
            \end{equation*}
            Sada postoje disjunktni Borelovi skupovi $B, \; B' \in \borel{\real}$ takvi da
            \begin{equation*}
                \begin{aligned}
                    X(\omega) &\in B\\
                    X(\omega') &\in B'.
                \end{aligned}
            \end{equation*}
            Tako\dj er vrijedi
            \begin{equation*}
                \{X \in B \cap B'\} = \underbrace{\{X \in B\}}_{\in \sigAlg{Y}} \cap \underbrace{\{X \in B'\}}_{\in \sigAlg{Y}} = \varnothing.
            \end{equation*}
            Sada znamo da postoje $A, \, A' \in \famE$ takvi da je
            \begin{equation*}
                \begin{aligned}
                    \omega &\in \{Y \in A\} = \{X \in B\}\\
                    \omega' &\in \{Y \in A'\} = \{X \in B'\}.
                \end{aligned}
            \end{equation*}
            Kako je $Y(\omega) = Y (\omega')$, slijedi da je $Y(\omega) = Y(\omega') \in A \cap A'$, me\dj utim vrijedi
            \begin{equation*}
                \{Y \in A\} \cup \{Y \in A'\} = \{Y \in A \cup A'\} = \varnothing,
            \end{equation*}
            \v sto je kontradikcija.
            Dakle $\tilde{f}$ je dobro definirana na $R_Y$, sada definiramo $f : E \to \real$ sa:
            \begin{equation*}
                f(\omega) :=
                \begin{cases}
                    \tilde{f} (\omega), &\omega \in R_Y\\
                    0, &\omega \notin R_Y.
                \end{cases}
            \end{equation*}
            Primjetimo da je sada za gotovo svaki $\omega \in \Omega$
            \begin{equation*}
                \begin{aligned}
                X(\omega) &= \lim\limits_{n \to \infty} f_n (Y(\omega)) = \tilde{f}(Y(\omega)) = f(Y(\omega))\\
                &= (f \circ Y) (\omega).
                \end{aligned}
            \end{equation*}
            \item Neka je $X$ slu\v cajna varijabla, tada postoje nenegativne slu\v cajne varijable $X^+$, $X^-$ takve da je $X = X^+ - X^-$, pa prema prethodnom koraku postoje funkcije $f^+$, $f^-$ takve da vrijedi:
            \begin{equation*}
                \begin{aligned}
                   X^+ = f^+ \circ Y\\
                   X^- = f^- \circ Y.
                \end{aligned}
            \end{equation*}
            Stavimo da je $f:= f^+ - f^-$, pa sada vidimo da je
            \begin{equation*}
                \begin{aligned}
                    X &= X^+ - X^- = f^+ \circ Y - f^- \circ Y = (f^+ - f^-) \circ Y\\
                    &= f \circ Y.
                \end{aligned}
            \end{equation*}
        \end{enumerate}
        
        \item[$\impliedby$]
        Neka je $X = f \circ Y$. Neka je $A \in \sigAlg{X}$, tada postoji $B \in \borel{\real}$ takav da $A = \{ X \in B\}$, sada vrijedi $\{  X \in B \} = \{ f \circ Y \in B \} = \{ Y \in \{ f \in B \} \}$, budu\' ci je $f$ izmjeriva slijedi da je $\{f \in B\} \in \famE$, pa je $\{Y \in \{ f \in B \}\} \in \sigAlg{Y}$, odnosno $\sigAlg{X} \subseteq \sigAlg{Y}$.
        Dakle $X$ je $\urePar{\sigAlg{Y}}{\borel{\real}}$-izmjeriva.
    \end{itemize}
\end{rj}