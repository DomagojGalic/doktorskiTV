% poglavalje 5.1 predavanje 16 - slaba konvergencija

\chapter{Slaba konvergencija}

Neka je
\begin{equation*}
    \inc := \bigSkup{f: \real \to \real}{x \leq y \implies f (x) \leq f(y)}.
\end{equation*}
Ako je $F$ d.f., tada je $F \in \inc$.
Ako je $F$ p.f.F., tada je $\restr{F}{\real} \in \inc$, pa u tom smislu kra\' ce pi\v semo da je $F \in \inc$.
O\v cito za svaku $G \in \inc$ i za svaki $x \in \real$ vrijedi:
\begin{equation}    \label{jed:16.1}
    \lim\limits_{y \nearrow x} G(y) := G(x -) \leq G (x) \leq G(x +) := \lim\limits_{y \searrow x} G(y).
\end{equation}
Za funkciju $G \in \inc$ ozna\v cimo sa $G^R$ funkciju definiranu s $G^R (x) := G(x+)$.
Tada je $G^R \in \inc$ te $G^R$ je neprekidna zdesna, pa je $G^R$ d.f.
Ako je $F$ p.d.F. tada je $F^R = F$.

\begin{defn}  \label{defn:16.1-1}
    Neka je $G \in \inc$.
    Ako za $x \in \real$ vrijedi $G(x-) = G(x) = G(x+)$, ka\v zemo da $x$ pripada skupu $C(G)$ \emph{svih to\v caka neprekidnosti} od $G$.     
\end{defn}

Uo\v cimo da je $\real \setminus C(G)$ najvi\v se prebrojiv, pa je $C(G)$ gust u $\real$.
Nadalje, vrijedi
\begin{equation}    \label{jed:16.2}
    \begin{gathered}
        G(G) \subseteq C(G^R)\\
        \restr{G^R}{C(G)} = \restr{G}{C(G)},
    \end{gathered}
\end{equation}
razmislite o jednakosti ovih skupova.
Lako se vidi da za $G \in \inc$ sljede\' ci limesi postoje u $\extReal$:
\begin{equation}    \label{jed:16.3}
    \begin{aligned}
        G(+\infty) &:= \lim\limits_{x \nearrow +\infty} G(x)\\
        G(-\infty) &:= \lim\limits_{x \searrow -\infty} G(x)\\
        G^R(+\infty) &= G(+\infty)\\
        G^R(-\infty) &= G(-\infty).
    \end{aligned}
\end{equation}

Za $0 < a < +\infty$ uvedimo oznake:
\begin{equation*}
        \begin{aligned}
            \Delta G(a) &:= G(a) - G(-a),\\
            \Delta G &:= G(+\infty) - G(-\infty),
        \end{aligned}
\end{equation*}
te je $\Delta G(a) \in \real$, $\Delta G \in \extReal$, tako\dj er vrijedi:
\begin{equation}    \label{jed:16.4}
    0 \leq \Delta G(a) \leq \Delta G = \lim\limits_{a \nearrow +\infty} \Delta G (a) = \lim\limits_{
        \begin{smallmatrix}
            a \nearrow +\infty\\
            a \in C(G)
        \end{smallmatrix}
    } \Delta G (a) \leq +\infty.
\end{equation}
Uo\v cimo da ako je $\Delta G = 1$, onda je $G^R$ u su\v stini p.d.F., modulo pomak za $G(-\infty)$.

\begin{defn}    \label{defn:16.5}
    Niz $\niz{G_n}{n \in \nat} \subseteq \inc$ \emph{konvergira slabo} prema $G \in \inc$, i pis\v emo:
    \begin{equation*}
        G_n \xrightarrow[n \to \infty]{w} G,
    \end{equation*}
    ako za svaki $x \in C(G)$ vrijedi:
    \begin{equation*}
        \lim\limits_{n \to \infty} G_n(x) = G(x).
    \end{equation*}
    Ako $G_n \xrightarrow{w} G$ i ako vrijedi $G_n (\pm \infty) \xrightarrow{} G(\pm \infty)$, tada ka\v zemo da $(G_n)$ \emph{konvergira potpuno} prema $G$, i pi\v semo:
    \begin{equation*}
        G_n \xrightarrow[n \to \infty]{c} G.
    \end{equation*}
    Ako je $X$ slu\v cajna varijabla na $(\Omega_1, \famF_1, \masP_1)$ i ako je $(X_n)$ niz slu\v cajnih varijabli na $(\Omega_2, \famF_2, \masP_2)$, tada ka\v zemo da $(X_n)$ \emph{konvergira po distribuciji} prema $X$ ako vrijedi
    \begin{equation*}
        F_{X_n} \xrightarrow[n \to \infty]{c} F_X
    \end{equation*}
    i pi\v semo:
    \begin{equation*}
        X_n \xrightarrow[n \to \infty]{d} X.
    \end{equation*}
\end{defn}

\begin{nap} \label{nap:16.6}
    \begin{enumerate}[label=(\alph*)]
        \item Neka je $(F_n)$ niz p.d.F., tada
        \begin{equation*}
            F_n \xrightarrow{w} G \centernot\implies F_n \xrightarrow{c} G.
        \end{equation*}
        Na primjer uzmimo $F_n := \karaktFja_{\desInt{n}{+\infty}}$, to jest $\masP(\{n\}) = 1$, daje $F_n \xrightarrow{w} 0$.
        \item
        \begin{equation*}
            X_n \xrightarrow[n \to \infty]{g.s.} X
            \begin{smallmatrix}
                \implies\\
                \notimpliedby
            \end{smallmatrix}
            X_n \xrightarrow[n \to \infty]{\masP} X
            \begin{smallmatrix}
                \implies\\
                \notimpliedby
            \end{smallmatrix}
            X_n \xrightarrow[n \to \infty]{d} X.
        \end{equation*}
        \item   \label{nap:16.6c}
        Ako $G_n \xrightarrow[n \to \infty]{w} G$, tada
            \begin{equation*}
                \limsup\limits_n G_n(-\infty) \leq G(-\infty) \leq G(+ \infty) \leq \liminf\limits_n G(+\infty).
            \end{equation*}
            Koriste\' ci \eqref{jed:16.4} dobijemo, za svaki $x \in C(G)$,
            \begin{equation*}
                \limsup\limits_n G_n (-\infty) \leq G(x) \leq \liminf\limits_n G_n (+\infty).
            \end{equation*}
        \item   \label{nap:16.6d}
        Ako $G_n \xrightarrow{w} G$, tada je $\Delta G \leq \liminf\limits_n \Delta G_n$.
        Slijedi direktno iz \ref{nap:16.6c}.
        \item   \label{nap:16.6e}
        Ako je $D \subseteq \real$ gust u $\real$ i za svaki $x \in D$ $G_n (x) \xrightarrow{} G (x)$, tada $G_n \xrightarrow{w} G$.
        Aproksimirajte $x \in C(G)$ to\v ckama iz $D$ odozdo i odozgo.
        \item   \label{nap:16.6f}
        Neka je $D \subseteq \real$ gust u $\real$ i neka je $(F_n)$ niz p.d.F. takav da za svaki $x \in D$ postoji limes $\lim\limits_n F_n (x)$.
        Tada definiramo $G : D \to \segment{0}{1}$, sa 
        \begin{equation*}
            G(x) := \lim\limits_{n \to \infty} F_n (x).
        \end{equation*}
        O\v cito je $G$ neopadaju\' ca i sli\v cno kao kod $G^R$ definiramo $F:\real \to \segment{0}{1}$ pomo\' cu
        \begin{equation*}
            F (x) := \inf\limits_{
                \begin{smallmatrix}
                    u \in D\\
                    x < u
                \end{smallmatrix}
            } G(u).
        \end{equation*}
        Tada je $F \in \inc$, te vrijedi:
        \begin{equation*}
            \begin{gathered}
                F^R = F\\
                C(G) \subseteq C(F)\\
                F(\pm \infty) = G(\pm \infty)\\
                \restr{F}{D \cap C(G)} = \restr{G}{C(G)},
            \end{gathered}
        \end{equation*}
        te po \ref{nap:16.6e} vrijedi
        \begin{equation*}
            F_n \xrightarrow[n \to \infty]{w} F.
        \end{equation*}
        Uo\v cimo da je
        \begin{equation*}
            0 \leq F(-\infty) \leq F(+\infty) \leq 1.
        \end{equation*}
    \end{enumerate}
\end{nap}

\begin{lm}  \label{lm:16.7}
    Ako $G_n \xrightarrow{w} G$ i $\Delta G_n < +\infty$, za svaki $n \in \nat$, tada su sljede\' ce tvrdnje ekvivalentne:
    \begin{enumerate}[label=(\roman*)]
        \item \label{lm:16.7.1}
        \begin{equation*}
            \lim\limits_{n \to \infty} G_n (\pm \infty) = G(\pm \infty) \in \real,
        \end{equation*}
        \item \label{lm:16.7.2}
        \begin{equation*}
            \lim\limits_{n \to \infty} \Delta G_n = \Delta G < +\infty,
        \end{equation*}
        \item \label{lm:16.7.3}
        \begin{equation*}
            \lim\limits_{a \nearrow +\infty} \sup\limits_{n \in \nat} [ \Delta G_n - \Delta G_n (a) ] = o(1).
        \end{equation*}
    \end{enumerate}
\end{lm}

\begin{proof}
    O\v cito $\ref{lm:16.7.1} \implies \ref{lm:16.7.2}$.
    Doka\v zimo da $\ref{lm:16.7.2} \implies \ref{lm:16.7.1}$.
    Uo\v cimo,
    \begin{equation*}
        \Delta G < +\infty \implies G (\pm \infty) \in \real.
    \end{equation*}
    Prema napomeni \ref{nap:16.6c} i $\Delta G_n < +\infty$ slijedi
    \begin{equation*}
        \begin{gathered}
            \begin{aligned}
                \limsup\limits_{n \to \infty} G_n (+\infty) &:= \limsup\limits_{n \to \infty} [ \Delta G_n + G_n (-\infty) ] = \Delta G + \limsup\limits_{n \to \infty} G_n (-\infty)\\
                &\leq \Delta G + G (-\infty) = G(+\infty) \leq \liminf\limits_{n \to \infty} G_n(+\infty) 
            \end{aligned}\\
            \implies G_n (+\infty) \xrightarrow[n \to \infty]{} G(+\infty).
        \end{gathered}
    \end{equation*}
    Sada "$-\infty$ dio" slijedi direktno.

    Doka\v zimo $\ref{lm:16.7.3} \implies \ref{lm:16.7.2}$. Za svaki $\varepsilon > 0$ takav da za svaki $a \geq a_0$ imamo
    \begin{equation*}
        \Delta G_n - \Delta G_n (a) < \varepsilon,
    \end{equation*}
    za svaki $n \in \nat$
    Promatramo samo takve $a$ za koje je $\pm a \in C(G)$, takav skup je gust.
    Slijedi
    \begin{equation*}
        \limsup\limits_{n \to \infty} \Delta G_n \leq \Delta G (a) + \varepsilon < + \infty.
    \end{equation*}
    Prema napomeni \ref{nap:16.6d} dobivamo da je $\Delta G$ kona\v can i da za svaki $\varepsilon > 0$
    \begin{equation*}
        \limsup\limits_{n \to \infty} \Delta G_n \leq \Delta G + \varepsilon \implies \liminf\limits_{n \to \infty} \Delta G_n \leq \Delta G.
    \end{equation*}
    Tvrdnja slijedi po napomeni \ref{nap:16.6d}.

    Doka\v zimo $\ref{lm:16.7.2} \implies \ref{lm:16.7.3}$.
    Za svaki $\varepsilon > 0$ postoji $n_1 = n_1 (\varepsilon) \in \nat$, takav da
    \begin{equation*}
        n \in \nat, \quad n \geq n_1 \implies \Delta G_n - \Delta G < \varepsilon.
    \end{equation*}
    Izaberimo $a_0 > 0$, takav da je $\pm a \in C(G)$ i $\Delta G - \Delta G (a_0) < \varepsilon$.
    Tada postoji $n_2 = n_2 (\varepsilon) \in \nat$ takav da
    \begin{equation*}
        n \in \nat, \quad n \geq n_2 \implies \Delta G (a_0) - \Delta G_n (a_0) < \varepsilon.
    \end{equation*}
    Uzmemo $n_0 := \max (n_1, n_2)$ i za $n \in \nat$, $n \geq n_0$ vrijedi
    \begin{equation*}
        \Delta G_n - \Delta G_n (a_0) < 3 \varepsilon.
    \end{equation*}
    Za $1 \leq j \leq n_0$ izaberimo $a_j$ takav da je
    \begin{equation*}
        \Delta G_j - \Delta G_j (a_j) < 3 \varepsilon.
    \end{equation*}
    Dakle, za $a \geq \max (a_0, a_1, \ldots, a_{n_0 - 1})$ vrijedi
    \begin{equation*}
        \sup\limits_{n \in \nat} [ \Delta G_n - \Delta G_n (a) ] < 3 \varepsilon.
    \end{equation*}
    \v Sto dokazuje \ref{lm:16.7.3}.
\end{proof}

\begin{tm}[1. verzija Helly-Bray-ovog teorema]  \label{tm:16.8}
    Ako je $(F_n)$ niz p.d.F. takav da $F_n \xrightarrow[n \to \infty]{w} F \in \inc$, tada za svaki $a, b \in C(F)$, $a < b$ i za svaku neprekidnu funkciju $g : \segment{a}{b} \to \real$, vrijedi
    \begin{equation*}
        \lim\limits_{n \to \infty} \int_a^b g \: d F_n = \int_a^b g \: d F.
    \end{equation*}
\end{tm}

\begin{proof}
    Za $a \in C(F)$ je $\mu_F (\{ a \}) = 0$, pa je
    \begin{equation*}
        \int_a^b g \: dF = \int_{\lijInt{a}{b}} g(x) \: d \mu_F (x).
    \end{equation*}
    Budu\' ci da je $\segment{a}{b}$ kompaktan funkcija $g$ je uniformno neprekidna
    \begin{equation*}
        (\forall \varepsilon > 0) (\exists \delta > 0) (x, y \in \segment{a}{b}, \quad |x - y| < \delta \implies |g(x) - g(y)| < \varepsilon).
    \end{equation*}
    Budu\' ci da je $C(F)$ gust, mo\v zemo na\' ci $k \in \nat$ i to\v cke $a = x_1 < x_2 < \ldots < x_{k + 1} = b$ u $C(F)$ takve da je $x_{i + 1} - x_i < \delta$, $1 \leq i \leq k$.
    Tada slijedi
    \begin{equation*}
        \begin{gathered}
            \begin{aligned}
                \int_a^b g \: d F_n - \int_a^b g \: d F &= \suma{i = 1}{k} \Big[ \int_{x_i}^{x_{i+1}} g(x) \: d F_n - \int_{x_i}^{x_{i + 1}} g(x_i) \: d F_n + \int_{x_i}^{x_{i+1}} g(x_i) \: d F_n\\
                &\quad \quad- \int_{x_i}^{x_{i + 1}} g(x_i) \: d F + \int_{x_i}^{x_{i + 1}} g(x_i) \: d F - \int_{x_i}^{x_{i + 1}} g(x) \: d F \Big]\\
                &= \suma{i = 1}{k} \Big[ \int_{x_i}^{x_{i + 1}} (g (x) - g(x_i)) \: d F_n + \int_{x_i}^{X_{i + 1}} (g(x_i) - g(x)) \: d F\\
                &\quad \quad+ g(x_i) [ F_n (x_{i + 1}) - F_n (x_i) - F (x_{i + 1}) + F(x_i) ] \Big].
            \end{aligned}\\
            \begin{aligned}
                \implies (\forall n \in \nat) \Big| \int_a^b g \: d F_n - \int_a^b g \: d F \Big| &\leq \varepsilon \cdot \underbrace{\masP_{F_n} (\lijInt{a}{b})}_{\leq 1} + \varepsilon \cdot \underbrace{\mu_{F} (\lijInt{a}{b})}_{\leq 1}\\
                &\quad + \suma{i = 1}{k} | g(x_i) | \cdot \bigg[| F_n (x_{i + 1}) - F (x_{i + 1}) |\\
                &\quad \quad \quad+ | F(x_i) - F_n (x_i) | \bigg].
            \end{aligned}\\
            \implies (\forall \varepsilon > 0) \quad \limsup\limits_{n \to \infty} \Big| \int_a^b g \: d F_n - \int_a^b g \: d F \Big| \leq 2 \varepsilon.
        \end{gathered}
    \end{equation*}
\end{proof}

U daljenjem tekstu $(F_n) = \niz{F_n}{n \in \nat}$ uvijek ozna\v cava niz u kojem je svaka $F_n$ p.d.F.

\begin{lm}  \label{lm:16.9}
    Neka $F_n \xrightarrow[n \to \infty]{w} F \in \inc$ i neka je $g: \real \to \desInt{0}{+\infty}$ neprekidna.
    Za svaki $n \in \nat$, $a \in C(F)$ i $x \in \extReal$, definiramo
    \begin{equation*}
        \begin{aligned}
            G_n (x) &:= \int_a^x g \: d F_n\\
            G(x) &:= \int_a^x g \: d F
        \end{aligned}
    \end{equation*}
    (uo\v cimo $G_n, G : \extReal \to \segment{0}{+\infty}$, jer ili integriramo $g \geq 0$, za $x > 0$, ili $-g \leq 0$, za $a < x$).
    Tada vrijedi:
    \begin{enumerate}[label=(\roman*)]
        \item \label{lm:16.9.1}
        \begin{equation*}
            \restr{G_n}{\real}, \restr{G}{\real} \in \inc,
        \end{equation*}
        \item \label{lm:16.9.2}
        \begin{equation*}
            G (x) = \lim_{n \to \infty} G_n(x), \quad \forall x \in C(G),
        \end{equation*}
        \item \label{lm:16.9.3}
        \begin{equation*}
            \begin{aligned}
                \liminf\limits_{n \to \infty} \int_a^{+\infty} g \: d F_n &\geq \int_a^{+\infty} g \: d F,\\
                \liminf\limits_{n \to \infty} \int_{-\infty}^a g \: d F_n &\geq \int_{-\infty}^a g \: d F.
            \end{aligned}
        \end{equation*}
    \end{enumerate}
\end{lm}

\begin{proof}
    \ref{lm:16.9.1} $g \geq 0 \implies$ $G$ i $G_n$ su neopadaju\' ce.
    Za $x \in \real$, $g$ je ome\dj ena na $\segment{a}{x}$ (odnosno $\segment{x}{y}$) i $\Delta F \leq 1 = \Delta F_n$ $\implies$ $G_n (x)$, $G(x) \in \real$.

    \ref{lm:16.9.2} Direktno iz teorema \ref{tm:16.8} i napomene \ref{nap:16.6e}.

    \ref{lm:16.9.3} Direktno iz \ref{lm:16.9.2} i napomene \ref{nap:16.6c}
\end{proof}

\begin{defn}    \label{defn:16.10}
    Neprekidna funkcija $g: \real \to \real$ je \emph{uniformno integrabilna u odnosu na $(F_n)$} (ili kra\' ce \emph{$(F_n)-\ui$}) ako vrijedi
    \begin{equation*}
        \lim\limits_{a \nearrow +\infty} \sup\limits_{n \in \nat} \int_{\{|y| \geq a\}} |g (y)| \: d F_n (y) = 0.
    \end{equation*}
    
    Re\' ci \' cemo da je $(F_n)$ \emph{napet} ako je konstantna funkcija $g \equiv 1$ $(F_n)-\ui$.
\end{defn}

\begin{zad} \label{zad:16.11}
    \begin{enumerate}[label=(\roman*)]
        \item \label{zad:16.11.1}
        Ako je $g$ $(F_n)-\ui$, tada je $g^+$ $(F_n)-\ui$
        \item \label{zad:16.11.2}
        Ako su $f, g$ $(F_n)-\ui$, te $\alpha, \beta \in \real$, tada je $\alpha f + \beta g$ tako\dj er $(F_n)-\ui$.
        \item \label{zad:16.11.3}
        Ako je $g$ $(F_n)-\ui$, $f$ neprekidna i $|f| \leq |g|$, tada je $f$ $(F_n)-\ui$ 
    \end{enumerate}
\end{zad}

Korsite\' ci zadatak \ref{zad:16.11} mo\v zemo "prevesti" lemu \ref{lm:16.7} u:

\begin{kor} \label{kor:16.12}
    Ako je $F_n \xrightarrow{w} F \in \inc$, tada su sljede\' ce tvrdnje ekvivalentne:
    \begin{enumerate}[label=(\roman*)]
        \item \label{kor:16.12.1}
        \begin{equation*}
            F_n \xrightarrow{c} F,
        \end{equation*}
        \item   \label{kor:16.12.2}
        \begin{equation*}
            (F_n) \textnormal{ je napet},
        \end{equation*}
        \item   \label{kor:16.12.3}
        Svaka ome\dj ena i neprekidna funkcija je $(F_n)-\ui$
    \end{enumerate}
\end{kor}

\begin{lm}  \label{lm:16.13}
    Ako $F_n \xrightarrow{w} F \in \inc$ i $g: \real \to \desInt{0}{+\infty}$ je neprekidna i takva da je $\int\limits_{\real} g \: d F_n < +\infty$, za svaki $n \in \nat$, tada je $g$ $(F_n)-\ui$ ako i samo ako
    \begin{equation*}
        \lim\limits_{n \to \infty} \int_\real g \: d F_n = \int_\real g \: d F < +\infty.
    \end{equation*}
\end{lm}

\begin{proof}
    Neka je $a \in C(F)$ i definiramo $G_n$ i $G$ kao u lemi \ref{lm:16.9}.
    Prema lemi \ref{lm:16.9} \ref{lm:16.9.2} $G_n (x) \to G(x)$, $\forall x \in C(G)$.
    
    Ako je $g$ $(F_n)-\ui$, tada vrijedi lema \ref{lm:16.7} \ref{lm:16.7.3}, pa prema istoj lemi vrijedi i lema \ref{lm:16.7} \ref{lm:16.7.1}, \v sto daje tra\v zenu tvrdnju.

    Ako vrijedi $\int\limits_{\real} g \: d F_n \to \int\limits_\real g \: d F < +\infty$, tada je ispunjena lema \ref{lm:16.7} \ref{lm:16.7.2}, stoga po istoj lemi vrijedi i lema \ref{lm:16.7} \ref{lm:16.7.3}, a to daje da je $g$ $(F_n)-\ui$.
\end{proof}

\begin{kor} \label{kor:16.14}
    Neka je $g: \real \to \real$ neprekidna i $(F_n)-ui$.
    Ako je $F_n \xrightarrow{w} f \in \inc$, tada je $\int\limits_\real |g| \: d F < +\infty$ i vrijedi
    \begin{equation*}
        \lim\limits_{n \to \infty} \int_\real g \: d F_n = \int_\real g \: d F \in \real.
    \end{equation*} 
\end{kor}

\begin{defn}    \label{defn:16.15}
    Niz $(F_n)$ je \emph{slabo kompaktan} ako postoje podniz $(F_{n_k})$ i $F \in \inc$ takvi da $F_{n_k} \xrightarrow{w} F$.

    Re\' ci \' cemo da je $(F_n)$ \emph{potpuno kompaktan} ako svaki podniz sadr\v zi podpodniz koji konvergira potpuno pream nekoj p.d.F..
\end{defn}

\begin{tm}  \label{tm:16.16}
    Svaki niz $(F_n)$ p.d.F. funkcija je slabo kompaktan.
\end{tm}

\begin{proof}
    Uzmimo $D = \{r_j\} \subseteq \real$ prebrojiv i gust.
    Zbog $0 \leq F_n (r_1) \leq 1$, za svaki $n \in \nat$ postoji podniz $\{ F_{n_j, 1} (r_1) \}$ koji je konvergentan.
    Zbog $0 \leq F_{n_j, 1 } (r_2) \leq 1$ postoji podniz tog podniza $\{ F_{n_j, 2} (r_2) \}$ koji je konvergentan.
    Induktivno nastavljamo dalje i uzmemo dijagnoalni podniz $(F_{n_j, j})$ i uo\v cimo da taj konvergira na $D$.
    Koriste\' ci napomenu \ref{nap:16.6f} dolazimo do $F \in \inc$ za koju
    \begin{equation*}
        F_{n_j, j} \xrightarrow{w} F.
    \end{equation*} 
\end{proof}

\begin{kor} \label{kor:16.17}
    Niz $(F_n)$ je potpuno kompaktan ako i samo ako je napet.
\end{kor}

\begin{proof}
    \begin{itemize}
        \item[$\impliedby$]
        Direktno iz teorema \ref{tm:16.16} i korolara \ref{kor:16.12}
        \item[$\implies$]
        Pretpostavimo suprotno, to jest da $1$ nije $(F_n)-\ui$.
        Tada
        \begin{equation*}
            (\exists \varepsilon > 0) (\exists a_n \nearrow + \infty) (\exists \textnormal{ podniz } \{n_j\})
        \end{equation*}
        takvi da je
        \begin{equation*}
            \Delta F_{n_j} - \Delta F_{n_j} (a_j) > \varepsilon, \quad j \in \nat.
        \end{equation*}
        Po lemi \ref{lm:16.7} slijedi da niti jedan podniz od $(F_{n_j})$ ne mo\v ze konvergirati potpuno, \v sto je kontradikcija.
    \end{itemize}    
\end{proof}

\begin{zad} \label{zad:16.18}
    \begin{equation*}
        F_n \xrightarrow{w} F \quad \quad (F_n \xrightarrow{c} F)
    \end{equation*}
    ako i samo ako svaki podniz od $(F_n)$ ima daljnji podniz koji konvergira slabo (potpuno) prema istoj funkciji $F$.
\end{zad}