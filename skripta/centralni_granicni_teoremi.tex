% poglavlje 5.4 - centralni granicni teoremi -> predavanje 19

\chapter{Centralni greni\v cni teoremi}

Po\v cnimo od najja\v cih pretpostavki.
Vidjeli smo u djelu \ref{dio:3} da je rubna vrijednost "brzine konvergencije" $\sqrt{n}$.
Prvo fundamentalno pitanje je: "Ako je $\niz{X:n}{n \in \nat}$ niz nezavisnih jednako distribuiranih slu\v cajnih varijabli i $X_1$ ima sve momente, \v sto se doga\dj a sa $\frac{S_n}{\sqrt{n}}$?"

Neka je $\niz{X_n}{n \in \nat}$ niz nezavisnih jednako distribuiranih slu\v cajnih varijabli na vjerojatnosnom prostoru $\vjerojatnosniProstor$.
Neka je $S_n = X_1 + \ldots + X_n$.
Pretpostavimo da $X_1$ imam momente svakog reda i ozna\v cimo sa $m:= \masE X_1$, te sa $\sigma^2 = \Var X_1$.
Uo\v cimo da u slu\v caju kada je $\Var X_1 = 0$ vrijedi $X_n \equiv m \; g.s.$, pa tada $\frac{S_n}{n} \equiv m$ i taj je slu\v caj trivijalno rje\v sen.
Stoga bez smanjenja op\' cenitosti mo\v zemo promatrati $0 < \sigma^2 < +\infty$.
Zbog nezavisnosti vrijedi
\begin{equation*}
    \varphi_{\frac{S_n - m n}{\sigma \sqrt{n}}} (t) = \produkt{j = 1}{n} \varphi_{\frac{X_j - m}{\sigma}} \Big( \frac{t}{\sqrt{n}} \Big) \overset{\iid}{=} \Big[ \varphi_{\frac{X_1 - m}{\sigma}} \Big( \frac{t}{\sqrt{n}} \Big) \Big]^n
\end{equation*}
Zbog postojanja drugog momenta, po korolaru \ref{kor:17.7} \ref{kor:17.7.3}, slijedi
\begin{equation*}
    \varphi_{\frac{X_1 - m}{\sigma}} \Big( \frac{t}{\sqrt{n}} \Big) = 1 - \frac{t^2}{2 n} + o \Big( \frac{t^2}{n} \Big), \quad \forall t \in \real.
\end{equation*}
Koriste\' ci lemu \ref{lm:17.10} dobivamo
\begin{equation*}
    \varphi_{\frac{S_n - n m}{\sigma \sqrt{n}}} (t) \xrightarrow[n \to \infty]{} e^{-\frac{t^2}{2}} = \varphi_{N(0,1)} (t).
\end{equation*}
Uo\v cimo da je sve spremno za primjenu teorema neprekidnosti, te da smo u izvodu koristili samo drugi moment.
To zna\v ci da smo pokazali sljede\' ci teorem.

\begin{tm}[CLT, $\iid$, $2^{\textnormal{nd}}$ mom.; Paul L\' evy]  \label{tm:19.1}
    \quad \newline
    Neka je $\niz{X_n}{n \in \nat}$ niz nezavisnih jednako distribuiranih slu\v cajnih varijabli i neka je $X_1 \in L^2 (\masP)$.
    Neka je $0 < \sigma^2 < +\infty$.
    Tada
    \begin{equation*}
        \frac{S_n - n m}{\sigma \sqrt{n}} \xrightarrow[n \to \infty]{d} N (0, 1).
    \end{equation*}    
\end{tm}

Ako su na primjer
\begin{equation*}
    X_n \sim
    \begin{pmatrix}
        0 & 1\\
        1 - p & p
    \end{pmatrix},
    \quad p \in \obInt{0}{1},
\end{equation*}
tada je $S_n \sim B (n, p)$, a u konvergenciji "$\xrightarrow{d}$" samo razdioba je va\v zna.
Stoga direktno dobijemo "majstariji CLT".

\begin{kor}[de Moivre-Laplace]  \label{kor:19.2}
    Neka je $0 < p < 1$ i $Z_n \sim B (0, 1)$, za svaki $n \in \nat$.
    Tada
    \begin{equation*}
        \frac{Z_n - n p}{\sqrt{n p (1 - p)}} \xrightarrow[n \to \infty]{d} N (0, 1).
    \end{equation*}
\end{kor}

\begin{nap} \label{nap:19.3}
    Nekoliko prirodnih puteva se sada name\' ce.
    \v Sto se doga\dj a ako $(X_n)$ nisu jednakno distribuirane?
    Mo\v ze li se dogoditi konvergencija prema nekom drugom limesu?
    Kada se de\v sava konvergencija ba\v s prema $N(0, 1)$?
\end{nap}

U prvom koraku oslabit \' cemo pretpostavke, tako da sa\' cuvamo samo nezavisnost, to jest, mi\v cemo zahtjev jednake distribuiranosti.

Neka je $\niz{X_n}{n \in \nat} \subseteq L^2 (\masP)$ niz \emph{nezavisnih} slu\v cajnih varijabli na $\vjerojatnosniProstor$.
Za $n \in \nat$ uvodimo oznake
\begin{equation*}
    \begin{aligned}
        m_n &:= \masE X_n,\\
        \sigma^2_n &:= \Var X_n,\\
        \varphi_n &:= \varphi_{X_n},\\
        s^2_n &:= \Var (S_n) = \suma{k = 1}{n} \Var X_k = \suma{k = 1}{n} \sigma^2_k,
    \end{aligned}
\end{equation*}
te pretpostavljamo da je $s_1 > 0$.
Drugo fundamentalno pitanje je: "Koji su nu\v zni i dovoljni uvjeti
da $\frac{S_n - \masE S_n}{s_n} \xrightarrow[n \to \infty]{d} N(0, 1)$?"
Sljede\' ci teorem je va\v zan u odgovoru na to pitanje.

\begin{tm}  \label{tm:19.4}
    Neka je $\indFamilija{X_{n, m}}{n \in \nat, \; q \leq m \leq n} \subseteq L^2 (\masP)$ familija slu\v cajnih varijabli na vjerojatnosnom prostru $\vjerojatnosniProstor$, takvih da je $\masE X_{n, m} = 0$, za sve $n$ i $m$, te da je za svaki $n \in \nat$, familija $\niz{X_{n, m}}{ 1 \leq m \leq n}$ nezavisna.
    Neka je $S_n := X_{n, 1} + \leq X_{n, m}$, $n \in \nat$.
    Ako vrijedi
    \begin{enumerate}[label=(\alph*)]
        \item \label{tm:19.4.1}
        \begin{equation*}
            \suma{m = 1}{n} \masE \big( X_{n, m}^2 \big) \xrightarrow[n \to \infty]{} \sigma^2, \quad 0 < \sigma^2 < +\infty,
        \end{equation*}
        \item \label{tm:19.4.2}
        \begin{equation*}
            (\forall \varepsilon > 0) \quad \lim\limits_{n \to \infty} \suma{m = 1}{n} \masE \Big( |X_{n, m}|^2 \cdot \karaktFja_{\{ |X_{n, m}| > \varepsilon \}} \Big) = 0,
        \end{equation*}
    \end{enumerate}
    tada
    \begin{equation*}
        S_n \xrightarrow[n \to \infty]{d}  N (0, \sigma^2).
    \end{equation*}
\end{tm}

\begin{proof}
    Neka je $\varphi_{n, m} := \varphi_{X_{n, m}}$ i neka je $\sigma^2_{n, m} := \masE [X_{n, m}^2] = \Var (X_{n, m}^2)$.
    Po teoremu o neprekidnosti, dovoljno je dokazati
    \begin{equation*}
        \produkt{m = 1}{n} \varphi_{n, m} (t) \xrightarrow[n \to \infty]{} e^{-\frac{t^2 \sigma^2}{2}}.
    \end{equation*}
    Ozna\v cimo, za $t \in \real$, te $m$ i $n$,
    \begin{equation*}
        \begin{aligned}
            z_{n, m} &:= \varphi_{n, m} (t),\\
            w_{n, m} &:= 1 - \frac{t^2 \sigma_{n, m}^2}{2}.
        \end{aligned}
    \end{equation*}
    Prema korolaru \ref{kor:17.7} slijedi
    \begin{equation*}
        \begin{aligned}
            |z_{n, m} - w_{n, m}| &= \Big| \masE (e^{i t X_{n, m}}) - \Big( 1 + \frac{\masE [ (i t X_{n, m})^2 ]}{2} \Big) \Big| \leq \masE \Big[ \frac{|t X_{n, m}|^3}{3!} \land \frac{2 |t X_{n, m}|^2}{2!} \Big]\\
            &\leq \masE \Big[ \frac{|t X_{n, m}|^3}{6} \karaktFja_{\{ |X_{n, m}| \leq \varepsilon \}} \Big] + \masE \Big[ |t X_{n, m}|^2 \karaktFja_{\{ |X_{n, m}| > \varepsilon \}} \Big]\\
            &\leq \frac{\varepsilon t^3}{6} \masE \Big[ |X_{n, m}|^2 \karaktFja_{\{ |X_{n, m}| \leq \varepsilon  \}} \Big] + t^2 \cdot \masE \Big[ |X_{n, m}|^2 \karaktFja_{\{ |X_{n, m}| > \varepsilon \}} \Big], \quad \varepsilon > 0.
        \end{aligned}
    \end{equation*}
    Stoga, $\forall \varepsilon > 0$ imamo
    \begin{equation*}
        \begin{aligned}
            \limsup\limits_{n \to \infty} \suma{m = 1}{n} |z_{n, m} - w_{n, m}| &\leq \frac{\varepsilon t^3}{6} \limsup\limits_{n \to \infty} \suma{m = 1}{n} \masE \Big[ |X_{n, m}|^2 \karaktFja_{\{ |X_{n, m}| \leq \varepsilon \}} \Big]\\
            &\quad \quad + t^2 \cdot \underbrace{\limsup\limits_{n \to \infty} \suma{m = 1}{n} \masE \Big[ |X_{n, m}|^2 \karaktFja_{\{|X_{n, m}| > \varepsilon \}} \Big]  }_{\overset{\ref{tm:19.4.2}}{=} 0}\\
            &\leq \frac{\varepsilon t^3}{6} \limsup\limits_{n \to \infty} \suma{m = 1}{n} \masE \Big[ |X_{n, m}|^2 \Big] \overset{\ref{tm:19.4.1}}{=} \frac{1}{6} \varepsilon t^3 \sigma^2.
        \end{aligned}
    \end{equation*}
    Zbog proizvoljnosti $\varepsilon > 0$ dobivamo
    \begin{equation}    \label{jed:19.5}
        \limsup\limits_{n \to \infty} \suma{m = 1}{n} |z_{n, m} - w_{n, m}| = 0.
    \end{equation}

    Uo\v cimo da je $|z_{n, m}| = |\varphi_{n, m} (t)| \leq 1$.
    Zbog
    \begin{equation*}
        \sigma_{n, m}^2 \leq \varepsilon^2 + \masE \Big[ |X_{n, m}|^2 \karaktFja_{\{ |X_{n, m}| > \varepsilon \}} \Big],    
    \end{equation*}
    pa koriste\' ci \ref{tm:19.4.2} dobivamo
    \begin{equation*}
        \lim\limits_{n \to \infty} \big( \sup\limits_{1 \leq m \leq n} \sigma_{n, m}^2 \big) = 0.
    \end{equation*}
    Dakle za $n$ dovoljno velik, $|w_{n, m}| \leq 1$, $1 \leq m \leq n$.
    Za dovoljno velik $n$ primjenimo lemu \ref{lm:17.9} uz $\alpha = 1$ te dobijemo
    \begin{equation*}
        \Big| \produkt{m = 1}{n} \varphi_{n, m} (t) - \produkt{m = 1}{n} \Big( 1 - \frac{t^2 \sigma_{n, m}^2}{2} \Big) \Big| \leq \suma{m = 1}{n} |z_{n, m} - w_{n, m}| \xrightarrow[n \to \infty]{} 0,
    \end{equation*}
    gdje konvergencija slijedi zbog \eqref{jed:19.5}.
    Sada koriste\' ci \ref{tm:19.4.1} jo\v s jednom, dobivamo
    \begin{equation*}
        \lim\limits_{n \to \infty} \suma{m = 1}{n} \frac{t^2 \sigma_{n, m}^2}{2} = \frac{t^2 \sigma^2}{2},
    \end{equation*}
    te budu\' ci da ve\' c imamo
    \begin{equation*}
        \sup\limits_{1 \leq m \leq n} \sigma_{n, m}^2 \xrightarrow[n \to \infty]{} 0,    
    \end{equation*}
    koriste\' ci sli\v can argument kao u lemi \ref{lm:17.10} dobijemo
    \begin{equation*}
        \produkt{m = 1}{n} \Big( 1 - \frac{t^2 \sigma_{n, m}^2}{2} \Big) \xrightarrow[n \to \infty]{} e^{- \frac{t^2 \sigma^2}{2}}.
    \end{equation*}
\end{proof}

\begin{kor}[dovoljan uvjet, J. Lindeberg 1922]  \label{kor:19.6}
    \quad \newline
    Ako za svaki $\varepsilon > 0$,
    \begin{equation}    \label{jed:19.7}
        \lim\limits_{n \to \infty} \frac{1}{s_n^2} \suma{k = 1}{n} \int_{\{|X_k - m_k| \geq \varepsilon s_n\}} (x - m_k)^2 \: d F_{X_k} (x) = 0,
    \end{equation}
    tada
    \begin{equation*}
        \frac{S_n - \masE S_n}{s_n} \xrightarrow[n \to \infty]{d} N(0,1).
    \end{equation*}
\end{kor}

\begin{proof}
    Za $n \in \nat$ te $1 \leq m \leq n$ stavimo
    \begin{equation*}
        X_{n, m} := \frac{X_m - m_m}{s_n}.
    \end{equation*}
    Sada imamo
    \begin{equation*}
        \suma{m = 1}{n} \masE [X_{n, m}^2] = \frac{1}{s_n^2} \suma{m = 1}{n} \Var X_m = 1,
    \end{equation*}
    pa je uvjet teorema \ref{tm:19.4} \ref{tm:19.4.1} ispunjen uz $\sigma = 1$.
    Primjetimo da je
    \begin{equation*}
        \suma{k = 1}{n} \int_{\{|X_k - m_k| \geq \varepsilon s_n\}} (x - m_k)^2 \: d F_{X_k} (x) = \suma{m = 1}{n} \masE \Big[ |X_{n, m}|^2 \cdot \karaktFja_{\{ |X_{n, m}| > \varepsilon \}} \Big].
    \end{equation*}
    Stoga je ispunjen i zahtjev \ref{tm:19.4.2} teorema \ref{tm:19.4}.
    Sada po teoremu \ref{tm:19.4} slijedi
    \begin{equation*}
        \frac{S_n - \masE S_n}{s_n} = X_{n, 1} + \ldots + X_{n, k} \xrightarrow[n \to \infty]{d} N(0, 1).
    \end{equation*}
\end{proof}

Vrijedi li obrat ovog teorema?
Ako da, to bi bio odgovor na na\v se fundamentalno pitanje.
Na odgovor se \v cekalo vi\v se od 10 godina.
Odgovor je pozitivan uz uvijet da je zadovoljeno jedno svojstvo koje je posljedica Lindebergovog uvjeta \eqref{jed:19.7}.

\begin{lm}  \label{lm:19.8}
    Ako $\niz{X_n}{n \in \nat}$ zadovoljava Lindebergov uvijet \eqref{jed:19.7} i $m_n = 0$, za svaki $n \in \nat$ i $s_n \nearrow +\infty$, tada
    \begin{equation}    \label{jed:19.9}
        \lim\limits_{n \to \infty} \max\limits_{1 \leq k \leq n} \frac{\sigma_k^2}{s_n^2} = 0.
    \end{equation}
\end{lm}

\begin{proof}
    Primjetimo
    \begin{equation*}
        \begin{aligned}
            \max\limits_{1 \leq k \leq n} \frac{\sigma_k^2}{s_n^2} &= \max\limits_{1 \leq k \leq n} \frac{1}{s_n^2} \Big[ \int_{\{ |X_k|\leq \varepsilon s_n \}} x^2 \: d F_k (x) + \int_{\{ |x| > \varepsilon s_n \}} x^2 \: d F_k (x) \Big]\\
            &\leq \max\limits_{1 \leq k \leq n} \frac{1}{s_n^2} \Big[ \varepsilon^2 s_n^2 + \int_{\{ |x| > \varepsilon s_n \}} x^2 \: d F_k (x) \Big]\\
            &\leq \varepsilon^2 + \frac{1}{s_n^2} \suma{k = 1}{n} \int_{\{ |x| > \varepsilon s_n \}} x^2 \: d F_k (x), \quad \forall \varepsilon > 0,
        \end{aligned}
    \end{equation*}
    sada \eqref{jed:19.7} daje \eqref{jed:19.9}.
\end{proof}

Uo\v cimo da u slu\v caju da $s_n \nearrow +\infty$, tada je \eqref{jed:19.9} ekvivalentno s
\begin{equation}    \label{jed:19.10}
    \lim\limits_{n \to \infty} \frac{\sigma_n}{s_n} = 0.
\end{equation}
Nadalje, podsjetimo se da za $|z| \leq 1$ (uz konvenciju $\ln 0 = 0$) imamo
\begin{equation}    \label{jed:19.11}
    | \ln z - (z - 1) | \leq |z - 1|^2,
\end{equation}
\v sto slijedi iz
\begin{equation*}
    \ln z = \suma{k = 1}{\infty} (-1)^{k-1} \frac{(z - 1)^2}{k}, \quad z \neq 0
\end{equation*}
i direktno za $z = 0$.

\begin{tm}[nu\v zan uvjet, W. Feller 1935]  \label{tm:19.12}
    Ako
    \begin{equation*}
        \begin{gathered}
            s_n \nearrow +\infty\\
            \lim\limits_{n \to \infty} \frac{\sigma_n}{s_n} = 0\\
            \frac{S_n - \masE S_n}{s_n} \xrightarrow[n \to \infty]{d} N(0,1),
        \end{gathered}
    \end{equation*}
    tada $(X_n)$ zadovoljava Lindebergov uvjet \eqref{jed:19.7}.
\end{tm}

\begin{proof}
    Bez smanjenja op\' cenitosti mo\v zemo gledati slu\' caj $m_k = 0$, $\forall k$.
    Koriste\' ci \eqref{jed:19.9}, \eqref{jed:19.10} , \eqref{jed:19.11} i korolar \ref{kor:17.7} dobivamo
    \begin{equation*}
        \begin{aligned}
            \suma{k = 1}{n} \Big| \ln \varphi_k \Big( \frac{t}{s_n} \Big) - \Big( \varphi_k \Big( \frac{t}{s_n} \Big) - 1  \Big) \Big| &\leq \suma{k = 1}{n} \Big| \varphi_k \Big( \frac{t}{s_n} \Big) - 1 \Big|\\
            &\leq \max_{1 \leq k \leq n} \Big| \varphi_k \Big( \frac{t}{s_k} \Big) - 1 \Big| \cdot \suma{k = 1}{n} \Big| \varphi_k \Big( \frac{t}{s_n} \Big) -1 \Big|\\
            &\leq \max\limits_{1 \leq k \leq n} \Big| \varphi_k \Big( \frac{t}{s_n} \Big) - 1 \Big| \cdot \frac{t^2}{2} \underbrace{\suma{k = 1}{n} \frac{\sigma_k^2}{s_n^2} }_{= 1} = o(1).    
        \end{aligned}
    \end{equation*}
    Po teoremu neprekidnosti i na\v si pretpostavkama
    \begin{equation*}
        \begin{gathered}
            \lim\limits_{n \to \infty} \produkt{k = 1}{n} \varphi_k \Big( \frac{t}{s_n} \Big) = e^{-\frac{t^2}{2}} \implies \suma{k = 1}{n} \ln \varphi_k \Big( \frac{t}{s_n} \Big) \xrightarrow[n \to \infty] -\frac{t^2}{2}\\
            \implies \quad -\frac{t^2}{2} = \suma{k = 1}{n} \Big( \varphi_k \Big( \frac{t}{s_n} \Big) -1 \Big) + o(1)
        \end{gathered}
    \end{equation*}
    \v sto daje
    \begin{equation*}
        \frac{t^2}{2} - \suma{k = 1}{n} \int_{\{ |X_k| \leq \varepsilon s_n \}} \Big( 1 - \cos \frac{tx}{s_n} \Big) \: d F_k (x) = \suma{k = 1}{n} \int_{\{ |X_k| > \varepsilon s_n \}} \Big( 1 - \cos \frac{tx}{s_n} \Big) \: d F_k (x) + o(1).
    \end{equation*}
    Na desnoj strani imamo ocjenu
    \begin{equation*}
        \Big( 1 - \cos \frac{tx}{s_n} \Big) \leq 2 = 2 \cdot 1 \leq  \frac{2x^2}{\varepsilon^2 s_n^2},
    \end{equation*}
    dok na lijevoj strani imamo ocjenu
    \begin{equation*}
        \Big( 1 - \cos \frac{tx}{s_n} \Big) \leq \frac{t^2 x^2}{2 s_n^2}.
    \end{equation*}
    Odakle slijedi
    \begin{equation*}
        \begin{aligned}
            \frac{t^2}{2} - \frac{t^2}{2 s_n^2} \suma{k = 1}{n} \int_{\{ |X_k| \leq \varepsilon s_n \}} x^2 \: d F_k (x) &\leq \suma{k = 1}{n} \int_{\{ |X_k| > \varepsilon s_n \}} \frac{2 x^2}{\varepsilon^2 s_n^2} \: d F_k (x) + o(1)\\
            &\leq \frac{2}{\varepsilon^2 s_n^2} \suma{k = 1}{n} \sigma_k^2 + o (1) \quad \Big/ \cdot \frac{2}{t^2},
        \end{aligned}
    \end{equation*}
    \v sto ima za posljedicu
    \begin{equation*}
        1 - \frac{1}{s_n^2} \suma{k = 1}{n} \int_{\{ |X_k| \leq \varepsilon s_n \}} x^2 \: d F_k (x) \leq \frac{4}{t^2 \varepsilon^2} + o(1) \quad \quad n \to \infty.
    \end{equation*}
    Sada, za svaki $t$ i za svaki $\varepsilon > 0$ vrijedi
    \begin{equation*}
        1 - \frac{4}{t^2 \varepsilon^2} \leq \liminf_{n \to \infty} \frac{1}{s_n^2} \suma{k = 1}{n} \int_{\{ |X_k| \leq \varepsilon s_n \}} x^2 \: d F_k (x),
    \end{equation*}
    pu\v stamo $t \to +\infty$ i vidimo
    \begin{equation*}
        (\forall \varepsilon > 0) \quad 1 \leq \liminf\limits_{n \to \infty} \frac{1}{s_n^2} \suma{k = 1}{n} \int_{\{ |X_k| \leq \varepsilon s_n \}} x^2 \: d F_k (x) \leq \frac{1}{s_n^2} \suma{k = 1}{n} \sigma_k^2 = 1.
    \end{equation*}
    Kona\v cno, imamo
    \begin{equation*}
        \lim\limits_{n \to \infty} \frac{1}{s_n^2} \suma{k = 1}{n} \int_{\{ |X_k| > \varepsilon s_n \}} x^2 \: d F_k (x) = 0.
    \end{equation*}
\end{proof}

\begin{kor}[Lindeberg-Fellerov teorem]  \label{kor:19.13}
    
\end{kor}