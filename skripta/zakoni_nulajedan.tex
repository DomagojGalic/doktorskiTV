% zakoni 0 - 1

\chapter{Zakoni 0-1}  \label{zakoni_01}

\begin{pr}[majmun za pisa\' cim strojem]  \label{pr:9.1}
    Zamislimo majmuna koji nasumica tipka po pisa\' cem stroju sa $M$ tipki.
    Mo\v zemo zamisliti da je to niz nezavisnih pokusa u kojima majmun (nasumice) poga\dj a pojedinu tipku sa vjerojatno\v s\' cu $\frac{1}{M}$.
    Zamislimo li da majmu tipka unedogled, koja je vjerojatnost da \' ce majmun u nekom "intervalu" natipkati "Dundo Maroje"?
    Uo\v cimo li da je "D.M." specifi\v can niz simbola duljine $T$, pa mo\v zemo niz pokusa promatrati kao niz simbola duljine $T$.
    Po korolaru \ref{kor:7.9} to su opet nezavisni slu\v cajni elementi i vjerojatnsot da u jednoj sekvenci majmun pogodi "D.M." je $p = \frac{1}{M^T}$.
    Iako je $p$ izrazito mali, va\v zno je da je $p > 0$.
    Ozna\v cimo li sekvencu sa $X_n$ dobivamo
    \begin{equation*}
        \begin{aligned}
            \vjeroj{\textnormal{bar jedna sekvenca "D.M."}}
            &= \vjeroj{\unija{n \in \nat}{} \{X_n = \textnormal{D.M.}\}}\\
            &= \vjeroj{\unija{n \in \nat}{} \{X_1 \neq \textnormal{D.M.}, \ldots, X_{n-1} \neq \textnormal{D.M.}, \: X_n = \textnormal{D.M.}\}}\\
            &= \textnormal{(disjunktna)}\\
            &= \suma{n \in \nat}{} (1 - p)^{n - 1} \cdot p = p + (1 - p) \: p + (1 - p)^2 \: p + \ldots\\
            &= \Big( \suma{n = 1}{\infty} (1 - p)^{n - 1} \Big) \cdot p = \frac{1}{1 - (1 - p)} \cdot p\\
            &= 1.
        \end{aligned}
    \end{equation*}
\end{pr}

\begin{nap} \label{nap:9.1-1}
    Prisjetimo se, neka je $\niz{A_n}{n \in \nat}$ niz skupova, tada su limes superior i limes inferior definirani sa:
    \begin{enumerate}[label=(\roman*)]
        \item $\liminf\limits_{n \to \infty} A_n := \unija{n = 1}{\infty} \Bigg( \presjek{k = n}{\infty} A_k \Bigg)$
        \item $\limsup\limits_{n \to \infty} A_n := \presjek{n = 1}{\infty} \Bigg( \unija{k = n}{\infty} A_k \Bigg)$
    \end{enumerate}
\end{nap}

Korisna je sljede\' ca karakterizacija limesa inferior i limesa superior

\begin{prop}    \label{prop:9.1-2}
    Neka je $\niz{A_n}{n \in \nat}$ niz podskupova od $\Omega$.
    Neka je za svaki $\omega \in \Omega$:
    \begin{equation*}
        K_\omega := \skup{n \in \nat}{\omega \in A_n},
    \end{equation*}
    tada vrijedi:
    \begin{enumerate}[label=(\arabic*)]
        \item $\omega \in \limsup\limits_{n \to \infty} A_n \iff \card{K_\omega} = \infty$,
        \item $\omega \in \liminf\limits_{n \to \infty} A_n \iff \card{\nat \setminus K_\omega} < +\infty$.
    \end{enumerate}
\end{prop}

Odnosno elementarni doga\dj aj se nalazi u limes superioru niza ako se nalazi u beskona\v cno mnogo doga\dj aja tog niza, dok se elementarni doga\dj aj nalazi u limes inferioru niza ako se nalazi u svima, osim eventualno kona\v cno mnogo elemenata tog niza.

Is tog razloga koristimo kraticu $\io$ (eng. infinitely often) za limes superior niza.

\begin{lm}[Borel - Cantelli]  \label{lm:9.2}
    Neka je $\vjerojatnosniProstor$ vjerojatnosni prostor i neka je $\niz{A_n}{n \in \nat} \subseteq \famF$ niz doga\dj aja za koje vrijedi
    \begin{equation*}
        \suma{n = 1}{\infty} \vjeroj{A_n} < +\infty.
    \end{equation*}
    Tada je
    \begin{equation*}
        \vjeroj{A_n \; i.o.} := \vjeroj{\limsup\limits_{n \to +\infty} A_n} = 0.
    \end{equation*}
\end{lm}

\begin{proof}
    Pretpostavimo da vrijedi
    $\suma{n = 1}{\infty} \vjeroj{A_n} < +\infty$.
    \begin{equation*}
        \begin{aligned}
            \suma{n = 1}{\infty} \masE [ \karaktFja_{A_n} ] &= \suma{n = 1}{\infty} \vjeroj{A_n}\\
            &= (\textnormal{prema LTMK-u})\\
            &= \masE \Big[ \suma{n = 1}{\infty} \karaktFja_{A_n} \Big] < +\infty.
        \end{aligned}
    \end{equation*}
    Tada $(\exists C \in \famF)$ takav da je $\vjeroj{C} = 1$ i da je $\suma{n = 1}{\infty} \karaktFja_{A_n}(\omega) < +\infty$, $\forall \omega \in C$
    \begin{equation*}
        \implies \vjeroj{A_n \; i.o.} \leq \vjeroj{C^c} = 0. 
    \end{equation*}
\end{proof}

\begin{tm}[Borelov zakon 0 - 1]   \label{tm:9.3}
    Neka je $\vjerojatnosniProstor$ vjerojatnosni prostor i neka je $\niz{A_n}{n \in \nat} \subseteq \famF$ niz nezavisnih doga\dj aja.
    Tada vrijedi:
    \begin{equation*}
        \begin{aligned}
            \vjeroj{A_n \; i.o.} = 0 &\iff \suma{n = 1}{\infty} \vjeroj{A_n} < +\infty\\
            \vjeroj{A_n \; i.o.} = 1 &\iff \suma{n = 1}{\infty} \vjeroj{A_n} = +\infty.
        \end{aligned}
    \end{equation*}
\end{tm}

\begin{proof}
    Prema lemi \ref{lm:9.2} dovoljno je pokazati tvrdnju
    \begin{equation*}
        \suma{n = 1}{\infty} \vjeroj{A_n} = + \infty \implies \vjeroj{A_n \; \io} = 1.
    \end{equation*}
    Diskusijom toka funkcije $h(x) = e^{-x} + x - 1$ mo\v ze se pokazati nejednakost
    \begin{equation*}
        1 - x \leq e^{-x}.
    \end{equation*}
    Sada imamo:
    \begin{equation*}
        \begin{aligned}
            \masP \Big(\unija{k \geq n}{} A_k \Big) &= 1 - \masP \Big( \presjek{k \geq n}{} A_k^c \Big) = (\textnormal{prema zadatku \ref{zad:6.14}})\\
            &= 1 - \produkt{k \geq n}{} (1 - \vjeroj{A_k})\\
            &\geq 1 - e^{- \suma{k \geq n}{} \vjeroj{A_k}}\\
            &= 1
        \end{aligned}
    \end{equation*}
    To povla\v ci:
    \begin{equation*}
        \begin{aligned}
            \masP ( A_n \; \io ) &= \masP \Big( \presjek{n \in \nat}{} \unija{k \geq n}{} A_k \Big) = \lim\limits_{n \to \infty} \masP \Big( \unija{k \geq n}{} A_k \Big)\\
            &= 1.
        \end{aligned}
    \end{equation*}
\end{proof}

\begin{nap} \label{nap:9.4}
    U primjeru \ref{pr:9.1}. $A_n = \{ X_n = \textnormal{D.M.} \}$ su nezavisni i $\vjeroj{A_n} = p = \frac{1}{M^T} > 0 \implies \sum \vjeroj{A_n} = +\infty \implies \vjeroj{A_n \; i.o.} = 1$ (!).
    Pa ipak prosje\v cno vrijeme (ili prvi $n$) da se dogodi $A_n$ je $\frac{1}{p} = M^T = 10^{10^7}$ (ako treba 1 sekunda po znaku, prije \' ce se sunce ohladiti).
\end{nap}

\begin{defn}    \label{defn:9.5}
    Ka\v zemo da je $\sigma$-algebra na vjerojatnosnom prostoru $\vjerojatnosniProstor$ \emph{trivijalna} ako vrijedi
    \begin{equation*}
        A \in \famG \implies \vjeroj{A} \in \{0, \; 1\}
    \end{equation*}
\end{defn}

Uo\v cimo da je $\famG$ trivijalna ako i samo ako je nezavisna sa samom sobom.
Dodavanje ili oduzimanje proizvoljnog broja trivijalnih $\sigma$-algebri ne\' ce mjenjati status nezavisnosti te familije.

\begin{lm}  \label{lm:9.6}
    Neka je $\vjerojatnosniProstor$ vjerojatnosni prostor, $\famG \subseteq \famF$ $\sigma$-algebra, $E$ potpun, separabilan metri\v cki prostor i $X : \Omega \to E$ $\famG$-izmjeriv slu\v cajni element.
    Ako je $\famG$ trivijalna, tada je $X$ degenerirana.
\end{lm}

\begin{proof}
    Za svaki $n \in \nat$, $E$ mo\v zemo prikazati kao
    \begin{equation*}
        E = \unija{j \in \nat}{} B_j^n,
    \end{equation*}
    pri \v cemu je ovo disjunktna unija Borelovih skupova i za svaki $j \in \nat$ vrijedi
    \begin{equation*}
        \diam{B_j^n} < \frac{1}{n}.
    \end{equation*}
    Tada su $\{ X \in B_j^n \}$ u $\sigma$-algebri $\famG$, pa su im vjerojatnosti $0$ ili $1$.
    \begin{equation*}
        \begin{aligned}
            1 &= \vjeroj{\Omega} = \suma{j \in \nat}{} \vjeroj{X \in B_{j(n)}^n} \implies\\
            \masP \Big( X \in \presjek{n \in \nat}{} B_{j(n)}^n \Big) &= 1 \implies\\
            \presjek{n \in \nat}{} B_{j(n)}^n &\neq \varnothing
        \end{aligned}
    \end{equation*}
    S druge strane
    \begin{equation*}
        \diam{\presjek{n \in \nat}{}B_{j(n)}^n} = 0 \implies (\exists x \in E) \; \presjek{n \in \nat}{} B_{j(n)}^n = \{ x \}.
    \end{equation*}
\end{proof}

\begin{defn}    \label{defn:9.7}
    Neka je $\indFamilija{\famF_n}{n \in \nat}$ niz $\sigma$-algebri na nepraznom skupu $\Omega$.
    Definiramo $\sigma$-algebru $\famT$ sa
    \begin{equation*}
        \famT := \presjek{n \in \nat}{} \indSigAlg{\famF_k}{k > n},
    \end{equation*}
    nazivamo \emph{repnom $\sigma$-algebrom} (u odnosu na $\indFamilija{\famF_n}{n \in \nat}$).

    Funkciju
    \begin{equation*}
        X : \Omega \to \urePar{E}{\famE}
    \end{equation*}
    koja je $\urePar{\famT}{\famE}$-izmjeriva nazivamo \emph{repnom funkcijom}.

    Ako je $\vjerojatnosniProstor$ vjerojatnosni prostor i $\famF_n \subseteq \famF$ za svaki $n \in \nat$, elemente $\sigma$-algebre $\famT$ nazivamo \emph{repnim doga\dj ajima}.
\end{defn}

\begin{pr}  \label{pr:9.8}
    Neka je $\niz{X_n}{n \in \nat}$ niz slu\v cajnih varijabli na vjerojatnosnom prostoru $\vjerojatnosniProstor$ i neka je $\famF_n = \sigAlg{X_n}$ za svaki $n \in \nat$.
    Tipi\v cni repni doga\dj aju su, za $a \in \extReal$,
    \begin{itemize}
        \item[] $\displaystyle \bigSkup{\omega \in \Omega}{(X_n(\omega))_{n \in \nat} \; \textnormal{ konvergira}}$
        \item[] $\displaystyle \bigSkup{\omega \in \Omega}{\lim\limits_{n \to \infty} X_n(\omega) = a}$
        \item[] $\displaystyle \bigSkup{\omega \in \Omega}{\suma{n = 1}{\infty} X_n(\omega) \; \textnormal{ konvergira}}$
    \end{itemize}
    dok
    \begin{itemize}
        \item[] $\displaystyle \bigSkup{\omega}{\suma{n = 1}{\infty} X_n (\omega) = a}$
    \end{itemize}
    nije repni doga\dj aj.
\end{pr}

\begin{tm}[Kolmogorovljev zakon 0-1]  \label{tm:9.9}
    Ako su $\niz{\famF_n}{n \in \nat}$, $\famF_n \subseteq \famF$, nezavisne $\sigma$-algebre na vjerojatnosnom prostrou $\vjerojatnosniProstor$, tada je repna $\sigma$-algebra $\famT$ trivijalna.
\end{tm}

\begin{proof}
    Za $n \in \nat$ definiramo $\famT_n := \indSigAlg{\famF_k}{k > n}$.
    Po korolaru \ref{kor:6.11} $\sigma$-algebre $\famF_1, \ldots, \famF_n, \; \famT_n$ su nezavisne.
    Budu\' ci da je $\famT \subseteq \famT_n$, po napomeni \ref{nap:6.9} \ref{nap:6.9d} slijedi da su $\famF_1, \ldots, \famF_n, \; \famT$ nezavisne za svaki $n \in \nat$.
    To, prema napomeni \ref{nap:6.9} \ref{nap:6.9c}, zna\v ci da su $\famT, \; \famF_1, \; \famF_2, \ldots$ nezavisne $\sigma$-algebre.
    Po korolaru \ref{kor:6.11} $\famT$ i $\indSigAlg{\famF_n}{n \in \nat}$ su nezavisne, a po napomeni \ref{nap:6.9} \ref{nap:6.9d} i $\famT \subseteq \indSigAlg{\famF_n}{n \in \nat}$ slijedi da su $\famT$ i $\famT$ nezavisne, to jest $\famT$ je trivijalna.
\end{proof}

Usporedimo li primjer \ref{pr:9.8} s teoremom \ref{tm:9.9}, jasno je da \' ce teorem \ref{tm:9.9} igrati va\v znu ulogu u analizi grani\v cnog pona\v sanja nezavisnih slu\v cajnih nizova.
Time \' cemo se detaljno bavit kasnije, a sad obratimo pa\v znju na jo\v s jedan klasi\v cni 0-1 zakon.

\begin{defn}    \label{defn:9.9-1}
    Neka je $p: \nat \to \nat$ bijekcija takva da postoji $m \in \nat$ sa svojstvom:
    \begin{equation*}
        n \geq m \implies p_n := p(n) = n.
    \end{equation*}
    Tada ka\v zemo da je $p$ \emph{kona\v cna permutacija} na $\nat$.
\end{defn}

Ako je $S$ neprazan skup i $p$ kona\v cna permutacija na $\nat$, tada $p$ inducira permutaciju $f_p$ na $S^\infty$ definiranu s
\begin{equation*}
    f_p (s_1, \; s_2, \ldots) = (s_{p_1}, \; s_{p_2}, \ldots).
\end{equation*}

\begin{defn}    \label{defn:9.9-2}
    Skup $A \subseteq S^\infty$ je \emph{simetri\v can u odnosu na kona\v cne permutacije} ako vrijedi:
    \begin{equation*}
        f_p^{-1} (A) = A,
    \end{equation*}
    za svaku kona\v cnu permutaciju $p$ na $\nat$.
\end{defn}

Uo\v cimo da na izmjerivom prostoru $\urePar{S}{\famS}$ familija simetri\v cnih skupova tvori $\sigma$-algebru, ozna\v cenu s $\famS_{\simetr}^\infty$, koja je sadr\v zana u $\sigma$-algebri $\famS^\infty$.
% dokaži tu tvrdnju!
Tu $\sigma$-algebru $\famS_{\simetr}^{\infty}$ obi\v cno zovemo \emph{permutacijski invarijantnom} na $S^\infty$.

\begin{nap} %interna napomena
    Doka\v zi gornju tvrdnju.
\end{nap}

\begin{zad} \label{zad:9.10}
    Neka su $\famF_1 \subseteq \famF_2 \subseteq \ldots \subseteq \famF$ $\sigma$-algebre na vjerojatnosnom prostoru $\vjerojatnosniProstor$.
    Za svaki $A \in \indSigAlg{\famF_n}{n \in \nat}$ postoji niz $\niz{A_n}{n \in \nat} \subseteq \unija{n \in \nat}{} \famF_n$, takav da je $\lim\limits_{n \to \infty} \vjeroj{A_n \triangle A} = 0$.
\end{zad}

Podsjetimo se da za niz $E$-zna\v cnih slu\v cajnih elemenata $(X_n)$, mo\v zemo promatrati $X:=(X_1, \; X_2, \ldots)$ kao preslikavanje
\begin{equation*}
    X: \urePar{\Omega}{\famF} \to \urePar{E^\infty}{\famE^\infty}.
\end{equation*}

\begin{tm}[Hewitt-Savagev zakon 0-1]    \label{tm:9.11}
    Neka su $X_1, \; X_2, \ldots$ nezavisni jednako distribuirani slu\v cajni elementi s vjerojatnostima u izmjerivom prostoru $\urePar{E}{\famE}$.
    Tada je
    \begin{equation*}
        \praslika{X}(\famE_{\simetr}^\infty) \subseteq \famF    
    \end{equation*}
    trivijalna $\sigma$-algebra.
\end{tm}

\begin{proof}
    $\praslika{X} (\famE^\infty) \subseteq \famF$, pa je $\praslika{X} (\famE_{\simetr}^\infty) \subseteq \famF$ i $\sigma$-algebra je.
    Treba dokazati da je trivijalna.
    Uo\v cimo da je $\masP_X$ vjerojatnost na $\urePar{E^\infty}{\famE^\infty}$, pa treba dokazati da je $\famE_{\simetr}^\infty$ trivijalno u odnosu na $\masP_X$.

    Neka je
    \begin{equation*}
        \famE_n := \sigAlg{\pi_1, \ldots, \pi_n}, \quad n \in \nat    
    \end{equation*}
    pri \v cemu su $\pi_i : E^\infty \to E$ projekcije.
    Uo\v cimo da vrijedi
    \begin{equation*}
        \begin{aligned}
            \famE^\infty &= \indSigAlg{\famE_n}{n \in \nat}\\
            \famE_{\simetr}^\infty &\subseteq \famE^\infty,
        \end{aligned}
    \end{equation*}
    pa za $A \in \famE_{\simetr}^\infty$ primijenimo zadatak \ref{zad:9.10}, znamo da mo\v zemo prona\' ci niz
    \begin{equation*}
        \niz{A_n}{n \in \nat} \subseteq \unija{n \in \nat}{} \famE_n
    \end{equation*}
    takav da je
    \begin{equation*}
        \lim\limits_{n \to \infty} \masP_X (A \triangle A_n) = 0.
    \end{equation*}
    Bez smanjenja op\' cenitosti mo\v zemo posti\' ci da je $A_n \in \famE_n$, za svaki $n \in \nat$.
    Uzmimo permutaciju takvu da
    \begin{equation*}
        \begin{aligned}
            \{1, \ldots, n\} &\to \{n+1, \ldots, 2n\}\\
            \{n+1, \ldots, 2n\} &\to \{1, \ldots, n\}.
        \end{aligned}
    \end{equation*}
    Tada je $f_p(A) = A$ (uo\v cimo i $f_p^{-1} = f_p$) i neka je
    \begin{equation*}
        B_n := f_p (A_n).
    \end{equation*}
    Budu\' ci da su $(X_n)$ nezavisni i jednako distribuirani slijedi fa je
    \begin{equation*}
        (\masP_X)_{f_p} = \masP_X    
    \end{equation*}
    pa je
    \begin{equation*}
        \begin{gathered}
            \masP_X (B_n) = \masP_X (A_n) \xrightarrow[n \to \infty]{} \masP_X (A)\\
            \masP_X (B_n \triangle A) \xrightarrow[n \to \infty]{} 0.
        \end{gathered}
    \end{equation*}
    Slijedi da je
    \begin{equation*}
        \masP_X (A \triangle (A_n \cap B_n)) \leq \masP_X (A \triangle A_n) + \masP_X (A \triangle B_n) \xrightarrow[n \to \infty]{} 0.
    \end{equation*}
    Uo\v cimo da su zbog nezavisnosti $(X_n)$ skupovi $A_n$ i $B_n$ nezavisni u odnosu na $\masP_X$, pa dobivamo:
    \begin{equation*}
        \begin{aligned}
            \masP_X (A)
            &= \lim\limits_{n \to \infty} \masP (A_n \cap B_n) = \lim\limits_{n \to \infty} \masP_X (A_n) \cdot \masP_X (B_n)\\
            &= \masP_X (A) \cdot \masP_X(A) \implies \masP(A) \in \{0, \; 1\}.
        \end{aligned}
    \end{equation*}
\end{proof}

\begin{nap} \label{nap:9.12}
    Slu\v cajna varijabla modelira numeri\v cke ishode nekog slu\v cajnog pokusa.
    
    Kona\v cno ponavljanje pokusa modeliramo slu\v cajnim vektorima.
    U oba slu\v caja je vjerojatnosna informacija sadr\v zana u funkciji distribucije koja je
    \begin{equation*}
        F: \real^d \to \segment{0}{1}
    \end{equation*}
    i za koju imamo razvijen analiti\v cki (ili kombinatorni) aparat.
    Vrlo \v cesto nam je (i s teorijskog i s primjenjenog stajali\v sta) va\v zno, barem konceptualno, razmi\v sljati o beskona\v cnom ponavaljanju pokusa.
    Prva najjednostavnija prepostavka je da se ti pokusi provode \emph{nezavisno}.
    Stoga osnovni objekt na\v se teorije postaje niz nezavisnih slu\v cajnih varijabli $\niz{X_n}{n \in \nat}$ na vjerojatnosnom prostoru $\vjerojatnosniProstor$.
    Uo\v cimo da rezultati iz poglavlja \ref{poglavlje3} pokazuju da se takav niz uvijek mo\v ze realizirati (i to \v cak na $\segment{0}{1}$).
    Dakle, za svaki izbor p.d.F. $F_1, \; F_2, \ldots$ postoji niz nezavisnih slu\v cajnih varijabli, na istom vjerojatnosnom prostoru, koje imaju to\v cno zadane distribucije.
    Jedno od najva\v znijih pitanja postaje asimptotsko pona\v sanje niza $(X_n)$.
    Osobito je va\v zan poseban slu\v caj u kojem postoji p.d.F. $F$ takav da je $F_{X_n} = F$, za svaki $n \in \nat$.
    To je slu\v caj niza \emph{nezavisnih i jednako distribuiranih slu\v cajnih varijabli}, koji \' cemo kra\' ce ozna\v cavati sa $\iid$
\end{nap}