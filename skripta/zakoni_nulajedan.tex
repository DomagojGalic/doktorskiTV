% zakoni 0 - 1

\chapter{Zakoni 0 - 1}

\begin{pr}[majmun za pisa\' cim strojem]  \label{pr:9.1}
    Zamislimo majmuna koji nasumica tipka po pisa\' cem stroju sa $M$ tipki.
    Mo\v zemo zamisliti da je to niz nezavisnih pokusa (nasumice) u kojima majmun poga\dj a pojedinu tipku sa vjerojatno\v s\' cu $\frac{1}{M}$.
    Zamislimo li da majmu tipka unedogled, koja je vjerojatnost da \' ce majmun u nekom "intervalu" natipkati "Dundo Maroje"?
    Uo\v cimo li da je "D.M." specifi\v can niz simbola duljine $T$, pa mo\v zemo niz pokusa promatrati kao niz simbolja duljine $T$.
    Po korolaru \ref{kor:7.9} to su opet nezavisni slu\v cajni elementi i vjerojatnsot da u jednoj sekvenci majmun pogodi "D.M." je $p = \frac{1}{M^T}$.
    Iako je $p$ izrazito mali, va\v zno je da je $p > 0$.
    Ozna\v cimo li sekvencu sa $X_n$ dobivamo
    \begin{equation*}
        \begin{aligned}
            \vjeroj{\textnormal{bar jedna sekvenca "D.M."}}
            &= \vjeroj{\unija{n \in \nat}{} (X_n = \textnormal{D.M.})}\\
            &= \vjeroj{\unija{n \in \nat}{} (X_1 \neq \textnormal{D.M.}), \ldots, X_{n-1} \neq \textnormal{D.M.}, \: X_n = \textnormal{D.M.})}\\
            &= \textnormal{(disjunktna)}\\
            &= \suma{n \in \nat}{} (1 - p)^{n - 1} = p + (1 - p) \: p + (1 - p)^2 \: p + \ldots\\
            &= \Big( \suma{n = 1}{\infty} \Big) \cdot p = \frac{1}{1 - (1 - p)} \cdot p\\
            &= 1.
        \end{aligned}
    \end{equation*}
\end{pr}

\begin{lm}[Borel - Cantelli]  \label{lm:9.2}
    Neka je $\vjerojatnosniProstor$ vjerojatnosni prostor i $\niz{A_n}{n \in \nat} \subseteq \famF$ niz doga\dj aja za koje vrijedi $\suma{n = 1}{\infty} \vjeroj{A_n} < +\infty$.
    Tada je $\vjeroj{A_n \; i.o.} = \vjeroj{\limsup\limits_{n \to +\infty} A_n} = 0$.
\end{lm}

\begin{proof}
    Po teoremu o monotonoj konvergenciji
    $+\infty > \suma{n = 1}{\infty} \vjeroj{A_n} = \suma{n = 1}{\infty} \masE [ \karaktFja_{A_n} ] = \masE [\suma{n = 1}{\infty} \karaktFja_{A_n}] \implies (\exists C \in \famF)$ takav da je $\vjeroj{C} = 1$ i da je $\suma{n = 1}{\infty} \karaktFja_{A_n}(\omega) < +\infty$, $\forall \omega \in C$
    \begin{equation*}
        \implies \vjeroj{A_n \; i.o.} \leq \vjeroj{C^c} = 0. 
    \end{equation*}
\end{proof}

\begin{tm}[Borelov zakon 0 - 1]   \label{tm:9.3}
    Neka je $\vjerojatnosniProstor$ vjerojatnosni prostor i neka je $\niz{A_n}{n \in \nat} \subseteq \famF$ niz nezavisnih doga\dj aja.
    Tada vrijedi:
    \begin{equation*}
        \begin{aligned}
            \vjeroj{A_n \; i.o.} = 0 &\iff \suma{n = 1}{\infty} \vjeroj{A_n} < +\infty\\
            \vjeroj{A_n \; i.o.} = 1 &\iff \suma{n = 1}{\infty} \vjeroj{A_n} = +\infty.
        \end{aligned}
    \end{equation*}
\end{tm}

\begin{proof}
    Po lemi \ref{lm:9.2} dovoljno je dokazati ($\Sigma = + \infty \implies \vjeroj{\cdot} = 1$).
    Koristimo $1 - x \leq e^{-x}$ (deriviraj $h(x) = e^{-x} + x - 1$).
    $\vjeroj{\unija{k \geq n}{} A_k} = 1 - \vjeroj{\presjek{k \geq n}{} A_k^c} = $ zadatak \ref{zad:6.14} $= \produkt{k \geq n}{} (1 - \vjeroj{A_k}) \geq 1 - \produkt{k \geq n}{} e^{-\suma{k \geq n}{} \vjeroj{A_k}} = 1 \implies \vjeroj{A_n \; i.o.} = \vjeroj{\presjek{n \in \nat}{} \unija{k \geq n}{} A_k} = \lim\limits_{n \to \infty} \vjeroj{\unija{k \geq n}{} A_k} = 1$.
\end{proof}

\begin{nap} \label{nap:9.4}
    U primjeru \ref{pr:9.1}. $A_n = \{ X_n = \textnormal{D.M.} \}$ su nezavisni i $\vjeroj{A_n} = p = \frac{1}{M^T} > 0 \implies \sum \vjeroj{A_n} = +\infty \implies \vjeroj{A_n \; i.o.} = 1$ (!).
    Pa ipak prosje\v cno vrijeme (ili prvi $n$) da se dogodi $A_n$ je $\frac{1}{p} = M^T = 10^{10^7}$ (ako treba 1 sekunda po znaku, prije \' ce se sunce ohladiti).
\end{nap}

\begin{defn}    \label{defn:9.5}
    Ka\v zemo da je $\sigma$-algebra na vjerojatnosnom prostoru $\vjerojatnosniProstor$ \emph{trivijalna} ako vrijedi
    \begin{equation*}
        A \in \famG \implies \vjeroj{A} \in \{0, \; 1\}
    \end{equation*}
\end{defn}

Uo\v cimo da je $\famG$ trivijalna ako i samo ako je nezavisna sa samom sobom.
Dodavanje ili oduzimanje proizvoljnog broja trivijalnih $\sigma$-algebri ne\' ce mjenjati status nezavisnosti te familije.

\begin{lm}  \label{lm:9.6}
    Neka je $\vjerojatnosniProstor$ vjerojatnosni prostor, $\famG \subseteq \famF$ $\sigma$-algebra, $E$ potpun, separabilan metri\v cki prostor i $X : \Omega \to E$ $\famG$-izmjeriv slu\v cajni element.
    Ako je $\famG$ trivijalna, tada je $X$ degenerirana.
\end{lm}

\begin{proof}
    Za svaki $n \in \nat$, $E = \unija{j \in \nat}{} B_j^n$, pri \v cemu je ovo disjunktna unija Borelovih skupova i $\diam{B_j^n} < \frac{1}{n}$, za svaki $j \in \nat$.
    Tada su $\{ X \in B_j^n \}$ u $\sigma$-algebri $\famG$, pa su im vjerojatnosti $0$ ili $1$.
    $1 = \vjeroj{\Omega} = \suma{j \in \nat}{} = \vjeroj{X \in B_{j(n)^n}} = 1 \implies \vjeroj{X \in \presjek{n \in \nat}{} B_{j(n)}^n} = 1 \implies \presjek{n \in \nat}{} B_{j(n)}^n \neq \varnothing$.
    S druge strane $\diam{\presjek{n \in \nat}{}B_{j(n)}^n} = 0 \implies (\exists x \in E) \; \presjek{n \in \nat}{} B_{j(n)}^n = \{ x \}$.
\end{proof}

\begin{defn}    \label{defn:9.7}
    Neka je $\indFamilija{\famF_n}{n \in \nat}$ niz $\sigma$-algebri na nepraznom skupu $\Omega$.
    Definiramo $\sigma$-algebru $\famT$ sa
    \begin{equation*}
        \famT := \presjek{n \in \nat}{} \indSigAlg{\famF_k}{k > n},
    \end{equation*}
    nazivamo \emph{repnom $\sigma$-algebrom} (u odnosu na $\{ \famF_n \}$).
    Funkciju $X : \Omega \to \urePar{E}{\famE}$ koja je $\urePar{\famT}{\famE}$-izmjeriva nazivamo \emph{repnom funkcijom}.
    Ako je $\vjerojatnosniProstor$ vjerojatnosni prostor i $\famF_n \subseteq \famF$ za svaki $n \in \nat$, elemente $\sigma$-algebre $\famT$ nazivamo \emph{repnim doga\dj ajima}.
\end{defn}

\begin{pr}  \label{pr:9.8}
    Neka je $\niz{X_n}{n \in \nat}$ niz slu\v cajnih varijabli na vjerojatnosnom prostoru $\vjerojatnosniProstor$ i neka je $\famF_n = \sigAlg{X_n}$ za svaki $n \in \nat$.
    Tipi\v cni repni doga\dj aju su $(a \in \extReal)$ $\skup{\omega \in \Omega}{(X_n(\omega))_{n \in \nat} \; \textnormal{ konvergira}}$, $\skup{\omega \in \Omega}{\lim\limits_{n \to \infty} X_n(\omega) = a}$, $\skup{\omega \in \Omega}{\suma{n = 1}{\infty} X_n(\omega) \; \textnormal{ konvergira}}$, dok $\skup{\omega}{\suma{n = 1}{\infty} X_n (\omega) = a}$ nije repni doga\dj aj.
\end{pr}