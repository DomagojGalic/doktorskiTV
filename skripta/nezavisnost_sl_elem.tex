% nezavisnost slučajnih elemenata

\chapter{Nezavisnost i slu\v cajni elementi}

I u ovom poglavlju $\vjerojatnosniProstor$ uvijek ozna\v cava proizvoljan vjerojatnosni prostor.

\begin{defn}   \label{defn:7.1}
    Neka je $\Lambda \neq \varnothing$ indeksni skup, $\indFamilija{\urePar{E_\lambda}{\famE_\lambda}}{\lambda \in \Lambda}$ familija izmjerivih prostora i $\niz{X_\lambda : \Omega \to E_\lambda}{\lambda \in \Lambda}$ familija slu\v cajnih elemenata.
    Familija $\niz{X_\lambda}{\lambda \in \Lambda}$ je \emph{nezavisna} (ili nepreciznije \emph{slu\v cajni elementi} $X_\lambda$ \emph{su nezavisni}) ako su $\sigma$-algebre $\sigAlg{X_\lambda}$, $\lambda \in \Lambda$ nezavisne.
\end{defn}

O\v cito, to je ekvivalentno svojstvu:
\begin{equation}    \label{jed:7.2}
    \begin{gathered}
        (\forall n \in \nat) (\forall \lambda_1, \ldots, \lambda_n \in \Lambda, \; \textnormal{ me\dj usobno razli\v citi}) (\forall F_1 \in \famE_1, \ldots, F_n \in \famE_n)\\ 
        \masP \Big( \presjek{i = 1}{n} \{ X_{\lambda_i} \in F_i \} \Big) = \produkt{i = 1}{n} \vjeroj{X_{\lambda_i} \in F_i}.
    \end{gathered}
\end{equation}

Nadalje, definicija \ref{defn:7.1} (odnosno \eqref{jed:7.2}) ekvivalentno je sa:
\begin{equation}    \label{jed:7.3}
    \begin{gathered}
        (\forall n \in \nat) (\forall \lambda_1, \ldots, \lambda_n \in \Lambda, \textnormal{ me\dj usobno razli\v cite})\\
        X_{\lambda_1}, \ldots, X_{\lambda_n} \textnormal{ su nezavisne}.
    \end{gathered}
\end{equation}

Tada postaje jasno da je vrlo korisno imati operativnu i preciznu karakterizaciju nezavisnosti kona\v cne familije slu\v cajnih elemenata.

\begin{tm}  \label{tm:7.4}
    Neka je $n \in \nat$, $\urePar{E_1}{\famE_1}, \ldots, \urePar{E_n}{\famE_n}$ izmjerivi prostori i
    \begin{equation*}
        (X_1 : \Omega \to E_1), \ldots, (X_n : \Omega \to E_n),    
    \end{equation*}
    slu\v cajni elementi.
    Tada su $X_1, \ldots, X_n$ nezavisni ako i samo ako je
    \begin{equation*}
        \masP_{\nvektor{X}} = \masP_{X_1} \otimes \ldots \otimes \masP_{X_n}.
    \end{equation*}
\end{tm}

\begin{proof}
    Uo\v cimo da tvrdnja ima smisla.
    
    Po propoziciji \ref{prop:4.9} slijedi da je $\nvektor{X}$ slu\v cajni element na $E_1 \times \ldots \times E_n$, to jest obje vjerojatnosti su definirane na $\famE_1 \otimes \ldots \otimes \famE_n$ koje su generirane $\pi$-sistemom cilindara oblika $A_1 \times \ldots \times A_n$, $A_1 \in \famE_1, \ldots, A_n \in \famE_n$.
    Stoga su navedene vjerojatnosti jednake ako i samo ako su jednake na cilindrima.
    Budu\' ci da je
    \begin{equation*}
        \masP_{\nvektor{X}}(A_1 \times \ldots \times A_n) = \masP \Big( \presjek{i = 1}{n} \{ X_i \in A_i \} \Big)
    \end{equation*}
    i da vrijedi
    \begin{equation*}
        \masP_{X_1} \otimes \ldots \otimes \masP_{X_n} (A_1 \times \ldots \times A_n) = \produkt{i = 1}{n} \vjeroj{X_i \in A_i}
    \end{equation*}
    stoga tvrdnja slijedi.
\end{proof}

Uzmemo li poseban slu\v caj $E_1 = \ldots = E_n = \real$ i iskoristimo li napomenu \ref{nap:5.8} \ref{nap:5.8a} i zadatak \ref{zad:5.11}, iz teorema \ref{tm:7.4} dobivamo:

\begin{kor} \label{kor:7.5}
    Slu\v cajne varijable $X_1, \ldots, X_n$ su nezavisne ako i samo ako, za svaki $x_1, \ldots, x_n \in \real$ vrijedi
    \begin{equation*}
        F_{\nvektor{X}}(x_1, \ldots, x_n) = F_{X_1} (x_1) \cdot \ldots \cdot F_{X_n} (x_n).
    \end{equation*}
\end{kor}

Primjenimo li teorem \ref{tm:7.4} na prebrojiv skup oblika $A \times \ldots \times A$, dobijemo jednostavan kriterij nezavisnosti diskretne za slu\v cajne varijable:

\begin{kor} \label{kor:7.6}
    Neka su $X_1, \ldots, X_n$, slu\v cajne varijable za koje postoji prebrojiv skup $A \subseteq \real$ takav da je $X_i \in A \; (g.s.)$, za svaki $i = 1, \ldots, n$.
    Tada su $X_1, \ldots, X_n$ nezavisne ako i samo ako za svaki izbor $\nBezZagVekt{a} \in A$, vrijedi
    \begin{equation*}
        \masP (X_1 = a_1, \ldots, X_n = a_n) = \vjeroj{X_1 = a_1} \cdot \ldots \cdot \vjeroj{X_n = a_n}.
    \end{equation*}
\end{kor}

Podsjetimo se da za $X, \; Y \in L^1$, $X \cdot Y$ ne mora biti u $L^1$. Ako su pak $X, \; Y \in L^2$, onda je $X \cdot Y \in L^1$, ali $\masE [X \cdot Y]$ ne mora biti jednako $\masE X \cdot \masE Y$.

\begin{tm}  \label{tm:7.7}
    Ako su $X, \; Y \in L^1 \vjerojatnosniProstor$ nezavisne slu\v cajne varijable, tada je $X \cdot Y \in L^1(\masP)$ i vrijedi
    \begin{equation*}
        \masE [X \cdot Y] = \masE X \cdot \masE Y.
    \end{equation*}
\end{tm}

\begin{proof}
    Uo\v cimo da je $Z:=\urePar{X}{Y}$ dvodimenzionalan slu\v cajan vektor i po teoremu \ref{tm:7.4} je $\masP_Z = \masP_X \otimes \masP_Y$.
    Po zadatku \ref{zad:4.15}
    \begin{align*}
        \int\limits_\Omega |X(\omega)| \cdot |Y(\omega)| \: d \masP
        &= \int\limits_{\real^2} |x| \cdot |y| \: d \masP_Z \urePar{x}{y} = \int\limits_\real |x| \: d \masP_X (x) \cdot \int\limits_{\real} |y| \: d \masP_Y (y)\\
        &= \masE [|X|] \cdot \masE [|Y|] < +\infty \implies X \cdot Y \in L^1.
    \end{align*}
    Sada na sli\v can na\v cin Fubinijev teorem (zadatak \ref{zad:4.16} i jednad\v zba \ref{jed:5.11-3-1}) daje
    \begin{equation*}
        \masE [X \cdot Y] = \masE X \cdot \masE Y.
    \end{equation*}
\end{proof}

\begin{zad} \label{zad:7.8}
    Ako su $X, \; Y \in L^2 \vjerojatnosniProstor$ nezavisne, tada je
    \begin{equation*}
        \Var (X + Y) = \Var X + \Var Y.
    \end{equation*}
\end{zad}

\begin{rj}[\ref{zad:7.8}]
    Doka\v zimo najprije ne\v sto generalniji rezultat.

    Neka su $X_1, \ldots, X_n \in L^2$, tada vrijedi
    \begin{equation*}
        \Var \Big( \suma{k = 1}{n} X_k \Big) = \suma{k = 1}{n} \Var X_k + 2 \cdot \suma{i < j}{} \masE \big[ (X_i - \masE X_i) (X_j - \masE X_j) \big].
    \end{equation*}

    Dokazujemo tvrdnju indukcijom po $n$.
    \begin{enumerate}
        \item[(B)]
        Dokazujemo tvrdnju za $n = 2$.
        \begin{equation*}
            \begin{aligned}
                \Var (X_1 + X_2) &= \masE \big[ (X_1 - X_2)^2 \big] - \big( \masE [ X_1 + X_2 ] \big)^2\\
                &= \masE [X_1^2] + \masE [X_2^2] + 2 \cdot \masE [X_1 \cdot X_2] - (\masE X_1)^2 - (\masE X_2)^2 - 2 \cdot \masE X_1 \cdot \masE X_2\\
                &= \big( \masE [X_1^2] - (\masE X_1)^2 \big) + \big( \masE [ X_2^2 ] - (\masE X_2)^2 \big) + 2 \cdot \big( \masE [ X_1 \cdot X_2 ] - \masE X_1 \cdot \masE X_2 \big)\\
                &= \Var X_1 + \Var X_2 + 2 \cdot \masE \big[ (X_1 - \masE X_1) (X_2 - \masE X_2) \big].
            \end{aligned}
        \end{equation*}
        \item[(K)]
        Primjetimo
        \begin{equation*}
            \Var \Big( \suma{k = 1}{n + 1} X_k \Big) = \Var \Big( \Big( \suma{k = 1}{n} X_k \Big) + X_{n + 1} \Big),
        \end{equation*}
        prema bazi indukcij slijedi
        \begin{equation*}
            \begin{aligned}
                \Var \Big( \suma{k = 1}{n + 1} X_k \Big) &= \Var \Big( \suma{k = 1}{n} X_k \Big) + \Var X_{n + 1} + 2 \cdot \masE \Big[ \Big(  \suma{k = 1}{n} X_k - \suma{k = 1}{n} \masE X_k \Big) (X_{n + 1} - \masE X_{n + 1}) \Big]\\
                &= \suma{k = 1}{n} \Var X_k + 2  \cdot \suma{i < j \leq n}{} \masE \big[ (X_i - \masE X_i) (X_j - \masE X_j) \big] + \Var X_{n + 1}\\
                &\quad \quad + 2 \cdot \masE \Big[ \Big( \suma{k = 1}{n} (X_k - \masE X_k) \Big) (X_{n + 1} - \masE X_{n + 1}) \Big]\\
                &= \suma{k = 1}{n + 1} \Var X_k + 2 \cdot \masE \Big[ \Big(  \suma{k = 1}{n} X_k - \suma{k = 1}{n} \masE X_k \Big) (X_{n + 1} - \masE X_{n + 1}) \Big]\\
                &\quad \quad + 2 \cdot \suma{k = 1}{n} \masE \big[ (X_k - \masE X_k) (X_{n + 1} - \masE X_{n + 1}) \big]\\
                &= \suma{k = 1}{n + 1} \Var X_k + 2 \cdot \suma{i < j \leq n + 1}{} \masE \big[ (X_i - \masE X_i) (X_j - \masE X_j) \big]
            \end{aligned}
        \end{equation*}
    \end{enumerate}
    Sada po teoremu \ref{tm:7.7} imamo da za nezavisne $X_1, \ldots, X_n$, te za $i \neq j$ vrijedi
    \begin{equation*}
        \masE \big[ (X_i - \masE X_i) (X_j - \masE X_j) \big] = \masE X_i \cdot \masE X_j - \masE X_i \cdot \masE X_j = 0.
    \end{equation*}
    Dakle za nezavisne slu\v cajne varijable dobivamo izraz
    \begin{equation*}
        \Var \big( \suma{k = 1}{n} X_k \Big) = \suma{k = 1}{n} \Var X_k,
    \end{equation*}
    odnosno u slu\v caju kada je $n = 2$ imamo
    \begin{equation*}
        \Var (X + Y) = \Var X + \Var Y.
    \end{equation*}
\end{rj}

Uo\v cimo da sada direktno iz korolara \ref{kor:6.11} direktno dobivamo:

\begin{kor} \label{kor:7.9}
    Neka je $\Lambda \neq  \varnothing$ indeksni skup, $\indFamilija{\urePar{E_\lambda}{\famE_\lambda}}{\lambda \in \Lambda}$ familija izmjerivih prostora i $\niz{X_\lambda : \Omega \to E_\lambda}{\lambda \in \Lambda}$ familija nezavisnih slu\v cajnih elemenata.
    Neka je $\famT$ particija od $\Lambda$ i neka je za svaki $T \in \famT$ zadan izmjeriv prostor $\urePar{H_T}{\famH_T}$ i izmjerivo preslikavanje
    \begin{equation*}
        f_T : \bigUrePar{\produkt{t \in T}{} E_t}{\dirProd{t \in T}{} \famE_t} \to \urePar{H_T}{\famH_T}.
    \end{equation*}
    Tada je
    \begin{equation*}
        \bigNiz{f_T \circ \niz{X_\lambda}{\lambda \in T}}{T \in \famT}    
    \end{equation*}
    nezavisna familija slu\v cajnih elemenata.
\end{kor}

%% koriste\' ci dokaz sli\v can onome iz \ref{tm:7.7} rje\v site.
\begin{zad} \label{zad:7.10}
    Neka su $X : \Omega \to \urePar{E}{\famE}$ i $Y : \Omega \to \urePar{H}{\famH}$ nezavisni slu\v cajni elementi.
    Neka je $f : \urePar{E \times H}{\famE \otimes \famH} \to \urePar{\real}{\borel{\real}}$ izmjeriva za koju vrijedi jedan od sljede\' cih uvjeta:
    \begin{itemize}
        \item $f \geq 0$
        \item $\masE [|f \urePar{X}{Y}|] < +\infty$.
    \end{itemize}
    Tada je
    \begin{align*}
        \masE [ f \urePar{X}{Y} ]
        &= \int\limits_{E} \int\limits_{H} f \urePar{x}{y} \: d \masP_Y (y) \: d \masP_X (x) \\
        &= \int\limits_H \int\limits_E f \urePar{x}{y} \: d \masP_X (x) \: d \masP_Y (y).
    \end{align*}
\end{zad}

\begin{defn}    \label{defn:7.10-1}
    Slu\v cajni element $X$ sa vrijednostima u $\urePar{E}{\famE}$ je \emph{degeneriran} ako postoji $a \in E$, takav da je
    \begin{equation*}
        X = a \; (g.s.).
    \end{equation*}
\end{defn}

Primjetimo u prethodnoj definiciji \emph{nije nu\v zno}
\begin{itemize}
    \item[] $\{a\} \in \famE$,
    \item[] $\{X = a\} \in \famF$.
\end{itemize}

\begin{zad} \label{zad:7.11}
    Ako je $X$ degenerirani slu\v cajni element, a $Y$ bilo koji slu\v cajni element (ne nu\v zno s vrijednostima u istom prostoru) tada su $X$ i $Y$ nezavisni.
    Ako su $X$ i $Y$ nezavisni slu\v cajni elementi takvi da je $X + Y$ degenerirana, tada su $X$ i $Y$ degenerirane.
\end{zad}

%   djelomićno rješenje
\begin{rj}[\ref{zad:7.11}]
    Zadatak \' cemo rje\v siti kroz tri leme, kroz koje dokazujemo da ako je $X$ degenerirana slu\v cajna varijabla, tada je $\sigAlg{X}$ degenerirana $\sigma$-algebra, stoga je ona nezavisan sa svakom drugom $\sigma$-algebrom.

    \begin{lm}   \label{lm:7.11-1-1}
        Neka su $A, B \in \famF$ takvi da je $\masP (B) = 1$, tada vrijedi
        \begin{equation*}
            \masP (A \cap B) = \masP (A).
        \end{equation*}
    \end{lm}

    \begin{proof}{\ref{lm:7.11-1-1}}
        Primjetimo da se skup $A$ mo\v ze prikazati kao disjunktna unija
        \begin{equation*}
            A = (A \cap B) \cup (A \setminus B).
        \end{equation*}
        Sada vidimo da vrijedi
        \begin{equation*}
            \masP (A) = \masP (A \cap B) + \masP (A \setminus B),
        \end{equation*}
        primjetimo da vrijedi
        \begin{equation*}
            A \ setminus B = A \cap B^c \subseteq B^c.
        \end{equation*}
        Sada po monotonosti vjerojatnosti, te prema vjerojatnosti suprotnog doga\dj aja imamo
        \begin{equation*}
            \masP (A \setminus B) \leq \masP (B^c) = 1 - \masP (B) = 0,
        \end{equation*}
        odakle sada slijedi
        \begin{equation*}
            \masP (A) = \masP (A \cap B).
        \end{equation*}
    \end{proof}

    \begin{lm}  \label{lm:7.11-1-2}
        Neka su $\famG, \famH \subseteq \famF$ $\sigma$-algebre za koje vrijedi
        \begin{equation*}
            \forall B \in \famG \quad \masP (B) \in \{0, 1 \},
        \end{equation*}
        tada su $\famG$ i $\famH$ nezavisne.
    \end{lm}

    \begin{proof}{\ref{lm:7.11-1-2}}
        Neka su $A \in \famH$, $B \in \famG$ proizvoljne.
        Budu\' ci je $B \in \famG$ imamo dva slu\' caja
        \begin{enumerate}[label=($\arabic*^\circ$)]
            \item Pretpostavimo da vrijedi $\masP (B) = 1$, sada prema lemi \ref{lm:7.11-1-1} imamo
            \begin{equation*}
                \masP (A \cap B) = \masP (A) = 1 \cdot \masP (A) = \masP (A) \cdot \masP (B).
            \end{equation*}
            \item Pretpostavimo da je $\masP (B) = 0$, sada zbog $A \cap B \subseteq B$ imamo
            \begin{equation*}
                \masP (A \cap B) \leq \masP (B) = 0 = \masP (A) \cdot 0 = \masP (A) \cdot \masP (B).
            \end{equation*}
        \end{enumerate}
        Budu\' ci su $A, B$ bili proizvoljni, $\famG$ i $\famH$ su nezavisne.
    \end{proof}

    \begin{lm}  \label{lm:7.11-3}
        Ako je slu\v cajna varijabla $Y$ degenerirana, tada je $\sigAlg{Y}$ degenerirana $\sigma$-algebra.
    \end{lm}

    \begin{proof}{\ref{lm:7.11-3}}
        Promatrajmo $A \in \sigAlg{Y}$, tada postoji $B \in \famE$ takav da vrijedi
        \begin{equation*}
            \{Y \in B\} = A.
        \end{equation*}
        Promatrajmo dva slu\v caja
        \begin{enumerate}[label=$(\arabic*^\circ)$]
            \item Neka je $a \in B$, sada imamo
            \begin{equation*}
                \{Y = a\} \subseteq \{Y \in B\} = A \quad \implies \quad \masP (A) = 1.
            \end{equation*}
            \item Alternativno, neka je $a \notin B$, sada imamo
            \begin{equation*}
                A = \{ Y \in B\} \subseteq \big\{ Y \in \{a\}^c \big\} = \{ Y \neq a \},
            \end{equation*}
            odakle imamo
            \begin{equation*}
                \masP (A) \leq \masP (Y \neq a) = 1 - \masP (Y = a) = 0.
            \end{equation*}
        \end{enumerate}
    \end{proof}

    Sada primjetimo da, prema lemi \ref{lm:7.11-3}, $\sigAlg{X}$ je trivijalna $\sigma$-algebra.
    Prema lemi \ref{lm:7.11-1-2} slijedi da su $\sigAlg{X}$ te $\sigAlg{Y}$ nezavisne,
    za proizvoljan $Y$, stoga su $X$ i $Y$ nezavisne kao slu\v cajne varijable.
\end{rj}

\begin{defn}    \label{defn:7.11-1}
    Grupa $\urePar{G}{\cdot}$ sa $\sigma$-algebrom $\famG$ je \emph{izmjeriva grupa} ako je grupovna operacija $\cdot : G \times G \to G$ izmjeriva u paru $\urePar{\famG \otimes \famG}{\famG}$.
\end{defn}

\begin{defn}    \label{defn:7.11-2}
    Neka su $\mu$ i $\nu$ $\sigma$-kon\v cne mjere na $\urePar{G}{\famG}$, te neka je $\mu \otimes \nu$ ($\sigma$-kona\v cna) produktna mjera na $\urePar{G \times G}{\famG \otimes \famG}$, pa mo\v zemo promatrati mjeru na $\urePar{G}{\famG}$ induciranu preslikavanjem $\cdot$ u odnosu na $\mu \otimes \nu$.
    
    Tu mjeru nazivamo \emph{konvolucijom} mjera $\mu$ i $\nu$ te ozna\v cavamo sa
    \begin{equation*}
        \mu * \nu.
    \end{equation*}
\end{defn}

Po zadatku \ref{zad:4.15} dobijemo da je za svaki $H \in \famG$
\begin{equation*}   \label{jed:7.12}
    (\mu * \nu)(H) := \int\limits_G \nu (a^{-1} \: H) \: d \mu (a).
\end{equation*}

\begin{nap} \label{nap:7.12-1}
    Definirajmo funkciju $f : G^2 \to G$, sa
    \begin{equation*}
        (a, b) \xmapsto{f} a \cdot b
    \end{equation*}
    i neka je $H \in \famG$.

    Primjetimo
    \begin{equation*}
        \begin{aligned}
            \praslika{f} (H) = \{f \in H \} &= \indFamilija{(a, b) \in G^2}{a \cdot b \in H}\\
            &= \indFamilija{(a, b) \in G^2}{b \in a^{-1} H}.
        \end{aligned}
    \end{equation*}
    Dakle vrijedi
    \begin{equation*}
        (a, b) \in \praslika{f} (H) \quad \iff \quad b \in a^{-1} H,
    \end{equation*}
    odnosno
    \begin{equation}    \label{jed:7.12-2}
        \karaktFja_{ \{ f \in H \} } (a, b) = \karaktFja_{a^{-1} H} (b).
    \end{equation}
    Sada imamo
    \begin{equation*}
        \begin{aligned}
            (\mu * \nu) (H) :&= (\mu \otimes \nu) (\{ f \in H \})\\
            &= \int\limits_{G^2} \karaktFja_{\{ f \in H \}} (a, b) \: d (\mu \otimes \nu) (a, b)\\
            &= \int\limits_{G^2} \karaktFja_{ a^{-1} H } (b) \: d (\mu \otimes \nu) (a, b)\\
            &= \int\limits_G \Big( \int\limits_G \karaktFja_{a^{-1} H} (b) \: d \nu (b) \Big) \: d \mu (a)\\
            &= \int\limits_G \nu (a^{-1} H) \: d \mu (a). 
        \end{aligned}
    \end{equation*}
\end{nap}

\begin{zad} \label{zad:7.13}
    Ako je $\urePar{G}{\cdot}$ Abelova grupa, tada je konvolucija komutativana operacija.
\end{zad}

\begin{rj}[\ref{zad:7.13}]
    Budu\' ci je $G$ Abelova vrijedi
    \begin{equation*}
        \forall a, b \in G \quad \quad a \cdot b = b \cdot a.
    \end{equation*}
    Sada, za $H \subseteq G$, imamo
    \begin{equation*}
        \begin{aligned}
            \bigSkup{(a, b) \in G^2}{b \in a^{-1} H} &= \bigSkup{(a, b) \in G^2}{a \cdot b \in H} = \bigSkup{(a, b) \in G^2}{b \cdot a \in H}\\
            &= \bigSkup{(a, b) \in G^2}{a \in b^{-1} H}.
        \end{aligned}
    \end{equation*}
    Odnosno vrijedi
    \begin{equation}    \label{jed:7.13-1}
        \karaktFja_{a^{-1} H} (b) = \karaktFja_{b^{-1} H} (a).
    \end{equation}
    Dakle, za $H \subseteq G$ imamo
    \begin{equation*}
        \begin{aligned}
            (\mu * \nu) (H) &= \int\limits_G \nu (a^-1 H) \: d \mu (a) = \int\limits_G \Big( \int\limits_G \karaktFja_{a^{-1} H} (b) \: d \nu (b) \Big) \: d \mu (a) =
                \begin{psmallmatrix}
                    \textnormal{Fubini}
                \end{psmallmatrix}\\
            &= \int\limits_{G^2} \karaktFja_{a^-1} (b) \: d ( \mu \otimes \nu ) (a, b) =
                \begin{psmallmatrix}
                    \ref{jed:7.13-1}
                \end{psmallmatrix}
            = \int\limits_{G^2} \karaktFja_{b^{-1} H} (a) \: d (\mu \otimes \nu) (a, b) =
                \begin{psmallmatrix}
                    \textnormal{Fubini}
                \end{psmallmatrix}\\
            &= \int\limits_G \Big( \int\limits_G \karaktFja_{b^{-1} H} (a) \: d \mu (a) \Big) \: d \nu (b) = \int\limits_G \mu (b^{-1} H) \: d \nu (b)\\
            &= (\nu * \mu) (H).
        \end{aligned}
    \end{equation*}
\end{rj}

\begin{nap} \label{nap:7.14}
    \begin{enumerate}[label=(\alph*)]
        \item Op\' cenito govore\' ci mjera $\mu * \nu$ ne mora biti $\sigma$-kona\v cna.
        Ako su $\mu$ i $\nu$ kona\v cne mjere, tada je i $\mu * \nu$ kona\v cna i
        \begin{equation*}
            (\mu * \nu) (G) = \mu (G) \: \nu (G),
        \end{equation*}
        \v sto slijedi iz \eqref{jed:7.12}.
        \item Ako su $\mu, \; \nu, \; \eta$ $\sigma$-kona\v cne mjere na $\urePar{G}{\famG}$, takve da su $\mu * \nu$ i $\nu * \eta$ $\sigma$-kona\v cne, tada vrijedi
        \begin{equation*}
            (\mu * \nu) * \eta = \mu * (\nu * \eta).
        \end{equation*}
        \item   \label{nap:7.14c}
        Ako na $\urePar{G}{\famG}$ postoji $\sigma$-kona\v cna mjera $\lambda$ takva da je $\mu \ll \lambda$ i $\nu \ll \lambda$, tada za $f = \frac{d \mu}{d \lambda}$ i $g = \frac{d \nu}{d \lambda}$ vrijedi
        \begin{equation*}
            (\mu * \nu) (H) = \int\limits_G \int\limits_{a^{-1} H} f(a) \: g(b) \: d \lambda (b) \: d \lambda (a).
        \end{equation*}
        \item Ako je $\lambda$ iz \ref{nap:7.14c} jo\v s i \emph{invarijantna} u odnosu na sve lijeve translacije (to jest za svaki $g \in G$ i za svaki $C \in \famG$ je $\lambda (g^{-1} C) = \lambda (C)$), tada je
        \begin{equation*}
            (\mu * \nu) (H) = \int\limits_H \Big( \int\limits_G f(a) \: g (a^{-1} b) \: d \lambda (a) \Big) \: d \lambda (b),
        \end{equation*}
        to jest $\mu * \nu \ll \lambda$ i postoji gusto\' ca
        \begin{equation*}
            \frac{d (\mu * \nu)}{d \lambda} (b) = \int\limits_G f(a) \: g (a^{-1} b) \: d \lambda (a)
        \end{equation*}
        koju ozna\v cavamo sa \emph{$f*g$}.
    \end{enumerate}
\end{nap}

\begin{tm}  \label{tm:7.15}
    Ako su $X$ i $Y$ nezavisni slu\v cajni elementi s vjerojatnostima u izmjerivoj grupi $\urePar{G}{\cdot}$, tada je
    \begin{equation*}
        \masP_{X \cdot Y} = \masP_X * \masP_Y.
    \end{equation*}
\end{tm}

\begin{proof}
    Uo\v cimo da je $Z:= \urePar{X}{Y}$, su\v cajni element s vrijednostima u $G \times G$ i da je $\masP_Z = \masP_X \otimes \masP_Y$.
    Budu\' ci da je $\cdot$ izmjerivo, slijedi da je $X \cdot Y$ slu\v cajni element s vrijednostima u $G$ i $\masP_{X \cdot Y}$ je vjerojatnost inducirana s $\cdot$ u odnosu na $\masP_Z$.
    To je po definiciji upravo
    $\masP_X * \masP_Y$.
\end{proof}

\begin{zad} \label{zad:7.16}
    Primjenite ove rezultate na Abelovu grupu $\urePar{\real^d}{+}$ i posebno obratite pa\v znju na slu\v caj $d = 1$.
    Za slu\v caj $d = 1$izvedite pravila za konvluciju d.f.
\end{zad}

\begin{rj}[\ref{zad:7.16}]
    Neka su $F$ i $G$ d.f. na $\real$.
    Pomo\' cu njih definiramo mjere sa
    \begin{equation*}
        \begin{aligned}
            \mu (\lijInt{a}{b}) &:= F(b) - F(a)\\
            \nu (\lijInt{a}{b}) &:= G(b) - G(a).
        \end{aligned}
    \end{equation*}
    Sada imamo
    \begin{equation*}
        \begin{aligned}
            (\mu * \nu) (\lijInt{a}{b}) = \int\limits_\real \nu (\lijInt{a}{b} - x) \: d \mu(x) = \int\limits_\real \nu (\lijInt{a - x}{b - x}) \: d  \mu (x)\\
            &= \int\limits_\real \big( G(b - x) - G(a - x) \big) \: d F (x).
        \end{aligned}
    \end{equation*}
    Ako je $\mu \ll \lambda$ te $\nu \ll \lambda$, tada stavimo
    \begin{equation*}
        \begin{aligned}
            \frac{d \mu}{d \lambda} &= f\\
            \frac{d \nu}{d \lambda} &= g.
        \end{aligned}
    \end{equation*}
    Sada vrijedi
    \begin{equation*}
        (\mu * \nu) (\lijInt{a}{b}) = \int\limits_a^b \Big( \int\limits_\real f(x) \cdot g (y - x) \: d \lambda (x) \Big) \: d \lambda (y),
    \end{equation*}
    odnosno
    \begin{equation*}
        (f * g) (y) = \int\limits_\real f(t) \cdot g(y - t) \: d \lambda (t).
    \end{equation*}
    Tako\dj er, ako imamo
    \begin{equation*}
        \begin{aligned}
            F (x) &= \masP (X \leq x)\\
            G (y) &= \masP (Y \leq y),
        \end{aligned}
    \end{equation*}
    tada vrijedi
    \begin{equation*}
        (F * G) (z) = \masP (X + Y \leq z) = \int\limits_\real F(z - y) \: d G(y).
    \end{equation*}
\end{rj}

Promotrimo sada nezavisne familije slu\v cajnih elemenata.
Po\v cnimo od kona\v cnog slu\v caja.
Uo\v cimo prvo da ako su $\nBezZagVekt{X}, \; n \in \nat$, \emph{nezavisni} slu\v cajni elementi s vrijednostima u (respektivno) $\urePar{E_1}{\famE_1}, \ldots, \urePar{E_n}{\famE_n}$, tada je distribucija slu\v cajnog elementa $\nvektor{X}$ (s vrijednostima u $\urePar{E_1 \times \ldots \times E_n}{\famE_1 \otimes \ldots \otimes \famE_n}$) u potpunosti odre\dj ena s $\masP_{X_1}, \ldots, \masP_{X_n}$; vidi teorem \ref{tm:7.4}

U statistici i primjenama \v cesto je obrnuta situacija.
Pretpostavimo da su nam zadane distribucije $\nBezZagVekt{\masP}$ na (respektivno) $\urePar{E_1}{\famE_1}, \ldots, \urePar{E_n}{\famE_n}$ i \v zelimo formirati slu\v cajni element $X = \nvektor{X}$ na $E_1 \times \ldots \times E_n$, tako da su $\nBezZagVekt{X}$ \emph{nezavisni} i $\masP_{X_1} = \masP_1, \ldots, \masP_{X_n} = \masP_n$.
Iz dokazanog lako slijedi da se to jednostavno posti\v ze na vjerojatnosnom prostoru \emph{kanonskog oblika}
\begin{equation} \label{jed:7.17}
    \begin{aligned}
        \Omega &:= E_1 \times \ldots \times E_n\\
        \famF &:= \famE_1 \otimes \ldots \otimes \famE_n\\
        \masP &:= \masP_1 \otimes \ldots \otimes \masP_n\\
        X &:= \id : \Omega \to \Omega\\
        X_i &= \pi_i : E_1 \times \ldots \times E_n \to E_i, \quad i = 1, \ldots, n.
    \end{aligned}
\end{equation}
U klasi\v cnoj situaciji, kada su $X_i$ slu\v cajne varijable, ovaj zahtjev mo\v zemo ispuniti i direktnije.

\begin{prop}    \label{prop:7.18}
    Neka su $\nBezZagVekt{F}$ proizvoljne p.d.F. i $\vjerojatnosniProstor$ vjerojatnosni prostor na kojem postoji slu\v cajni vektor $U = \nvektor{U}$, takav da su $\nBezZagVekt{U} $ nezavisne i jednako distribuirane slu\v cajne varijable s uniformnom razdiobom na $\obInt{0}{1}$.
    Tada na $\vjerojatnosniProstor$ postoji slu\v cajni vektor $X = \nvektor{X}$ takav da su $\nBezZagVekt{X}$ nezavisne slu\v cajne varijable i
    \begin{equation*}
        F_{X_1} = F_1, \ldots F_{X_n} = F_n.
    \end{equation*}
\end{prop}

\begin{proof}
    Neka je $h_i : \obInt{0}{1} \to \real, \; i = 1, \ldots n$, zadan sa $h_i (x) := \sup \skup{r \in \real}{F_i (r) < x}$ i neka je
    \begin{equation*}
        X := (h_1 \circ U_1, \ldots, h_n \circ U_n).
    \end{equation*} 
    Tada je
    \begin{equation*}
        \skup{w \in \obInt{0}{1}}{h_i(w) \leq x} = \skup{w \in \obInt{0}{1}}{w \leq F_i (x)}
    \end{equation*}
    pa je $F_{h_i \circ U_i} = F_i$.
    Nezavisnost slijedi iz korolara \ref{kor:7.9}
\end{proof}

\begin{nap} \label{nap:7.19}
    Lako je posti\' ci uvijete iz propozicije \ref{prop:7.18}.
    Neka je
    \begin{equation*}
        \Omega := \segment{0}{1}^n, \; \famF := \borel{\segment{0}{1}^n}, \; \masP := \restr{\lambda^n}{\famF},
    \end{equation*}
    te neka je
    \begin{equation*}
        U_i := \pi_i : \Omega \to \segment{0}{1}, \; i = 1, \ldots, n.
    \end{equation*}
\end{nap}

\begin{zad} \label{zad:7.20}
    Uzmimo $n = 1$ u napomeni \ref{nap:7.19}. Napi\v simo svaki $U(x) = x \in \obInt{0}{1}$ u binarnom prikazi i ozna\v cimo $n$-tu znamenku s $h_n$, $n \in \nat$.
    Doka\v zi da su $\niz{h_n}{n \in \nat}$ nezavisne i jednako distribuirane slu\v cajne varijable i vrijedi
    \begin{equation*}
        h_n \sim
        \begin{pmatrix}
            0& 1\\
            \frac{1}{2}& \frac{1}{2}    
        \end{pmatrix}.
    \end{equation*}
\end{zad}