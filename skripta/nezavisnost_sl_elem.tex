% nezavisnost slučajnih elemenata

\chapter{Nezavisnost i slu\v cajni elementi}

I u ovom poglavlju $\vjerojatnosniProstor$ uvijek ozna\v cava proizvoljan vjerojatnosni prostor.

\begin{defn}   \label{defn:7.1}
    Neka je $\Lambda \neq \varnothing$ indeksni skup, $\indFamilija{\urePar{E_\lambda}{\famE_\lambda}}{\lambda \in \Lambda}$ familija izmjerivih prostora i $\niz{X_\lambda : \Omega \to E_\lambda}{\lambda \in \Lambda}$ familija slu\v cajnih elemenata.
    Familija $\niz{X_\lambda}{\lambda \in \Lambda}$ je \emph{nezavisna} (ili nepreciznije \emph{slu\v cajni elementi} $X_\lambda$ \emph{su nezavisni}) ako su $\sigma$-algebre $\sigAlg{X_\lambda}$, $\lambda \in \Lambda$ nezavisne.
\end{defn}

O\v cito, to je ekvivalentno svojstvu:
\begin{equation}    \label{jed:7.2}
    \begin{gathered}
        (\forall n \in \nat) (\forall \lambda_1, \ldots, \lambda_n \in \Lambda, \; \textnormal{ me\dj usobno razli\v cite}) (\forall F_1 \in \famE_1, \ldots, F_n \in \famE_n)\\ 
        \vjeroj{\presjek{i = 1}{n} \{ X_{\lambda_i} \in F_i \}} = \produkt{i = 1}{n} \vjeroj{X_{\lambda_i} \in F_i}.
    \end{gathered}
\end{equation}

Nadalje, definicija \ref{defn:7.1} (odnosno \eqref{jed:7.2}) ekvivalentno je sa:
\begin{equation}    \label{jed:7.3}
    \begin{gathered}
        (\forall n \in \nat) (\forall \lambda_1, \ldots, \lambda_n \in \Lambda, \textnormal{ me\dj usobno razli\v cite})\\
        X_{\lambda_1}, \ldots, X_{\lambda_n} \textnormal{ su nezavisne}.
    \end{gathered}
\end{equation}

Tada postaje jasno da je vrlo korisno imati operativnu i preciznu karakterizaciju nezavisnosti kona\v cne familije slu\v cajnih elemenata.

\begin{tm}  \label{tm:7.4}
    Neka je $n \in \nat$, $\urePar{E_1}{\famE_1}, \ldots, \urePar{E_n}{\famE_n}$ izmjerivi prostori i $X_1 : \Omega \to E_1, \ldots, X_n \to E_n$, slu\v cajni elementi.
    Tada su $X_1, \ldots, X_n$ nezavisni ako i samo ako je
    \begin{equation*}
        \masP_{\nvektor{X}} = \masP_{X_1} \otimes \ldots \otimes \masP_{X_n}.
    \end{equation*}
\end{tm}

\begin{proof}
    Uo\v cimo da tvrdnja ima smisla, jer po propoziciji \ref{prop:4.9} slijedi da je $\nvektor{X}$ slu\v cajni element na $E_1 \times \ldots \times E_n$, to jest obje vjerojatnosti su definirane na $\famE_1 \otimes \ldots \otimes \famE_n$ koje su generirane $\pi$-sistemom cilindrova oblika $A_1 \times \ldots \times A_n$, $A_1 \in \famE_1, \ldots, A_n \in \famE_n$.
    Stoga su navedene vjerojatnosti jednake ako i samo ako su jednake na cilindrima.
    Budu\' ci da je
    \begin{equation*}
        \masP_{\nvektor{X}}(A_1 \times \ldots \times A_n) = \vjeroj{\presjek{i = 1}{n} \{ X_i \in A_i \}}
    \end{equation*}
    i da vrijedi
    \begin{equation*}
        \masP_{X_1} \otimes \ldots \otimes \masP_{X_n} (A_1 \times \ldots \times A_n) = \produkt{i = 1}{n} \vjeroj{X_i \in A_i}
    \end{equation*}
    stoga tvrdnja slijedi.
\end{proof}

Uzmemo li poseban slu\v caj $E_1 = \ldots = E_n = \real$ i iskoristimo li napomenu \ref{nap:5.8} \ref{nap:5.8a} i zadatak \ref{zad:5.11}, iz teorema \ref{tm:7.4} dobivamo:

\begin{kor} \label{kor:7.5}
    Slu\v cajne varijable $X_1, \ldots, X_n$ su nezavisne ako i samo ako, za svaki $x_1, \ldots, x_n \in \real$ vrijedi
    \begin{equation*}
        F_{\nvektor{X}}(x_1, \ldots, x_n) = F_{X_1} (x_1) \cdot \ldots \cdot F_{X_n} (x_n).
    \end{equation*}
\end{kor}

Primjenimo li teorem \ref{tm:7.4} na prebrojiv skup oblika $A \times \ldots \times A$, dobijemo jednostavan kriterij nezavisnosti diskretne za slu\v cajne varijable:

\begin{kor} \label{kor:7.6}
    Neka su $X_1, \ldots, X_n$, slu\v cajne varijable za koje postoji prebrojiv skup $A \subseteq \real$ takav da je $X_i \in A \; (g.s.)$, za svaki $i = 1, \ldots, n$.
    Tada su $X_1, \ldots, X_n$ nezavisne ako i samo ako za svaki izbor $\nBezZagVekt{a} \in A$, vrijedi
    \begin{equation*}
        \masP (X_1 = a_1, \ldots, X_n = a_n) = \vjeroj{X_1 = a_1} \cdot \ldots \cdot \vjeroj{X_n = a_n}.
    \end{equation*}
\end{kor}

Podsjetimo se da za $X, \; Y \in L^1$, $X \cdot Y$ ne mora biti u $L^1$. Ako su pak $X, \; Y \in L^2$, onda je $X \cdot Y \in L^1$, ali $\masE [X \cdot Y]$ ne mora biti jednako $\masE X \cdot \masE Y$.

\begin{tm}  \label{tm:7.7}
    Ako su $X, \; Y \in L^1 \vjerojatnosniProstor$ nezavisne slu\v cajne varijable, tada je $X \cdot Y \in L^1(\masP)$ i vrijedi
    \begin{equation*}
        \masE [X \cdot Y] = \masE X \cdot \masE Y.
    \end{equation*}
\end{tm}

\begin{proof}
    Uo\v cimo da je $Z:=\urePar{X}{Y}$ dvodimenzionalan slu\v cajan vektor i po teoremu \ref{tm:7.4} je $\masP_Z = \masP_X \otimes \masP_Y$.
    Po zadatku \ref{zad:4.15}
    \begin{align*}
        \int_\Omega |X(\omega)| \cdot |Y(\omega)| \: d \masP
        &= \int_{\real^2} |x| \cdot |y| \: d \masP_Z \urePar{x}{y} = \int_\real |x| \: d \masP_X (x) \cdot \int_{\real} |y| \: d \masP_Y (y)\\
        &= \masE [|X|] \cdot \masE [|Y|] < +\infty \implies X \cdot Y \in L^1.
    \end{align*}
    Sada na slu\v can na\v cin Fubinije teorem (zadatak \ref{zad:4.16}) daje
    \begin{equation*}
        \masE [X \cdot Y] = \masE X \cdot \masE Y.
    \end{equation*}
\end{proof}

\begin{zad} \label{zad:7.8}
    Ako su $X, \; Y \in L^2 \vjerojatnosniProstor$ nezavisne, tada je
    \begin{equation*}
        \Var (X + Y) = \Var X + \Var Y.
    \end{equation*}
\end{zad}

Uo\v cimo da sada direktno iz korolara \ref{kor:6.11} direktno dobivamo:

\begin{kor} \label{kor:7.9}
    Neka je $\Lambda \neq  \varnothing$ indeksni skup, $\indFamilija{\urePar{E_\lambda}{\famE_\lambda}}{\lambda \in \Lambda}$ familija izmjerivih prostora i $\niz{X_\lambda : \Omega \to E_\lambda}{\lambda \in \Lambda}$ familija nezavisnih slu\v cajnih elemenata.
    Neka je $\famT$ particija od $\Lambda$ i neka je za svaki $T \in \famT$ zadan izmjeriv prostor $\urePar{H_T}{\famH_T}$ i izmjerivo preslikavanje $f_T : \urePar{\produkt{t \in T}{} E_t}{\dirProd{t \in T}{} \famE_t} \to \urePar{H_T}{\famH_T}$.
    Tada je $\niz{f_T \circ \niz{X_\lambda}{\lambda \in T}}{T \in \famT}$ nezavisna familija slu\v cajnih elemenata.
\end{kor}

%% koriste\' ci dokaz sli\v can onome iz \ref{tm:7.7} rje\v site.
\begin{zad} \label{zad:7.10}
    Neka su $X : \Omega \to \urePar{E}{\famE}$ i $Y : \Omega \to \urePar{H}{\famH}$ nezavisni slu\v cajni elementi.
    Neka je $f : \urePar{E \times H}{\famE \otimes \famH} \to \urePar{\real}{\borel{\real}}$ izmjeriva za koju vrijedi:
    ili je $f \geq 0$ ili je $\masE [|f \urePar{X}{Y}|] < +\infty$.
    Tada je
    \begin{align*}
        \masE [ f \urePar{X}{Y} ]
        &= \int_{E} \int_{H} f \urePar{x}{y} \: d \masP_Y (y) \: \masP_X (x) \\
        &= \int_H \int_E f \urePar{x}{y} \: d \masP_X (x) \: \masP_Y (y).
    \end{align*}
\end{zad}

\begin{zad} \label{zad:7.11}
    Slu\v cajni element sa vrijednostima u $\urePar{E}{\famE}$ je \emph{degeneriran} ako postoji $a \in E$, takav da je $X = a \; (g.s.)$ (u ovoj situaciji \emph{nije nu\v zno} da je $\{a\} \in \famE$, to jest da je $\{X = a\} \in \famF$).
    Ako je $X$ degenerirani slu\v cajni element, a $Y$ bilo koji slu\v cajni element (ne nu\v zno s vrijednostima u istom prostoru) tada su $X$ i $Y$ nezavisni.
    Ako su $X$ i $Y$ nezavisni slu\v cajni elementi takvi da je $X + Y$ degenerirana, tada su $X$ i $Y$ degenerirane.
\end{zad}


Grupa $\urePar{G}{\cdot}$ sa $\sigma$-algebrom $\famG$ je \emph{izmjeriva grupa} ako je grupovna operacija $\cdot : G \times G \to G$ izmjeriva u paru $\urePar{\famG \otimes \famG}{\famG}$.
Neka su $\mu$ i $\nu$ $\sigma$-kon\v cne mjere na $\urePar{G}{\famG}$.
Tada je i $\mu \otimes \nu$ $\sigma$-kona\v cna mjera na $\urePar{G \times G}{\famG \otimes \famG}$, pa mo\v zemo promatrati mjeru na $\urePar{G}{\famG}$ induciranu preslikavanjem $\cdot$ u odnosu na $\mu \otimes \nu$.
Tu mjeru nazivamo \emph{konvolucijeom} mjera $\mu$ i $\nu$ te ozna\v cavamo sa $\mu * \nu$.
Po zadatku \ref{zad:4.15} dobijemo da je za svaki $H \in \famG$
\begin{equation*}   \label{jed:7.12}
    (\mu * \nu)(H) := \int_G \nu (a^{-1} \: H) \: d \mu (a).
\end{equation*}

\begin{zad} \label{zad:7.13}
    Ako je $\urePar{G}{\cdot}$ Abelova grupa, tada je konvolucija komutativana operacija.
\end{zad}

\begin{nap} \label{nap:7.14}
    \begin{enumerate}[label=(\alph*)]
        \item Op\' cenito govore\' ci mjera $\mu * \nu$ ne mora biti $\sigma$-kona\v cna.
        Ako su $\mu$ i $\nu$ kona\v cne mjere, tada je i $\mu * \nu$ kona\v cna i
        \begin{equation*}
            (\mu * \nu) (G) = \mu (G) \: \nu (G),
        \end{equation*}
        \v sto slijedi iz \eqref{jed:7.12}.
        \item Ako su $\mu, \; \nu, \; \eta$ $\sigma$-kona\v cne mjere na $\urePar{G}{\famG}$, takve da su $\mu * \nu$ i $\nu * \eta$ $\sigma$-kon\v cne, tada vrijedi
        \begin{equation*}
            (\mu * \nu) * \eta = \mu * (\nu * \eta).
        \end{equation*}
        \item   \label{nap:7.14c}
        Ako na $\urePar{G}{\famG}$ postoji $\sigma$-kona\v cna mjera $\lambda$ takva da je $\mu << \lambda$ i $\nu << \lambda$, tada za $f = \frac{d \mu}{d \lambda}$ i $g = \frac{d \nu}{d \lambda}$ vrijedi
        \begin{equation*}
            (\mu * \nu) (H) = \int_G \int_{a^{-1} H} f(a) \: g(b) \: d \lambda (b) \: d \lambda (a).
        \end{equation*}
        \item Ako je $\lambda$ iz \ref{nap:7.14c} jo\v s i \emph{invarijantna} u odnosu na sve lijeve translacije (to jest za svaki $g \in G$ i za svaki $C \in \famG$ je $\lambda (g^{-1} C) = \lambda (C)$), tada je
        \begin{equation*}
            (\mu * nu) (H) = \int_H \Big( \int_G f(a) \: g (a^{-1} b) \: d \lambda (a) \Big) \: d \lambda (b),
        \end{equation*}
        to jest $\mu * \nu << \lambda$ i postoji gusto\' ca
        \begin{equation*}
            \frac{d (\mu * \nu)}{d \lambda} (b) = \int_G f(a) \: g (a^{-1} b) \: d \lambda (a)
        \end{equation*}
        koju ozna\v cavamo sa \emph{$f*g$}.
    \end{enumerate}
\end{nap}

\begin{tm}  \label{tm:7.15}
    Ako su $X$ i $Y$ nezavisni slu\v cajni elementi s vjerojatnostima u izmjerivoj grupi $\urePar{G}{\cdot}$, tada je $\masP_{X \cdot Y} = \masP_X * \masP_Y$.
\end{tm}

\begin{proof}
    Uo\v cimo da je $Z:= \urePar{X}{Y}$, su\v cajni element s vrijednostima u $G \times G$ i da je $\masP_Z = \masP_X \otimes \masP_Y$.
    Budu\' ci da je $\cdot$ izmjerivo, slijedi da je $X \cdot Y$ slu\v cajni element s vrijednostima u $G$ i $\masP_{X \cdot Y}$ je vjerojatnost inducirana s $\cdot$ u odnosu na $\masP_Z$.
    To je po definiciji to\v cno $\masP_X * \masP_Y$.
\end{proof}

\begin{zad} \label{zad:7.16}
    Primjenite ove rezultate na Abelovu grupu $\urePar{\real^d}{+}$ i posebno obratite pa\v znju na slu\v caj $d = 1$.
    Za slu\v caj $d = 1$izvedite pravila za konvluciju d.f.
\end{zad}