% distribucije slucajnih elemenata

\chapter{Distribucije slu\v cajnih elemenata} \label{dist_sl_elem}

\begin{pr}  \label{pr:5.1}
    Promatrajmo vjerojatnosni prostor
    \begin{equation*}
        \vjerojatnosniProstor = \Big(\segment{0}{1}, \: \borel{\segment{0}{1}}, \: \restr{\lambda}{\borel{\segment{0}{1}}}\Big)
    \end{equation*}
    i dvije slu\v cajne varijable $X (\omega) := \omega$, $Y (\omega) := 1 - X (\omega)$.
    O\' cito $X(\omega) \neq Y(\omega)$, za svaki $\omega \in \Omega \setminus \{ \frac{1}{2} \}$, to jest kao funkcije u skupovnom smislu $X$ i $Y$ su "potpuno" razli\v cite.
    \v Sto mo\v zemo re\' ci o "vjerojatnosnim informacijama" koje mo\v zemo saznati o $X$ i $Y$?
    Za proizvoljan $A \in \famF = \borel{\segment{0}{1}}$ imamo $1-A = 1 + (-A) \in \borel{\segment{0}{1}}$ i $\lambda (1 - A) = \lambda (A) \implies \vjeroj{X \in A} = \vjeroj{Y \in A}, \; \forall A \in \famF$.
    Dakle iako su $X$ i $Y$ vrlo razli\v cite slu\v cajne varijable, njihove "vjerojatnosne distribucije" su jednake.
\end{pr}

\begin{defn}    \label{defn:5.2}
    Neka je $\vjerojatnosniProstor$ vjerojatnosni prostor na kojem je definiran slu\v cajan element $X$ s vrijednostima u izmjerivom prostoru $\urePar{E}{\famE}$.
    \emph{Vjerojatnosna distribucija} slu\v cajnog elementa $X$ je vjerojatnosna mjera $\masP_X$, definirana na $\urePar{E}{\famE}$
    sa:
    \begin{equation*}
        \masP_X (A) := \masP(X \in A), \quad \forall A \in \famE.
    \end{equation*}
\end{defn}

Uo\v cimo $\masP_X$ je poseban slu\v caj tako zvane mjere inducirane izmjerivim preslikavanjem.

\begin{nap} \label{nap:5.3}
    \begin{enumerate}[label=(\alph*)]
        \item Uo\v cimo da je u $\masP$ sadr\v zana "sva vjerojatnosna informacija" o $X$.
        \item Bavljenje slu\v cajnim elementima s vjerojatnostima u $\urePar{E}{\famE}$ (bez obzira na kojem vjerojatnosnom prostoru bili definirani) mo\v ze se svesti na tako zvani \emph{kanonski slu\v caj}.
        \item Preciznije, mi fiksiramo "vjerojatnosni prostor" i "slu\v cajni element", to jest uzmemo $\widetilde{\Omega} := E$, $\widetilde{\famF} := \famE$, $\widetilde{X} := id : E \to E$ i dr\v zimo uvijek fiksirano, a slu\v cajni element "realiziramo" postavljanjem odgovaraju\' ce vjerojatnosne mjere $\masP_X$ na taj prostor.
        Tada $X$ (uz $\masP$) i $\widetilde{X}$ (uz $\masP_X$) imaju istu distribuciju (i to je $\masP_X$).
        \item Time se uvodi nova ideja o pojmu jednakosti u vjerojatnosti.
        Vidjeli smo da za slu\v cajne elemente (definirane na istom prostoru s vjerojatnostima u istom prostoru) mo\v zemo promatrati obi\v cnu skupovnu jednakost $X = Y$, jednakost gotovo sigurno $(g.s.)$, a sada i jednakost po distribuciji.
        Preciznije, ako su $X$ i $Y$ slu\v cajni elementi s vrijednostima u $\urePar{E}{\famE}$, definirani na vjerojatnosnim prostorima $(\Omega_1, \: \famF_1, \: \masP)$ i $(\Omega_1, \: \famF_2, \: \masO)$, re\' ci \' cemo da su \emph{jednaki po distribuciji}, i pisati $X \distJed Y$, ako je $\masP_X = \masO_Y$.
        Lako se vidi (primjer \ref{pr:5.1})
        \begin{equation*}
            X = Y
            \begin{smallmatrix}
                \implies&  \\
                \notimpliedby&
            \end{smallmatrix}
            X = Y \; (g.s.)
            \begin{smallmatrix}
                \implies&  \\
                \notimpliedby&
            \end{smallmatrix}
            X \distJed Y.
        \end{equation*}
    \end{enumerate}
\end{nap}

U slu\v caju $\urePar{E}{\famE} = \urePar{\extReal}{\borel{\extReal}}$ informaciju o $\masP_X$ mo\v zemo prevesti na jezik obi\v cnih realnih funkcija realne varijable.

\begin{defn}    \label{defn:5.4}
    Neka je $X$ pro\v sirena slu\v cajna varijabla na $\vjerojatnosniProstor$.
    Funkcija $F_X : \extReal \to \segment{0}{1}$, definiran sa
    \begin{equation*}
        F_X (a) := \masP (X \leq a) = \masP (\skup{\omega \in \Omega}{X(\omega) \leq a}) = \masP_X (\segment{-\infty}{a}), \quad a \in \real
    \end{equation*}
    nazivamo \emph{vjerojatnosnom funkcijom distribucije} slu\v cajnog elementa $X$.
\end{defn}

Funkcija $F_X$ je neopadaju\' ca (jer $a \leq b \implies \segment{-\infty}{a} \subseteq \segment{-\infty}{b} \implies \masP_X (\segment{-\infty}{a}) \leq \masP_X (\segment{-\infty}{b}) $) i neprekidna zdesna (jer $a_n \searrow a \implies \segment{-\infty}{a} = \presjek{n = 1}{\infty} \segment{-\infty}{a_n}$).
Uo\v cimo da je $F_X (-\infty) \geq 0$ i da je
\begin{equation*}
    F_X (-\infty) = 0 \iff \masP (X = -\infty) = 0,
\end{equation*}
a zbog neprekidnosti zdesna je $F_X (-\infty) = \lim\limits_{a \ \searrow -\infty} F_X (a)$.
S druge strane postoji i $\lim\limits_{a \nearrow +\infty} F_X (a) \leq F_X(+\infty) = 1$, te je
\begin{equation*}
    F_X(+\infty) = \lim\limits_{a \nearrow +\infty} F_X (a) \iff \masP (X = +\infty) = 1.
\end{equation*}

Posebno, $X$ je slu\v cajna varijabla ako i samo ako vrijedi:
\begin{align}    \label{jed:5.5}
    \begin{split}
        F_X(-\infty) &= 0, \\
        \lim\limits_{a \nearrow +\infty} F_X(a) &= 1
    \end{split}
\end{align}

Mo\v zemo gledati s druge strane i re\' ci da je $F: \extReal \to \segment{0}{1}$ \emph{p.d.F.} ako je $F$ neopadaju\' ca, neprekidna zdesna i zadovoljava uvjet \eqref{jed:5.5}.
Time dolazimo do fundamentalog pitanja: "Je li svaka p.d.F. vjerojatnosna funkcija distribucije neke slu\v cajne varijable?"
Odgovor je: Da!

\begin{tm}  \label{tm:5.6}
    Neka je $F: \extReal \to \segment{0}{1}$ neopadaju\' ca, neprekidna zdesna i zadovoljava uvijet \eqref{jed:5.5}, tada postoji vjerojatnosni prostor $\vjerojatnosniProstor$ i slu\v cajana varijabla $X$ na tom vjerojatnosnom prostoru, takva da je $F$ vjerojatnosna funkcija distribucije od $X$.
\end{tm}

\begin{proof}
    Uzmemo kanonski slu\v caj $\Omega = \real, \; \famF = \borel{\real}$ i $X = id : \real \to \real$.
    Sjetimo se da je $\famI$ poluprsten i $\sigAlg{\famI} = \famF$, pri \v cemu je $\famI$ iz primjera \ref{pr:2.3}.
    Za $\lijInt{a}{b} \in\famI$ definiramo
    \begin{equation}    \label{jed:5.7}
        \masP_F (\lijInt{a}{b}) := F (b) - F(a)
    \end{equation}
    i nije prete\v sko pokazati da je $\masP_F$ $\sigma$-aditivna na $\famI$.
    Po korolaru \ref{kor:2.8} i po teoremu \ref{tm:2.11} postoji jedinstveno pro\v sirenje $\masP_F$ koje je vjerojatnost na $\borel{\real}$.
    Dakle, $X$ je slu\v cajna varijabla na $(\Omega, \: \famF, \: \masP_F)$ i $(\masP_F)_X = \masP_F$, pa je $F_X = F$.
\end{proof}

Dakle vrijedi
\begin{equation*}
    X \distJed Y \iff F_X = F_Y.
\end{equation*}
%zar ne fali tu neki uvijet tipa ako su F_X i F_Y jednake u svim točkama neprekidnosti?

\begin{nap} \label{nap:5.8}
    \begin{enumerate}[label=(\alph*)]
        \item   \label{nap:5.8a}
        Za slu\v cajnu varijablu $X$ postoji $1-1$ korespondencija izme\dj u $\masP_X$ i $F_X$, posebno, sav "vjerojatnosni sadr\v zaj" o $X$ sadr\v zan je i u $F_X$.
        \item Teorem \ref{tm:5.6} opravdava \v cesto kori\v stenje funkcija distribucija u praksi.
        Istra\v ziva\v cima u raznim podru\v cjima mo\v ze biti va\v zno samo prona\' ci "distribuciju" i oni se dalje bave vjerojatnostima bez da se bave pitanjem egzistencije istih.
        \item Usporedi \eqref{jed:5.7} sa \eqref{jed:2.14}.
        \item Radi se u stvari o istoj konstrukciji samo je u \eqref{jed:5.7} na\v sa $F$ "normirana" na poseban na\v cin, to jest $F(+\infty)$ je stavljeno da bude $1$.
        Jasno, za svaki $c \in \real$, $F + c$ bi dala istu vjerojatnost $\masP_{F + c}$, kao i $\masP_F$.
        Op\' cenito, ako je $F: \real \to \real$ neopadaju\' ca i neprekidna zdesna, tada \eqref{jed:2.14} i isti dokaz kao u teoremu \ref{tm:5.6} (samo koristimo zadatak \ref{zad:2.9}) daje tvrdnju o egzistenciji i jedinstvenosti $\sigma$-kona\v cne mjere $\mu$ na $\urePar{\real}{\borel{\real}}$ takve da vrijedi \eqref{jed:2.14}. Svaku funkciju $F : \real \to \real$, koja je neopadaju\' ca i neprekidna zdesna zvat \' cemo \emph{d.f.}.
        Dakle \eqref{jed:2.14} daje $1-1$ korespondenciju izme\dj u
        \begin{equation}    \label{jed:5.9}
            \begin{matrix}
                &\skup{F + c}{c \in \real}\\
                &F \; \textnormal{ d.f.}
            \end{matrix}
            \longleftrightarrow
            \begin{matrix}
                &\mu_F \;\; \sigma \textnormal{-kona\v cna}\\
                &\textnormal{na } \; \urePar{\real}{\borel{\real}}
            \end{matrix}
        \end{equation}
        Za d.f. $F$ postoje (u $\extReal$) limesi:
        \begin{enumerate}[label=]
            \item $F(-\infty) = \lim\limits_{a \searrow -\infty} F(a)$
            \item $F(+\infty) = \lim\limits_{a \nearrow \infty} F(a)$
            \item $\mu_F(\real) = F(+\infty) - F(-\infty)$.
        \end{enumerate}
    \end{enumerate}
\end{nap}

\begin{zad} \label{zad:5.10}
    Opi\v site korespondenciju kao u \eqref{jed:5.9} na $\urePar{\real}{\borel{\real}}$i poka\v zite da je korespondencija u \eqref{jed:5.9} poseban slu\v caj korespondencije na $\extReal$.
\end{zad}

\begin{zad} \label{zad:5.11}
    Neka je $X = (X_1, \ldots, X_d)$ slu\v cajan vektor na $\vjerojatnosniProstor$.
    \begin{enumerate}[label=(\alph*)]
        \item Opi\v si funkciju distribucije $F_x$ u ovom slu\v caju.
        \item Iska\v zite i doka\v zite analogon teorema \ref{tm:5.6}
        %uputa: umjesto \eqref{jed:5.7} promatrajte 
        %\begin{equation*}
        %    \Delta^{a, \: b}_{d} F := \suma{
        %    \begin{smallmatrix}
        %        (x_1, \ldots, x_d) = x \\
        %        x_i \in \{a_i, \; b_i\}
        %    \end{smallmatrix}}{}
        %    (-1)^{\#(x)} F(x_1, \ldots, x_d)
        %\end{equation*}
        %gdje je \# (x) := \card{\skup{1 \leq i \leq d}{x_i = a_i}} i usporedite sa \masP_X (\lijInt{a}{b}) := \vjeroj{a_i < X_i \leq b_i \: : \: \textnormal{ za sve } i = 1, \ldots, d}.  
    \end{enumerate}
\end{zad}

Dobro su nam poznati primjeri brojnih razli\v citih distribucija na $\real^d$, osobito za $d = 1$.
Korisna je i sljede\' ca klasifikacija, gledamo $d = 1$ zbog jednostavnosti zapisa.

\begin{defn}    \label{defn:5.11-1}
    Slu\v cajna varijabla $X$ je \emph{diskretna} ako postoji prebrojiv skup $S$ takav da je
    \begin{equation*}
        X \in S \; (g.s.),
    \end{equation*}
    dok je p.d.F. $G$ \emph{diskretna} ako postoji prebrojiv skup $\skup{\urePar{x_j}{p_j}}{j \in J} \subseteq \real \times \segment{0}{1}$, takav da je
    \begin{equation*}
        G(x) = \suma{
            \begin{smallmatrix}
                j \in J\\
                x_j \leq x
            \end{smallmatrix}
            }{} p_j = \suma{j \in J}{} p_j \: \karaktFja_{\desInt{x_j}{+\infty}} (x).
    \end{equation*}
\end{defn}

Lako se vidi da je $X$ diskretna ako i samo ako je $F_X$ diskretna.
Ako jo\v s imamo i $(j_1 \neq j_2 \implies x_{j_1} \neq x_{j_2})$ tada je $p_j = \vjeroj{X = x_j}, \; \forall j \in J$.
Uo\v cimo da skup $\skup{\urePar{x_j}{p_j}}{j \in J}$ u potpunosti opisuje $F_X$.

\begin{defn}    \label{defn:5.11-2}
Ka\v zemo da je p.d.F. $G$ \emph{apsolutno neprekidna} ako postoji izmjeriva funkcija $g : \real \to \desInt{0}{+\infty}$, takva da vrijedi
\begin{equation*}
    G(x) = \int\limits_{-\infty}^{x} g(t) \: d \lambda (t).
\end{equation*}
Ka\v zemo da je slu\v cajna varijabla $X$ \emph{apsolutno neprekidna} ako je $F_X$ aposlutno neprekidna.
\end{defn}

\begin{nap} \label{nap:5.11-3}
    Prisjetimo se, neka su $\mu$, $\nu$ dvije mjere na $\izmjerivProstor$. Ka\v zemo da je $\nu$ \emph{apsolutno neprekidna} u odnosu na $\mu$ ako vrijedi:
    \begin{equation*}
        A \in \famF, \; \mu(A) = 0 \implies \nu (A) = 0.
    \end{equation*}
    Tada pi\v semo $\nu \ll \mu$.
\end{nap}

Po Radon-Nykondimovom teoremu znamo da je $X$ apsolutno neprekidna ako i samo ako je $\masP_X \ll \lambda$, te da tada postoji i $\lambda-(g.s.)$ jedinstvena Radon-Nykondimova derivacija $\frac{d \masP_X}{d \lambda}$ koja igra ulogu funkcije $g$ odozgo.
U ovom slu\v caju nazivamo je funkcijom gusto\' ce i ozna\v cavamo sa $f_X$.
Postoji jako puno funkcija gusto\' ce jer je $f: \real \to \desInt{0}{+\infty}$ funkcija gusto\' ce neke slu\v cajne varijable $X$ ako i samo ako je $f$ izmjeriva i vrijedi $\int\limits_{\real} f(t) \: d \lambda (t) = 1$.

Uo\v cimo nadalje da se za $\sigma$-kona\v cne mjere $\mu$ na $\urePar{\real}{\borel{\real}}$ i d.f. $F_{\mu}$ pripadni integrali potpuno podudaraju, to jest:
\begin{equation*}
    \int_{\real} f(x) \: d \mu (x) = \int_{-\infty}^{\infty} f(x) \: d F_{\mu} (x).
\end{equation*}

Posebno, ako je $X$ slu\v cajni element na $\vjerojatnosniProstor$ s vrijednostima u $\urePar{E}{\famE}$ i $f:E \to \real$ izmjeriva funkcija, tada je
\begin{equation*}
    f \circ X \in L^1 \vjerojatnosniProstor \iff f \in L^1 (E, \: \famE, \: \masP_X)
\end{equation*}
i vrijedi:
\begin{equation*}
    \int_{\Omega} (f \circ X) (\omega) \: d \masP (\omega)
    = \int_E f(x) \: d \masP_X (x)
    = \int_\real \id (t) \: d \masP_{f \circ X} (t)
    = \int_{-\infty}^{\infty} u \: d F_{f \circ X} (u).
\end{equation*}

\begin{figure}[H]
    \centering
    \begin{tikzpicture}
        \matrix (m) [matrix of math nodes,row sep=5em,column sep=8em,minimum width=3em]
        {
        \vjerojatnosniProstor & (E, \: \famE, \: \masP_X) \\
        & (\real, \: \borel{\real}, \: \masP_{f \circ X})\\
        & (\real, \: \borel{\real})\\};
        \path[-stealth]
        (m-1-1) edge node [above] {$X$} (m-1-2)
        (m-1-2) edge node [right] {$f$} (m-2-2)
        (m-1-1) edge node [below, xshift=-9pt] {$f \circ X$} (m-2-2)
        (m-2-2) edge node [right] {$\id$} (m-3-2)
        (m-1-1) edge node [below, xshift=-9pt] {$f \circ X$} (m-3-2);
    \end{tikzpicture}
\end{figure}
U tom slu\v caju ka\v zemo da postoji o\v cekivanje od $f \circ X$ i bilo koje od gornjih integrala ozna\v cavamo sa $\ocek{f \circ X}$
(uo\v cimo da $f \circ X \in L^1 \vjerojatnosniProstor$ zna\v ci da je $\ocek{|f \circ X|} < +\infty$).

Zanimaju nas neki posebni slu\v cajevi gornjih integrala.

\begin{defn}    \label{defn:5.11-4}
    Neka je $X$ slu\v cajna varijabla  na $\vjerojatnosniProstor$ te neka je $p > 0$, tada $\ocek{|X|^p} \in \segment{0}{+\infty}$ i zovemo ga \emph{$p$-ti apsolutni moment} od $X$.
    
    Ako je $\ocek{|X|^p} < +\infty$, tada je $X \in L^p := L^p \vjerojatnosniProstor$, \v sto je normirani prostor s normom $\norma{X}_p = (\masE [|X|^p])^{\frac{1}{p}}$ te postoji i $\ocek{X^p}$ \v sto nazivamo \emph{$p$-ti moment} od $X$.
\end{defn}

\begin{nap} \label{nap:5.11-5}
    Tehni\v cki $L^p \vjerojatnosniProstor$ je kvocjentni prostor svih slu\v cajnih varijabli (op\' cenito izmjerivih funkcija) \v ciji je $p$-ti apsolutni moment kona\v can kvocjenitano po relaciji ekvivalencije $\sim$ definirane sa
    \begin{equation*}
        f \sim g \iff f = g \; (g.s.),
    \end{equation*}
    to nam je potrebno kako bi dobili jedno od definicijskih svojstava norme
    \begin{equation*}
        \norma{X}_p = 0 \implies X = 0,
    \end{equation*}
    budu\' ci vrijedi
    \begin{equation*}
        \masE [|X|^p] = 0 \implies X = 0 \; (g.s.)
    \end{equation*}
    Pa onda zaklju\v cujemo da je na $L^p \vjerojatnosniProstor$ $X = 0 \; (g.s.)$.
\end{nap}

Ako je $p \leq q$, tada je $L^q \subseteq L^p$ (i $\norma{X}_p \leq \norma{X}_q$), pa postojanje $q$-tog momenta implicira postojanje $p$-tog momenta.
Ako je $X \geq 0$ i $p > 0$ jednostavnim ra\v cunom daje
\begin{align}   \label{jed:5.12}
    \begin{split}
        \ocek{X^p} =& \int_0^{+\infty} t^p \: d \masP_X (t) = \int_0^{\infty} \Big( \int_0^t p \cdot u^{p - 1} \: d u \Big) \: d \masP_X(t) = (\textnormal{Fubini})\\
        =& p \int_0^{\infty} \: du \int_u^t u ^{p-1} \: d \masP_X (t) = p \int_0^{+\infty} u^{p-1} \masP_X (\segment{u}{+\infty}) \: du \\
        =& \int_0^{+\infty} p \cdot u^{p-1} \vjeroj{X \geq u} \: du.
    \end{split}
\end{align}

Ako je $X = (X_1, \ldots, X_d)$ $d$-dimenzionalni slu\v cajni vektor i $X_k \in L^1, \; (\forall k \in \{1, \ldots, d\})$, tada i $X$ ima o\v cekivanje i pi\v semo $\masE X := (\masE X_1, \ldots, \masE X_d)$.

\begin{zad} \label{zad:5.13}
    Neka je $X$ $d$-dimenzionalni slu\v cajni vektor koji ima o\v cekivanje i $f: \real^d \to \real$ konveksna funkcija, to jest funkcija za koju vrijedi:
    \begin{equation*}
        (\forall x, \: y \in \real^d)(\forall \lambda \in \segment{0}{1}) \quad f (\lambda x + (1 - \lambda) y) \leq \lambda f(x) + (1 - \lambda) f(y).
    \end{equation*}
    Ako  postoji (u \v sirem smislu) $\masE [ f(X) ]$, tada je
    \begin{equation*}
        f (\masE X) \leq \masE [ f(X) ].
    \end{equation*}
    To nazivamo \emph{Jensenovom nejednakosti}.
\end{zad}

Osim prvog momenta slu\v cajne varijable, kojeg jo\v s nazivamo i o\v cekivanjem, va\v zan je i drugi moment to jest \emph{drugi centrali moment} $\masE [ (X - \masE X)^2 ]$, kojeg nazivamo \emph{varijancom}.
Uo\v cimo da varijanca postoji ako i samo ako postoji drugi moment od $X$ i tada je
\begin{equation*}
    Var \: X := \masE [ (X - \masE X)^2 ] = \masE [ X^2 ] - (\masE X)^2.
\end{equation*}