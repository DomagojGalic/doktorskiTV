% poglavlje 5.1 uvod u levyjeve procese -> predavanje 21

\chapter{Uvod u L\' evyjeve procese}

Neka je $X$ slu\v cajna varijabla s karakteristi\v cnom funkcijom $\varphi = \varphi_X$.
Budu\' ci da vrijedi
\begin{equation*}
    \begin{aligned}
        x \sim 0 \quad &\implies \quad \frac{i t x}{1 + x^2} \sim i t x,\\
        x \sim \infty \quad &\implies \quad \frac{i t x}{1 + x^2} \sim 0.
    \end{aligned}
\end{equation*}
Sada iz \eqref{jed:20.17} slijedi da je $\varphi$ beskona\v cno dijeljiva ako i samo ako postoji ure\dj ena trojka $(\gamma, \sigma^2, \nu)$, pri \v cemu je $\gamma \in \real$, $\sigma^2 \geq 0$, $\nu$ je mjera na $\real \setminus \{0\}$ za koju vrijedi
\begin{equation*}
    \int\limits_{\real \setminus \{0\}} (|x|^2 \land 1) \: d \nu (x) < +\infty,
\end{equation*}
takvi da je $\varphi = e^\psi$ i da je
\begin{equation}    \label{jed:21.1}
    \psi (u) = i \gamma u - \frac{\sigma^2}{2} u^2 + \int\limits_{\real \setminus \{0\}} \Big( e^{i u x }- 1 - i u x \karaktFja_{\{|x| < 1\}} \Big) \: d \nu (x).
\end{equation}

Koriste\' ci veliki Kolmogorovljev teorem, za svaki $n \in \nat$ mo\v zemo na\' ci niz slu\v cajnih varijabli $\niz{X_{m, n}}{m \in \nat}$ koje su nezavisne i jednako distribuirane i za koje vrijedi
\begin{equation*}
    \big[ \varphi_{X_{m, n}} \big]^n = \varphi.
\end{equation*}
Stavimo li
\begin{equation*}
    Y_{\frac{m}{n}} := \suma{k = 1}{m} X_{k, n},
\end{equation*}
dobijemo diskretni stohasti\v cki proces kod kojeg je
\begin{equation*}
    \varphi_{Y_{\frac{m}{n}}} = e^{\frac{m}{n} \psi}.
\end{equation*}
Posebno za $m = n$ imamo $\varphi_{Y_1} = \varphi$.

Postavlja se prirodno pitanje: "Mo\v ze li se napraviti proces po neprekidnom vremenu koji \' ce biti na sli\v can na\v cin  izveden iz $X$ i \v cuvati dobra svojstva?"
Jedno od dobrih svojstava koje o\v cekujemo je
\begin{equation*}
    \varphi_{Y_t} \xrightarrow[t \to t_0]{} \varphi_{Y_{t_0}},
\end{equation*}
to jest po teoremu o neprekidnosti o\v cekujemo bar neku vrstu konvergencije i za pripadne slu\v cajne varijable.
S druge strane ho\' cemo posti\' ci neku prirodnu generalizaciju niza nezavisnih jednako distribuiranih slu\v cajnih varijabli.

\begin{nap} \label{nap:21.2}
    Na prvi pogled moglo bi se pomisliti da treba posti\' ci da se $\niz{X_t}{t \geq 0}$ sastoji od \emph{nezavisnih} slu\v cajnih varijabli i da, recimo, bar vrijedi
    \begin{equation*}
        X_{t_n} \xrightarrow[t_n \to t_0]{\masP} X_{t_0}.
    \end{equation*}
    Lako se vidi da je to pogre\v san pristup, jer bi iz nezavisnosti moralo slijediti da je $X_{t_0}$ gotovo sigurno jednako konstanti.
    Budu\' ci da to vrijedi za svaki $t_0$, to bi zna\v cilo da na\v s proces jedino mo\v ze biti deterministi\v cka funkcija.
    Stoga nije ideja da su $(X_t)$ nezavisne, ve \' c pogledamo li $(Y_{\frac{m}{n}})$, trebamo zahtjevati da prirasti budu nezavisni.
\end{nap}

\begin{defn}    \label{defn:21.3}
    Stohasti\v cki proces $\niz{X_t}{t \geq 0}$ s vrijednostima u $\real$ je \emph{L\' evijev proces} ako vrijedi:
    \begin{enumerate}[label=(\alph*)]
        \item   \label{defn:21.3.1}
        Za svaki $n \in \nat$ i za svaki izbor $0 \leq t_0 < t_1 \ldots < t_n$ su slu\v cajne varijable
        \begin{equation*}
            X_{t_n} - X_{t_{n - 1}}, X_{t_{n - 1}} - X_{t_{n - 2}}, \ldots, X_{t_1} - X_{t_0}
        \end{equation*}
        nezavisne.
        \item   \label{defn:21.3.2}
        za svaki $h > 0$ i za svaki $t, s \geq 0$ je
        \begin{equation*}
            X_{t + h} - X_t \distJed X_{s + h} - X_s.
        \end{equation*}
        \item   \label{defn:21.3.3}
        za gotovo sve $\omega \in \Omega$ trajektorija
        \begin{equation*}
            t \mapsto X_t (\omega)
        \end{equation*}
        je c\' adl\' ag (franc. continue \' a droite, limite \' a gauche - neprekidna zdesna, s limesom slijeva)
        \item   \label{defn:21.3.4}
        $X_0 \equiv 0$.
    \end{enumerate}
\end{defn}

Zbog \ref{defn:21.3.1}, \ref{defn:21.3.2} i \ref{defn:21.3.4} slijedi da je za svaki $n \in \nat$
\begin{equation*}
    X_1 = (X_1 - X_{\frac{n - 1}{n}}) + (X_\frac{n - 1}{n} - X_\frac{n - 2}{n}) + \ldots + (X_\frac{1}{n} - X_0),
\end{equation*}
to jest $X_1$ je suma $n$ nezavisnih jednako distribuiranih slu\v cajnih varijabli.
Dakle $\varphi_{X_1}$ je beskon\v cno djeljiva, pa je
\begin{equation}    \label{jed:21.4}
    \varphi_{X_1} = e^\psi,
\end{equation}
Pri \v cemu je $\psi$ oblika \eqref{jed:21.1}.
Kao i ranije lako se vidi
\begin{equation}
    \varphi_{X_\frac{m}{n}} = e^{\frac{m}{n} \psi},
\end{equation}
a zbog \ref{defn:21.3.3} i teorema neprekidnosti je
\begin{equation}    \label{jed:21.5}
    \varphi_{X_t} = e^{t \psi},
\end{equation}
pa se $\varphi$ naziva \emph{karakteristi\v cnim eksponentom} L\' evyjevog procesa (uo\v cimo da je $\varphi$ odre\dj en pripadnom trojkom $(\gamma, \sigma^2, \nu)$).
Daleko je slo\v zenije pokazati obrat (i to ovdje ne\' cemo dokazivati) koji dalje fundamentalni teorem

\begin{tm}  \label{tm:21.6}
    \begin{enumerate}[label=(\alph*)]
        \item   \label{tm:21.6.1}
        Ako je $\niz{X_t}{t \geq 0}$ L\' evyjev proces, tada je $\varphi_{X_t}$ beskona\v cno djeljiva, za svaki $t \geq 0$ i vrijedi \eqref{jed:21.5}.
        \item   \label{tm:21.6.2}
        Ako je $\varphi$ beskona\v cno djeljiva karakteristi\v cna funkcija, tada postoji L\' evyjev proces $\niz{X_t}{t \geq 0}$ takav da je $\varphi_{X_1} = \varphi$.
    \end{enumerate}
\end{tm}

\begin{pr}  \label{pr:21.7}
    Neka je $\psi$ zadan sa $(\gamma, 0, 0)$.
    Tada je
    \begin{equation*}
        \varphi_{X_t} = e^{i \gamma t u} \quad \implies \quad X_t = \gamma t
    \end{equation*} 
    \v sto je deterministi\v cko gibanje po pravcu $t \mapsto \gamma t$, $t \geq 0$.
\end{pr}