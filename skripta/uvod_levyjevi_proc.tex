% poglavlje 5.1 uvod u levyjeve procese -> predavanje 21

\chapter{Uvod u L\' evyjeve procese}

Neka je $X$ slu\v cajna varijabla s karakteristi\v cnom funkcijom $\varphi = \varphi_X$.
Budu\' ci da vrijedi
\begin{equation*}
    \begin{aligned}
        x \sim 0 \quad &\implies \quad \frac{i t x}{1 + x^2} \sim i t x,\\
        x \sim \infty \quad &\implies \quad \frac{i t x}{1 + x^2} \sim 0.
    \end{aligned}
\end{equation*}
Sada iz \eqref{jed:20.17} slijedi da je $\varphi$ beskona\v cno dijeljiva ako i samo ako postoji ure\dj ena trojka $(\gamma, \sigma^2, \nu)$, pri \v cemu je $\gamma \in \real$, $\sigma^2 \geq 0$, $\nu$ je mjera na $\real \setminus \{0\}$ za koju vrijedi
\begin{equation*}
    \int\limits_{\real \setminus \{0\}} (|x|^2 \land 1) \: d \nu (x) < +\infty,
\end{equation*}
takvi da je $\varphi = e^\psi$ i da je
\begin{equation}    \label{jed:21.1}
    \psi (u) = i \gamma u - \frac{\sigma^2}{2} u^2 + \int\limits_{\real \setminus \{0\}} \Big( e^{i u x }- 1 - i u x \karaktFja_{\{|x| < 1\}} \Big) \: d \nu (x).
\end{equation}

Koriste\' ci veliki Kolmogorovljev teorem, za svaki $n \in \nat$ mo\v zemo na\' ci niz slu\v cajnih varijabli $\niz{X_{m, n}}{m \in \nat}$ koje su nezavisne i jednako distribuirane i za koje vrijedi
\begin{equation*}
    \big[ \varphi_{X_{m, n}} \big]^n = \varphi.
\end{equation*}
Stavimo li
\begin{equation*}
    Y_{\frac{m}{n}} := \suma{k = 1}{m} X_{k, n},
\end{equation*}
dobijemo diskretni stohasti\v cki proces kod kojeg je
\begin{equation*}
    \varphi_{Y_{\frac{m}{n}}} = e^{\frac{m}{n} \psi}.
\end{equation*}
Posebno za $m = n$ imamo $\varphi_{Y_1} = \varphi$.

Postavlja se prirodno pitanje: "Mo\v ze li se napraviti proces po neprekidnom vremenu koji \' ce biti na sli\v can na\v cin  izveden iz $X$ i \v cuvati dobra svojstva?"
Jedno od dobrih svojstava koje o\v cekujemo je
\begin{equation*}
    \varphi_{Y_t} \xrightarrow[t \to t_0]{} \varphi_{Y_{t_0}},
\end{equation*}
to jest po teoremu o neprekidnosti o\v cekujemo bar neku vrstu konvergencije i za pripadne slu\v cajne varijable.
S druge strane ho\' cemo posti\' ci neku prirodnu generalizaciju niza nezavisnih jednako distribuiranih slu\v cajnih varijabli.

\begin{nap} \label{nap:21.2}
    Na prvi pogled moglo bi se pomisliti da treba posti\' ci da se $\niz{X_t}{t \geq 0}$ sastoji od \emph{nezavisnih} slu\v cajnih varijabli i da, recimo, vrijedi
    \begin{equation*}
        X_{t_n} \xrightarrow[t_n \to t_0]{\masP} X_{t_0}.
    \end{equation*}
    Lako se vidi da je to pogre\v san pristup, jer bi iz nezavisnosti moralo slijediti da je $X_{t_0}$ gotovo sigurno jednako konstanti.
    Budu\' ci da to vrijedi za svaki $t_0$, to bi zna\v cilo da na\v s proces jedino mo\v ze biti deterministi\v cka funkcija.
    Stoga nije ideja da su $(X_t)$ nezavisne, ve \' c pogledamo li $(Y_{\frac{m}{n}})$, trebamo zahtjevati da prirasti budu nezavisni.
\end{nap}

\begin{defn}    \label{defn:21.3}
    Stohasti\v cki proces $\niz{X_t}{t \geq 0}$ s vrijednostima u $\real$ je \emph{L\' evyjev proces} ako vrijedi:
    \begin{enumerate}[label=(\alph*)]
        \item   \label{defn:21.3.1}
        Za svaki $n \in \nat$ i za svaki izbor $0 \leq t_0 < t_1 \ldots < t_n$ su slu\v cajne varijable
        \begin{equation*}
            X_{t_n} - X_{t_{n - 1}}, X_{t_{n - 1}} - X_{t_{n - 2}}, \ldots, X_{t_1} - X_{t_0}
        \end{equation*}
        nezavisne.
        \item   \label{defn:21.3.2}
        za svaki $h > 0$ i za svaki $t, s \geq 0$ je
        \begin{equation*}
            X_{t + h} - X_t \distJed X_{s + h} - X_s.
        \end{equation*}
        \item   \label{defn:21.3.3}
        za gotovo sve $\omega \in \Omega$ trajektorija
        \begin{equation*}
            t \mapsto X_t (\omega)
        \end{equation*}
        je \cadlag  \: (franc. continue \' a droite, limite \' a gauche - neprekidna zdesna, s limesom slijeva)
        \item   \label{defn:21.3.4}
        $X_0 \equiv 0$.
    \end{enumerate}
\end{defn}

Zbog \ref{defn:21.3.1}, \ref{defn:21.3.2} i \ref{defn:21.3.4} slijedi da je za svaki $n \in \nat$
\begin{equation*}
    X_1 = (X_1 - X_{\frac{n - 1}{n}}) + (X_\frac{n - 1}{n} - X_\frac{n - 2}{n}) + \ldots + (X_\frac{1}{n} - X_0),
\end{equation*}
to jest $X_1$ je suma $n$ nezavisnih jednako distribuiranih slu\v cajnih varijabli.
Dakle $\varphi_{X_1}$ je beskon\v cno djeljiva, pa je
\begin{equation}    \label{jed:21.4}
    \varphi_{X_1} = e^\psi,
\end{equation}
Pri \v cemu je $\psi$ oblika \eqref{jed:21.1}.
Kao i ranije lako se vidi
\begin{equation}
    \varphi_{X_\frac{m}{n}} = e^{\frac{m}{n} \psi},
\end{equation}
a zbog \ref{defn:21.3.3} i teorema neprekidnosti je
\begin{equation}    \label{jed:21.5}
    \varphi_{X_t} = e^{t \psi},
\end{equation}
pa se $\psi$ naziva \emph{karakteristi\v cnim eksponentom} L\' evyjevog procesa (uo\v cimo da je $\psi$ odre\dj en pripadnom trojkom $(\gamma, \sigma^2, \nu)$).
Daleko je slo\v zenije pokazati obrat (i to ovdje ne\' cemo dokazivati) koji dalje fundamentalni teorem

\begin{tm}  \label{tm:21.6}
    \begin{enumerate}[label=(\alph*)]
        \item   \label{tm:21.6.1}
        Ako je $\niz{X_t}{t \geq 0}$ L\' evyjev proces, tada je $\varphi_{X_t}$ beskona\v cno djeljiva, za svaki $t \geq 0$ i vrijedi \eqref{jed:21.5}.
        \item   \label{tm:21.6.2}
        Ako je $\varphi$ beskona\v cno djeljiva karakteristi\v cna funkcija, tada postoji L\' evyjev proces $\niz{X_t}{t \geq 0}$ takav da je $\varphi_{X_1} = \varphi$.
    \end{enumerate}
\end{tm}

\begin{pr}  \label{pr:21.7}
    Neka je $\psi$ zadan sa $(\gamma, 0, 0)$.
    Tada je
    \begin{equation*}
        \varphi_{X_t} (u) = e^{i \gamma t u} \quad \implies \quad X_t = \gamma t
    \end{equation*} 
    \v sto je deterministi\v cko gibanje po pravcu $t \mapsto \gamma t$, $t \geq 0$.
\end{pr}

\begin{pr}  \label{pr:21.8}
    Neka je $\psi$ zadan sa $(0,1,0)$.
    Tada je $\niz{X_t}{t \geq 0}$ L\' evyjev proces za koji vrijedi $X_t \sim N(0, t)$.
    Taj se proces naziva \emph{standardnim Brownovim gibanjem}.
    Neka je $\niz{B_t}{t \geq 0}$ standardno Brownovo gibanje, neka je $\gamma \in \real$ i $\sigma > 0$.
    Definiramo proces $X_t := \sigma B_t + \gamma t$.
    Budu\' ci je $(B_t)$ L\' evyjev proces lako se direktno provjeri da je i $(X_t)$ L\' evyjev proces, te da je
    \begin{equation}    \label{jed:21.9}
        \varphi_{X_t} (u) = e^{\big( i \gamma t u - \frac{\sigma^2 t}{2} u^2 \big)}.
    \end{equation}
    Dakle, svaki L\' evyjev proces koji ima svojstvo $\nu \equiv 0$ se mo\v ze realizirati kao linearna translacija Brownovog gibanja.
\end{pr}

\begin{pr}  \label{pr:21.9}
    Neka je $\lambda > 0$ i $\psi$ je zadana sa $(0,0,\lambda \delta_{\{1\}})$.
    Tada je $\niz{X_t}{t \geq 0}$ L\' evyjev proces za koji vrijedi
    \begin{equation*}
        \varphi_{X_t} (u) = e^{t \lambda (e^{i u} - 1)},
    \end{equation*}
    to jest $X_t$ je Poissonova razdioba s parametrom $\lambda t$, pa se i ovaj proces naziva \emph{Poissonovim procesom s parametrom $\lambda$}.
    Prema tome, za svaki $n \in \nat_0$ vrijedi
    \begin{equation}    \label{jed:21.10}
        \masP (X_t = n) = e^{- \lambda t} \frac{(\lambda t)^n}{n !}.
    \end{equation}
    Budu\' ci da je ovaj proces \cadlag, a mo\v ze poprimiti i njegov prirast (po \ref{defn:21.3.1}) i sam proces samo vrijednosti u $\nat_0$ i kretat \' ce se po cijelobrojnim skokovima.
    Kada se dogodi skok?
    \begin{equation*}
        \masP (X_{t + h} - X_t > 0) = \masP (X_h > 0) = 1 - e^{- \lambda h},
    \end{equation*}
    dakle vrijeme do prvog skoka je uvijek eksponencijalno, s parametrom $\lambda$.
    Budu\' ci da je eksponencijalna razdioba neprekidna, to zna\v ci da je vjerojatnost da se skok dogodi u nekom fiksnom trenutku nula.
    Dakle, skokovi se doga\dj aju u nekim kontinuirano distribuiranim slu\v cajnim momentima vremena.
    Ujedno mo\v ze se pokazati da skokovi moraju biti veli\v cine $1$.
    Ovo sugerira da mjera $\nu$ daje va\v znu informaciju o skokovima.
\end{pr}

\begin{pr}  \label{pr:21.10}
    Neka su familije $\niz{N_t}{t \geq 0}$ i $\niz{Y_n }{n \in \nat}$ nezavisne (u smislu generiraju\' cih $\sigma$-algebri) i neka je $(N_t)$ Poissonov proces s parametrom $\lambda$, a $(Y_n)$ niz nezavisnih jednako distribuiranih slu\v cajnih varijabli s pripadnom p.d.F. $F$.
    Po velikom Kolmogorovljevom teoremu  mogu\' ce je konstruirati takve familije na istom vjerojatnosnom prostoru $\vjerojatnosniProstor$.
    Definiranom proces $\niz{X_t}{t \geq 0}$ pomo\' cu
    \begin{equation}    \label{jed:21.11}
        X_t := \suma{j = 1}{N_t} Y_j, \quad t \geq 0.
    \end{equation}
    Za svaki $\omega$ je $N_t(\omega) \in \nat_0$ (za $N_t (\omega) = 0$ sumu u \eqref{jed:21.11} interpretiramo kao $0$), pa formalno ima smisla.
    O\v cito je i $X_0 \equiv 0$.
    Uo\v cimo da $(X_t)$ ima skokove to\v cno kada se dogode skokovi za $(N_t)$, samo je veli\v cina skoka $Y_j$ umjesto $1$.
    Posebno, sve trajektorije su \cadlag.
    Budu\' ci da je $(N_t)$ rastu\' c, te su $(N_t)$ i $(Y_n)$ nezavisne i $(Y_n)$ niz nezavisnih slu\v cajnih varijabli, iz
    \begin{equation*}
        X_{t + h} - X_t = \suma{j = N_t}{N_{t+ h}} Y_j, \quad X_t - X_0 = \suma{j = 1}{N_t} Y_j,
    \end{equation*}
    se lako dobije nezavisnost prirasta.
    Nadalje,
    \begin{equation}    \label{jed:21.12}
        \begin{aligned}
            \varphi_{X_{t + h} - X_t} (u) &= \masE \Big[ e^{i u (X_{t + h} - X_t)} \Big] =
            \begin{psmallmatrix}
                N_{t + h} - N_t \sim N_h\\
                \textnormal{i nezavisnost od } (Y_n)
            \end{psmallmatrix} = \masE \Big[ e^{i u \suma{j = 1}{N_h} Y_j} \Big]\\
            &= \masE \Big[ \suma{n = 0}{\infty} \karaktFja_{\{ N_h = n\}} \cdot \produkt{j = 1}{n} e^{i u Y_j} \Big] =
            \begin{psmallmatrix}
                \textnormal{nezavisnost}
            \end{psmallmatrix}\\
            &= \suma{n = 0}{\infty} \masP (N_h = n) \cdot [ \varphi_F (u) ]^n = \suma{n = 0}{\infty} \frac{(\lambda h)^n}{n!} e^{-\lambda h} [\varphi_F (u)]^n\\
            &= e^{-\lambda h} \cdot e^{\lambda h \varphi_F (u)} = e^{\lambda h \int \big( e^{i u x} - 1 \big) \: d F (x)}.
        \end{aligned}
    \end{equation}
    Odavde slijedi da je $\niz{X_t}{t \geq 0}$ L\' evyjev proces i $\psi$ je odre\dj en s
    \begin{equation}    \label{jed:21.13}
        \psi (u) = \int\limits_\real \lambda \cdot \big( e^{i u x} - 1 \big) \: d F (x),
    \end{equation}
    to zna\v ci da se $\nu$ dobije tako da se oduzme $\masP_F (\{0\}) \geq 0$, pa je $\nu$ kona\v cna mjera na $\real \setminus \{0\}$ (zapravo subvjerojatnosna) i integral
    \begin{equation*}
        \int\limits_{\{ 0 < |x| < 1 \}} i u x \: d \nu (x) = u i \int\limits_{\{ 0 < |x| < 1 \}} x \: d \nu (x)
    \end{equation*}
    i on "izlazi" kao $\gamma$ da bi se dobila formula \eqref{jed:21.1}.
    Uo\v cimo da ovaj primjer jasno pokazuje smisao mjere $\nu$ vezano uz skokove, $\nu$ nazivamo \emph{L\' evyjevom mjerom}.
    Proces $\niz{X_t}{t \geq 0}$ zovemo \emph{slo\v zenim Poissonovim procesom}.
\end{pr}

Podsjetimo se da je proces neprekidan ako su mu gotovo sve trajektorije neprekidne funkcije.
Mo\v ze se pokazati sljede\' ci teorem.

\begin{tm}  \label{tm:21.14}
    L\' evyjev proces $\niz{X_t}{t \geq 0}$ je neprekidan ako i samo ako je $\nu \equiv 0$.
    Posebno, Brownovo gibanje je neprekidan proces.
\end{tm}

\begin{nap} \label{nap:21.15}
    \v Sto mo\v zemo re\' ci o diferencijabilnosti trajektorija Brownovog gibanja?
    Budu\' ci je $\niz{B_t}{t \geq 0}$ L\' evyjev proces, slijedi
    \begin{equation*}
        \frac{B_{t + h} - B_t}{h} \distJed \frac{B_h}{h}.
    \end{equation*}
    Uo\v cimo da je
    \begin{equation*}
        \Var \Big( \frac{B_h}{h} \Big) = \frac{1}{h^2} \Var B_h = \frac{h}{h^2} = \frac{1}{h} \xrightarrow[h \searrow 0]{} +\infty,
    \end{equation*}
    \v sto sugerira da trajektorije Brownovog gibanja nisu diferencijabilne.
    Mo\v ze se rigorozno pokazati da su nigdje diferencijabilne.
\end{nap}

\begin{nap} \label{nap:21.16}
    Obratimo pa\v znju na jedan fundamentalni aspekt.
    Vidjeli smo da se niz nezavisnih jednako distribuiranih slu\v cajnih varijabli $\niz{X_n}{n \in \nat}$ ne generalizira prirodno na nezavisnu familiju $\niz{X_t}{t \geq 0}$, ve\' c na L\' evyjeve procese.
    Posebno,
    nezavisne familije $(X_t)$ s velikim nivoom neprekidnosti mogu samo biti trivijelne i nisu osobito interesantne.
    To zna\v ci da su u prirodnim situacijama $X_t$ i $X_s$ (na primjer) zavisne slu\v cajne varijable.
    Drugim rje\v cima \emph{zavisnost} i razni oblici zavisnosti \' ce igrati fundamentalnu ulogu u prou\v cavanju stohasti\v ckih procesa.
    \v Sto sugeriraju L\' evyjevi procesi?
    Kako je $X_{t + h} - X_t$ nezavisno od $X_t - X_{t - \varepsilon_1}$, $X_{t - \varepsilon_1} - X_{t - \varepsilon_2}$, $\ldots$, $X_{t - \varepsilon_n} - X_0$, za sve mogu\' ce izbore rastu\' cih $\varepsilon_1, \ldots, \varepsilon_n$.
    Iz toga direktno dobijemo da je $X_{t + h} - X_t$ nezavisan od $\sigma$-algebre $\indSigAlg{X_s}{0 \leq s \leq t} = \famF_t$.
    Dakle kod L\' evyjevih procesa je "budu\' cnost zavisna od sada\v snjosti" ($X_{t + h}$ ovisi o $X_t$), ako je "prirast od sada\v snjosti do budu\' cnosti" nezavisno od "pro\v slosti".
    U srednjem, taj prirast mo\v ze doprinjeti "ni\v sta novo" kao kod, na primjer, Brownovog gibanja
    \begin{equation*}
        \masE [X_{t + h} - X_t] = \masE [X_h] = 0,
    \end{equation*}
    ili mo\v ze "narasti" ili "pasti" u srednjem, na primjer, za Poissonov proces
    \begin{equation*}
        \masE [X_{t + h} - X_t] = \masE [X_h] = \lambda h > 0.
    \end{equation*}
\end{nap}