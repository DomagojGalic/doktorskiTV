% Skripta iz kolegija teorija vjerojatnosti sa doktorskog studija
% 2018/2019

\documentclass[a4paper,twoside,12pt]{report}

% paketi
\usepackage{classNotes}
\usepackage[languagenames,fixlanguage,croatian]{babelbib}
\usepackage[pdftex]{hyperref}

\usepackage{txfonts} 
\usepackage[pdftex]{graphicx}
\usepackage{enumitem}

\usepackage{graphicx}
\usepackage{subcaption}
\usepackage{float}
\usepackage{tikz}
\usetikzlibrary{matrix}
\usetikzlibrary{patterns}

\usepackage{fourier}
\usepackage{heuristica}

\usepackage{amsmath}
\usepackage{cancel}

%\usepackage[lite]{mtpro2}

%\usepackage{amsthm}

%\usepackage{mathtools}

% naredbe
\newcommand{\partitive}[1]{\mathcal{P}({#1})}
\newcommand{\mjera}[1]{\mu({#1})}
\newcommand{\vjeroj}[1]{\mathbb{P}({#1})}
\newcommand{\ocek}[1]{\mathbb{E}({#1})}
\newcommand{\unija}[2]{\bigcup\limits_{#1}^{#2}}
\newcommand{\presjek}[2]{\bigcap\limits_{#1}^{#2}}
\newcommand{\suma}[2]{\sum\limits_{#1}^{#2}}
\newcommand{\produkt}[2]{\prod\limits_{#1}^{#2}}
\newcommand{\dirProd}[2]{\bigotimes\limits_{#1}^{#2}}
\newcommand{\card}[1]{\textnormal{card}({#1})}

\newcommand{\notimpliedby}{\centernot\impliedby}

\newcommand{\segment}[2]{[{#1}, \:{#2}]}
\newcommand{\niz}[2]{({#1} \: | \: {#2})}
\newcommand{\indFamilija}[2]{\{{#1} \: | \: {#2}\}}
\newcommand{\sigAlg}[1]{\sigma({#1})}
\newcommand{\prsten}[1]{\mathcal{R}({#1})}
\newcommand{\indSigAlg}[2]{\sigma({#1} \: | \: {#2})}
\newcommand{\dynk}[1]{\mathcal{D}({#1})}

\newcommand{\skup}[2]{\{{#1} \: | \: {#2}\}}
\newcommand{\lijInt}[2]{\langle {#1}, \: {#2}]}
\newcommand{\desInt}[2]{[ {#1}, \: {#2} \rangle}
\newcommand{\obInt}[2]{\langle {#1}, \: {#2} \rangle}
\newcommand{\urePar}[2]{({#1}, \: {#2})}

\newcommand{\restr}[2]{\left. \kern- \nulldelimiterspace {#1}
    \vphantom{\big|} \right|_{#2}}
\newcommand{\praslika}[1]{{#1}^{\leftarrow}}
\newcommand{\distJed}{\overset{d}{=}}

%skupovi
\newcommand{\N}{\mathbb{N}}
\newcommand{\nat}{\mathbb{N}}
\newcommand{\Z}{\mathbb{Z}}
\newcommand{\Q}{\mathbb{Q}}
\newcommand{\R}{\mathbb{R}}
\newcommand{\real}{\mathbb{R}}
\newcommand{\extReal}{\overline{\mathbb{R}}}
\newcommand{\Pp}{\mathbb{P}}
\newcommand{\Ee}{\mathbb{E}}

\newcommand{\F}{\mathcal{F}}
\newcommand{\E}{\mathcal{E}}
\newcommand{\hh}{\mathcal{H}}
\newcommand{\ttop}{\mathcal{T}}

\newcommand{\famU}{\mathcal{U}}
\newcommand{\famV}{\mathcal{V}}
\newcommand{\famA}{\mathcal{A}}
\newcommand{\famB}{\mathcal{B}}
\newcommand{\famC}{\mathcal{C}}
\newcommand{\famD}{\mathcal{D}}
\newcommand{\famE}{\mathcal{E}}
\newcommand{\famF}{\mathcal{F}}
\newcommand{\famG}{\mathcal{G}}
\newcommand{\famH}{\mathcal{H}}
\newcommand{\famI}{\mathcal{I}}
\newcommand{\famK}{\mathcal{K}}
\newcommand{\famN}{\mathcal{N}}
\newcommand{\famO}{\mathcal{O}}
\newcommand{\famS}{\mathcal{S}}
\newcommand{\famT}{\mathcal{T}}

\newcommand{\masE}{\mathbb{E}}
\newcommand{\masP}{\mathbb{P}}
\newcommand{\masO}{\mathbb{O}}

\newcommand{\karaktFja}{\mathbb{1}}

\newcommand{\Var}{\mathrm{Var}}
\newcommand{\Cov}{\mathrm{Cov}}
\newcommand{\diam}[1]{\mathrm{diam}(#1)}

\newcommand{\simetr}{\mathrm{sim}}
\newcommand{\iid}{\mathrm{i.i.d.}}
\newcommand{\io}{\mathrm{i.o.}}

\newcommand{\id}{\mathrm{id}}

\newcommand{\borel}[1]{\mathcal{B}_{#1}}

\newcommand{\izmjerivProstor}{(\Omega, \: \F)}
\newcommand{\topProstor}{(X, \: \mathcal{T})}
\newcommand{\prostorMjere}{(\Omega, \: \F, \: \mu)}
\newcommand{\vjerojatnosniProstor}{(\Omega, \: \F, \: \mathbb{P})}

\newcommand{\norma}[1]{\left\lVert {#1} \right\rVert}

\newcommand{\nvektor}[1]{({#1}_1, \ldots, {#1}_n)}
\newcommand{\dvektor}[1]{({#1}_1, \ldots, {#1}_d)}

\newcommand{\nBezZagVekt}[1]{{#1}_1, \ldots, {#1}_n}
\newcommand{\dBezZagVekt}[1]{{#1}_1, \ldots, {#1}_d}

\DeclarePairedDelimiter\ceil{\lceil}{\rceil}
\DeclarePairedDelimiter\floor{\lfloor}{\rfloor}

\DeclareMathOperator{\sign}{sign}

\newcommand{\conv}[1]{\mathrm{Conv}_{#1}}


% renderiranje tikz pattern-a
\pgfdeclarepatternformonly{south west lines}{\pgfqpoint{-0pt}{-0pt}}{\pgfqpoint{3pt}{3pt}}{\pgfqpoint{3pt}{3pt}}{
        \pgfsetlinewidth{0.4pt}
        \pgfpathmoveto{\pgfqpoint{0pt}{0pt}}
        \pgfpathlineto{\pgfqpoint{3pt}{3pt}}
        \pgfpathmoveto{\pgfqpoint{2.8pt}{-.2pt}}
        \pgfpathlineto{\pgfqpoint{3.2pt}{.2pt}}
        \pgfpathmoveto{\pgfqpoint{-.2pt}{2.8pt}}
        \pgfpathlineto{\pgfqpoint{.2pt}{3.2pt}}
        \pgfusepath{stroke}}

% i.o. -> infinately often
\begin{document}

    %\title{Teorija vjerojatnosti - bilje\v ske}
    %\author{Domagoj Gali\' c}

    %\maketitle

    \tableofcontents

    \part{Uvod}

    %%%%%%%%%%%%%%%%%%%%%%%%%%%%%
    %%  Vjerojatnosni prostor  %%
    %%%%%%%%%%%%%%%%%%%%%%%%%%%%%
    
    % o vjerojatnosnom prostoru


\chapter{Vjerojatnosni prostor}

Sustavno razmi\v sljanje o vjerojatnosnim pojmovima po\v cinje u 16. stolje\' cu nove ere u Italiji (Cardano, Tartaglia). Ve\' c do 19. stolje\' ca razvijaju se slo\v zenije ideje, kao uvijetna vjerojatnost, Laplace-ov model, te prvi grani\v cni teoremi. Odnos intuitivnog poimanja vjerojatnost i pripadnog matemati\v ckog pojma suptilno je pitanje.
Ilustrirajmo jedan aspekt tog odnosa uspore\dj uju\' ci pitanje tipa "Koja je vjerojatnost da na Marsu postoji \v zivot?", s pitanjem "Koja je vjerojatnost da subotom izme\dj u 12 i 13 sati autoputom Zagreb - Karlovac pro\dj e barem tisu\' cu automobila?".
U prvom slu\' caju htjeli bismo pridru\v ziti odre\dj enu "mjeru" (dakle broj) stupnju vjerovanja da na Marsu postoji \v zivot.
To vodi na ideju tako zvane \emph{subjektivne vjerojatnosti} (Keynes 1921.).
Uo\v cimo da je uz prvo pitanje vrlo te\v sko vezati neki pokus, a nemogu\' ce ideju "ponavljanja pokusa".
S druge strane, u drugom slu\v caju mo\v zemo jednostavno provoditi mjerenja svake subote (ponavljanje pokusa) i formirati "distribuciju" vezanu uz taj slu\v cajni pokus.
Ovo nas vodi na takozvani \emph{Objektivni pristup} koji se tako\dj er razvija u prvoj polovici 20. stolje\' ca (von Mises 1928. i Kolmogorov 1933.)
Iako ovaj pristup konceptualno ograni\v cavaju\' ce djeluje na teoriju, matemati\v cnost ovog pristupa postaje klju\v cnim razlogom njegove op\' ce prihva\' cenosti.
Osobito je za nas va\v zan Kolmogorovljev pristup teoriji vjerojatnosti, koji ne razmi\v slja o tome kako pojedinom stvarnom doga\dj aju dati odre\dj enu dati odre\dj enu vjerojatnost $p \in [0, \: 1]$, ve\' c uz pretpostavku da takvi brojevi postoje, razvija pravila o njihovim odnosima.
Time sama priroda doga\dj aja postaje sekundarna u toj teoriji (jasno, u primjenama je i dalje od prvorazredne va\v znosti), a bitna postaje analiza "distribucije" i njenih pravila.
S druge strane, za bolje razumjevanje koncepata teorije, a osobito u njenim primjenama, \v cest uzimamo u obzir konkretne primjere.
Primjeri iz hazardnih igara (od arapskog al-zahr za igra\v cu kocku) \v cesto donose boljem razumijevanju same teorije.
Na primjer, jako je jednostavno modelirati jedno bacanje simetri\v cne kovanice ($50\%$ \v sanse za ishod glave i $50\%$ za ishod pisma).
Dvije kovanice ve\' c mogu predstavljati problem.

\begin{pr}[D'Alambert 1754] \label{pr:1.1}
    Kolika je vjerojatnost da prilikom jednog bacanja dvije simetri\v cne kovanice bar jednom "padne pismo"?

    D'Alambert ka\v ze da imamo 3 mogu\' cnosti (dvije glave,
    dva pisma i po jedno od svakoga), od kojih su za nas povoljne
    dvije, stoga je vjerojatnost $\frac{2}{3}$!?
    
    Ovaj primjer pokazuje koliko je va\v zno precizno odrediti \emph{osnovne (elementarne) ishode}.
    Uzmemo li 2G, 2P i 1G1P kao osnovne ishode, onda moramo dobro razmisliti koje su njihove vjerojatnosti (nisu $\frac{1}{3}$). Zamislimo da smo jednu kovanicu obojili.
    Jesmo li time promjenili pokus?
    Nismo.
    Sada ishode pokusa mo\v zemo prikazati kao ure\dj ene parove, ako obojanu kovanicu stavio na prvo mjesto, vidimo da su mogu\' ci ishodi:
    (G, G), (G, P), (P, G), (P, P)
    
    Odakle vidimo da je tra\v zena vjerojatnost zapravo $\frac{3}{4}$.
\end{pr}

Prvi va\v zan objekt je takozvani \emph{prostor elementarni doga\dj aja} (sample space).
U jeziku teorije skupova dovoljno je zahtjevati da to bude neprazan skup $\Omega \neq \varnothing$.

U primjeru \ref{pr:1.1} imamo $ \Omega = \{(G, \: G), (G, \: P), (P, \: G), (P, \: P)\}$.
\v Sto su doga\dj aji?
Na primjer doga\dj aj iz primjera \ref{pr:1.1} je opisan sa \emph{"palo je barem jedno pismo"}, to jest, to je podskup $A = \{(G, \: P),(P, \: G), (P, \: P)\} \subseteq \Omega$.

Dakle, doga\dj aje je prirodno promatrati kao elemente neke familije $\F \subseteq \partitive{\Omega}$.
Zapravo, najbolje bi bilo uzeti sve podskupove, dakle familiju $\partitive{\Omega}$, kao doga\dj aje.
\v Sto \' ce biti vjerojatnosti?
Ve\' c primjeri sugeriraju da su vjerojatnosti brojevi pridru\v zeni doga\dj ajima.
Dosta je o\v cito da od tih brojeva zahtjevamo bar sljede\' ca svojstva: ako su $A$, $B$ doga\dj aji tada za vjerojatnosti $\vjeroj{A}$, $\vjeroj{B}$ vrijedi:

\begin{equation} \label{jed:1.2}
    0 \leq \vjeroj{A} \leq 1
\end{equation}

\begin{align} \label{jed:1.3}
    \begin{split}
        \vjeroj{\varnothing} &= 0 \\
        \vjeroj{\Omega} &= 1
    \end{split}
\end{align}

\begin{equation} \label{jed:1.4}
    A \cap B = \varnothing \implies \vjeroj{A \cup B} = \vjeroj{A} + \vjeroj{B}
\end{equation}

\begin{pr}[Laplace-ov model] \label{pr:1.5}
    Neka je $\Omega \neq \varnothing$ kona\v can skup.
    Ideja je modelirati slu\v caj u kojem su svi osnovni ishodi \emph{jednako vjerojatni}.
    Lako se vidi da taj zahtjev zajedno sa pravilima \eqref{jed:1.2}, \eqref{jed:1.3} i \eqref{jed:1.4}, nu\v zno vodi na to\v cno jedan model, onaj u kojem je svaki $A \subseteq \Omega$ doga\dj aj i vrijedi:

    \begin{equation} \label{jed:1.6}
        \vjeroj{A} = \frac{\card{A}}{\card{\Omega}}
    \end{equation}
\end{pr}

U ovom primjeru imamo zadane vjerojatnosti $\vjeroj{\{ \omega\}} = \frac{1}{\card{\Omega}}$, za svaki $\omega \in \Omega$ i onda nas pravilo \eqref{jed:1.4} vodi do formule \eqref{jed:1.6}.

To se mo\v ze i poop\' citi, te je svaki model nad kon\v cnim $\Omega$ mogu\' ce opisati na sljede\' ci na\v cin.

\begin{pr} \label{pr:1.7}
    Neka je $\card{\Omega} = n \in \N$.
    Neka su $p_1, \dots p_n$ realni brojevi za koje vrijedi:
    \begin{equation}    \label{jed:1.8}
        p_1, \dots, p_n \in [0, \: 1]
    \end{equation}

    \begin{equation}    \label{jed:1.9}
        \suma{i = 1}{n} p_i = 1
    \end{equation}

    Tada je sa

    \begin{equation} \label{jed:1.10}
        \vjeroj{A} = \suma{\omega_i \in A}{} p_i
    \end{equation}
    dana vjerojatnost na $\partitive{\Omega}$ koja zadovoljava pravila \eqref{jed:1.2}, \eqref{jed:1.3} i \eqref{jed:1.4} i svaka vjerojatnost na $\partitive{\Omega}$ koja zadovoljava \eqref{jed:1.2}, \eqref{jed:1.3} i \eqref{jed:1.4} mo\v ze se prikazati na ovaj na\v cin.
\end{pr}

Primjetimo da se ovaj pristup direktno prenosi i na prebrojive $\Omega$ ako dodamo poop\' cenje pravila \eqref{jed:1.4}:

\begin{equation} \label{jed:1.11}
    (A_n, \: n \in \N) \subseteq \partitive{\Omega}, \; A_i \cap A_j = \varnothing, \; i \neq j \implies \vjeroj{\unija{n}{} A_n} = \suma{n}{} \vjeroj{A_n}
\end{equation}

\begin{zad} \label{zad:1.12}
    Doka\v zi sljede\' ce tvrdnje.
    \begin{enumerate}[label=(\alph*)]
        \item Doka\v zi da \eqref{jed:1.2}, \eqref{jed:1.3} i \eqref{jed:1.11} $\implies$ \eqref{jed:1.4}
        \item Doka\v zi da \eqref{jed:1.2}, \eqref{jed:1.3}, \eqref{jed:1.4} i pravilo:
        \begin{equation} \label{jed:1.13}
            (A_n, \: n \in \N) \subseteq \partitive{\Omega}, \; A_1 \supseteq A_2 \supset \dots, \quad \presjek{n}{} A_n = \varnothing \implies \lim_{n \to \infty} \vjeroj{A_n} = 0
        \end{equation}
        impliciraju \eqref{jed:1.11}
    \end{enumerate}
\end{zad}

%%
%% rješenje zadatka 12
%%

\begin{rj}[\ref{zad:1.12}]
    \begin{enumerate}[label=(\alph*)]
        \item Definiramo niz $\niz{C_n}{n \in \N}$, sa $C_1 = A, \; C_2 = B, \; C_{n + 2} = \varnothing, \; n \in \N$.
            Vidimo da je to niz disjunktnih doga\dj aja.
            Sada vrijedi
            \begin{equation*}
                \vjeroj{\unija{n \in \N}{} C_n} \overset{\eqref{jed:1.11}}{=} \suma{n \in \N}{} \vjeroj{C_n} = \vjeroj{C_1} + \vjeroj{C_2} + \underbrace{\suma{n = 3}{\infty} \vjeroj{C_n}}_{=0}
            \end{equation*}
            Odakle je lako vidjeti:
            \begin{align*}
                \vjeroj{A \cup B \cup \unija{n = 3}{\infty} \varnothing} &= \vjeroj{A \cup B} \\
                &= \vjeroj{A} + \vjeroj{B}
            \end{align*}
        \item Neka je $\niz{A_n}{n \in \N} \subseteq \F$ niz disjunktnih doga\dj aja.
        Definiramo $B_n := \unija{k = n}{\infty} A_k$, te primjetimo da je $B_1 \supseteq B_2 \supseteq \dots$, tako\dj er primjetimo da je $\presjek{n = 1}{\infty} B_n = \varnothing$.
        Naime prepostavimo da je $x \in \presjek{k = n_0}{\infty} B_n$, tada je $x \in B_n \; \forall n \in \N$, to jest, $ \forall n \in \N \; x \in \unija{k = n}{\infty} A_k$.
        Odabreimo neki $n_0 \in \N$, sada $x \in \unija{k = n_0}{\infty} A_k \implies \exists ! \: k_0 \geq n_0$ takav da $x \in A_{k_0}$, jer je $\niz{A_n}{n \in \N}$ familija disjunktnih doga\dj aja, ali tada $x \in A_k$, za $k > k_0 \implies x  \in \unija{k = n_1}{\infty} A_k, \; n_1 > k_0$ stoga $x \in B_n$ za $n > k_0$, \v sto je kontradikcija pa je $\presjek{n = 1}{\infty} B_n = \varnothing$.\\
        Sada imamo
        \begin{align*}
            \vjeroj{\unija{k = 1}{\infty} A_k}
            &= \vjeroj{\unija{k = 1}{n} A_k}
            + \vjeroj{\unija{k = n + 1}{\infty} A_k} \\
            &= \suma{k = 1}{n} \vjeroj{A_k}
            + \vjeroj{\unija{k = n + 1}{\infty} A_k}. 
        \end{align*}
        Pu\v stanjem limesa dobijemo:
        \begin{align*}
            \lim_n \vjeroj{\unija{k = n + 1}{\infty} A_k}
            \overset{\eqref{jed:1.13}}{=}& \lim_n
            \vjeroj{\presjek{n + 1}{\infty}( \unija{k = n + 1}{\infty} A_k)}\\
            =& 0.
            \end{align*}
        I prelaskom na limes dobijemo tra\v zenu tvrdnju.
    \end{enumerate}
\end{rj}

\begin{pr} \label{pr:1.14}
    Neka je $\card{\Omega} = \aleph_0$ i poredajmo $\Omega$ u niz $\{ \omega_1, \: \omega_2, \dots \}$.
    Svaki model koji zadovoljava \eqref{jed:1.2}, \eqref{jed:1.3} i \eqref{jed:1.11} na $\partitive{\Omega}$ mo\v ze se dobiti preko niza realnih brojeva $\{p_n, \: n \in \N \}$ tako da vrijedi:
    \begin{equation}   % možda promjeniti format ovoga? 
        p_n \in [0, \: 1], \quad \forall n \in \N
    \end{equation}

    \begin{equation}
        \suma{n \in \N}{} p_n = 1
    \end{equation}

    \begin{equation} \label{jed:1.17}
        \vjeroj{A} = \suma{\omega_n \in A}{} p_n, \quad A
            \subseteq \Omega
    \end{equation}
\end{pr}

To zna\v ci da prakti\v cki sve situacije u kojima je $\Omega$ najvi\v se prebrojiv mo\v zemo svesti na \emph{vjerojatnosti prostor} $(\Omega, \: \partitive{\Omega}, \: \Pp)$, pri \v cemu je $\Pp$ opisan sa \eqref{jed:1.10} ili \eqref{jed:1.17}

\v Sto ako je $\Omega$ neprebrojiv?

\begin{pr}  \label{pr:1.18}
    Neka je $\Omega = \segment{a}{b} \subseteq \R$ i opi\v simo pokus "slu\v cajnog odabira to\v cke u $\Omega$".
    Prirodno je o\v cekivati da za $\segment{c}{d} \subseteq \segment{a}{b}$ vrijedi:
    \begin{equation} \label{jed:1.19}
        \vjeroj{\segment{c}{d}} = \frac{d - c}{b - a}
    \end{equation}
    Te za doga\dj aj $A \subseteq \Omega$ i $x \in \R$, takve da je
    $x + A \subseteq \Omega$, vrijedi
    \begin{equation} \label{jed:1.20}
        \vjeroj{x + A} = \vjeroj{A}.
    \end{equation}
    Postoji li takav model na $\partitive{\Omega}$
\end{pr}

\begin{tm}
    Ne postoji funkcija $\Pp: \partitive{\segment{a}{b}} \to \R$ koja zadovoljava \eqref{jed:1.2}, \eqref{jed:1.3}, \eqref{jed:1.11}, \eqref{jed:1.19} i \eqref{jed:1.20}.
\end{tm}

%% možda dodati kao zadatak teorem o osnovnim svojstvima mjere?
%% iz njega slijede sve ove stvari + pokazat da iz normiranosti
%% vjerojatnosti slijedi da je vjerojatnost praznog skupa 0

\begin{proof}
    Bez smanjanja op\' cenitosti uzmemo $\segment{a}{b} = \segment{-3}{3}$ i pretpostavimo suprotno, to jest da postoji $\Pp$ s tra\v zenim svojstvima.
    Iz \eqref{jed:1.2}, \eqref{jed:1.3}, \eqref{jed:1.11} po zadatku \ref{zad:1.12} slijedi \eqref{jed:1.4}, te se lako poka\v ze i
    \begin{equation} \label{monotonstVjeroj}
        A \subseteq B \implies \vjeroj{A} \leq \vjeroj{B}
    \end{equation}
    Na $\segment{-1}{1}$ definiramo relaciju ekvivalencije $\sim$ sa $x \sim y \iff x - y \in \Q$.
    Po \emph{aksiomu izbora} postoji $A \subseteq \segment{-1}{1}$, takav da je za svaki $x \in \segment{-1}{1}$ (i pripadni $[x] \in \segment{-1}{1} \big/_{\sim}$), $\card{A \cap [x]} = 1$ (u $A$ smo stavili samo jedan element iz svake klase ekvivalencije).
    Skup $\Q \cap \segment{-2}{2}$ je prebrojiv pa ga mo\v zemo poredati u niz $\niz{q_n}{n \in \N}$. 
    Definirajmo $A_n = q_n + A \subseteq \segment{-3}{3}, \; n \in
    \N$.
    Neka je $H := \unija{n=1}{\infty} A_n \subseteq \segment{-3}{3}$. 
    Iz definicije relacije $\sim$ slijedi da su $A_n$ me\dj usobno disjunktni, pa \eqref{jed:1.11} daje:
    \begin{equation*}
        \vjeroj{H} = \suma{n=1}{\infty} \vjeroj{A_n}
            \overset{\eqref{jed:1.20}}{=} \suma{n=1}{\infty}
            \vjeroj{A}.
    \end{equation*}
    Kada bi $\vjeroj{A} > 0$, onda bi vrijedilo $\suma{n=1}{\infty} \vjeroj{A} = + \infty \implies \vjeroj{H} = +\infty$, \v sto je kontradikcija sa \v cinjenicom da je $\vjeroj{H} \in \segment{0}{1}$, stoga nu\v zno vrijedi $\vjeroj{A} = 0$, pa onda i $\vjeroj{H} = 0$.
    Za svaki $x \in \segment{-1}{1}$, vrijedi $A \cap [x] = \{y\}, \; y \in \segment{-1}{1}, \; x - y \in \Q \cap \segment{-2}{2}$.
    Dakle $x \in q_n + A \subseteq H$, za neki $n$.
    Pa mora vrijediti $\segment{-1}{1} \subseteq H$, pa po \eqref{jed:1.19} i \eqref{monotonstVjeroj} slijedi
    \begin{equation*}
        \frac{1}{3} = \vjeroj{\segment{-1}{1}} \leq \vjeroj{H} = 0,
    \end{equation*}
    \v sto je kontradikcija, stoga funkcija $\Pp$ sa tra\v zenim svojstvima ne postoji.
\end{proof}

Priroda problema je takva da nemo\v zemo odustati niti od jednog od zahtjeva \eqref{jed:1.2}, \eqref{jed:1.3}, \eqref{jed:1.11}, \eqref{jed:1.19}, \eqref{jed:1.20}, stoga moramo odustati od zahtjeva definicije funkcije $\Pp$ na $\partitive{\Omega}$.

\v Zelimo sa\v cuvati osnovne operacije i to nas vodi na sljede\' cu definiciju.

\begin{defn}
    Vjerojatnosni prostor je ure\dj ena trojka $\vjerojatnosniProstor$ koja se sastoji od nepraznog skupa $\Omega$, $\sigma$-algebre $\F$ doga\dj aja na $\Omega$ te funkcije $\Pp: \F \to \R$ koja zadovoljava \eqref{jed:1.2}, \eqref{jed:1.3} i \eqref{jed:1.11}.
    Funkciju $\Pp$ nazivamo \emph{vjerojatnosnom mjerom}.
\end{defn}

\begin{nap} \label{nap:1.24}
    \begin{enumerate}[label=(\alph*)]
        \item Podsjetimo se da je $\F \subseteq \partitive{\Omega}$ $\sigma$-algebra, ako je $\varnothing \in \F$, $\F$ je zatvoren na komplemente i prebrojive unije. Posljedica toga je da je $\F$ zatvorena na kona\v cne i prebrojive upotrebe uobi\v cajenih skupovnih operacija.
        \emph{Doga\dj aji}  su samo oni podskupovi od $\Omega$ koji su elementi $\sigma$-algebre $\F$.
        \item Za $\sigma$-algebru $\famF$ na $\Omega \neq  \varnothing$, dovoljno je tra\v ziti:
        \begin{enumerate}[label=(\roman*)]
            \item $\varnothing \in \famF$
            \item $A \in \famF \implies A^c \in \famF$
            \item $\niz{A_n}{n \in \nat} \subseteq \famF \implies \unija{n \in \nat}{} A_n \in \famF$.
        \end{enumerate}
        \item Ure\dj en par $\izmjerivProstor$ koji se sastoji od nepraznog skupa $\Omega$ i $\sigma$-algebre $\F$ na $\Omega$ nazivamo \emph{izmjerivim prostorom}.
        Ako na $\F$ imamo funkciju $\mu : \F \to \segment{0}{+ \infty}$ koja zadovoljava \eqref{jed:1.11} i $\mjera{\varnothing} = 0$, onda ka\v zemo da je $\mu$ \emph{(pozitivna) mjera}.
        Ako postoje $\niz{E_n}{n \in \N} \subseteq \F$, takvi da je $\unija{n \in \N}{} E_n = \Omega$ i $\mjera{E_n} < +\infty, \; \forall n \in \N$, tada ka\v zemo da je $\mu$ \emph{$\sigma$-kona\v cna (pozitivna) mjera}.
        Ako je $\mjera{\Omega} < +\infty$, tada ka\v zemo da je $\mu$ kona\v cna (pozitivna) mjera.
        Dakle Kolmogorov je definirao vjerojatnosni prostor kao poseban slu\v caj prostora kona\v cne mjere.
        \item Za vjerojatnosnu mjeru nije potrebno zahtjevati $\vjeroj{\varnothing} = 0$.
        Minimalni zahtjevi da bi na izmjerivom prostoru $\izmjerivProstor$ funkcija $\masP: \famF \to \real$ bila vjerojatnost su:
        \begin{enumerate}[label=(\roman*)]
            \item $\vjeroj{A} \geq 0, \quad \forall A \in \famF$
            \item $\vjeroj{\Omega} = 1$
            \item $\niz{A_n}{n \in \nat}$ niz disjunktnih doga\dj aja,
            \begin{equation*}
                \vjeroj{\unija{n \in \nat}{} A_n} = \suma{n \in \nat}{} \vjeroj{A_n}.
            \end{equation*}
        \end{enumerate}
    \end{enumerate}
\end{nap}

\begin{tm}[O osnovnim svojstvima mjere] \label{tm:1.24-1}
    Neka je $\prostorMjere$ prostor mjere, tada vrijedi:
    \begin{enumerate}[label={(\roman*)}]
        \item \label{tm:1.24-1.1}
        \emph{Monotonost}: Ako je $A \subseteq B$, vrijedi 
            $\mjera{A} \leq \mjera{B}$
        \item \emph{Subaditivnost}: Ako je $A \subseteq \unija{n=1}{\infty} A_n$, tada
            vrijedi $\mjera{A} \leq \suma{n=1}{\infty} \mjera{A_n}$
        \item \label{tm:1.24-1.3}
        \emph{Neprekidnost odozdo}: Ako $A_n \nearrow A$, 
            tj. $(A_n)_{n \in \N}$ td. $A_1 \subseteq A_2 \subseteq \dots$ i vrijedi
            $\unija{n \in \N}{} A_n = A$, tada vrijedi $\mjera{A} =
            \lim\limits_{n} \mjera{A_n}$.
        \item \emph{Neprekidnost odozgo}: Ako $A_n \searrow A$, tj. $(A_n)_{n \in \N}$ td.
            $A_1 \supseteq A_2 \supseteq \dots$, ako postoji $n_0 \in \N$ tako da
            $\mjera{A_{n_0}} < \infty$ i vrijedi $\presjek{n \in \N}{} A_n = A$,
            tada vrijedi $\mjera{A} = \lim\limits_{n} \mjera{A_n}$.
    \end{enumerate}
\end{tm}

\begin{proof}
    \begin{enumerate}[label={(\roman*)}]
        \item Neka je $A \subseteq B$, tada je $B = A \dot{\cup} (B \setminus A)$, a sada
            vrijedi $\mjera{B} = \mjera{A} + \mjera{B \setminus A}$, budu\' ci je
            $\mjera{B \setminus A} \geq 0$, vrijedi da je $\mjera{A} \leq \mjera{B}$.

        \item Definiramo $B_1 := A_1, \: B_k := A_k \setminus (\unija{i=1}{k} A_i),
            \: k \geq 1$. Tvrdim da je $(B_n)_{n \in \N}$ niz disjunktnih skupova.
            Neka su $m, \: n \in \N$, takvi da $m \neq n$, bez smanjenja op\' cenitosti
            mo\v zemo pretpostaviti da je $m < n$. Neka je $x \in B_n \implies x \in A_n 
            \land x \notin A_k, \: k < n \implies x \notin B_m$.
            Dakle $B_m \cap B_n = \emptyset.$ Tako\dj er tvrdimo da je $\unija{k=1}{n} A_k
            = \unija{k=1}{n} B_k, \: \forall n n \in \N$.
            Dokaz vr\v simo matemati\v ckom indukcijom po $n \in \N$.
            \begin{enumerate}
                \item[(B)] $A_1 = B_1$.
                \item[(P)] Neka je $n \in \N$, pretpostavimo da vrijedi $\unija{k=1}{n} A_k
                = \unija{k=1}{n} B_k$.
                \item[(K)] $\unija{k=1}{n+1} B_k = (\unija{k=1}{n} B_k) \cup B_{n+1}
                    = (\unija{k=1}{n} A_k) \cup \underbrace{(A_n \setminus
                    \unija{k=1}{n} A_k)}_{\textnormal{po definiciji}}
                    = \unija{k=1}{n+1} A_k$.
            \end{enumerate}
            Po principu matemati\v cke indukcije tvrdnja vrijedi za svaki $n \in \N$.
            Sada vrijedi $\unija{n \in \N}{} A_n = \unija{n \in \N}{} B_n$, primjetimo
            $x \in \unija{n \in \N}{} A_n \implies \exists n_0 \in \N$, takav da
            $x \in A_{n_0} \implies x \in \unija{k=1}{n_0} A_k = \unija{k=1}{n_0} B_k
            \implies x \in \unija{n \in \N}{} B_n$. Obratna inkluzija se dokazuje identi\v
            cno.
            Po pretpostavci je $A \subset \unija{n \in \N}{} A_n = \unija{n \in \N}{} B_n$,
            sada po \ref{tm:1.24-1.1} vrijedi $\mjera{A} \leq
            \mjera{\unija{n \in \N}{} B_n} = \suma{n=1}{\infty} B_n$. Tako\dj er budu\' ci
            vrijedi $B_k \subseteq A_k$, vrijedi i $\mjera{B_k} \leq \mjera{A_k}$, stoga
            imamo $\suma{n=1}{\infty} \mjera{B_n} \leq \suma{n=1}{\infty} \mjera{A_n}$.
            Odavde vidimo da vrijedi:
            \begin{equation*}
                \mjera{A} \leq \suma{n=1}{\infty} A_n.
            \end{equation*}
        \item Definiramo niz $(B_n)_{n \in \N}$ sa $B_1 := A_1, \: B_n = A_n \setminus
            A_{n-1}, \: n > 1$. Tvrdimo da je $(B_n)_{n \in \N}$ niz disjunktnih skupova
            te vrijedi $\unija{k=1}{n} A_k = \unija{k=1}{n} B_k$. Neka $m, \: n \in \N
            \: m \neq n$, bez smanjenja op\' cenitosti mo\v zemo pretpostaviti $m < n$.
            Neka je $x \in B_m \cap B_n, \: x \in B_n \implies x \notin A_{n-1}
            \implies x \in A_k, \: k \leq n-1$, jer je $(A_n)_{n \in \N}$ rastu\' c,
            a kako je $B_m \subseteq A_{n-1}$, \v sto je kontradikcija, dakle vrijedi
            $B_n \cap B_m = \emptyset$. Drugu tvrdnju dokazujemo po indukciji.
            \begin{enumerate}
                \item[(B)] $B_1 = A_1$.
                \item[(P)] Neka je $n \in \N$, tada vrijedi $\unija{k=1}{n} A_k
                    = \unija{k=1}{n} B_k$.
                \item[(K)] $\unija{k=1}{n+1} B_k = (\unija{k=1}{n} A_k) \cup B_{n+1}
                    = (\unija{k=1}{n} A_k) \cap (A_{n+1} \setminus A_n)
                    = \unija{k=1}{n+1} A_k$.
            \end{enumerate}
            Po Principu matemati\v cke indukcije tvrdnja vrijedi za svaki $n \in \N$.
            Vrijedi $A_n = \unija{k=1}{n} A_k$, jer je rije\v c o rastu\' cim nizu,
            dakle vrijedi $\unija{k=1}{n} A_k = \unija{k = 1}{n} B_k = A_n$.
            Sada imamo $\mjera{A} = \mjera{\unija{n \in \N}{} A_n} = \mjera{
                \unija{n \in \N}{} B_n} = \suma{n=1}{\infty} \mjera{B_n}
                = \lim\limits_{n} \suma{k=1}{n} \mjera{B_k} = \lim\limits_{n}
                \mjera{\unija{k=1}{n} B_k} = \lim\limits_{n} \mjera{A_n}$.
        \item Definirajmo niz $(B_n)_{n \in \N}$ sa $B_n := A_{n_0} \setminus
            A_{n_0 + n}$, tvrdimo da vrijedi $B_1 \subseteq B_2 \subseteq \dots$ te
            $\unija{n=1}{\infty} B_n = A_{n_0} \setminus \presjek{n=1}{\infty} A_n$.
            Neka je $x \in B_m = A_{n_0} \setminus A_{n_0+m} \implies x \in A_{n_0}
            \land x \notin A_{n_0+m} \implies x \in A_{n_0} \land x \in (A_{n_0+m})^c$.
            Budu\' ci je $A_{n_0+m} \supseteq A_{n_0+m+k} \implies (A_{n_0+m})^c
            \subseteq (A_{n_0+m+k})^c$.
            Dakle $x \in A_{n_0} \land x \in (A_{n_0+m})^c \implies x \in A_{n_0}
            \land x \in (A_{n_0+m+k})^c \implies x \in A_{n_0} \setminus A_{n_0+(m+k)}
            = B_{m+k}$.
            Posebno za $k=1$ tvrdnja slijedi to jest $B_n \subseteq B_{n+1},
            \forall n \in \N$, pa vrijedi $B_1 \subseteq B_2 \subseteq \dots$.
            Nadalje vrijedi $\unija{n=1}{\infty}
            B_n = A_{n_0} \setminus \presjek{n=1}{\infty} A_{n_0+n}$.
            Budu\' ci je $(A_n)_{n \in \N}$ padaju\' ci niz, onda vrijedi
            $\presjek{n=1}{\infty} A_n = \presjek{n=1}{\infty} A_{n_0+n}$, stoga je
            $\unija{n=1}{\infty} B_n = A_{n_0} \setminus \presjek{n=1}{\infty} A_n$.
            Sada iz tvrdnje \ref{tm:1.24-1.3} vrijedi $\mjera{A_{n_0} \setminus
            \presjek{n=1}{\infty} A_n} = \mjera{\unija{n=1}{\infty} B_n}
            = \lim\limits_{n} \mjera{B_n} = \lim\limits_{n} \mjera{A_{n_0} \setminus
            A_{n_0+n}}$.
            Sada, zbog $\mjera{A_{n_0}} < \infty$, vrijedi $\mjera{A_{n_0}}
            - \mjera{\presjek{n=1}{\infty} A_n} = \lim\limits_{n}
            (\mjera{A_{n_0}} - \mjera{A_{n_0+n}}) = \mjera{A_{n_0}}
            - \lim\limits_{n} \mjera{A_{n_0+n}} = \mjera{A_{n_0}} - \lim\limits_{n}
            \mjera{A_n}$, dakle vrijedi
            \begin{equation*}
                \mjera{\presjek{n=1}{\infty} A_n} = \lim\limits_{n} \mjera{A_n}.
            \end{equation*}
    \end{enumerate}
\end{proof}


    %%%%%%%%%%%%%%%%%%%%%%%%%%%%%%%%%%%%%%%%%%%%%%%%%%
    %%  Jedinstvenost i egzistencija vjerojatnosti  %%
    %%%%%%%%%%%%%%%%%%%%%%%%%%%%%%%%%%%%%%%%%%%%%%%%%%

    % o egzistenciji i vjerojatnosti vjerojatnosne mjere

\chapter{Egzistencija i jedinstvenst vjerojatnosti}

Mo\v zemo li u ovoj poop\' cenoj situacij dobro modelirati primjer
\ref{primjer1.18}? Ovo pitanje nas vodi na pitanje egzistencije
(a time i jedinstvenosti) vjerojatnosti. Budu\' ci je vjerojatnost
poseban slu\v caj mjere, teoremi iz teorije mjere vrijedi i ovdije.
Podsjetimo se.

Najmanja $\sigma$-algebra na nepraznom skupu $\Omega$ je
$\{\varnothing, \Omega\}$, a najve\' ca $\sigma$-algebra je
$\partitive{\Omega}$.
Dakle svaka familija $\E \subseteq \partitive{\Omega}$ je sadr\v zana
u barem jednoj $\sigma$-algebri na $\Omega$.
Lako se vidi da je presjek neprazne familije $\sigma$-algebri ponovo
$\sigma$-algebra, pa je i
\begin{equation} \label{jed:2.1}
    \sigAlg{\E} := \presjek{\shortstack{$\E \subseteq \hh \subseteq
    \partitive{\Omega}$ \\ $\hh$ $\sigma$-algebra}}{} 
    %\presjek{\shortstack{\E \subseteq \hh \subseteq
        %\partitive{\Omega}\\ \hh \: \sigma \textnormal{-algebra}}}{}
        \hh
\end{equation}

tako\dj er $\sigma$-algebra i to najmanje $\sigma$-algebra koja
sadr\v zi $\E$ (ka\v zemo jo\v s da je $\sigma$-algebra
$\sigAlg{\E}$, $\sigma$-algebra generirana s $\E$).
O\v cito je $\sigAlg{\{\Omega\}} = \{ \varnothing, \; \Omega\}$,
te tako\dj er
$\sigAlg{\{A\}} = \{ \varnothing, \; A, \; A^c, \; \Omega \}$.

\begin{zad} \label{zad:2.2}
    Neka je $\E = \indFamilija{\{ \omega \}}{\omega \in \Omega}$.
    Tada je $\sigAlg{\E} =\{ A \subseteq \Omega \: |$ $A$ ili $A^c$
    prebrojiv $\}$. Tada je $\sigAlg{\E} = \partitive{\Omega}$ ako
    i samo ako je $\Omega$ kona\v can ili prebrojiv.
\end{zad}

%
% rješi i nadopiši zadatak
%

Ako je $\topProstor$ topolo\v ski prostor s topologijom $\ttop$, tada
$\sigAlg{\ttop}$ ozna\v cavamo s $\borel{X}$ i nazivamo
$\sigma$-algebrom \emph{Borelovih skupova}.

\begin{pr}  \label{pr:2.3}
    Ako je $X = \R$ ili je $X = \segment{A}{B}$, za $A, \: B \in \R$,
    $A < B$, s euklidskom topologijom, onda se $\borel{X}$ mo\v ze
    prikazati kao $\sigAlg{\E}$, za razne familije $\E$.
    Na primjer za $\E$ mo\v zemo uzeti
    \begin{itemize}
        \item $\mathcal{I} := \skup{\lijInt{a}{b} \cap X}{a < b, \:
            a, \: b \in \R} \bigcup \{\varnothing\}$
        \item $\mathcal{I}_{\Q} := \skup{\lijInt{a}{b} \cap X}{a < b,
            \: a, \: b \in \Q} \bigcup \{ \varnothing \}$
        \item $\mathcal{O}_{\Q} := \skup{\obInt{a}{b} \cap X}{a < b,
            \: a, \: b \in \Q} \bigcup \{ \varnothing \}$
        \item $\mathcal{K} := \skup{\lijInt{-\infty}{b} \cap X}{b
            \in \R}$
        \item $\mathcal{K}_{\Q} := \skup{\lijInt{-\infty}{b}
            \cap X}{b \in \Q}$.
    \end{itemize}
\end{pr}

Sve familije u primjeru \ref{pr:2.3} imaju zanimljivo svojstvo:
\begin{equation}    \label{jed:2.4}
    A, \: B \in \E \implies A \cap B \in \E.
\end{equation}

Svaka neprazna familija koja ima svojstvo \eqref{jed:2.4} naziva se
\emph{$\pi$-sistem}. Svaka familija $\E \subseteq \partitive{\Omega}$,
Koja sadr\v zi $\Omega$, zatvorena je na prave skupovne razlike i
rastu\' ce prebrojive unije naziva se \emph{Dynkinovom klasom}.

Presjek Dynkinovih klasa je Dynkinova klasa. Svaka $\sigma$-algebra
je i $\pi$-sistem i Dynkinova klasa. Posebno $\partitive{\Omega}$
je Dynkinova klasa, pa se kao i u \eqref{jed:2.1} mo\v ze napraviti
najmanja Dynkinova klasa generirana klasom $\E$; oznaka je
$\dynk{\E}$. Iz teorije mjere poznat nam je sljede\' ci zadatak.

\begin{zad}   \label{zad:2.5}
    Neka je $\E$, $\pi$-sistem na $\Omega$.
    Tada je $\sigAlg{\E} = \dynk{\E}$.    
\end{zad}

%
%  rješi zadatak
%

\begin{tm}  \label{tm:2.6}
    Neka je $\izmjerivProstor$ izmjeriv prostor i $\E$ $\pi$-sistem
    na $\Omega$ sa svojstvom $\F = \sigAlg{\E}$.
    Ako su $\mu$ i $\nu$ kona\v cne pozitivne mjere na
    $\izmjerivProstor$ takve da je $\mjera{\Omega}
    = \nu(\Omega)$ i $\restr{\mu}{\E} = \restr{\nu}{\E}$, tada je
    $\mu = \nu$.   
\end{tm}

\begin{proof}
    Neka je $\hh := \skup{A \in \F}{\mjera{A} = \nu(A)}$.
    O\v cito je $\E \subseteq \hh$. S druge strane $\Omega \in \hh$,
    te je $\hh$ zatvoren rastu\' ce prebrojive unije zbog
    neprekidnosti pozitivnih mjera na rastu\' ce unije.
    Za $A, \: B \in \hh, \: A \subseteq B$, zbog kona\v cnosti mjera
    $mu$ i $nu$, vrijedi $\mjera{B \setminus A} = \mjera{B}
    - \mjera{A} = \nu(B) - \nu(A) = \nu(B \setminus A)$.
    Dakle $\hh$ je Dynkinova klasa, pa je $\F = \underbrace{
    \sigAlg{\E} = \dynk{\E}}_{\textnormal{zadatak \ref{zad:2.5}}}
    \subseteq \hh$, to jest $\F = \hh$.
\end{proof}

\begin{kor} \label{kor:2.7}
    Neka je $\izmjerivProstor$ i $\E$ $\pi$-sistem na $\Omega$ sa
    svojstvom $\F = \sigAlg{\E}$. Ako su $\Pp$ i $\overline{\Pp}$
    dvije vjerojatnosti na $\izmjerivProstor$ takve da je
    $\restr{\Pp}{\E} = \restr{\overline{\Pp}}{\E}$, tada je $\Pp
    = \overline{\Pp}$.
\end{kor}

\begin{kor} \label{kor:2.8}
    Neka je $X = \real$ ili $X = \segment{A}{B}$, $A, \; B \in \real, \; A < B$, s euklidskom topologijom.
    Neka su $\masP$ i $\mathbb{O}$ vjerojatnosti na $\urePar{X}{\borel{X}}$.
    Neka je $\famE$ bilokoja od 5 klasa iz primjera \ref{pr:2.3}.
    Ako je $\restr{\masP}{\famE} = \restr{\mathbb{O}}{\famE}$, tada je $\masP = \mathbb{O}$.
\end{kor}

\begin{zad} \label{zad:2.9}
    Neka je $\izmjerivProstor$ izmjeriv prostor i $\E$ $\pi$-sistem
    na $\Omega$ sa svojstvom $\F = \sigAlg{\E}$.
    Ako su $\mu$ i $\nu$ dvije mjere na $\izmjerivProstor$ takve
    da je $\restr{\mu}{\E} = \restr{\nu}{\E}$ i postoji rastu\' ci
    niz skupova $\niz{C_n}{n \in \N} \subseteq \E$ takav da je
    $\unija{n}{}C_n = \Omega$ i $\mjera{C_n} = \nu(C_n) < +\infty, \;
    \forall n \in \N$, tada je $\mu = \nu$.
\end{zad}

\begin{nap} \label{nap:2.10}
    Grubo govore\' ci ovi rezultati pokazuju da je $\pi$-sistem
    dovoljno bogata struktura da bi se osigurala jedinstvenost mjere
    (a time i vjerojatnosti). To je vrlo sna\v zan rezultat,
    budu\' ci da velik broj klasa \v cini $\pi$-sistem, te je lako iz
    svake klase napraviti $\pi$-sistem; uzme se proizvoljna klasa
    $\E \subseteq \partitive{\Omega}$, klasa koja se sastoji iz svih
    kona\v cnih presjeka elementat iz $\E$.

    Prirodno je pitanje mo\v ze li se mjera sa $\pi$-sistema
    pro\v siriti na pripadnu $\sigma$-algebru. Uzmemo li klasu
    $\mathcal{K}$ iz primjera \ref{pr:2.3} i za svaki $a \in \R_+$
    izabermo $\mu_a$ takav da je $\mu_a(A) = a, \; \forall A \in
    \mathcal{K}$, takva \' ce funkcija zadovoljavati
    \eqref{vjerPrav4} na $\mathcal{K}$ (jer na $\mathcal{K}$ nemamo
    niti dva disjunktna skupa). O\v cito se $\mu_a$ ne mo\v ze
    pro\v siriti do mjere na $\borel{\R}$.
    Dakle za egzistenciju  (pro\v sirivanje) mjere trebamo ne\v sto
    bogatiju klasu od $\pi$-sistema.
\end{nap}

Re\' ci cemo da je $\E \subseteq \partitive{\Omega}$ \emph{poluprsten}
na $\Omega$ ako vrijedi:
\begin{enumerate}[label=(\roman*)]
    \item $\varnothing \in \E$
    \item $\E$ je $\pi$-sistem
    \item $A, \: B \in \E, \: A \subseteq B \implies B \setminus
        A \in \E$ je kona\v cna disjunktna unija elemenata iz $\E$.
\end{enumerate}

Osnovni teorem o pro\v sirenju mjere (Caratheodoryjeva konstrukcija)
ka\v ze:

\begin{tm}  \label{tm:2.11}
    Neka je $\E$ poluprsten skupova na nepraznom skupu $\Omega$. Ako
    je funkcija $\nu: \E \to [0, \: + \infty]$ $\sigma$-aditivna
    (u smislu da zadovoljava svojstvo \eqref{vjerPrav4} na $\E$)
    i $\nu(\varnothing) = 0$, tada postoji mjera $\mu$ na $(\Omega,
    \: \partitive{\Omega})$, takva da je $\restr{\mu}{\E} = \nu$.
\end{tm}

\begin{nap} \label{nap:2.12}
    Neka je $\Omega$ neprazan skup, $\E := \{\varnothing\} \cup
    \indFamilija{\{ \omega \}}{\omega in \Omega}$. Tada je $\E$
    poluprsten i svaka funkcija $\nu: \E \to [0, \: +\infty]$
    za koju je $\nu(\varnothing) = 0$ je $\sigma$-aditivna;
    preciznije potpuno je opisana familijom "brojeva"
    $\nu(\{ \omega \}) \in [0, \: +\infty], \; \omega \in \Omega$.

    Po zadatku \ref{zad:2.2} i teoremu \ref{tm:2.11} postoji mjera
    $\mu$ na $\sigAlg{\E}$ koja je pro\v sirenje od $\nu$ i za svaki
    $A \subseteq \Omega$, $A$ prebrojiv je $\mjera{A}
    = \suma{\omega \in A}{} \nu(\{ \omega \})$.

    ako je i $\Omega$ prebrojiv, onda je ovim pro\v sirenjem
    jedinstveno odre\dj ena mjera $\mu$ na $\partitive{\Omega}$.
    U su\v stini situacija je sli\v cna kao i u primjeru
    \ref{primjer7} i primjeru \ref{primjer14}.
    \v Sto ako je $\Omega$ neprebrojiv?
    %
    % provedi detaljniju analizu
    %
    Pogledajmo dva ekstremna slu\v caja:
    \begin{enumerate}[label=(\roman*)]
        \item $\nu(\{ \omega \}) = 0, \; \forall \omega \in \Omega$.
            Uzmemo bilo koji $c \in \segment{0}{+\infty}$ i defnirajmo
            $\mu_c$ pomo\' cu
            \begin{equation*}
                \mu_c(A) = 
                \begin{cases}
                    0, &A \;\; \textnormal{prebrojiv}\\
                    c, &A^c \;\; \textnormal{prebrojiv}
                \end{cases}
            \end{equation*}
            Svaki $\mu_c$ je pro\v sirenje od $\nu$. Uo\v cimo da
            uvijeti iz zadatka \ref{zad:2.9} nisu ispunjeni pa nema
            jedinstvenosti pro\v sirenja. Ako je $\Omega
            = \segment{a}{b}, \; \sigAlg{\E}
            \subsetneq \borel{\Omega}$ i pro\v sirenje sa $\E$ na
            $\borel{\Omega}$ ne vodi ka nekoj konkretnoj mjeri.
        \item Ako je $\nu(\{\omega\}) > 0, \; \forall \omega \in
            \Omega$ Onda postoji $\varepsilon > 0$ takav da za
            neprebrojivo $\omega$ iz zadanog neprebrojivog skupa
            imamo $\nu(\{\omega\}) \geq \varepsilon$.
            Posebno, to zna\v ci da je $\mjera{A} = +\infty$ za svaki
            $A$, za koji je $A^c$ prebrojiv. Ponovo ne osobito
            korisna konstrukcija.
    \end{enumerate}
\end{nap}

U slu\v caju $X = \R$ ili $X = \segment{A}{B}$, kao i u drugim
neprebrojivim slu\v cajevima korisniji je sljede\' ci postupak.
Neka je $\mu$ (pozitivna) mjera na $(\R, \: \borel{\R})$, takva da je
$\mjera{K} < +\infty$, za svaki kompaktan skup $K \subseteq \R$
(posebno, takva mjera je \emph{regularna}). Budu\' ci da je za
$a < 0$ (i za $b \geq 0$) skup $\lijInt{a}{0}$ ($\lijInt{0}{b}$),
sadr\v zan u kompaktu, sljede\' ca definicija opisuje neopdaju\' cu
funkciju $F_{\mu}: \R \to \R$ danu sa
\begin{equation}    \label{jed:2.13}
    F_{\mu}(x) :=
        \begin{cases}
            \mjera{\lijInt{0}{x}}, &x > 0\\
            0,   &x = 0\\
            - \mjera{\lijInt{x}{0}}, &x < 0.
        \end{cases}
\end{equation}
Uo\v cimo da je za svaki $a, \: b \in \R, \: a < b$,
\begin{equation}    \label{jed:2.14}
    \mjera{\lijInt{a}{b}} = F_{\mu}(b) - F_{\mu}(a).
\end{equation}

\begin{zad} \label{zad:2.15}
    Doka\v zite da je $F_{\mu}$ neopadaju\' ca neprekidna zdesna i da
    postoje (u skupu $\overline{\R}$).
    \begin{align*}
        F_{\mu}(-\infty) :=& \lim_{x \searrow -\infty} F_{\mu}(x)\\
        F_{\mu}(+\infty) :=& \lim_{x \nearrow +\infty} F_{\mu}(x).
    \end{align*}
\end{zad}

Neka je sada $F$ funkcija sa svojstvima iz zadatka \ref{zad:2.15}.
Pomo\' cu \eqref{jed:1.14} definiramo "mjeru" $\mu$ na klasi
$\mathcal{I}$ iz primjera \eqref{pr:2.3}. Nije te\v sko pokazati
da je $\mu$ $\sigma$-aditivna na $\mathcal{I}$, pa prema teoremu
\ref{tm:2.11} i zadatku \ref{zad:2.9} postoji jedinstveno
pro\v sirenje od $\mu$ na $\borel{\R}$ koje daje regularnu mjeru
$\mu$. Za funkciju $F(x) = x$ ova konstrukcija daje tako zvanu
\emph{Lebesgue-ovu mjeru} $\lambda$ \v cija restrikcija na
$\borel{\segment{0}{1}}$ zadovoljava primjer \ref{primjer1.18}.
Uo\v cimo da funkcije s istim prirastima \eqref{jed:1.14}
daju istu mjeru.



    
    %%%%%%%%%%%%%%%%%%%%%%%%
    %%  Slucajni elmenti  %%
    %%%%%%%%%%%%%%%%%%%%%%%%

    % poglavlje o slučajnim elementima
\usetikzlibrary{arrows,chains,matrix,positioning,scopes}

\chapter{Slu\v cajni elementi} \label{poglavlje3}

Neka su $D$, $K$ neprazni podskupovi i $f: D \to K$ preslikavanje.
Postoji bitna razlika izme\dj u slika i inverznih slika kada govorimo o skupovnim operacijama. Ako je $*$ bilo koja od operacija $\cup, \: \cap, \: \setminus$, a $B_1, \: B_2 \subseteq K$, tada je
\begin{equation} \label{jed:3.1}
    \praslika{f} (B_1 * B_2) = \praslika{f}(B_1) * \praslika{f}(B_2),
\end{equation}
ali ako su $A_1, \: A_2 \subseteq D$, tada je
\begin{align*}
    f(A_1 \cap A_2) \subsetneq& f(A_1) \cap f(A_2) \\
    f(A_1 \setminus A_2) \supsetneq& f(A_1) \setminus f(A_2). 
\end{align*}
Posebno to zna\v ci da ako je $\famD \subseteq \partitive{D}$ $\sigma$-algebra na $D$, $f(\famD) := \skup{f(A)}{A \in \famD}$ nije nu\v zno $\sigma$-algebra na $K$.
S druge strane iz \eqref{jed:3.1} (i sli\v cnih relacija za prebrojive operacije) direktno slijedi:

Ako je $\famK \subseteq \partitive{K}$ "struktura", tada je i
\begin{equation} \label{jed:3.2}
    \praslika{f}(\famK) := \skup{\praslika{f}(B)}{B \in
        \famK}
\end{equation}
tako\dj er "struktura", kao i ako je $\famD \subseteq \partitive{D}$, tada je i
\begin{equation} \label{jed:3.3}
    \skup{B \in \partitive{K}}{\praslika{f}(B) \in \famD},
\end{equation}
tako\dj er "struktura"; pri \v cemu pod "struktura" mislimo $\pi$-sistem, poluprsten, prsten, Dynkinova klasa, algebra, $\sigma$-prsten ili $\sigma$-algebra.

Od posebnog su interesa preslikavanja koja po\v stuju odre\dj en
izbor $\sigma$-algebri.
\begin{defn}    \label{defn:3.3-1}
    Ako su $(D, \: \famD)$ i $(K, \: \famK)$ izmjerivi prostori i $f: D \to K$ preslikavanje, tada ka\v zemo da je $f$ \emph{izmjerivo} (ili \emph{$(\famD, \: \famK)$-izmjerivo}) ako je
    \begin{equation*}
        \praslika{f}(\famK) \subseteq \famD.
    \end{equation*}
\end{defn}

Uo\v cimo da iz \eqref{jed:3.2} slijedi da je $\praslika{f}(\famK)$ uvijek $\sigma$-algebra i da je to najmanja $\sigma$-algebra na $D$ u odnosu na koju je $f$ izmjeriva (u odnosu na $\famK$).
Zato ka\v zemo da je $\praslika{f}(\famK)$ \emph{$\sigma$-algebra generirana sa $f$} i \v cesto ju ozna\v cavamo sa \emph{$\sigAlg{f}$}.
Dakle $f$ je izmjeriva ako i samo ako vrijedi:
\begin{equation*}
    \sigAlg{f} \subseteq \famD.
\end{equation*}
Ovaj pojam se lako poop\' ci.
Neka je $\indFamilija{(K_t, \: \famK_t)} {t \in T}$ proizvoljna indeksirana familija izmjerivih prostora i neka za svaki $t \in T$ imamo preslikavanje $f_t : D \to K_t$.
Tada je  $\indSigAlg{f_t}{t \in T} := \sigAlg{\unija{t \in T}{} \sigAlg{f_t}}$ \emph{$\sigma$-algebra generirana familijom $\indFamilija{f_t}{t \in T}$} i to je najmanja $\sigma$-algebra na $D$ u odnosu na koju su sva preslikavanja $f_t$ izmjeriva.

Ako su $(X, \: \famU)$ i $(Y, \: \famV)$ topolo\v ski prostori, onda $(\borel{X}, \: \borel{Y})$-izmjerivo preslikavanje $f: X \to Y$ nazivamo \emph{Borelovim preslikavanjem}.
Podsjetimo se da je $f: X \to Y$ neprekidno ako je $\praslika{f}(\famV) \subseteq \famU$.
Kakva je veza izme\dj u ovih pojmova?

\begin{lm}  \label{lm:3.4}
    Ako je $f: D \to K$ preslikavanje i $\famC \subseteq \partitive{K}$, tada je $\sigAlg{\praslika{f}(\famC)} = \praslika{f} (\sigAlg{\famC})$.
\end{lm}

\begin{proof}
    Po \eqref{jed:3.2} $\praslika{f}(\sigAlg{\famC})$ je $\sigma$-algebra, koja o\v cito sadr\v zi $\praslika{f}(\famC)$, \v sto povla\' ci $\sigAlg{\praslika{f}(\famC)} \subseteq \praslika{f}(\sigAlg{\famC})$.
    Sada po \eqref{jed:3.3} slijedi da je $\skup{B \in \partitive{K}} {\praslika{f}(B) \in \sigAlg{\praslika{f}(\famC)}}$ je $\sigma$-algebra, koja o\v cito sadr\v zi $\famC$, pa onda nu\v zno mora sadr\v zavati i $\sigAlg{\famC}$.
    Slijedi da je $\praslika{f}(\sigAlg{\famC}) \subseteq \sigAlg{\praslika{f}(\famC)}$.
\end{proof}

Zbog $\praslika{f}(\borel{Y}) = \praslika{f}(\sigAlg{\famV}) = \sigAlg{\praslika{f}(\famV)} \subseteq \sigAlg{\famU} = \borel{X}$, slijedi:

\begin{kor} \label{kor:3.5}
    Ako je $f: (X, \: \famU) \to (Y, \: \famV)$ neprekidna, tada je $f$ Borelova.
\end{kor}

Uo\v cimo nadalje da direktno iz definicije slijedi:

\begin{lm}  \label{lm:3.6}
    Neka su $(A, \: \famA)$, $(B, \: \famB)$ i $(C, \: \famC)$ izmjerivi prostori. Ako su $f: A \to B$ i $g:B \to C$ izmjeriva preslikavanja, tada je izmjeriva i $g \circ f$.
\end{lm}

\begin{defn}    \label{defn:3.7}
    Neka je $\vjerojatnosniProstor$ vjerojatnosni prostor i $(E, \: \E)$ izmjeriv prostor. Preslikavanje $f: \Omega \to E$ je \emph{slu\v cajni element} (\emph{s vrijednostima u $E$}) ako je $f$
    $(\F, \: \E)$-izmjerivo.
\end{defn}

\begin{pr}  \label{pr:3.8}
    Neka je $\vjerojatnosniProstor$ vjerojatnosti prostor i $(E, \: \E) := (\R, \: \borel{\extReal})$, pri \v cemu je $\extReal := \R \cup \{-\infty, \: +\infty\}$, s odgovaraju\' com topologijom skupovi $\desInt{-\infty}{a}$, $\lijInt{b}{+\infty}$ su otvoreni).
    Slu\v cajne elemente s vrijednostima u $\extReal$ nazivamo \emph{pro\v sirenim slu\v cajnim varijablama}.
    Uo\v cimo da su $\{ -\infty \}, \; \{+\infty\} \in \borel{\extReal}$ (na primjer $\{+\infty\} = \presjek{n \in \N}{} \lijInt{n}{+\infty}$).
    Ako je $X$ pro\v sirena slu\v cajana varijabla i $\vjeroj{|X| = \infty} = 0$, tada ka\v zemo da je $X$ \emph{slu\v cajna varijabla}.
    Posebno, slu\v cajni element s vrijednostima u $\R$ je slu\v cajna varijabla.
    Na sli\v can na\v cin mo\v zemo promatraiti i slu\v cajne elemente s vrijednostima u $\R_+ := \desInt{0}{+\infty}$, ili $\extReal_+ := \segment{0}{+\infty}$. 
\end{pr}

\begin{nap} \label{nap:3.8.1}
    $\borel{\extReal}$ mo\v zemo promatrati i kao najmanju $\sigma$-algebru koja sadr\v zi sve elemente iz $\borel{\R}$, kao i skupove $\{-\infty\}$, $\{+\infty\}$.
    Odnosno vrijedi:
    \begin{equation*}
        \borel{\extReal} = \sigAlg{\borel{\R} \cup \{\{-\infty\}, \: \{+\infty\}\}}
    \end{equation*}
\end{nap}

\begin{nap} \label{nap:3.9}
    Neka je $\vjerojatnosniProstor$ vjerojatnosni prostor i $S$ neka je neko svojstvo, takvo da za svaki $\omega \in \Omega$ mo\v zemo utvrditi (u principu) da "$\omega$ zadovoljava $S$" ili da "$\omega$ ne zadovoljava $S$".
    Ka\v zemo da je $S$ zadovoljeno \emph{gotovo sigurno} ($g.s.$) ako postoji skup $A \in \F$ takvo da je $\vjeroj{A} = 1$ i $A \subseteq \skup{\omega \in \Omega}{\omega \; \textnormal{zadovoljava} \; S}$.
    Uo\v cimo da skup $\vjeroj{A} = 1$ i $A \subseteq \skup{\omega \in \Omega}{\omega \; \textnormal{zadovoljava} \; S}$ ne mora biti doga\dj aj.
    Sjetimo se da je vjerojatnosni prostor \emph{potpun} ako vrijedi:
    \begin{equation*}
        A \subseteq E, \; E \in \F, \; \vjeroj{E} = 0 \implies A \in \F.
    \end{equation*}
    O\v cito, ako je $\vjerojatnosniProstor$ potpun i $S$ je zadovoljeno gotovo sigurno, tada je $\vjeroj{A} = 1$ i $A \subseteq \skup{\omega \in \Omega}{\omega \; \textnormal{zadovoljava} \; S} \in \F$.

    Uz ove pojmove mo\v zemo re\' ci da je pro\v sirena slu\v cajna varijabla $X$ slu\v cajna varijabla ako i samo ako je $X \in \R \; (g.s.)$
\end{nap}

\begin{nap} \label{nap:3.9-1}
    Neka je $\vjerojatnosniProstor$ vjerojatnosni prostor, \emph{upotpunjenje vjerojatnosnog prostora} je potpun vjerojatnosni prostor $(\Omega, \: \overline{\famF}, \overline{\masP})$, gdje je $\famF \subseteq \overline{\famF}$ i gdje je $\overline{\masP}$ pro\v sireneje od $\masP$, odnosno vrijedi:
    \begin{equation*}
        \forall A \in \famF \quad \vjeroj{A} = \overline{\masP}(A).
    \end{equation*}
\end{nap}

\begin{zad} \label{zad:3.10}
    Doka\v zite da se svaki vjerojatnosni prostor mo\v ze upotpuniti.
    Nadalje, ako je $f$ slu\v cajni element na polaznom vjerojatnosnom
    prostoru, tada je $f$ slu\v cajni element i na upotpunjenu.
\end{zad}

%
%   rješi zadatak
%



\begin{rj}  \label{rj:3.10}
    Doka\v zimo prvo da je svaki vjerojatnosni prostor mogu\' ce upotpuniti. Neka je $\vjerojatnosniProstor$ vjerojatnosni prostor, tada postoji (minimalano) upotpunjenje $(\Omega, \: \overline{\famF}, \overline{\masP})$.

    Definirajmo $\famN := \skup{E \subset F}{F \in \famF \; \land \; \vjeroj{F} = 0}$ te $\overline{\famF} = \skup{A \cup E}{A \in \famF, \; E \in \famN}$ te tako\dj er $\overline{\masP} : \overline{\famF} \to \segment{0}{+\infty}$ na elementima $\overline{\famF}$ sa $\overline{\masP}(A \cup E) = \vjeroj{A}$.
    Dokazujemo sljede\' ce tvrdnje.
    \begin{enumerate}[label=(\arabic*)]
        \item   \label{rj:3.10.1}
        $\overline{\famF}$ je $\sigma$-algebra
        \item   \label{rj:3.10.2}
        $\overline{\masP}$ je vjerojatnosti
        \item   \label{rj:3.10.3}
        $(\Omega, \: \overline{\famF}, \overline{\masP})$ je potpun vjerojatnosni prostor
        \item   \label{rj:3.10.4}
        $(\Omega, \: \overline{\famF}, \overline{\masP})$ je minimalno potpuno pro\v sirenje of $\vjerojatnosniProstor$.
    \end{enumerate}

    Krenimo redom, poka\v zimo \ref{rj:3.15.1}.
    Primjetimo na po\v cetku da je $\forall A \in \famF, \; A = A \cup \varnothing$, stoga vrijedi $\famF \subseteq \overline{\famF}$
    \begin{enumerate}[label=(\roman*)]
        \item O\v cito je $\varnothing \in \overline{\famF}$
        \item Neka je $A \cap E \in \overline{\famF}$, sada postoji $E \in \famF$ takav da $\vjeroj{F} = 0$ i $F \subseteq E$.
        Vrijedi:
        \begin{equation*}
            \begin{aligned}
                (A \cup E)^c &= (A \cup (E \cap F))^c = (A \cup (F \setminus (F \setminus E)))^c = (A \cup (F \cap (F \setminus E)^c))^c\\
                &= A^c \cap (F \cap (F \setminus E)^c)^c = A^c \cap (F^c \cup (F \setminus E))\\
                &= (A^c \cap F^c) \cup (A^c \cap (F \setminus E))
            \end{aligned}
        \end{equation*}
        Budu\' ci su $A, \; F \in \famF$, i da je $A^c \cap (F \setminus E) \subseteq F$, vrijedi da je $\vjeroj{F} = 0$, vidimo da je $(A \cup E)^c \in \overline{\famF}$.
        \item Neka je $\niz{A_n \cup E_n}{n \in \nat}$ niz u $\overline{\famF}$, gdje je $A_n \in \famF$, a $E_n \subseteq F_n$ za neke $F_n \in \famF$, za koje vrijedi $\vjeroj{F_n} = 0$.
        Sada vidimo:
        \begin{equation*}
            \unija{n \in \nat}{} (A_n \cup E_n) = \unija{n \in \nat}{} A_n \cup \unija{n \in \nat}{} E_n.
        \end{equation*}
        Vrijedi da je $\unija{n \in \nat}{} A_n \in \famF$ i tako\dj er vrijedi $\unija{n \in \nat}{} E_n \subseteq \unija{n \in \nat}{} F_n$, gdje je $\unija{n \in \nat}{} F_n \in \famF$ i uz to vrijedi $\vjeroj{\unija{n \in \nat}{} F_n} \leq \suma{n \in \nat}{} \vjeroj{F_n} = \suma{n \in \nat}{} 0 = 0$.
        Dakle vidimo da je $\unija{n \in \nat}{} (A_n \cup E_n) \in \overline{\famF}$
    \end{enumerate}
    pa je $\overline{\famF}$ $\sigma$-algebra na $\Omega$.

    Poka\v zimo sada \ref{rj:3.15.2}.
    Prvo poka\v zimo da je $\overline{\masP}$ dobro definirana, odnosno da ne ovisiti o reprezentaciji, to jest ako imamo skupove $A \cup E$, $A' \cup E'$ takve da $A, \; A' \in \famF$, $E \subseteq F, \; E' \subseteq F'$ takvi da $F, \; F' \in \famF$, sa $\vjeroj{F} = \vjeroj{F'} = 0$ te uz to jo\v s i vrijedi $A \cup E = A' \cup E'$, tada mora biti $\overline{\masP} (A \cup E) = \overline{\masP} (A' \cup E')$.
    Primjetimo vrijedi:
    \begin{equation*}
        (A \cup E) \cup (F \cup F') = (A' \cup E') \cup (F \cup F') \iff A \cup F \cup F' = A' \cup F \cup F',
    \end{equation*}
    dakle,
    \begin{equation*}
        \begin{aligned}
            \vjeroj{A} \leq \vjeroj{A \cup F \cup F'} = \vjeroj{A' \cup F \cup F'} \leq \vjeroj{A'} + \vjeroj{F \cup F'} = \vjeroj{A'} + 0 = \vjeroj{A'}\\
            \vjeroj{A'} \leq \vjeroj{A' \cup F \cup F'} = \vjeroj{A \cup F \cup F'} \leq \vjeroj{A} + \vjeroj{F \cup F'} = \vjeroj{A} + 0 = \vjeroj{A}.
        \end{aligned}
    \end{equation*}
    Odakle vidimo da nu\v zno vrijedi $\vjeroj{A} = \vjeroj{A'}$, samim time mora vrijediti i $\overline{\masP} (A \cup E) = \overline{\masP} (A' \cup E')$, pa je $\overline{\masP}$ dobro definirana.
    Da je $\overline{\masP}$ ekstenzija od $\masP$ vidimo tako da za svaki $A \in \famF$ vrijedi $A = A \cup \varnothing$, pa prema tome $\vjeroj{A} = \overline{\masP} (A)$.
    Poka\v zimo da je $\overline{\masP}$ vjerojatnost.
    \begin{enumerate}[label=(\roman*)]
        \item $\overline{\masP} (A \cup E) = \vjeroj{A} \geq 0, \quad \forall (A \cup E) \in \overline{\famF}$
        \item $\overline{\masP} (\Omega) = \vjeroj{\Omega} = 1$
        \item Neka je $\niz{A_n \cup E_n}{n \in \nat} \subseteq \overline{\famF}$ tada vrijedi $A_n \in \famF$ i za svaki $E_n$, postoji $F_n \in \famF$ takav da vrijedi $E_n \subseteq F_n, \; \vjeroj{F_n} = 0$. Sada imamo
        \begin{equation}    \label{jed:3.10-1}
            \begin{aligned}
                \unija{n \in \nat}{} E_n &\subseteq \unija{n \in \nat}{} F_n\\
                \vjeroj{\unija{n \in \nat}{} F_n} &\leq \suma{n \in \nat}{} \vjeroj{F_n} = 0.
            \end{aligned}
        \end{equation}
        Odakle dobijemo:
        \begin{equation*}
            \begin{aligned}
                \overline{\masP} (\unija{n \in \nat}{} (A_n \cup E_n)) &= \overline{\masP} (\unija{n \in \nat}{} A_n \cup \unija{n \in \nat}{} E_n) \overset{\eqref{jed:3.10-1}}{=} \vjeroj{\unija{n \in \nat}{} A_n} = \suma{n \in \nat}{} \vjeroj{A_n}\\ &= \suma{n \in \nat}{} \overline{\masP} (A_n \cup E_n).
            \end{aligned}
        \end{equation*}
    \end{enumerate}
    Pa je $\overline{\masP}$ mjera na $\overline{\famF}$.

    Poka\v zimo sada \ref{rj:3.10.3}.
    Neka je $F \in \overline{\famF}$ takav da je $\overline{\masP} (F) = 0$ i neka je $E \subseteq F$.
    \v Zelimo pokazati da je $E \in \overline{\famF}$.
    Kako je $F \in \overline{\famF}$, postoje $A \in \famF, \; E' \in \famN$ takvi da vrijedi $F = A \cup E'$.
    Po definicijskom uvijetu je $\overline{\masP} (F) = \vjeroj{A} = 0$.
    Kako je $E' \in \famN$, postoji $F' \in \famF$ takav da je $E' \subseteq F'$, $\vjeroj{F'} = 0$.
    Sada imamo $\vjeroj{A \cup F'} \leq \vjeroj{A} + \vjeroj{F'} = 0 + 0$, a kako imamo:
    \begin{equation*}
        E \subseteq F = A \cup E' \subseteq A \cup F',
    \end{equation*}
    te vrijedi $\vjeroj{A \cup F'} = 0$, pa slijedi da je $E \in \famN$.
    Kako je $\famE \subseteq \overline{\famF}$, slijedi $E \in \overline{\famF}$, pa je po monotonosti vjerojatnosti $\overline{\masP} (E) = 0$.
    Pa je $(\Omega, \: \overline{\famF}, \: \overline{\masP})$ potpun vjerojatnosni prostor.

    Na poslijetku poka\v zimo da vrijedi \ref{rj:3.10.4}.
    Neka je $(\Omega, \: \famG, \: \masO)$ potpun vjerojatnosni prostor, takav da je $\famF \subseteq \famG$ te $\restr{\masO}{\famF} = \masP$.
    Neka je $A \cup E \in \overline{\famF}$, takav da je $A \in \famF, \; E \in \famN$, tada postoji $F \in \famF$ takav da je $E \subseteq F$, $\vjeroj{F} = 0$.
    Kako je $\masO$ pro\v sirenje od $\masP$ nu\v zno vrijedi $\masO (F) = 0$.
    Kako je $(\Omega, \: \famG, \: \masO)$ potpun, tada je $E \in \famG$, pa posljedi\v cno $A \cup E \in \famG$, stoga je $\overline{\famF} \subseteq \famG$.
    Dakle $(\Omega, \; \overline{\famF}, \; \overline{\masP})$ je minimalno upotpunjenje.
    
    Komentirajmo jo\v drugi dio zadatka, 
\end{rj}

\begin{nap} \label{nap:3.11}
    Kod slu\v cajnih elemenata treba provjeriti da vrijedi $\praslika{f}(\famE) \subseteq \famF$. Zbog leme \ref{lm:3.4} dovljno je provjeriti da je $\praslika{f}(\famC) \subseteq \famF$, ako je $\famC$ generiraju\' ca familija za $\famE$, to jest ako je $\famE = \sigAlg{\famC}$.
\end{nap}

Koriste\' ci napomenu \ref{nap:3.11} i primjer \ref{pr:2.3} dobivamo:

\begin{kor} \label{kor:3.12}
    Neka je $X : \Omega \to \extReal$ preslikavanje i $\famC$ bilo koja klasa iz primjera \ref{pr:2.3} (za slu\v caj $\real$).
    Tada je $X$ pro\v sirena slu\v cajna varijabla ako i samo ako je $\praslika{X}(\famC) \subseteq \famF$.
\end{kor}

Podsjetimo se da je za $x, \; y \in \extReal$, $x \lor y := \max \{x, \: y\}$ i $x \land y := \min \{x, \: y\}$.
Podsjetimo se i da koristimo konvencije $0 \cdot \pm \infty = 0$, te $c \cdot (\pm \infty) = \pm \infty$ za $c > 0$ i $(-1) \cdot (\pm \infty) = \mp \infty$. Koriste\' ci korolar \ref{kor:3.5} i lemu \ref{lm:3.6} dobijemo:

\begin{kor} \label{kor:3.13}
    Ako je $X$ (pro\v sirena) slu\v cajan varijabla, tada su i
    \begin{itemize}[label=]
        \item $c \cdot X, \; c \in \real$,
        \item $|X|$,
        \item $X^+ := X \lor 0$,
        \item $X^- := (-X) \land 0$
    \end{itemize}
    pro\v sirene slu\v cajne varijable.
\end{kor}

\begin{kor} \label{kor:3.14}
    Ako je $\niz{X_n}{n \in \nat}$ niz pro\v sirenih slu\v cajnih varijabli, tada su:
    \begin{itemize}[label=]
        \item $\sup\limits_{n} X_n$,
        \item $\inf\limits_{n} X_n$,
        \item $\limsup\limits_{n} X_n$,
        \item $\liminf\limits_{n} X_n$,
    \end{itemize}
    pro\v sirena slu\v cajne varijable.
\end{kor}

\begin{proof}
    Po korolaru \ref{kor:3.12} i $\famC = \famK$ iz primjera \ref{pr:2.3} slijedi da je $\sup\limits_n X_n$ pro\v sirena slu\v cajna varijabla, jer $\{ \sup\limits_n X_n \leq b \} = \presjek{n \in \nat}{} \{ X_n \leq b \}$.
    Iz korolara \ref{kor:3.13} slijedi tvrdnja za infimum, zbog $\inf\limits_n X_n = - (\sup\limits_n - X_n)$.
    Nadalje, $\liminf\limits_n X_n = \sup\limits_k (\inf\limits_{n \geq k}) X_n$, $\limsup\limits_n X_n = \inf\limits_k (\sup\limits_{n\geq k} X_n)$.
\end{proof}

Mo\v ze se dogoditi da slu\v cajni element $f:\Omega \to E$ ima vrijednosti u nekom podskupu $A \subseteq E$ koji mo\v ze i ne mora biti u $\famE$. Koriste\' ci \eqref{jed:3.2},\eqref{jed:3.3} i lemu \ref{lm:3.4} te inkluziju $i: A \to E$, mo\v zemo jednostavno opisati takve situacije.

\begin{zad} \label{zad:3.15}
    Neka je $(E, \: \famE)$ izmjeriv prostor i $A \subseteq E$.
    Poka\v zi da je
    \begin{equation*}
        A \cap \famE := \skup{A \cap B}{B \in \famE}
    \end{equation*}
    $\sigma$-algebra na $A$.
    Pove\v zi s pojmom relativne topologije ako je $E$ topolo\v ski prostor i $\famE = \borel{E}$.
    Ako je $f$ slu\v cajni element u $(A, \: A \cap \famE)$ mo\v zete li ga interpretirati kao slu\v cajni element u $E$?
    S druge strane, ako je $f$ slu\v cajni element s vrijednostima u $E$ i $f \in A \; (g.s.)$, mo\v zete li interpretirati $f$ kao slu\v cajni element u $\urePar{A}{A \cap \famE}$?
\end{zad}

\begin{nap} \label{nap:3.15-1}
    Sjetimo se, ako je $\urePar{X}{\famT}$ topolo\v ski prostor, te ako je $Y \subseteq X$, ne nu\v zno otvoren, tada familiju
    \begin{equation*}
        \skup{Y \cap U}{U \in \famT}
    \end{equation*}
    nazivamo \emph{relativnom topologijom na $Y$} (induciranom topologijom $\famT$).
\end{nap}

\begin{rj}  \label{rj:3.15}
    Neka je $\urePar{E}{\famT}$ topolo\v ski prostor i $\borel{E} = \sigAlg{\famT}$.
    Ovaj zadatak se sastoji od nekoliko tvrdnji.
    \begin{enumerate}[label=(\arabic*)]
        \item \label{rj:3.15.1}
        Treba pokazati da je $ A \cap \famE$ $\sigma$-algebra.
        \item \label{rj:3.15.2}
        Vrijedi li $A \cap \famE = \sigAlg{A \cap \famT}$?
        \item \label{rj:3.15.3}
        Ako je $f$ slu\v cajni element u $\urePar{A}{A \cap \famE}$, je li onda i slu\v cajni element u $E$?
        \item \label{rj:3.15.4}
        Ako je $f$ slu\v cajni element u $E$ i $f \in A \; (g.s.)$, je li $f$ i slu\v cajni element u $\urePar{A}{A \cap \famE}$?
    \end{enumerate}
    Idemo redom.
    Poka\v zimo prvo \ref{rj:3.15.1}.
    \begin{enumerate}[label=(\roman*)]
        \item Primjetimo $\varnothing \in \famE \implies \varnothing \in A \cap \famE$.
        \item U ovom slu\v caju komplement se podrazumjeva u odnosu na $A$.
        Neka je $B \in A \cap \famE$, dakle postoji $C \in \famE$ takav da $B = A \cap C$, a kako je $C \in \famE$, tada slijedi $C^c \in \famE$ odakle imamo $A \cap C^c = B^c \in A \cap \famE$.
        \item Neka su $\niz{B_n}{n \in \nat} \subseteq A \cap \famE$, tada za svaki $n \in \nat$ postoji $C_n \in \famE$, takav da $B_n = A \cap C_n$.
        Tada imamo $\unija{n \in \nat}{} B_n = \unija{n \in \nat}{} (A \cap C_n) = A \cap \unija{n \in \nat}{} C_n$, a kako je $\unija{n \in \nat}{} C_n \in \famE$, tada je nu\v zno $\unija{n \in \nat}{} B_n = A \cap \unija{n \in \nat}{} C_n \in A \cap \famE$.
    \end{enumerate}

    Poka\v zimo sada \ref{rj:3.15.2}.
    \begin{itemize}
        \item[$\subseteq$] Neka je $i : A \to E$ inkluzija.
        To je neprekidna funckija, pa je i izmjeriva u paru Borelovih $\sigma$-algebri.
        Dakle za $B \in \famT$ je $\praslika{i}(B)$ Borelov u $\sigma$-algebri generiranoj relativnom topologijom i vrijedi $\praslika{i}(B) = A \cap B$, pa je\\
        $\skup{A \cap B}{A \in \sigAlg{\famT}} = A \cap \famE \subseteq \sigAlg{A \cap \famT}$.
        \item[$\supseteq$] Primjetimo $\famT \subseteq \famE$, dakle vrijedi i $A \cap \famT \subseteq \famE$ pa $A \cap \famE$ sadr\v zi sve otvorene podskupove od $A$ i prema \ref{rj:3.15.1} je $\sigma$-algebra, stoga vrijedi:\\
        $\sigAlg{A \cap \famT} \subseteq \sigAlg{A \cap \famE} = A \cap \famE$.
    \end{itemize}
    
    Promotrimo \ref{rj:3.15.3}.
    Neka je $f: \Omega \to A \cap E$ slu\v cajni element u $\urePar{A}{A \cap \famE}$ te neka je $i : A \cap E \to E$ inkluzija.
    Funkcija $i$ je neprekidna, pa samim time i izmjeriva, pa je
    $X := i \circ f : \Omega \to E$ slu\v cajni element na $\urePar{E}{\famE}$ koji mo\v zemo promatrati kao pro\v sirenje slu\v cajnog elementa $f$ na $E$.
    Na taj na\v cin mo\v zemo $f$ promatrati kao slu\v cajni element na $E$.

    Te na kraju promotrimo \ref{rj:3.15.4}.
    Promatrajmo restrikciju slu\v cajnog elementa $f$ na $A$, $\restr{f}{A} : \Omega \to A$.
    %%
    %%  doradi ovo - upotpunjenje prostora i restrikcija.
    %%
\end{rj}

% Ovo revidiraj, primjer nije točan

\begin{pr}  \label{pr:3.16}
    Neka su $(A, \: \famA)$, $(B, \: \famB)$ izmjerivi prostori i $f: A \to B$ $(\famA, \: \famB)$-izmjerivo preslikavanje.
    Ka\v zemo da je $f$ \emph{jednostavno} ako je $f(A)$ kona\v can skup. To o\v cito vrijedi ako i samo ako postoje $n \in \nat$, $\{ b_1, \: \dots, \: b_n \} \subseteq B$, particija $\{ A_1, \: \dots, \: A_n \} \subseteq \famA$ skupa $A$, takvi da je $\restr{f}{A_j} = b_j$, za $j = 1, \dots, n$.
    
    U slu\v caju da je $(B, \: \famB) = (\extReal, \: \borel{\extReal})$ vrijedi:
    \begin{equation*}
        f = \suma{j = 1}{n} b_j \mathbb{1}_{A_j}.
    \end{equation*}
    Posebno, jednostave slu\v cajne varijable tvore vektorski prostor nad $\extReal$.
\end{pr}

\begin{lm}  \label{lm:3.17}
    Ako je $X$ nenegativan pro\v sirena slu\v cajna varijabla, tada postoji niz jednostavnih slu\v cajnih varijabli $\niz{X_n}{n \in \nat}$ takav da, za svaki $\omega \in \Omega$ vrijedi:
    \begin{equation*}
        0 \leq X_1(\omega) \leq X_1(\omega) \leq  \dots \nearrow X(\omega).
    \end{equation*} 
\end{lm}

\begin{proof}
    Za $\omega \in \Omega$ i $n \in \nat$ definiramo
    \begin{equation*}
        X_n(\omega) := \frac{1}{2^n} \floor*{2^n \cdot X(\omega)} \land n.
    \end{equation*}
\end{proof}

Iz leme \ref{lm:3.17} i iz korolara \ref{kor:3.14} slijedi:

\begin{tm}  \label{tm:3.18}
    Neka je $\vjerojatnosniProstor$ vjerojatnosni prostor i $X: \Omega \to \real$. Tada je $X$ slu\v cajna varijabla ako i samo ako je $X$ limes niza jednostavnih slu\v cajnih varijabli.
\end{tm}

\begin{zad} \label{zad:3.19}
    Neka je $\vjerojatnosniProstor$ vjerojatnosni prostor, $X, \; Y$ pro\v sirene slu\v cajne varijable na $\Omega$ i $\alpha, \; \beta \in \real$.
    \begin{enumerate}[label=(\alph*)]
        \item Ako su $X \geq 0$, $Y \geq 0$, $\alpha \geq 0$, $\beta    \geq 0$, tada je $\alpha X + \beta Y$ pro\v sirena nenegativan slu\v cajna varijabla.
        \item Ako su $X$ i $Y$ slu\v cajne  varijable, tada je $\alpha X + \beta Y$ gotovo sigurno dobro definirana i mo\v ze se interpretirati kao slu\v cajna varijabla.
    \end{enumerate}
\end{zad}

\begin{zad}
    Neka je $X$ slu\v cajna varijabla i $Y$ slu\v cajni element s vrijednostima u $(E, \: \famE)$.
    Tada je $X$ $(\sigAlg{Y}, \: \borel{\real})$-izmjeriva ako i samo ako postoji funckija $f: E \to \real$ $(\famE, \: \borel{\real})$-izmjeriva takva da je $X = f \circ Y$.
\end{zad}

%
% rješi i dodaj ovdje.
%

    %%%%%%%%%%%%%%%%%%%%%%%%%%
    %%  produktni prostori  %%
    %%%%%%%%%%%%%%%%%%%%%%%%%%

    % produktni prostori

\chapter{Produktni prostori}

Neka je $T \neq \varnothing$ skup indeksa i $\indFamilija{\Omega_t}{t \in T}$ familija nepraznih skupova.
Skup svih funkcija $f: T \to \unija{t \in T}{} \Omega_t$, takvih da je $f(t) \in \Omega_t, \; \forall t \in T$, ozna\v cavamo sa $\produkt{t \in T}{} \Omega_t$.
Aksiom izbora garantira da je $\produkt{t \in T}{} \Omega_t$ neprazan (jer su svi $\Omega_t \neq \varnothing$).

Definicija se mo\v ze pro\v siriti na proizvoljne $\Omega_t$ i tada dobivamo da je $\produkt{t \in T}{} \Omega_t$ \v ci je barem jedan $\Omega_t$ prazan.
Uz \emph{kartezijev produkt} $\produkt{t \in T}{} \Omega_t$, prirodno su vezane \emph{koordinaten projekcije $pi_t$}, $t \in T$, pri \v cemu je $\pi_{t_0}: \produkt{t \in T}{} \to \Omega_{t_0}$ definiramo sa
\begin{equation*}
    \pi_{t_0} (f) := f(t_0).
\end{equation*}

Podsjetimo se da su $(X_t, \: \famU_t), \; t \in T$, topolo\v ski prostori, tada se na $\produkt{t \in T}{} X_t$ definira \emph{produktna (ili Tihonovljeva) topologija} $\dirProd{t \in T}{} \famU_t$, kao najmanja topologija u odnosu na koju su sve $\pi_t$ neprekidne.
Ako su svi $X_t$ kompaktni,tada je i $\produkt{t \in T}{} X_t$ kompaktan u produktnoj topologiji.
Analogno postupamo u izmjerivoj situaciji.

Neka su $(\Omega_t, \: \famF_t), \; t \in T$ izmjerivi prostori.
Produktna $\sigma$-algebra $\dirProd{t \in T}{} \famF_t$ na $\produkt{t \in T}{} \Omega_t$, definirana je sa:
\begin{equation}    \label{jed:4.1}
    \dirProd{t \in T}{} \famF_t := \indSigAlg{\pi_t}{t \in T}.
\end{equation} 
Element $\produkt{t \in T}{} A_t \in \partitive{\produkt{t \in T}{}} \Omega_t$, takav da postoji kona\v can $J \subseteq T$ sa svojstvima:
\begin{itemize}[label=]
    \item $t \in J \implies A_t \in \famF_t$
    \item $ t \in T \setminus J \implies A_t = \Omega_t $
\end{itemize}
nazivamo \emph{izmjerivi cilindri\v cni pravokutnik}.
Lako se vidi da svi takvi skupovi tvore poluprsten skupova $\prsten{\produkt{t \in T}{} \Omega_t}$ i da vrijedi:
\begin{equation}    \label{jed:4.2}
    \sigAlg{\prsten{\produkt{t \in T}{} \Omega_t}} = \dirProd{t \in T}{} \famF_t.
\end{equation}
Uo\v cimo da se za topolo\v ske prostore javlja pitanje veze $\borel{\produkt{t \in T}{} \Omega_t}$ (\v sto je $= \sigAlg{\dirProd{t \in T}{} \famU_t}$) i $\dirProd{t \in T}{} \borel{X_t}$.
Lako se vidi da za topolo\v ske prostore $(X_t, \: \famU_t), \; t \in T$, vrijedi:
\begin{equation}    \label{jed:4.3}
    \dirProd{t \in T}{} \borel{X_t} \subseteq \borel{\produkt{t \in T}{} X_t},
\end{equation}
a nije osobito te\v sko konstruirati kontraprimjere koji pokazuju da u \eqref{jed:4.3} op\' cenito ne vrijedi jednakost.

\begin{zad} \label{zad:4.4}
    Ako je $T$ najvi\v se prebrojiv i svaki $X_t$ je separabilan metri\v cki prostor, tada u \eqref{jed:4.3} vrijedi jednakost.
    Posebno, za $d \in \nat$ vrijedi:
    \begin{equation}    \label{jed:4.5}
        \borel{\real^d} = \borel{\real} \otimes \dots \otimes \borel{\real}.
    \end{equation}
\end{zad}

\begin{zad} \label{zad:4.6}
    Za svaki skup $A \in \dirProd{t \in T}{} \famT_t$, postoji prebrojiv skup indeksa $J = J(A) \subseteq T$, takav da je $A \in \indSigAlg{\pi_t}{t \in J}$. 
\end{zad}

Ako je $I \subseteq T, \; I \neq \varnothing$, onda za svaki $A \subseteq \produkt{t \in T}{} \Omega_t$ i $x \in \produkt{t \in T}{} \Omega_t$ mogu\' ce promarati \emph{prerez od $A$ po $x$}; to jest skup $A_x := \skup{y \in \produkt{t \in T \setminus J}{} \Omega_t}{(x, \: y) \in A \; (\subseteq \produkt{t \in T}{} \Omega_t)}$.
Uo\v cimo da je prerez (po $x$) svakog izmjerivog cilindri\v cnog pravokutnika ponovo ili $\varnothing$ ili izmjerivi cilindri\v cni pravokutnik u $\prsten{\produkt{t \in T \setminus I}{} \Omega_t}$, pa slijedi:
\begin{equation}    \label{jed:4.7}
    A \in \dirProd{t \in T}{} \famF_t \implies A_x \in \dirProd{t \in T \setminus I}{} \famF_t.
\end{equation}

\begin{zad} \label{zad:4.8}
    Opi\v site detaljno svojstva prereza u slu\v caju kada je $T = \{1, \: 2\}$.
\end{zad}

Neka je $\izmjerivProstor$ izmjeriv prostor, $T \neq \varnothing$ skup indeksa i $\indFamilija{(E_t, \: \famE_t)}{t \in T}$ familija izmjerivih prostora.
Iz \eqref{jed:4.1} i \eqref{jed:4.2} i napomnene \ref{nap:3.11} direktno slijedi:

\begin{prop} \label{prop:4.9}
    Preslikavanje $X : \to \produkt{t \in T}{} E_t$ je $(\famF, \: \dirProd{t \in T}{} \famE_t)$-izmjerivo ako i samo ako je za svaki $t \in T$ preslikavanje $\pi_t \circ X$, $(\famF, \: \famE_t)$-izmjerivo.
\end{prop}

\begin{nap} \label{nap:4.10}
    \begin{enumerate}[label=(\alph*)]
        \item U slu\v caju kada je $(E_t, \: \famE_t) = (E, \: \famE)$, za svaki $t \in T$, koristimo oznake $E^T := \produkt{t \in T}{} E_t$, $\famE^T := \dirProd{t \in T}{} \famE_t$, ako je uz to jo\v s i $\card{T} = \aleph_0$, koristimo oznake $E^\infty, \; \famE^\infty$.
        \item Ako je $T = \{1, \: 2, \dots, \: d\}$ i $(E, \: \famE) = \urePar{\extReal}{\borel{\extReal}}$ onda je $E^d = \extReal^d$, $(\borel{\extReal})^d = \borel{\extReal^d}$.
        Slu\v cajni element s vrijednostima u $\extReal^d$ nazivamo \emph{pro\v sirenim slu\v cajnim vektorom}.
        Na $\extReal^d$ imamo $d$ projekcija $\pi_1, \dots, \pi_d$ i uvodimo oznaku, za $X : \Omega \to \extReal^d$, $X_i := \pi_i \circ X, \; i = 1, \dots, d$.
        Re\' ci cemo da je pro\v sireni slu\v cajni vektor $X$ slu\v cajni vektor ako je $X \in \real \; (g.s.)$.
        Iz propozicije \ref{prop:4.9} slijedi da je $X$ (pro\v sireni) slu\v cajni vektor ako i samo ako su $X_1, \dots X_d$ (pro\v sirene) slu\v cajne varijable.
    \end{enumerate}
\end{nap}

Neka je $d \in \nat$ i $(\Omega_i, \: \famF_i, \: \mu_i), \; i = 1, \dots, d$ prostori $\sigma$-kona\v cnih mjera.
Za $A = A_1 \times \dots \times A_d \in \prsten(\Omega_1 \times \dots \times \Omega_d)$ definiramo
\begin{equation}    \label{jed:4.11}
    \nu (A) := \mu_1 (A_1) \cdot \dots \cdot \mu_d (A_d),
\end{equation}
te se mo\v ze pokazati da je $\nu$ $\sigma$-kona\v cna funkcija.
Po zadatku \ref{zad:2.9} i teoremu \ref{tm:2.11} slijedi:

\begin{tm}  \label{tm:4.12}
    Postoji i jedinstveno je pr\v sirenje funkcije $\nu$ na mjeru na $\famF_1 \otimes \dots \otimes \famF_d$.
    To pro\v sirenje je $\sigma$-kona\v cna mjera koju nazivamo \emph{produktnom mjerom} i ozna\v cavamo sa $\mu_1 \otimes \dots \otimes \mu_d$.
    Bez smanjenja op\' cenitosti niti jedna od mjera $\mu_i$ nije trivijalna (to jest vrijedi $\mu_i (\Omega_i) \neq 0 \; i=1, \dots, d$).
    Tada je $\mu_1 \otimes \dots \otimes \mu_d$ kona\v cana ako i samo ako su $\mu_1, \dots, \mu_d$ kona\v cne.
    Ako su $\mu_1, \dots, \mu_d$ vjerojatnosne, tada je i $\mu_1 \otimes \dots \otimes \mu_d$ vjerojatnost. 
\end{tm}

\begin{pr}  \label{pr:4.13}
    Neka je $\mu_1 = \dots = \mu_d = \lambda$ (Lebesgueova mjera).
    Tada je $\mu_q \otimes \dots \mu_d$ $\sigma$-kona\v cna mjera na $\urePar{\real^d}{\borel{\real^d}}$, koju ozna\v cavamo $\lambda^d$ i zovemo \emph{$d$-dimenzionalana Lebesgueova mjera}.
\end{pr}

\begin{zad} \label{zad:4.15}
    Neka su $(\Omega_i, \: \famF_i, \: \mu_i), \; i = 1, \; 2$, prostori $\sigma$-kona\v cne mjere.
    Tada vrijedi:
    \begin{enumerate}[label=(\roman*)]
        \item Za svaki $A \in \famF_1 \otimes \famF_2$, preslikavanje
        \begin{equation*}
            \Omega_1 \ni x \mapsto \mu_2 (A_x)
        \end{equation*}
        je $\famF_1$-izmjerivo, a preslikavanje
        \begin{equation*}
            \Omega_2 \ni y \mapsto \mu_1 (A_y),
        \end{equation*}
        je $\famF_2$-izmjerivo.
        \item Za svaki $A \in \famF_1 \otimes \famF_2$ vrijedi:
        \begin{equation*}
            (\mu_1 \otimes \mu_2) (A) = \int_{\Omega_1} \mu_2 (A_x) \: d \mu_1(x) = \int_{\Omega_2} \mu_1 (A_y) \: d \mu_2 (y).
        \end{equation*}
        \item Za svaki $f: \Omega_1 \times \Omega_2 \to \segment{0}{+\infty}$ koji je $\urePar{\famF_1 \otimes \famF_2}{\borel{\segment{0}{+\infty}}}$-izmjeriv
        \begin{align*}
            \int_{\Omega_1 \times \Omega_2} f(x, \: y) \: d (\mu_1 \otimes \mu_2)(x, \:y) &= \int_{\Omega_2} \Big( \int_{\Omega_1} f(x, \: y) \: d \mu_1 (x) \Big) \: d \mu_2 (y)\\
            &= \int_{\Omega_1} \Big( \int_{\Omega_2} f(x, \: y) \: d \mu_2 (y) \Big) \: d \mu_1 (x)
        \end{align*}
    \end{enumerate} 
\end{zad}

\begin{zad} \label{zad:4.16}
    Neka su $(\Omega_i, \: \famF_i, \: \mu_i), \; i = 1, \: 2$, prostori $\sigma$-kona\v cne mjere i $f : \Omega_1 \times \Omega_2 \to \extReal$ $\urePar{\famF_1 \otimes \famF_2}{\borel{\extReal}}$-izmjeriva, takva da je $|f|$ $\mu_1 \otimes \mu_2$-integrabilna.
    Tada vrijedi (\emph{Fubinijev teorem}):
    \begin{enumerate}[label=(\roman*)]
        \item Funkcija $y \mapsto f(x, \: y)$ je $\mu_2$-integrabilna za gotovo svaki $x \in \Omega_1$, a funkcija $x \mapsto f(x, y)$ $\mu_1$-integrabilna za gotovo svaki $y \in \Omega_2$.
        \item Funkcija
        \begin{equation*}
            x \mapsto
            \begin{cases}
                \int_{\Omega_2} f(x, \: y) \: d \mu_2 (y), &\textnormal{ako je } y \mapsto f(x, \: y) \textnormal{ integrabilna}\\
                0, &\textnormal{ina\' ce} 
            \end{cases}
        \end{equation*}
        je $\mu_1$-integrabilna, a funkcija
        \begin{equation*}
            y \mapsto
            \begin{cases}
                \int_{\Omega_1} f(x, \: y) \: d \mu_1 (x), &\textnormal{ako je } x \mapsto f(x, \: y) \textnormal{ integrabilna}\\
                0, &\textnormal{ina\' ce}
            \end{cases}
        \end{equation*}
        je $\mu_2$-integrabilna.
        \item
        \begin{align*}
            \int_{\Omega_1 \times \Omega_2}  f(x, \: y) \: d (\mu_1 \otimes \mu_2)(x, \: y)  &= \int_{\Omega_2} \Big( \int_{\Omega_1} f(x, \: y) \: d \mu_1 (x) \Big) \: d \mu_2(y)\\
            &= \int_{\Omega_1} \Big( \int_{\Omega_2} f(x, \: y) \: d \mu_2 (y) \Big) \: d \mu_1(x).
        \end{align*}
    \end{enumerate}
\end{zad}

\begin{defn}    \label{defn:4.17}
    Neka su $\urePar{\Omega_1}{\famF_1}$ i $\urePar{\Omega_2}{\famF_2}$ izmerivi prostori. Preslikavanje $k : \Omega_1 \times \famF_2 \to \segment{0}{+\infty}$ je \emph{uniformno $\sigma$-kon\v cna jezgra} ako vrijedi:
    \begin{enumerate}[label=(\alph*)]
        \item $(\forall \omega \in \Omega_1 ) \; \famF_2 \ni A \mapsto k(\omega, \: A)$ je mjera,
        \item $(\forall A \in \famF_2 ) \; \Omega_1 \ni \omega \mapsto k(\omega, \: A)$ je $\famF_1$-izmjeriva,
        \item $(\exists B_n \in \famF)(\exists c_n > 0)$ takvi da $\Omega_2 = \unija{n = 1}{\infty} B_n$ i $(\forall n \in \nat)(\forall \omega \in \Omega_1)$ $k(\omega, \: B_n) \leq c_n$.\\
        Ako je $c_n = c$, $\forall n \in \nat$, $B_n = \Omega_2$, $\forall n \in \nat$, ka\v zemo da je $k$ \emph{uniformno kona\v cna jezgra}. Ako je $k(\omega, \: B) \leq 1$, $\forall \omega \in \Omega_1$, $\forall B \in \famF_2$ ka\v zemo da je $k$ \emph{subvjerojatnosna jezgra}. Ako je $k(\omega, \: \Omega_1) = 1$, $\forall \omega \in \Omega_1$, ka\v zemo da je $k$ \emph{vjerojatnosna jezgra}.
    \end{enumerate}
\end{defn}

\begin{zad} \label{zad:4.18}
    Ako je $\mu$ $\sigma$-kona\v cna mjera na $\urePar{\Omega_1}{\famF_1}$ i $k$ uniformno kona\v cna jezgra na $\Omega_1 \times \famF_2$, tada postoji i jedinstvena je $\sigma$-kona\v cna mjera $nu$ na $\urePar{\Omega_1 \times \Omega_2}{\famF_1 \otimes \famF_2}$ takva da je
    \begin{equation*}
        \nu (A \times B) = \int_A k(\omega, \: B) \: d \mu (\omega),
    \end{equation*}
    za svaki $A \times B \in \prsten{\Omega_1 \times \Omega_2}$.
    Ako je funkcija $f:\Omega_1 \times \Omega_2 \to \segment{0}{+\infty}$ $\urePar{\famF_1 \otimes \famF_2}{\borel{\segment{0}{+\infty}}}$-izmjeriva, tada je
    \begin{equation*}
        \int_{\Omega_1 \times \Omega_2} f(\omega_1, \: \omega_2) \: d \nu (\omega_1, \: \omega_2) = \int_{\Omega_1} \Big( \int_{\Omega_2} f(\omega_1, \: \omega_2) \: k(\omega_1, \: d \omega_2) \Big) \: d \mu (\omega_1).
    \end{equation*}
\end{zad}

\begin{zad} \label{zad:4.19}
    Iska\v zite i doka\v zite tvrdnje u zadacima \ref{zad:4.15}, \ref{zad:4.16} i \ref{zad:4.18} za slu\v caj vjerojatnosnih mjera i vjerojatnosnih jezgara.
\end{zad}

\v Sto mo\v zemo re\' ci o beskona\v cnom produktu?
Prvo gledamo slu\v aj $T = \nat$. Neka su $\urePar{\Omega_n}{\famF_n}$, $n \in \nat$, izmjerivi prostori. Po \eqref{jed:4.2}, znamo da je $\prsten{\produkt{n \in \nat}{} \Omega_n}$ generiraju\' ci poluprsten za $\dirProd{n \in \nat}{} \famF$, pa bi bilo dovoljno mjere zadati na $\prsten{\produkt{n \in \nat}{} \Omega_n}$.
Ako na svakom $\urePar{\Omega_n}{\famF_n}$ imamo $\sigma$-kona\v cnu mjeru $\mu_n$, prirodno bi se moglo o\v cekivati da na $A_1 \times \dots \times A_n \times \Omega_{n + 1} \times \dots$ zadamo mjeru pomo\' cu $\mu_1(A), \; mu_1(A) \times \mu_2(\Omega_2), \: mu_1(A) \times \mu_2(\Omega_2) \times \mu_3 (\Omega_3)$, i tako dalje.
U pravilu one \' ce biti jednake, jedino ako su to vjerojatnosne mjere.
No mo\v zemo postupati druga\' cije.
Neka je, za svaki $n \in \nat$, $\famG_n := \sigAlg{\pi_1, \dots, \pi_n}$, pa je $\famG_1 \subseteq \famG_2 \subseteq \dots$, tvore uzlazni niz $\sigma$-algebri takav da je $\famG_{\infty} := \unija{n = 1}{\infty} \famG_n$ algebra, ali ne nu\v zno i $\sigma$-algebra, i $\sigAlg{\famG_{\infty}} = \dirProd{n \in \nat}{} \famF_n$.
% nisam siguran što je ovaj G točno :/
Sada bi se na $\famG$ mogli promatrati nizovi $\niz{(\mu_1 \otimes \dots \otimes \mu_2)(G)}{n \in \nat}$ i u slu\v caju da postoje limesi $\lim\limits_{n \to \infty} (\mu_1 \otimes \dots \otimes \mu_2)(G)$, taj bi se limes mogao uzeti za $\nu (G)$.
Ali i taj put vodi na probleme, ilustrirajmo jedan od njih.

\begin{pr}  \label{pr:4.20}
    Neka je $(\Omega_n, \: \famF_n, \: \mu_n) = (\{0, \: 1\}, \: \partitive{\Omega_n}, \: \mu_n(\{0\}) = \mu_n (\{1\}) = 1)$.
    Uzmemo li bilo koju to\v cku $a \in \{ 0, \: 1 \}^{\nat}$, postoji $\lim\limits_{n \to \infty} (\mu_1 (\{ a_1 \}) \cdot \mu_2 (\{ a_2 \}) \cdot \dots \mu_n (\{ a_n \}) ) = 1$ pa bi "mjera" na $\{ 0, \: 1 \}^{\infty}$ koja bi bila kandidat za $\dirProd{n \in \nat}{} \mu_n$ imala neprebrojivo jedno\v clanih skupova mjere $1$, to jest nikako ni bi mogla biti $\sigma$-kona\v cna. 
\end{pr}

U vjerojatnosnom slu\v caju sve je u redu.

\begin{tm}  \label{tm:4.21}
    Neka je $(\Omega_n, \: \famF_n, \: \Pp_n), \; n \in \nat$, niz vjerojatnosnih prostora.
    Postoji i jedinstvena je vjerojatnosna mjera $\masP$ na $\urePar{\produkt{n \in \nat}{} \Omega_n}{\dirProd{n \in \nat}{} \famF_n}$ takva da je za svaki $n \in \nat$, za svaki izbor $A_1 \in \famF_1, \dots, A_n \in \famF_n$,
    \begin{equation*}
        %\masP (\praslika{\pi_1}(A_1) \cap \dots \cap \praslika{\pi_n}(A_n))
        \masP(\pi_1 \in A_1, \: \dots, \pi_n \in A_n)
        = \produkt{k = 1}{n} \masP_k (A_k).
    \end{equation*}
    Tu vjerojatnost nazivamo \emph{beskona\v cnim produktom} i ozna\v cavamo sa $\dirProd{n \in \nat}{} \masP_n$.
\end{tm}

\begin{zad} \label{zad:4.22}
    Formulirajte op\' ci teorem o produktu vjerojatnosnih mjera i poka\v zite da se dokazuje u su\v stini potpuno isto kao kao i teorem \ref{tm:4.21}.
    % uputa: vidi zadatak 6, a u dokazu koristi zadatak 1.12 b i Cantorov diagonalni postupak.
\end{zad}

    %%%%%%%%%%%%%%%%%%%%%%%%%%%%%%%%%%%%%%%%
    %%  distribucije slucajnih elemenata  %%
    %%%%%%%%%%%%%%%%%%%%%%%%%%%%%%%%%%%%%%%%

    % distribucije slucajnih elemenata

\chapter{Distribucije slu\v cajnih elemenata}

\begin{pr}  \label{pr:5.1}
    Promatrajmo vjerojatnosni prostor $\vjerojatnosniProstor = (\segment{0}{1}, \: \borel{\segment{0}{1}}, \: \restr{\lambda}{\borel{\segment{0}{1}}})$ i dvije slu\v cajne varijable $X (\omega) := \omega$, $Y (\omega) := 1 - X (\omega)$.
    O\' cito $X(\omega) = Y(\omega)$, za svaki $\omega \in \Omega \setminus \{ \frac{1}{2} \}$, to jest kao funkcije u skupovnom smislu $X$ i $Y$ su "potpuno" razli\v cite.
    \v Sto mo\v zemo re\' ci o "vjerojatnosnim informacijama" koje mo\v zemo saznati o $X$ i $Y$?
    Za proizvoljan $A \in \famF = \borel{\segment{0}{1}}$ imamo $1-A = 1 + (-A) \in \borel{\segment{0}{1}}$ i $\lambda (1 - A) = \lambda (A) \implies \vjeroj{X \in A} = \vjeroj{Y \in A}, \; forall A \in \famF$.
    Dakle iako su $X$ i $Y$ vlo razli\v cite slu\v cajne varijable, njihove "vjerojatnosne distribucije" su jednake.
\end{pr}

\begin{defn}    \label{defn:5.2}
    Neka je $\vjerojatnosniProstor$ vjerojatnosni prostor na kojem je definiran slu\v cajan element $X$ s vrijednostima u izmjerivom prostoru $\urePar{E}{\famE}$.
    \emph{Vjerojatnosna distribucija} slu\v cajnog elementa $X$ je vjerojatnosna mjera $\masP_X$, definirana na $\urePar{E}{\famE}$
    sa:
    \begin{equation*}
        \masP_X (A) := \masP(X \in A), \; A \in \famE.
    \end{equation*}
\end{defn}

Uo\v cimo $\masP_X$ je poseban slu\v caj tako zvane mjere inducirane izmjerivim preslikavanjem.

\begin{nap} \label{nap:5.3}
    \begin{enumerate}[label=(\alph*)]
        \item Uo\v cimo da je u $\masP$ sadr\v zana "sva vjerojatnosna informacija" o $X$.
        \item Bavljenje slu\v cajnim elementima s vjerojatnostima u $\urePar{E}{\famE}$ (bez obzira na kojem vjerojatnosnom prostoru bili definirani) mo\v ze se svesti na tako zvani \emph{kanonski slu\v caj}.
        \item Preciznije, mi fiksiramo "vjerojatnosni prostor" i "slu\v cajni element", to jest uzmemo $\tilde{\Omega} := E$, $\tilde{\famF} := \famE$, $\tilde{X} := id : E \to E$ i dr\v zimo uvijek fiksirano, a slu\v cajni element "realiziramo" postavljanjem odgovaraju\' ce vjerojatnosne mjere $\masP_X$ na taj prostor.
        Tada $X$ (uz $\masP$) i $\tilde{X}$ (uz $\masP_X$) imaju istu distribuciju (i to je $\masP_X$).
        \item Time se uvodi nova ideja o pojmu jednakosti u vjerojatnosti.
        vidjeli smo da za slu\v cajne elemente (definirane na istom prostoru s vjerojatnostima u istom prostoru) mo\v zemo promatrati obi\v cnu skupovnu jednakost $X = Y$, jednakost gotovo sigurno $(g.s.)$, a sada i jednakost po distribuciji.
        Preciznije, ako su $X$ i $Y$ slu\v cajni elementi s vjerojatnostima u $\urePar{E}{\famE}$, definirani na vjerojatnosnim prostorima $(\Omega_1, \: \famF_1, \: \masP)$ i $(\Omega_1, \: \famF_2, \: \masO)$, re\' ci \' cemo da su \emph{jednaki po distribuciji}, i pisati $X \distJed Y$, ako je $\masP_X = \masO_Y$.
        Lako se vidi (primjer \ref{pr:5.1})
        \begin{equation*}
            X = Y
            \begin{smallmatrix}
                \implies&  \\
                \notimpliedby&
            \end{smallmatrix}
            X = Y \; (g.s.)
            \begin{smallmatrix}
                \implies&  \\
                \notimpliedby&
            \end{smallmatrix}
            X \distJed Y.
        \end{equation*}
    \end{enumerate}
\end{nap}

U slu\v caju $\urePar{E}{\famE} = \urePar{\extReal}{\borel{\extReal}}$ informaciju o $\masP_X$ mo\v zemo prevesti na jezik obi\v cnih realnih funkcija realne varijable.

\begin{defn}    \label{defn:5.4}
    Neka je $X$ pro\v sirena slu\v cajna varijabla na $\vjerojatnosniProstor$.
    Funkcija $F_X : \extReal \to \segment{0}{1}$, definiran sa
    \begin{equation*}
        F_X (a) := \masP (X \leq a) = \masP (\skup{\omega \in \Omega}{X(\omega) \leq a}) = \masP_X (\segment{-\infty}{a}), \; a \in \real
    \end{equation*}
    nazivamo \emph{vjerojatnosnom funkcijom distribucije} slu\v cajnog elementa $X$.
\end{defn}

Funkcija $F_X$ je neopadaju\' ca (jer $a \leq b \implies \segment{-\infty}{a} \subseteq \segment{-\infty}{b} \implies \masP_X (\segment{-\infty}{a}) \leq \masP_X (\segment{-\infty}{b}) $) i neprekidna zdesna (jer $a_n \searrow a \implies \segment{-\infty}{a} = \presjek{n = 1}{\infty} \segment{-\infty}{a_n}$).
Uo\v cimo da je $F_X (-\infty) \geq 0$ i da je
\begin{equation*}
    F_X (-\infty) = 0 \iff \masP (X = -\infty) = 0,
\end{equation*}
a zbog neprekidnosti zdesna je $F_X (-\infty) = \lim\limits_{a \ \searrow -\infty} F_X (a)$.
S druge strane postoji i $\lim\limits_{a \nearrow +\infty} F_X (a) \leq F_X(+\infty) = 1$, te je
\begin{equation*}
    F_X(+\infty) = \lim\limits_{a \nearrow +\infty} F_X (a) \iff \masP (X = +\infty) = 1.
\end{equation*}

Posebno, $X$ je slu\v cajna varijabla ako i samo ako vrijedi:
\begin{align}    \label{jed:5.5}
    \begin{split}
        F_X(-\infty) &= 0, \\
        \lim\limits_{a \nearrow +\infty} F_X(a) &= 1
    \end{split}
\end{align}

Mo\v zemo gledati s druge strane i re\' ci da je $F: \extReal \to \segment{0}{1}$ \emph{p.d.F.} ako je $F$ neopadaju\' ca, neprekidna zdesna i zadovoljava uvjet \eqref{jed:5.5}.
Time dolazimo do fundamentalog pitanja: "Je li svaka p.d.F. vjerojatnosna funkcija distribucije neke slu\v cajne varijable?"
Odgovor je: Da!

\begin{tm}  \label{tm:5.6}
    Neka je $F: \extReal \to \segment{0}{1}$ neopadaju\' ca, neprekidna zdesna i zadovoljava uvijet \eqref{jed:5.5}, tada postoji vjerojatnosni prostor $\vjerojatnosniProstor$ i slu\v cajana varijabla $X$ na tom vjerojatnosnom prostoru, takva da je $F$ vjerojatnosna funkcija distribucije od $X$.
\end{tm}

\begin{proof}
    Uzmemo kanonski slu\v caj $\Omega = \real, \; \famF = \borel{\real}$ i $X = id : \real \to \real$.
    Sjetimo se da je $\famI$ poluprsten i $\sigAlg{\famI} = \famF$, pri \v cemu je $\famI$ iz primjera \ref{pr:2.3}.
    Za $\lijInt{a}{b} \in\famI$ definiramo
    \begin{equation}    \label{jed:5.7}
        \masP_F (\lijInt{a}{b}) := F (b) - F(a)
    \end{equation}
    i nije prete\v sko pokazati da je $\masP_F$ $\sigma$-aditivna na $\famI$.
    Po korolaru \ref{kor:2.8} i po teoremu \ref{tm:2.11} postoji jeedinstveno pre\v sirenje $\masP_F$ koje je vjerojatnost na $\borel{\real}$.
    Dakle, $X$ je slu\v cajna varijabla na $(\Omega, \: \famF, \: \masP_F)$ i $(\masP_F)_X = \masP_F$, pa je $F_X = F$.
\end{proof}

Dakle vrijedi
\begin{equation*}
    X \distJed Y \iff F_X = F_Y.
\end{equation*}

\begin{nap} \label{nap:5.8}
    \begin{enumerate}[label=(\alph*)]
        \item Za slu\v cajnu varijablu $X$ postoji $1-1$ korespondencija izme\dj u $\masP_X$ i $F_X$, posebno, sav "vjerojatnosni sadr\v zaj" o $X$ sadr\v zan je i u $F_X$.
        \item Teorem \ref{tm:5.6} opravdava \v cesto kori\v stenje funkcija distribucija u praksi.
        Istra\v ziva\v cima u raznim podru\v cjima mo\v ze biti va\v zno samo pron\' ci "distribuciju" i oni se dalje bave vjerojatnostima bez da se bave pitanjem egzistencije istih.
        \item Usporedi \eqref{jed:5.7} sa \eqref{jed:2.14}.
        \item Radi se u stvari o istoj konstrukciji samo je u \eqref{jed:5.7} na\v sa $F$ "normirana" na poseban na\v cin, to jest $F(+\infty)$ je stavljeno da bude $1$.
        Jasno, za svaki $c \in \real$, $F + c$ bi dala isto vjerojatnost $\masP_{F + c}$, kao i $\masP_F$.
        Op\' cenito, ako je $F: \real \to \real$ neopadaju\' ca i neprekidna zdesna, tada \eqref{jed:2.14} i isti dokaz kao u teoremu \ref{tm:5.6} (samo koristimo zadatak \ref{zad:2.9}) daje tvrdnju o egzistenciji i jedinstvenosti $\sigma$-kona\v cne mjere $\mu$ na $\urePar{\real}{\borel{\real}}$ takve da vrijedi \eqref{jed:2.14}. Svaku funkciju $F : \real \to \real$, koja je neopadaju\' ca i neprekidna zdesna zvat \' cemo \emph{d. f.}.
        Dakle \eqref{jed:2.14} daje $1-1$ korespondenciju izme\dj u
        \begin{equation}    \label{jed:5.9}
            \begin{matrix}
                &\skup{F + c}{c \in \real}\\
                &F \; \textnormal{d. f.}
            \end{matrix}
            \longleftrightarrow
            \begin{matrix}
                &\mu_F \;\; \sigma \textnormal{-kona\v cna}\\
                &\textnormal{na } \; \urePar{\real}{\borel{\real}}
            \end{matrix}
        \end{equation}
        Za d. f. $F$ postoje (u $\extReal$) limesi $F(-\infty) = \lim\limits_{a searrow -\infty} F(a)$ i $F(+\infty) = \lim\limits_{a nearrow \infty} F(a)$ i $\mu_F(\real) = F(+\infty) - F(-\infty)$.
    \end{enumerate}
\end{nap}

\begin{zad} \label{zad:5.10}
    Opi\v site koresponednciju kao u \eqref{jed:5.9} na $\urePar{\real}{\borel{\real}}$i poka\v zite da je koresponedencija u \eqref{jed:5.9} poseban slu\v caj koresponedencije na $\extReal$.
\end{zad}

\begin{zad} \label{zad:5.11}
    Neka je $X = (X_1, \ldots, X_d)$ slu\v cajan vektor na $\vjerojatnosniProstor$.
    \begin{enumerate}[label=(\alph*)]
        \item Opi\v si funkciju distribucije $F_x$ u ovom slu\v caju.
        \item Iska\v zite i doka\v zite analogon teorema \ref{tm:5.6}
        %uputa: umjesto \eqref{jed:5.7} promatrajte 
        %\begin{equation*}
        %    \Delta^{a, \: b}_{d} F := \suma{
        %    \begin{smallmatrix}
        %        (x_1, \ldots, x_d) = x \\
        %        x_i \in \{a_i, \; b_i\}
        %    \end{smallmatrix}}{}
        %    (-1)^{\#(x)} F(x_1, \ldots, x_d)
        %\end{equation*}
        %gdje je \# (x) := \card{\skup{1 \leq i \leq d}{x_i = a_i}} i usporedite sa \masP_X (\lijInt{a}{b}) := \vjeroj{a_i < X_i \leq b_i \: : \: \textnormal{ za sve } i = 1, \ldots, d}.  
    \end{enumerate}
\end{zad}

Dobro su nam poznati primjeri brojnih raznih distribucija na $\real^d$, osobito za $d = 1$.
Korisna je i sljede\' ca klasifikacija, gledamo $d = 1$ zbog jednostavnosti zapisa.
Slu\v cajna varijabla je $X$ \emph{diskretna} ako postoji prebrojiv skup $S$ takva da je $X \in S \; (g.s.)$, dok je p.d.F. $G$ \emph{diskretna} ako postoji prebrojiv skup $\skup{\urePar{x_j}{p_j}}{j \in J} \subseteq \real \times \segment{0}{1}$, takav da je $G(x) = \suma{\begin{smallmatrix} j \in y\\ x_j \leq x \end{smallmatrix}}{} p_j$.
Lako se vidi da je $X$ diskretna ako i samo ako je $F_X$ diskretna.
Ako jo\v s imamo i $(j_1 \neq j_2 \implies x_{j_1} \neq x_{j_2})$ tada je $p_j = \vjeroj{X = x_j}, \; \forall j \in J$.
Uo\v cimo da skup $\skup{\urePar{x_j}{p_j}}{j \in J}$ u potpunosti opisuje $F_X$.

Sli\v cno, ka\v zemo da je p.d.F. $G$ \emph{apsolutno neprekidna} ako postoji izmjeriva funkcija $g : \real \to \desInt{0}{+\infty}$, takva da je $G(x) = \int\limits_{-\infty}^{x} g(t) \: d \lambda (t)$.
Ka\v zemo da je slu\v cajna varijabla $X$ \emph{apsolutno neprekidna} ako je $F_X$ aposlutno neprekidna.
Po Radon-Nykondimovom teoremu znamo da je $X$ apsolutno neprekidna ako i samo ako je $\masP_X << \lambda$, te da tada postoji i $\lambda-(g.s.)$ jedinstvena Radon-Nykondimovom derivacija $\frac{d \masP_X}{d \lambda}$ koja igra ulogu funkcije $g$ odeozgo.
U ovom slu\v caju nazivamo je funkcijom gusto\' ce i ozna\v cavamo sa $f_X$.
Postoji jako puno funkcija gusto\' ce jer je $f: \real \to \desInt{0}{+\infty}$ funkcija gusto\' ce neke slu\v cajne varijable $X$ ako i samo ako je $f$ izmjeriva i vrijedi $\int\limits_{\real} f(t) \: d \lambda (t) = 1$.

Uo\v cimo nadalje da se za $\sigma$-kona\v cne mjere $\mu$ na $\urePar{\real}{\borel{\real}}$ i d.f. $F_{\mu}$ pripadni integrali potpuno podudaraju, to jest:
\begin{equation*}
    \int_{\real} f(x) \: d \mu (x) = \int_{-\infty}^{\infty} f(x) \: d F_{\mu} (x).
\end{equation*}

Posebno, ako je $X$ slu\v cajni element na $\vjerojatnosniProstor$ s vrijednostima u $\urePar{E}{\famE}$ i $f:E \to \real$ izmjeriva funkcija, tada je
\begin{equation*}
    f \circ X \in L^1 \vjerojatnosniProstor \iff f \in L^1 (E, \: \famE, \: \masP_X)
\end{equation*}
i vrijedi:
\begin{equation*}
    \int_{\Omega} (f \circ X) (\omega) \: d \masP (\omega) = \int_E f(x) \: d \masP_X (x) = \int_{-\infty}^{\infty} u \: d F_{f \circ X} (u).
\end{equation*}
% u tikz-u dodaj dijagram
U tom slu\v caju ka\v zemo da pstoji o\v cekivanje od $f \circ X$ i bilo koje od gornjih integrala ozna\v cavamo sa $\ocek{f \circ X}$
(uo\v cimo da $f \circ X \in L^1 \vjerojatnosniProstor$ zna\v ci da je $\ocek{|f \circ X|} < +\infty$).

Za slu\v cajnu varijablu $X$ na $\vjerojatnosniProstor$ zanimaju nas neki posebni slu\v cajevi ovih integrala.
Ako je $p > 0$, tada $\ocek{|X|^p} \in \segment{0}{+\infty}$ i zovemo ga \emph{$p$-ti apsolutni moment} od $X$.
Ako je $\ocek{|X|^p} < +\infty$, tada je $X \in L^p := L^p \vjerojatnosniProstor$ i $\norma{X}_p = [\ocek{|X|^p}]^{\frac{1}{p}}$ te postoji i $\ocek{X^p}$ \v sto nazivamo \emph{$p$-ti moment} od $X$.
Ako je $p \leq q$, tada je $L^q \subseteq L^p$ (i $\norma{X}_p \leq \norma{X}_q$), pa postojanje $q$-tog momenta implicira postojanje $p$-tog momenta.
Ako je $X \geq 0$ i $p > 0$ jednostavnim ra\v cunom daje
\begin{align}   \label{jed:5.12}
    \begin{split}
        \ocek{X^p} =& \int_0^{+\infty} t^p \: d \masP_X (t) = \int_0^{\infty} \Big( \int_0^t p \cdot u^{p - 1} \: d u \Big) \: d \masP_X(t) = (\textnormal{Fubini})\\
        =& p \int_0^{\infty} \: du \int_u^t u ^{p-1} \: d \masP_X (t) = p \int_0^{+\infty} u^{p-1} \masP_X (\segment{u}{+\infty}) \: du \\
        =& \int_0^{+\infty} p \cdot u^{p-1} \vjeroj{X \geq u} \: du.
    \end{split}
\end{align}

Ako je $X = (X_1, \ldots, X_d)$ $d$-dimenzionalni slu\v cajni vektor i $X_k \in L^1, \; (\forall k \in \{1, \ldots, d\})$, tada i $X$ ima o\v cekivanje i pi\v semo $\masE X := (\masE X_1, \ldots, \masE X_d)$.

\begin{zad} \label{zad:5.13}
    Neka je $X$ $d$-dimenzionalni slu\v cajni vektor koji ima o\v cekivanje i $f: \real^d \to \real$ konveksna funkcija, to jest funkcija za koju vrijedi:
    \begin{equation*}
        (\forall x, \: y \in \real^d)(\forall \lambda \in \segment{0}{1}) \quad f (\lambda y + (1 - \lambda) y) \leq \lambda f(x) + (1 - \lambda) f(y).
    \end{equation*}
    Ako  postoji (u \v sirem smislu) $\masE [ f(X) ]$, tada je
    \begin{equation*}
        f (\masE X) \leq \masE [ f(X) ].
    \end{equation*}
    To nazivamo \emph{Jensenovom nejednaskosti}.
\end{zad}

Osim prvog momenta slu\v cajen varijable, kojeg jo\v nazivamo i o\v cekivanjem, va\v zan je i drugi moment to jest \emph{drugi centrali moment} $\masE [ (X - \masE X)^2 ]$, kojeg nazivamo \emph{varijancom}.
Uo\v cimo da varijanca postoji ako i samo ako postoji drugi moment od $X$ i tada je
\begin{equation*}
    Var \: X := \masE [ (X - \masE X)^2 ] = \masE [ X^2 ] - (\masE X)^2.
\end{equation*}


    \part{Nezavisnost}

    %%%%%%%%%%%%%%%%%%
    %%  definicije  %%
    %%%%%%%%%%%%%%%%%%

    % nezavisnost definicije

\chapter{Definicije}

U ovom cijelom poglavlju $\vjerojatnosniProstor$ je vjerojatnosni prostor.
Doga\dj aji $A, \; B \in \famF$ su nezavisni ako vrijedi:
\begin{equation*}
    \vjeroj{A \cap B} = \vjeroj{A} \cdot \vjeroj{B}.
\end{equation*}

\begin{pr}  \label{pr:6.1}
    Iako se nezavisnost jako \v cesto koristi, zapravo se "rijetko" doga\dj a
    % slika
    \begin{figure}[H]
        \centering
        \begin{subfigure}[b]{0.45\textwidth}
            \begin{tikzpicture}
                \draw (0, 0) rectangle (6, 4);
                \draw (0.5, 1) rectangle (2.5, 3);
                \draw (3.25, 0.75) rectangle (5.75, 3.25);
                \node at (0.5, 2) [label=left:$a$, xshift=5pt] {};
                \node at (1.5, 1) [label=below:$a$, yshift=5pt] {};
                \node at (4.5, 0.75) [label=below:$b$, yshift=5pt] {};
                \node at (3.25, 2) [label=left:$b$, xshift=5pt] {};
                \node at (1.5, 2) [label=center:$A$] {};
                \node at (4.5, 2) [label=center:$B$] {};
            \end{tikzpicture}
        \end{subfigure}
        %%%%%%%
        \begin{subfigure}[b]{0.45\textwidth}
            \begin{tikzpicture}
                \draw (0, 0) rectangle  (6, 4);
                \draw (1.25, 1) rectangle (3.25, 3);
                \draw (2.75, 0.75) rectangle (5.25, 3.25);
                \fill[pattern=south west lines] (2.75, 1) rectangle (3.25, 3);
                \node at (1.25, 2) [label=left:$a$, xshift=5pt] {};
                \node at (2.25, 1) [label=below:$a$, yshift=5pt] {};
                \node at (4, 0.75) [label=below:$b$, yshift=5pt] {};
                \node at (5.25, 2) [label=right:$b$, xshift=-5pt] {};
                \node at (2.25, 2) [label=center:$A$] {};
                \node at (4, 2) [label=center:$B$] {};
                \node at (3, 1) [label=above:$x$, yshift=-5pt] {};
            \end{tikzpicture}
        \end{subfigure}
    \end{figure}
    \begin{equation*}
        \begin{aligned}
            \vjeroj{A} &= a^2\\
            \vjeroj{B} &= b^2\\
            \vjeroj{A \cap B} &= a \cdot x
        \end{aligned}
    \end{equation*}
    Odakle vidimo da slijedi:
    \begin{equation*}
        \vjeroj{A \cap B} = \vjeroj{A} \cdot \vjeroj{B} \iff x = a b^2.
    \end{equation*}
\end{pr}

Lako se vidi:\\
%$A$ je nezavisan sam sa sobom ako i samo ako je $\vjeroj{A} = 0 \lor %\vjeroj{A} = 1$ ako i samo ako $A$ i $B$ su nezavisni za svaki $B \in \famF$.
\begin{align}   \label{jed:6.2}
    \begin{split}
        \begin{matrix}
            A \textnormal{ je nezavisan}\\
            \textnormal{sa samim sobom}
        \end{matrix}
        &\iff
        \vjeroj{A} \in \{0, \; 1\}\\
        &\iff
        \begin{matrix}
            \textnormal{$A$ i $B$ su nezavisni}\\
            \textnormal{za svaki } B \in \famF
        \end{matrix}
    \end{split}
\end{align}

Laki ra\v cun tipa
\begin{align*}
    \vjeroj{A \cap B^c} &= 1 - \vjeroj{A^c \cup B} = 1 - \vjeroj{A^c \cup (A \cap B)}\\
    &= 1 - \vjeroj{A^c} - \vjeroj{A \cap B} = (\textnormal{nezavisnost})\\
    &= \vjeroj{A} - \vjeroj{A} \cdot \vjeroj{B}\\
    &= \vjeroj{A} \cdot \vjeroj{B^c}
\end{align*}
daje:
\begin{align}   \label{jed:6.3}
    \begin{split}
        A, \; B \textnormal{ nezavisni} &\iff A, \; B^c \textnormal{ nezavisni}\\
        &\iff A^c, \; B \textnormal{ nezavisni}\\
        &\iff A^c, \; B^c \textnormal{ nezavisni}.
    \end{split}
\end{align}

\v Sto ako imamo 3 ili vi\v se doga\dj aja?
Treba biti oprezan.

\begin{defn}    \label{defn:6.3-1}
    Doga\dj aji $A_1, \ldots, A_n \in \famF$ su nezavisni ako za svaki $k \in \{1, \ldots, n\}$ i svaki izbor me\dj usobno razli\v citih $i_1, \ldots, i_k \in \{1, \ldots, n\}$ vrijedi
    \begin{equation*}
        \vjeroj{A_{i_1} \cap \ldots  \cap A_{i_k}} = \vjeroj{A_{i_1}} \cdot \ldots \cdot \vjeroj{A_{i_k}}.
    \end{equation*}    
\end{defn}

\begin{zad} \label{zad:6.4}
    Na\dj ite primjere tri doga\dj aja $A, \: B, \: C \in \famF$ koji nisu nezavisni, ali za njih vrijedi:
    \begin{enumerate}[label=(\roman*)]
        \item U parovima su nezavisni
        \item $\vjeroj{A \cap B \cap C} = \vjeroj{A} \cdot \vjeroj{B} \cdot \vjeroj{C}$.
    \end{enumerate}
\end{zad}

\begin{rj}[\ref{zad:6.4}]
    Definirajmo skup elementarnih doga\dj aja $\Omega$ sa
    \begin{equation*}
        \Omega = \{0, 1\}^3 = \bigSkup{(\omega_1, \omega_2, \omega_3)}{\omega_k \in \{0, 1\}, \;  k = 1, 2, 3},
    \end{equation*}
    $\sigma$-algebru $\partitive{\Omega}$ te vjerojatnost definiranu sa
    \begin{equation*}
        \masP (\omega) = \frac{1}{8}, \quad \forall \omega \in \Omega.
    \end{equation*}
    Sada definiramo slu\v cajne varijable $X_1, X_2, X_3$ sa
    \begin{equation*}
        (\omega_1, \omega_2, \omega_3) \xmapsto{X_k} \omega_k, \quad k = 1, 2, 3.
    \end{equation*}

    Definirajmo na poslijetku doga\dj aje
    \begin{equation*}
        \begin{aligned}
            A_1 &:= \{ X_1 = X_2 \}\\
            A_2 &:= \{ X_2 = X_3 \}\\
            A_1 &:= \{ X_3 = X_1 \}.
        \end{aligned}
    \end{equation*}
    Primjetimo da za $i \neq j$ imamo
    \begin{equation*}
        \begin{aligned}
            \masP (X_i = 0, X_j = 0) &= \masP (X_i = 0) \cdot \masP (X_j = 0) = \frac{1}{2} \cdot \frac{1}{2} = \frac{1}{4}\\
            \masP (X_i = 1, X_j = 1) &= \masP (X_i = 1) \cdot \masP (X_j = 1) = \frac{1}{2} \cdot \frac{1}{2} = \frac{1}{4}.
        \end{aligned}
    \end{equation*}
    Dakle vidimo da vrijedi
    \begin{equation*}
        \masP (A_i) = \frac{1}{2}, \quad i = 1, 2, 3.
    \end{equation*}
    Odavde slijedi
    \begin{equation*}
        \masP (A_i \cap A_j) = \masP (X_1 = X_2 = X_3) = \frac{1}{4}.
    \end{equation*}
    S druge pak strane imamo
    \begin{equation*}
        \masP (A_1 \cap A_2 \cap A_3) = \masP (X_1 = X_2 = X_3) = \frac{1}{4} \neq \frac{1}{8} = \masP (A_1) \cdot \masP (A_2) \cdot \masP (A_3).
    \end{equation*}
    Dakle, dani skupovi su u parovima nezavisni, me\dj utim nisu nezavisni.
\end{rj}

\begin{zad} \label{zad:6.5}
    Doka\v zite da dodavanjem ili oduzimanjem doga\dj aja vjerojatnosti $0$ ili $1$, ne\' cemo promjeniti "status" nezavisnosti kona\v cne familije doga\dj aja.
\end{zad}

Je li uvijek jednostavno "intuitivno" prepoznati nezavisnost?

\begin{pr}  \label{pr:6.6}
    Promatramo obitelj s troje djece i pretpostavljamo da su svih 8 mogu\' cnosti (\v z\v z\v z, \v z\v zm, ..., mmm) jednako vjerojatne.
    Lako se vidi da su
    \begin{itemize}
        \item[] $A =$ "postoje oba spola"
        \item[] $B = $ "postoji najvi\v se jedno \v zensko djete"
    \end{itemize}
    nezavisni ($\vjeroj{A} = \frac{6}{8}, \; \vjeroj{B} = \frac{4}{8}, \; \vjeroj{A \cap B} = \frac{3}{8}$).
    \v Sto ako iste doga\dj aje promatramo za obitelji s 4 djece? Tada vi\v se nisu.
\end{pr}

Vrijedi li ne\v sto poput \eqref{jed:6.3} za slu\v caj $n$ doga\dj aja?
Zapravo da. Nije te\v sko za pokazati:
\begin{equation}    \label{jed:6.7}
    \begin{matrix}
        A_1, \ldots, A_n \in \famF\\
        \textnormal{su nezavisni doga\dj aji}
    \end{matrix}
    \iff
    \begin{matrix}
        \textnormal{za svaki od $2^n$ izbora doga\dj aja}\\
        B_1 \in \{A_1, \: A_1^c\}, \ldots, B_1 \in \{A_n, \: A_n^c\}\\
        \vjeroj{B_1 \cap \ldots \cap B_n} = \vjeroj{B_1} \: \ldots \: \vjeroj{B_n}
    \end{matrix}
    .
\end{equation}

Uo\v cimo da \eqref{jed:6.3} i \eqref{jed:6.7} navode na to da su nezavisnosti doga\dj aja zapravo posebni slu\v cajevi nezavisnosti familija doga\dj aja.

\begin{defn}    \label{defn:6.8}
    Neka je $\Lambda \neq \varnothing$ i $\indFamilija{\famG_\lambda}{\lambda \in \Lambda}$ indeksirani skup familija doga\dj aja, to jest $\famG_\lambda \subseteq \famF, \; \forall \lambda \in \Lambda$.
    Tada je $\indFamilija{\famG_\lambda}{\lambda \in \Lambda}$ \emph{nezavisna} (ili nepreciznije, \emph{familije} $\famG_\lambda$ \emph{su me\dj usobno nezavisne}), ako, za svaki $n \in \nat \setminus \{1\}$, za svaki izbor me\dj usobno razli\v citih $\lambda_1, \ldots, \lambda_n \in \Lambda$ i za svaki izbor $A_1 \in \famG_{\lambda_1}, \ldots, A_n \in \famG_{\lambda_n}$, vrijedi:
    \begin{equation*}
        \vjeroj{A_1 \cap \ldots \cap A_n} = \vjeroj{A_1} \cdot \ldots \cdot \vjeroj{A_n}.
    \end{equation*}
\end{defn}

\begin{nap} \label{nap:6.9}
    \quad
    \begin{enumerate}[label=(\alph*)]
        \item Od posebnog interesa je slu\v caj kada su sve $\famG_{\lambda}$ $\sigma$-algebre.
        \item
        \begin{equation*}
            \begin{aligned}
                A, \; B \textnormal{ su nezavisni} &\iff \{A\}, \; \{B\} \textnormal{ su nezavisne}\\
                &\iff \sigma \textnormal{-algebre } \{ \varnothing, \: A, \: A^c, \: \Omega \}, \; \{ \varnothing, \: B, \: B^c, \: \Omega \} \textnormal{ su nezavisne}.
            \end{aligned}
        \end{equation*}
        \item   \label{nap:6.9c}
        Sada se mo\v ze tretirati proizvoljna familija doga\dj aja.
        Ako je $\indFamilija{A_t}{t \in T} \subseteq \famF$, tada su $A_t, t \in T$, me\dj usuobno nezavisne ako su familije $\{A_t\}, \; t \in T$, me\dj usobno nezavisne.
        Lako se vidi da je to ispunjeno ako i samo ako je svaka kona\v can potfamilija na\v se familije sastavljena od me\dj usobno nezavisnih doga\dj aja.
        \item \label{nap:6.9d}
        Ako (u definiciji \ref{defn:6.8}) imamo $\famH_\lambda \subseteq \famG_\lambda, \; \forall \lambda \in \Lambda$, tada nezavisnost familija $\famG_\lambda, \; \lambda \in \Lambda$, implicira nezavisnost familija $\famH_\lambda, \; \lambda \in \Lambda$ (dokaz je trivijalan).
        \item Ako (u definiciji \ref{defn:6.8}) imamo $\tilde{\Lambda} \subseteq \Lambda$ i $\tilde{\Lambda} \neq \varnothing$, tada nezavisnost familija $\famG_\lambda, \; \lambda \in \Lambda$, implicira nezavisnost familija $\famG_\lambda, \; \lambda \in \tilde{\Lambda}$.
        \item \label{nap:6.9f}
        Uo\v cimo da definicija \ref{defn:6.8} uklju\v cuje i slu\v caj kada je neka od $\famG_\lambda = \varnothing$.
        Dodavanje ili oduzimanje takve familije unutar indeksiranog skupa $\{\famG_\lambda\}$ ne mjenja "status nezavisnosti" istog.
        Ako gledamo nadklase takvih $\famG_\lambda$, onda obi\v cno koristimo konvenciju $\sigAlg{\varnothing} = \{ \varnothing, \; \Omega\}$.
    \end{enumerate}
\end{nap}

O\' cito, smanjivati klase na razne na\v cine ne\' ce ugroziti nezavisnost.
\v Sto ako pove\' camo?

\begin{tm}  \label{tm:6.10}
    Neka je $\Lambda \neq \varnothing$ i $\indFamilija{\famG_\lambda}{\lambda \in \Lambda}$ indeksirani skup $\pi$-sistema doga\dj aja, to jest $\famG_\lambda \subseteq \famF$, za svaki $\lambda \in \Lambda$ i $\famG_\lambda$ je $\pi$-sistem, za svaki $\lambda \in \Lambda$.
    Ako je $\indFamilija{\famG_\lambda}{\lambda \in \Lambda}$ nezavisna, tada je i $\indFamilija{\sigAlg{\famG_\lambda}}{\lambda \in \Lambda}$ nezavisna.
\end{tm}

\begin{proof}
    Bez smanjenja op\' cenitosti (vidi napomenu \ref{nap:6.9} \ref{nap:6.9f}), uzmimo $\famG_\lambda \neq \varnothing$, za svaki $\lambda \in \Lambda$.
    Neka je $n \geq 2$, $\lambda_1, \ldots, \lambda_n \in \Lambda$, me\dj usobno razli\v citi i fiksirajmo izbor $A_1 \in \famG_{\lambda_1}, \ldots, A_n \in \famG_{\lambda_n}$.
    Promatramo familiju:
    \begin{equation*}
        \famD_1 := \skup{B \in \famF}{\vjeroj{B \cap A_2 \cap \ldots \cap A_n} = \vjeroj{B} \: \vjeroj{A_2} \ldots \vjeroj{A_n}}.
    \end{equation*}
    Po pretpostavci je $\famG_{\lambda_1} \subseteq \famD_1$.
    O\v cito je i $\Omega \in \famD_1$.
    Ako su $B_1, \; B_2 \in \famD_1$ i $B_1 \subseteq B_2$, tada vrijedi
    \begin{align*}
        \vjeroj{(B_2 \setminus B_1) \cap A_2 \cap \ldots \cap A_n} &= \vjeroj{B_2 \cap A_2 \cap \ldots \cap A_n} - \vjeroj{B_1 \cap A_2 \cap \ldots \cap A_n} \\
        &= [\vjeroj{B_2} - \vjeroj{B_1}] \: \vjeroj{A_2} \ldots \vjeroj{A_n}\\
        &= \vjeroj{B_2 \setminus B_1} \: \vjeroj{A_2} \ldots \vjeroj{A_n},
    \end{align*}
    odakle vidimo da je $B_2 \setminus B_1 \in \famD_1$.
    Sli\v cno dobijemo i
    \begin{equation*}
        \niz{B_n}{n \in \nat} \subseteq \famD_1, \quad B_n \nearrow B \implies B \in \famD_1.
    \end{equation*}
    Slijedi da je $\famD_1$ Dynkinova klasa koja sadr\v zi $\pi$-sistem $\famG_{\lambda_1}$.
    Stoga $\sigAlg{\famG_{\lambda_1}} \subseteq \famD_1$.
    Neka je sada $C_1 \in \sigAlg{\famG_{\lambda_1}}$ proizvoljan i:
    \begin{equation*}
        \famD_2 := \skup{B \in \famF}{\vjeroj{C_1 \cap B \cap A_3 \cap \ldots \cap A_n} = \vjeroj{C_1} \: \vjeroj{B} \: \vjeroj{A_3} \ldots \vjeroj{A_n}}.
    \end{equation*}
    Kako je $A_2 \in \famG_{\lambda_2}$ bio proizvoljan, slijedi da je $\famG_{\lambda_2} \subseteq \famD_2$, a lako se poka\v ze da je $\famD_2$ Dynkinova klasa.
    Slijedi $\sigAlg{\famG_{\lambda_2}} \subseteq \famD_2$.
    Time smo pokazali da za svaki izbor $B_1 \in \sigAlg{\famG_{\lambda_1}}, \; B_2 \in \sigAlg{\famG_{\lambda_2}}, \: A_3 \in \famG_{\lambda_3} \ldots, A_n \in \famG_{\lambda_n}$ vrijedi:
    \begin{equation*}
        \vjeroj{B_1 \cap B_2 \cap A_3 \cap \ldots \cap A_n} = \vjeroj{B_1} \: \vjeroj{B_2} \: \vjeroj{A_3} \ldots \vjeroj{A_n}.
    \end{equation*}
    Sada je jasno kako induktivno nastavoljamo s ovim dokazom i nakon kona\v cno mnogo koraka dobijemo \v zeljenu tvrdnju.
\end{proof}

\begin{kor} \label{kor:6.11}
    Neka je $\Lambda \neq \varnothing, \: \famF_\lambda \subseteq \famF$ $\sigma$-algebra za svaki $\lambda \in \Lambda$ i $\indFamilija{\famF_\lambda}{\lambda \in \Lambda}$ nezavisna.
    Ako je $\famT$ particija skupa $\Lambda$ i za svaki $T \in \famT$,
    \begin{equation*}
        \famF_T := \indSigAlg{\famF_\lambda}{\lambda \in T},
    \end{equation*}
    tada je $\indFamilija{\famF_T}{T \in \famT}$ nezavisna.
\end{kor}

\begin{nap}
    Prethodni korolar govori da disjunktni komadi ostaju nezavisni.
    \begin{equation*}
        \underbrace{\famF_{\lambda_1}, \ldots, \famF_{\lambda_n}} , \underbrace{\famF_{\lambda_{n + 1}}, \ldots}, \ldots
    \end{equation*}
\end{nap}

\begin{proof}
    Za $T \in \famT$ definiramo $\pi$-sistem
    \begin{equation*}
        \famG_T := \bigIndFamilija{A_1 \cap \ldots \cap A_n}{n \in \nat, \; A_i \in \unija{\lambda \in T}{}\famF_\lambda}.
    \end{equation*}
    Budu\' ci da su $\famF_\lambda$ $\sigma$-algebre, slijedi
    \begin{align*}
        \famG_T := \Big\{B_1 \cap \ldots \cap B_n \: \Big| \: &n \in \nat, \; n \leq \card{T}, \; \lambda_1, \ldots, \lambda_n \in T \textnormal{ me\dj usobno razli\v citih},\\ &B_1 \in \famF_{\lambda_1},\ldots, B_n \in \famF_{\lambda_n} \Big\},
    \end{align*}
    \v sto daje da su $\indFamilija{\famG_T}{T \in \famT}$ nezavisne.
    Sada tvrdnja slijedi iz teorema \ref{tm:6.10} i
    \begin{equation*}
        \sigAlg{\famG_T} = \famF_T.
    \end{equation*}
\end{proof}

\begin{prop}    \label{prop:6.12}
    Neka je $\Lambda \neq \varnothing$, $\famF_\lambda \subseteq \famF$ $\sigma$-algebra za svaki $\lambda \in \Lambda$.
    Neka je $\famT \subseteq \partitive{\Lambda}$ separiraju\' ca klasa na $\Lambda$ (to jest za svaki $\mu, \; \lambda \in \Lambda, \; \lambda \neq \mu$, postoji $T \in \famT$ takav da $\card{T \cap \{\lambda, \; \mu\}} = 1$).
    Tada je $\indFamilija{\famF_\lambda}{\lambda \in \Lambda}$, nezavisna ako i samo ako je, za svako $T \in \famT$,
    \begin{equation*}
        \Big\{ \indSigAlg{\famT}{t \in T}, \; \indSigAlg{\famF_t}{t \in T^c} \Big\},
    \end{equation*}
    nezavisna.
\end{prop}

\begin{proof}
    \quad
    \begin{enumerate}
        \item[$\implies$] Prema korolaru \ref{kor:6.11}.
        \item[$\impliedby$] Uo\v cimo da je dovoljno pokazati da za svaki kona\v can skup $S \subseteq \Lambda$ vrijedi $\indFamilija{\famF_\lambda}{\lambda \in S}$ nezavisna.
        
        Dokaz provodimo indukcijom po $\card{S}$.
        Za $\card{S} = 1$ tvrdnja trivijalno vrijedi.
        Pretpostavljamo da vrijedi za $\card{S} \leq n$ i uz tu pretpostavku promatramo skup $V \subseteq \Lambda$, $\card{V} = n + 1$.
        Posebno, $V$ sadr\v zi barem dva razli\v cita elementa iz $\Lambda$, recimo $\lambda$ i $\mu$, pa postoji $T \in \famT$ koji ih razlikuje.
        Posebno $V \cap T$ i $V \setminus T$ su neprazni skupovi.
        Po pretpostavci su $\indSigAlg{\famF_t}{t \in T}$ i $\indSigAlg{\famT}{t \in T^c}$ nezavisne, pa prema napomeni \ref{nap:6.9}  \ref{nap:6.9d} su $\indSigAlg{\famF_t}{t \in V \cap T}$ i $\indSigAlg{\famF_t}{t \in V \setminus T}$ nezavisne.
        Tada je
        \begin{equation*}
            \presjek{\lambda \in V}{} A_\lambda = B \cap C,
        \end{equation*}
        pri \v cemu su
        \begin{equation*}
            \begin{aligned}
                B &= \presjek{\lambda \in V \cap T}{} A_\lambda,\\
                C &= \presjek{\lambda \in V \setminus T}{} A_\lambda.
            \end{aligned}
        \end{equation*}
        Tada je $\vjeroj{B \cap C} = \vjeroj{B} \cdot \vjeroj{C}$, a $\card{V \cap T} \leq n$ i $\card{V \setminus T} \leq n$.
        Po pretpostavci indukcije dobivamo
        \begin{equation*}
            \masP \Big( \presjek{\lambda \in V}{} A_\lambda \Big) = \produkt{\lambda \in V}{} \vjeroj{A_\lambda}.
        \end{equation*}
    \end{enumerate}
\end{proof}

\begin{zad} \label{zad:6.13}
    Neka je $\famF_n \subseteq \famF$ $\sigma$-algebra, za svaki $n \in \nat$.
    Ako su, za svaki $n \in \nat$, nezavisne $\sigma$-algebre $\indSigAlg{\famF_k}{k \leq n}$ i $\famF_{n + 1}$, je li tada i $\indFamilija{\famF_n}{n \in \nat}$ nezavisna?
    \v Sto ako pretpostavimo da su $\famF_m$ i $\famF_n$ nezavisne, \v cim su $n, \; m \in \nat$ i $n \neq m$?
\end{zad}

\begin{zad} \label{zad:6.14}
    Neka je $\indFamilija{A_n}{n \in \nat} \subseteq \famF$ familija nezavisnih doga\dj aja.
    Vrijedi li
    \begin{equation*}
        \masP \Big( \presjek{n \in \nat}{} A_n \Big) = \produkt{n \in \nat}{} \vjeroj{A_n}?
    \end{equation*}
\end{zad}

    %%%%%%%%%%%%%%%%%%%%%%%%%%%%%%%%%%%%%%
    %% nezavisnost i slucajni elementi  %%
    %%%%%%%%%%%%%%%%%%%%%%%%%%%%%%%%%%%%%%

    % nezavisnost slučajnih elemenata

\chapter{Nezavisnost i slu\v cajni elementi}

I u ovom poglavlju $\vjerojatnosniProstor$ uvijek ozna\v cava proizvoljan vjerojatnosni prostor.

\begin{defn}   \label{defn:7.1}
    Neka je $\Lambda \neq \varnothing$ indeksni skup, $\indFamilija{\urePar{E_\lambda}{\famE_\lambda}}{\lambda \in \Lambda}$ familija izmjerivih prostora i $\niz{X_\lambda : \Omega \to E_\lambda}{\lambda \in \Lambda}$ familija slu\v cajnih elemenata.
    Familija $\niz{X_\lambda}{\lambda \in \Lambda}$ je \emph{nezavisna} (ili nepreciznije \emph{slu\v cajni elementi} $X_\lambda$ \emph{su nezavisni}) ako su $\sigma$-algebre $\sigAlg{X_\lambda}$, $\lambda \in \Lambda$ nezavisne.
\end{defn}

O\v cito, to je ekvivalentno svojstvu:
\begin{equation}    \label{jed:7.2}
    \begin{gathered}
        (\forall n \in \nat) (\forall \lambda_1, \ldots, \lambda_n \in \Lambda, \; \textnormal{ me\dj usobno razli\v citi}) (\forall F_1 \in \famE_1, \ldots, F_n \in \famE_n)\\ 
        \masP \Big( \presjek{i = 1}{n} \{ X_{\lambda_i} \in F_i \} \Big) = \produkt{i = 1}{n} \vjeroj{X_{\lambda_i} \in F_i}.
    \end{gathered}
\end{equation}

Nadalje, definicija \ref{defn:7.1} (odnosno \eqref{jed:7.2}) ekvivalentno je sa:
\begin{equation}    \label{jed:7.3}
    \begin{gathered}
        (\forall n \in \nat) (\forall \lambda_1, \ldots, \lambda_n \in \Lambda, \textnormal{ me\dj usobno razli\v cite})\\
        X_{\lambda_1}, \ldots, X_{\lambda_n} \textnormal{ su nezavisne}.
    \end{gathered}
\end{equation}

Tada postaje jasno da je vrlo korisno imati operativnu i preciznu karakterizaciju nezavisnosti kona\v cne familije slu\v cajnih elemenata.

\begin{tm}  \label{tm:7.4}
    Neka je $n \in \nat$, $\urePar{E_1}{\famE_1}, \ldots, \urePar{E_n}{\famE_n}$ izmjerivi prostori i
    \begin{equation*}
        (X_1 : \Omega \to E_1), \ldots, (X_n : \Omega \to E_n),    
    \end{equation*}
    slu\v cajni elementi.
    Tada su $X_1, \ldots, X_n$ nezavisni ako i samo ako je
    \begin{equation*}
        \masP_{\nvektor{X}} = \masP_{X_1} \otimes \ldots \otimes \masP_{X_n}.
    \end{equation*}
\end{tm}

\begin{proof}
    Uo\v cimo da tvrdnja ima smisla.
    
    Po propoziciji \ref{prop:4.9} slijedi da je $\nvektor{X}$ slu\v cajni element na $E_1 \times \ldots \times E_n$, to jest obje vjerojatnosti su definirane na $\famE_1 \otimes \ldots \otimes \famE_n$ koje su generirane $\pi$-sistemom cilindara oblika $A_1 \times \ldots \times A_n$, $A_1 \in \famE_1, \ldots, A_n \in \famE_n$.
    Stoga su navedene vjerojatnosti jednake ako i samo ako su jednake na cilindrima.
    Budu\' ci da je
    \begin{equation*}
        \masP_{\nvektor{X}}(A_1 \times \ldots \times A_n) = \masP \Big( \presjek{i = 1}{n} \{ X_i \in A_i \} \Big)
    \end{equation*}
    i da vrijedi
    \begin{equation*}
        \masP_{X_1} \otimes \ldots \otimes \masP_{X_n} (A_1 \times \ldots \times A_n) = \produkt{i = 1}{n} \vjeroj{X_i \in A_i}
    \end{equation*}
    stoga tvrdnja slijedi.
\end{proof}

Uzmemo li poseban slu\v caj $E_1 = \ldots = E_n = \real$ i iskoristimo li napomenu \ref{nap:5.8} \ref{nap:5.8a} i zadatak \ref{zad:5.11}, iz teorema \ref{tm:7.4} dobivamo:

\begin{kor} \label{kor:7.5}
    Slu\v cajne varijable $X_1, \ldots, X_n$ su nezavisne ako i samo ako, za svaki $x_1, \ldots, x_n \in \real$ vrijedi
    \begin{equation*}
        F_{\nvektor{X}}(x_1, \ldots, x_n) = F_{X_1} (x_1) \cdot \ldots \cdot F_{X_n} (x_n).
    \end{equation*}
\end{kor}

Primjenimo li teorem \ref{tm:7.4} na prebrojiv skup oblika $A \times \ldots \times A$, dobijemo jednostavan kriterij nezavisnosti diskretne za slu\v cajne varijable:

\begin{kor} \label{kor:7.6}
    Neka su $X_1, \ldots, X_n$, slu\v cajne varijable za koje postoji prebrojiv skup $A \subseteq \real$ takav da je $X_i \in A \; (g.s.)$, za svaki $i = 1, \ldots, n$.
    Tada su $X_1, \ldots, X_n$ nezavisne ako i samo ako za svaki izbor $\nBezZagVekt{a} \in A$, vrijedi
    \begin{equation*}
        \masP (X_1 = a_1, \ldots, X_n = a_n) = \vjeroj{X_1 = a_1} \cdot \ldots \cdot \vjeroj{X_n = a_n}.
    \end{equation*}
\end{kor}

Podsjetimo se da za $X, \; Y \in L^1$, $X \cdot Y$ ne mora biti u $L^1$. Ako su pak $X, \; Y \in L^2$, onda je $X \cdot Y \in L^1$, ali $\masE [X \cdot Y]$ ne mora biti jednako $\masE X \cdot \masE Y$.

\begin{tm}  \label{tm:7.7}
    Ako su $X, \; Y \in L^1 \vjerojatnosniProstor$ nezavisne slu\v cajne varijable, tada je $X \cdot Y \in L^1(\masP)$ i vrijedi
    \begin{equation*}
        \masE [X \cdot Y] = \masE X \cdot \masE Y.
    \end{equation*}
\end{tm}

\begin{proof}
    Uo\v cimo da je $Z:=\urePar{X}{Y}$ dvodimenzionalan slu\v cajan vektor i po teoremu \ref{tm:7.4} je $\masP_Z = \masP_X \otimes \masP_Y$.
    Po zadatku \ref{zad:4.15}
    \begin{align*}
        \int\limits_\Omega |X(\omega)| \cdot |Y(\omega)| \: d \masP
        &= \int\limits_{\real^2} |x| \cdot |y| \: d \masP_Z \urePar{x}{y} = \int\limits_\real |x| \: d \masP_X (x) \cdot \int\limits_{\real} |y| \: d \masP_Y (y)\\
        &= \masE [|X|] \cdot \masE [|Y|] < +\infty \implies X \cdot Y \in L^1.
    \end{align*}
    Sada na sli\v can na\v cin Fubinijev teorem (zadatak \ref{zad:4.16} i jednad\v zba \ref{jed:5.11-3-1}) daje
    \begin{equation*}
        \masE [X \cdot Y] = \masE X \cdot \masE Y.
    \end{equation*}
\end{proof}

\begin{zad} \label{zad:7.8}
    Ako su $X, \; Y \in L^2 \vjerojatnosniProstor$ nezavisne, tada je
    \begin{equation*}
        \Var (X + Y) = \Var X + \Var Y.
    \end{equation*}
\end{zad}

\begin{rj}[\ref{zad:7.8}]
    Doka\v zimo najprije ne\v sto generalniji rezultat.

    Neka su $X_1, \ldots, X_n \in L^2$, tada vrijedi
    \begin{equation*}
        \Var \Big( \suma{k = 1}{n} X_k \Big) = \suma{k = 1}{n} \Var X_k + 2 \cdot \suma{i < j}{} \masE \big[ (X_i - \masE X_i) (X_j - \masE X_j) \big].
    \end{equation*}

    Dokazujemo tvrdnju indukcijom po $n$.
    \begin{enumerate}
        \item[(B)]
        Dokazujemo tvrdnju za $n = 2$.
        \begin{equation*}
            \begin{aligned}
                \Var (X_1 + X_2) &= \masE \big[ (X_1 - X_2)^2 \big] - \big( \masE [ X_1 + X_2 ] \big)^2\\
                &= \masE [X_1^2] + \masE [X_2^2] + 2 \cdot \masE [X_1 \cdot X_2] - (\masE X_1)^2 - (\masE X_2)^2 - 2 \cdot \masE X_1 \cdot \masE X_2\\
                &= \big( \masE [X_1^2] - (\masE X_1)^2 \big) + \big( \masE [ X_2^2 ] - (\masE X_2)^2 \big) + 2 \cdot \big( \masE [ X_1 \cdot X_2 ] - \masE X_1 \cdot \masE X_2 \big)\\
                &= \Var X_1 + \Var X_2 + 2 \cdot \masE \big[ (X_1 - \masE X_1) (X_2 - \masE X_2) \big].
            \end{aligned}
        \end{equation*}
        \item[(K)]
        Primjetimo
        \begin{equation*}
            \Var \Big( \suma{k = 1}{n + 1} X_k \Big) = \Var \Big( \Big( \suma{k = 1}{n} X_k \Big) + X_{n + 1} \Big),
        \end{equation*}
        prema bazi indukcij slijedi
        \begin{equation*}
            \begin{aligned}
                \Var \Big( \suma{k = 1}{n + 1} X_k \Big) &= \Var \Big( \suma{k = 1}{n} X_k \Big) + \Var X_{n + 1} + 2 \cdot \masE \Big[ \Big(  \suma{k = 1}{n} X_k - \suma{k = 1}{n} \masE X_k \Big) (X_{n + 1} - \masE X_{n + 1}) \Big]\\
                &= \suma{k = 1}{n} \Var X_k + 2  \cdot \suma{i < j \leq n}{} \masE \big[ (X_i - \masE X_i) (X_j - \masE X_j) \big] + \Var X_{n + 1}\\
                &\quad \quad + 2 \cdot \masE \Big[ \Big( \suma{k = 1}{n} (X_k - \masE X_k) \Big) (X_{n + 1} - \masE X_{n + 1}) \Big]\\
                &= \suma{k = 1}{n + 1} \Var X_k + 2 \cdot \masE \Big[ \Big(  \suma{k = 1}{n} X_k - \suma{k = 1}{n} \masE X_k \Big) (X_{n + 1} - \masE X_{n + 1}) \Big]\\
                &\quad \quad + 2 \cdot \suma{k = 1}{n} \masE \big[ (X_k - \masE X_k) (X_{n + 1} - \masE X_{n + 1}) \big]\\
                &= \suma{k = 1}{n + 1} \Var X_k + 2 \cdot \suma{i < j \leq n + 1}{} \masE \big[ (X_i - \masE X_i) (X_j - \masE X_j) \big]
            \end{aligned}
        \end{equation*}
    \end{enumerate}
    Sada po teoremu \ref{tm:7.7} imamo da za nezavisne $X_1, \ldots, X_n$, te za $i \neq j$ vrijedi
    \begin{equation*}
        \masE \big[ (X_i - \masE X_i) (X_j - \masE X_j) \big] = \masE X_i \cdot \masE X_j - \masE X_i \cdot \masE X_j = 0.
    \end{equation*}
    Dakle za nezavisne slu\v cajne varijable dobivamo izraz
    \begin{equation*}
        \Var \big( \suma{k = 1}{n} X_k \Big) = \suma{k = 1}{n} \Var X_k,
    \end{equation*}
    odnosno u slu\v caju kada je $n = 2$ imamo
    \begin{equation*}
        \Var (X + Y) = \Var X + \Var Y.
    \end{equation*}
\end{rj}

Uo\v cimo da sada direktno iz korolara \ref{kor:6.11} direktno dobivamo:

\begin{kor} \label{kor:7.9}
    Neka je $\Lambda \neq  \varnothing$ indeksni skup, $\indFamilija{\urePar{E_\lambda}{\famE_\lambda}}{\lambda \in \Lambda}$ familija izmjerivih prostora i $\niz{X_\lambda : \Omega \to E_\lambda}{\lambda \in \Lambda}$ familija nezavisnih slu\v cajnih elemenata.
    Neka je $\famT$ particija od $\Lambda$ i neka je za svaki $T \in \famT$ zadan izmjeriv prostor $\urePar{H_T}{\famH_T}$ i izmjerivo preslikavanje
    \begin{equation*}
        f_T : \bigUrePar{\produkt{t \in T}{} E_t}{\dirProd{t \in T}{} \famE_t} \to \urePar{H_T}{\famH_T}.
    \end{equation*}
    Tada je
    \begin{equation*}
        \bigNiz{f_T \circ \niz{X_\lambda}{\lambda \in T}}{T \in \famT}    
    \end{equation*}
    nezavisna familija slu\v cajnih elemenata.
\end{kor}

%% koriste\' ci dokaz sli\v can onome iz \ref{tm:7.7} rje\v site.
\begin{zad} \label{zad:7.10}
    Neka su $X : \Omega \to \urePar{E}{\famE}$ i $Y : \Omega \to \urePar{H}{\famH}$ nezavisni slu\v cajni elementi.
    Neka je $f : \urePar{E \times H}{\famE \otimes \famH} \to \urePar{\real}{\borel{\real}}$ izmjeriva za koju vrijedi jedan od sljede\' cih uvjeta:
    \begin{itemize}
        \item $f \geq 0$
        \item $\masE [|f \urePar{X}{Y}|] < +\infty$.
    \end{itemize}
    Tada je
    \begin{align*}
        \masE [ f \urePar{X}{Y} ]
        &= \int\limits_{E} \int\limits_{H} f \urePar{x}{y} \: d \masP_Y (y) \: d \masP_X (x) \\
        &= \int\limits_H \int\limits_E f \urePar{x}{y} \: d \masP_X (x) \: d \masP_Y (y).
    \end{align*}
\end{zad}

\begin{defn}    \label{defn:7.10-1}
    Slu\v cajni element $X$ sa vrijednostima u $\urePar{E}{\famE}$ je \emph{degeneriran} ako postoji $a \in E$, takav da je
    \begin{equation*}
        X = a \; (g.s.).
    \end{equation*}
\end{defn}

Primjetimo u prethodnoj definiciji \emph{nije nu\v zno}
\begin{itemize}
    \item[] $\{a\} \in \famE$,
    \item[] $\{X = a\} \in \famF$.
\end{itemize}

\begin{zad} \label{zad:7.11}
    Ako je $X$ degenerirani slu\v cajni element, a $Y$ bilo koji slu\v cajni element (ne nu\v zno s vrijednostima u istom prostoru) tada su $X$ i $Y$ nezavisni.
    Ako su $X$ i $Y$ nezavisni slu\v cajni elementi takvi da je $X + Y$ degenerirana, tada su $X$ i $Y$ degenerirane.
\end{zad}

\begin{defn}    \label{defn:7.11-1}
    Grupa $\urePar{G}{\cdot}$ sa $\sigma$-algebrom $\famG$ je \emph{izmjeriva grupa} ako je grupovna operacija $\cdot : G \times G \to G$ izmjeriva u paru $\urePar{\famG \otimes \famG}{\famG}$.
\end{defn}

\begin{defn}    \label{defn:7.11-2}
    Neka su $\mu$ i $\nu$ $\sigma$-kon\v cne mjere na $\urePar{G}{\famG}$, te neka je $\mu \otimes \nu$ ($\sigma$-kona\v cna) produktna mjera na $\urePar{G \times G}{\famG \otimes \famG}$, pa mo\v zemo promatrati mjeru na $\urePar{G}{\famG}$ induciranu preslikavanjem $\cdot$ u odnosu na $\mu \otimes \nu$.
    
    Tu mjeru nazivamo \emph{konvolucijom} mjera $\mu$ i $\nu$ te ozna\v cavamo sa
    \begin{equation*}
        \mu * \nu.
    \end{equation*}
\end{defn}

Po zadatku \ref{zad:4.15} dobijemo da je za svaki $H \in \famG$
\begin{equation*}   \label{jed:7.12}
    (\mu * \nu)(H) := \int\limits_G \nu (a^{-1} \: H) \: d \mu (a).
\end{equation*}

\begin{zad} \label{zad:7.13}
    Ako je $\urePar{G}{\cdot}$ Abelova grupa, tada je konvolucija komutativana operacija.
\end{zad}

\begin{nap} \label{nap:7.14}
    \begin{enumerate}[label=(\alph*)]
        \item Op\' cenito govore\' ci mjera $\mu * \nu$ ne mora biti $\sigma$-kona\v cna.
        Ako su $\mu$ i $\nu$ kona\v cne mjere, tada je i $\mu * \nu$ kona\v cna i
        \begin{equation*}
            (\mu * \nu) (G) = \mu (G) \: \nu (G),
        \end{equation*}
        \v sto slijedi iz \eqref{jed:7.12}.
        \item Ako su $\mu, \; \nu, \; \eta$ $\sigma$-kona\v cne mjere na $\urePar{G}{\famG}$, takve da su $\mu * \nu$ i $\nu * \eta$ $\sigma$-kona\v cne, tada vrijedi
        \begin{equation*}
            (\mu * \nu) * \eta = \mu * (\nu * \eta).
        \end{equation*}
        \item   \label{nap:7.14c}
        Ako na $\urePar{G}{\famG}$ postoji $\sigma$-kona\v cna mjera $\lambda$ takva da je $\mu \ll \lambda$ i $\nu \ll \lambda$, tada za $f = \frac{d \mu}{d \lambda}$ i $g = \frac{d \nu}{d \lambda}$ vrijedi
        \begin{equation*}
            (\mu * \nu) (H) = \int\limits_G \int\limits_{a^{-1} H} f(a) \: g(b) \: d \lambda (b) \: d \lambda (a).
        \end{equation*}
        \item Ako je $\lambda$ iz \ref{nap:7.14c} jo\v s i \emph{invarijantna} u odnosu na sve lijeve translacije (to jest za svaki $g \in G$ i za svaki $C \in \famG$ je $\lambda (g^{-1} C) = \lambda (C)$), tada je
        \begin{equation*}
            (\mu * \nu) (H) = \int\limits_H \Big( \int\limits_G f(a) \: g (a^{-1} b) \: d \lambda (a) \Big) \: d \lambda (b),
        \end{equation*}
        to jest $\mu * \nu \ll \lambda$ i postoji gusto\' ca
        \begin{equation*}
            \frac{d (\mu * \nu)}{d \lambda} (b) = \int\limits_G f(a) \: g (a^{-1} b) \: d \lambda (a)
        \end{equation*}
        koju ozna\v cavamo sa \emph{$f*g$}.
    \end{enumerate}
\end{nap}

\begin{tm}  \label{tm:7.15}
    Ako su $X$ i $Y$ nezavisni slu\v cajni elementi s vjerojatnostima u izmjerivoj grupi $\urePar{G}{\cdot}$, tada je
    \begin{equation*}
        \masP_{X \cdot Y} = \masP_X * \masP_Y.
    \end{equation*}
\end{tm}

\begin{proof}
    Uo\v cimo da je $Z:= \urePar{X}{Y}$, su\v cajni element s vrijednostima u $G \times G$ i da je $\masP_Z = \masP_X \otimes \masP_Y$.
    Budu\' ci da je $\cdot$ izmjerivo, slijedi da je $X \cdot Y$ slu\v cajni element s vrijednostima u $G$ i $\masP_{X \cdot Y}$ je vjerojatnost inducirana s $\cdot$ u odnosu na $\masP_Z$.
    To je po definiciji upravo
    $\masP_X * \masP_Y$.
\end{proof}

\begin{zad} \label{zad:7.16}
    Primjenite ove rezultate na Abelovu grupu $\urePar{\real^d}{+}$ i posebno obratite pa\v znju na slu\v caj $d = 1$.
    Za slu\v caj $d = 1$izvedite pravila za konvluciju d.f.
\end{zad}

Promotrimo sada nezavisne familije slu\v cajnih elemenata.
Po\v cnimo od kona\v cnog slu\v caja.
Uo\v cimo prvo da ako su $\nBezZagVekt{X}, \; n \in \nat$, \emph{nezavisni} slu\v cajni elementi s vrijednostima u (respektivno) $\urePar{E_1}{\famE_1}, \ldots, \urePar{E_n}{\famE_n}$, tada je distribucija slu\v cajnog elementa $\nvektor{X}$ (s vrijednostima u $\urePar{E_1 \times \ldots \times E_n}{\famE_1 \otimes \ldots \otimes \famE_n}$) u potpunosti odre\dj ena s $\masP_{X_1}, \ldots, \masP_{X_n}$; vidi teorem \ref{tm:7.4}

U statistici i primjenama \v cesto je obrnuta situacija.
Pretpostavimo da su nam zadane distribucije $\nBezZagVekt{\masP}$ na (respektivno) $\urePar{E_1}{\famE_1}, \ldots, \urePar{E_n}{\famE_n}$ i \v zelimo formirati slu\v cajni element $X = \nvektor{X}$ na $E_1 \times \ldots \times E_n$, tako da su $\nBezZagVekt{X}$ \emph{nezavisni} i $\masP_{X_1} = \masP_1, \ldots, \masP_{X_n} = \masP_n$.
Iz dokazanog lako slijedi da se to jednostavno posti\v ze na vjerojatnosnom prostoru \emph{kanonskog oblika}
\begin{equation} \label{jed:7.17}
    \begin{aligned}
        \Omega &:= E_1 \times \ldots \times E_n\\
        \famF &:= \famE_1 \otimes \ldots \otimes \famE_n\\
        \masP &:= \masP_1 \otimes \ldots \otimes \masP_n\\
        X &:= \id : \Omega \to \Omega\\
        X_i &= \pi_i : E_1 \times \ldots \times E_n \to E_i, \quad i = 1, \ldots, n.
    \end{aligned}
\end{equation}
U klasi\v cnoj situaciji, kada su $X_i$ slu\v cajne varijable, ovaj zahtjev mo\v zemo ispuniti i direktnije.

\begin{prop}    \label{prop:7.18}
    Neka su $\nBezZagVekt{F}$ proizvoljne p.d.F. i $\vjerojatnosniProstor$ vjerojatnosni prostor na kojem postoji slu\v cajni vektor $U = \nvektor{U}$, takav da su $\nBezZagVekt{U} $ nezavisne i jednako distribuirane slu\v cajne varijable s uniformnom razdiobom na $\obInt{0}{1}$.
    Tada na $\vjerojatnosniProstor$ postoji slu\v cajni vektor $X = \nvektor{X}$ takav da su $\nBezZagVekt{X}$ nezavisne slu\v cajne varijable i
    \begin{equation*}
        F_{X_1} = F_1, \ldots F_{X_n} = F_n.
    \end{equation*}
\end{prop}

\begin{proof}
    Neka je $h_i : \obInt{0}{1} \to \real, \; i = 1, \ldots n$, zadan sa $h_i (x) := \sup \skup{r \in \real}{F_i (r) < x}$ i neka je
    \begin{equation*}
        X := (h_1 \circ U_1, \ldots, h_n \circ U_n).
    \end{equation*} 
    Tada je
    \begin{equation*}
        \skup{w \in \obInt{0}{1}}{h_i(w) \leq x} = \skup{w \in \obInt{0}{1}}{w \leq F_i (x)}
    \end{equation*}
    pa je $F_{h_i \circ U_i} = F_i$.
    Nezavisnost slijedi iz korolara \ref{kor:7.9}
\end{proof}

\begin{nap} \label{nap:7.19}
    Lako je posti\' ci uvijete iz propozicije \ref{prop:7.18}.
    Neka je
    \begin{equation*}
        \Omega := \segment{0}{1}^n, \; \famF := \borel{\segment{0}{1}^n}, \; \masP := \restr{\lambda^n}{\famF},
    \end{equation*}
    te neka je
    \begin{equation*}
        U_i := \pi_i : \Omega \to \segment{0}{1}, \; i = 1, \ldots, n.
    \end{equation*}
\end{nap}

\begin{zad} \label{zad:7.20}
    Uzmimo $n = 1$ u napomeni \ref{nap:7.19}. Napi\v simo svaki $U(x) = x \in \obInt{0}{1}$ u binarnom prikazi i ozna\v cimo $n$-tu znamenku s $h_n$, $n \in \nat$.
    Doka\v zi da su $\niz{h_n}{n \in \nat}$ nezavisne i jednako distribuirane slu\v cajne varijable i vrijedi
    \begin{equation*}
        h_n \sim
        \begin{pmatrix}
            0& 1\\
            \frac{1}{2}& \frac{1}{2}    
        \end{pmatrix}.
    \end{equation*}
\end{zad}

    %%%%%%%%%%%%%%%%%%%%%%%%%%%%%%%%%%%%%%%%%%%%%%%
    %%  beskonacne familije slucajnih elemenata  %%
    %%%%%%%%%%%%%%%%%%%%%%%%%%%%%%%%%%%%%%%%%%%%%%%

    % beskonačne familije slučajnih elemenata

\chapter{Beskona\v cne familije slu\v cajnih elemenata}

Podsjetimo se na oznake u napomeni \ref{nap:4.10}.

\begin{defn}    \label{defn:8.1}
    Neka je $\vjerojatnosniProstor$ vjerojatnosni prostor, $\urePar{E}{\famE}$ izmjeriv prostor, $T \neq \varnothing$ skup indeksa.
    \emph{$E$-zna\v cni stohasti\v cki} (ili \emph{slu\v cajni}) \emph{proces} je svako preslikavanje
    \begin{equation*}
        X : \Omega \to E^T
    \end{equation*}
    koje je $\urePar{\famF}{\famE^T}$-izmjerivo.
    Ako je $T$ beskona\v can i prebrojiv, obi\v cno ka\v zemo da je $X$ \emph{$E$-zna\v cni slu\v cajni niz}, a ako je $T$ kona\v can, ka\v zemo da je $X$ \emph{$E$-zna\v cni slu\v cajni vektor}.
    Ako je $E = \extReal$, $\famE = \borel{\extReal}$, umjesto rije\v ci "$E$-zna\v cni" koristimo rije\v c "pro\v sireni", a ako jo\v s postoji $A \in \famF, \; \vjeroj{A} = 0$, takav da je
    \begin{equation*}
        \unija{t \in T}{} \{ |\pi_t \circ X| = \infty \} \subseteq A,
    \end{equation*}
    tada se rije\v c "$E$-zna\v cni" u gornjoj definicij izbacuje.
\end{defn}

\begin{nap} \label{nap:8.2}
    \begin{enumerate}[label=(\alph*)]
        \item Umjesto $\pi_t \circ X$ naj\v ce\v s\' ce koristimo oznaku $X_t$ i stohasti\v cki proces promatramo kao familiju $E$-zna\v cnih funkcija $\niz{X_t}{t \in T}$.
        Po propoziciji \ref{prop:4.9}, $X$ je $E$-zna\v cni stohasti\v cki proces ako i samo ako su sve $X_t$ $E$-zna\v cni slu\v cajni elementi.
        \item Mo\v ze se dogoditi da $\urePar{T}{\famT}$ ima strukturu izmjerivog prostora.
        Tada je mogu\' ca jo\v s jedna interpretacija stohasti\v ckog procesa $\niz{X_t}{t \in T}$.
        Ponovo se (neprecizno!) koristi slovo $X$ za preslikavanje
        \begin{equation*}
            X : T \times \Omega \to E,
        \end{equation*}
        definirano sa
        \begin{equation*}
            X \urePar{t}{\omega} := X_t (\omega).
        \end{equation*}
        Re\' ci \' cemo da je stohasti\v cki proces $\niz{X_t}{t \in T}$ \emph{izmjeriv} ako je ovako definirani $X$ $\urePar{\famT \otimes \famF}{\famE}$-izmjeriv.
        Uo\v cimo (po \eqref{jed:4.7}) da za svako $\urePar{\famT \otimes \famF}{\famE}$-preslikavanje $X$ vrijedi da su sve $X_t$ slu\v cajni elementi u $E$ to jest $\niz{X_t}{t \in T}$ je $E$-zna\v cni stohasti\v cki proces.
        Obrat op\' cenito ne vrijedi.
        \item Ako je $T \subseteq \overline{\Z}$ ili $T \subseteq \extReal$, obi\v cno $t \in T$ interpretiramo kao vrijeme.
        Za svaki $\omega \in \Omega$ promatramo preslikavanje
        \begin{equation*}
            T \ni t \mapsto X_t (\omega) \in E,
        \end{equation*}
        koje nazivamo \emph{trajektorijom} ("sample path") procesa $X$.
        Ako $E$ ima i topolo\v sku strukturu obi\v cno se promatraju i neka "analiti\v cka" svojstva trajektorije (neprekidnost, neprekidnost slijeva, neprekidnost zdesna, limesi i sli\v cno).
        Naj\v ces\v s\' ci su slu\v cajevi
        \begin{itemize}
            \item $T = \nat_0$ - \emph{diskretni vremenski parametar} (trajektorije su nizovi u $E$);
            \item $T = \desInt{0}{+\infty}$ - \emph{neprekidni vremenski parametar} (trajektorije su funkcije realne varijable).
        \end{itemize}
        U oba slu\v caja $0$ ozna\v cava trenutak po\v cetka  (promatranja) slu\v cajnog procesa.
        \item Razni su nivoi jednakosti kod slu\v cajnih procesa.
        Ako su $\niz{X_t}{t \in T}$, $\niz{Y_t}{t \in T}$ $E$-zna\v cni stohasti\v cki procesi na $\vjerojatnosniProstor$, imamo dvije vrste $(g.s.)$ jednakosti.
        Re\' ci \' cemo da su $X$ i $Y$ \emph{nerazlu\v civi} ako postoje $A \in \famF, \; \vjeroj{A} = 0$, takvi da je
        \begin{equation*}
            \skup{\omega \in \Omega}{(\exists t \in T) \; X_t (\omega) \neq Y_t (\omega)} \subseteq A.
        \end{equation*}
        Re\' ci \' cemo da su $X$ i $Y$ \emph{modifikacije} (jedan drugoga) ako je za svaki $t \in T$ $X_t = Y_t \; (g.s.)$.
        Nerazlu\v civost uvijek povla\' ci modificiranost, ali obrat op\' cenito ne vrijedi.
        Ako je $T$ prebrojiv, pojmovi su ekvivalentni.
    \end{enumerate}
\end{nap}

U skladu s poglavljem \ref{dist_sl_elem}, $E$-zna\v cni stohasti\v cki procesi $X$ i $Y$ (ne nu\v zno definirani na istom vjerojatnosnom prostoru) imaju istu distribuciju ($X \distJed Y$) ako je $\masP_X = \masP_Y$.
Op\' cenito vrijedi
\begin{equation*}
    \textnormal{nerazlu\v civost}
    \begin{smallmatrix}
        \implies\\
        \notimpliedby
    \end{smallmatrix}
    \textnormal{modifikacija}
    \begin{smallmatrix}
        \implies\\
        \notimpliedby
    \end{smallmatrix}
    \textnormal{jednaka distribuiranost}
\end{equation*}
Za\v sto vrijedi ova zadnja implikacija?
Kada su $\masP_X$ i $\masP_Y$ jednaki?
Uo\v cimo da su $\masP_X$ i $\masP_Y$ definirane na $\urePar{E^T}{\famE^T}$ i da je $\prsten{E^T}$ generiraju\' ci poluprsten za $\famE^T$.
Slijedi da je
\begin{equation}    \label{jed:8.3}
    \masP_X = \masP_Y
    \iff
    \restr{\masP_X}{\prsten{E^T}} = \restr{\masP_Y}{\prsten{E^T}}.
\end{equation}

Relacija \eqref{jed:8.3} sugerira da se mo\v zemo ograni\v citi na kona\v cne koordinate kako bismo postigli jedakost po distribuciji.
Preciznije, neka je $J \subseteq T$, $J$ kona\v can i neprazan.

\begin{defn}
    Ozna\v cimo sa $\famE^{T, \: \pi_J} = \indSigAlg{\pi_t}{t \in J}$ $\sigma$-algebru (na $E^T$) i sa $\masP_{X, \: J} = \restr{\masP_X}{\famE^{T, \: \pi_J}}$ restrikciju od $\masP_X$ na $\famE^{T, \: \pi_J}$.
\end{defn}

Uo\v cimo da je jednostavno direktno poistovjetiti mjeru $\masP_{X, \: J}$ sa mjerom na kona\v cnom produktu $\urePar{E^J}{\famE^J}$.
Zato ka\v zemo da je
\begin{equation}    \label{jed:8.4}
    \skup{\masP_{X, \: J}}{J \subseteq T, \; J \neq \varnothing, \; \card{J} < \infty},
\end{equation}
familija \emph{kona\v cno dimenzionalnih distribucija stohasti\v ckog procesa $X$}.
Iz \eqref{jed:8.3} direktno slijedi:

\begin{tm}  \label{tm:8.5}
    $E$-zna\v cni stohasti\v cki procesi $X$ i $Y$ su jednaki po distribuciji ako i samo ako su im sve kona\v cno-dimenzionalne distribucije jednake.
\end{tm}

Promatramo li $\masP_{X, \: J}$ kao mjeru na $\urePar{E^J}{\famE^J}$ mo\v zemo iskoristiti teorem \ref{tm:7.4} i teorem \ref{tm:8.5} kako bismo opisali jednakost po distribuciji u nezavisnom slu\v caju (za jedno\v clane $J = \{j\}$ uvodimo oznaku $\masP_{X, \: J} = \masP_{X, \: j}$ i uo\v cimo da se $\masP_{X, \: j}$ mo\v ze poistovjetiti sa $\masP_{X_j}$).

\begin{kor} \label{kor:8.6}
    Neka je $\niz{X_t}{t \in T}$ $E$-zna\v cni stohasti\v cki proces i nezavisna familija slu\v cajnih elemenata.
    Neka je $\niz{Y_t}{t \in T}$ $E$-zna\v cni stohasti\v cki proces i nezavisna familija slu\v cajnih elemenata.
    Tada je $X \distJed Y$ ako i samo ako je $X_t \distJed Y_t$, za svaki $t \in T$.
\end{kor}

\v Sto re\' ci o egzistenciji procesa uz zadane distribucije?
Ako su za svaki $\varnothing \neq J \subseteq T$, $J$ kona\v can, zadane vjerojatnosti $\masP_J$ na $\famE^{T, \: \pi_J}$, potrebna nam je barem konzistentnost.

\begin{defn}    \label{defn:8.6-1}
    Neka je $\niz{\masP_J}{J \subseteq T, \; J \neq \varnothing, \; \card{J} < \infty}$ familija vjerojatnosti, ka\v zemo da je ona \emph{konzistentna} ako za svake $I$, $J$ vrijedi $\varnothing \neq J \subseteq  I \subseteq T$, te su $J$, $I$ kona\v cni,
    \begin{equation}    \label{jed:8.7}
        \restr{\masP_I}{\famE^{T, \: \pi_J}} = \masP_J.
    \end{equation}
\end{defn}
 
Zaista, ako vrijedi \eqref{jed:8.7} nije te\v sko vidjeti da je sa
\begin{equation*}
    \nu (A) := \masP_J (A), \quad A \in \prsten{E^T} \cap \famE^{T, \: \pi_j},
\end{equation*}
zadane, (to jest dobro definirana) kona\v cno aditivna funkcija
\begin{equation*}
    \nu : \prsten{E^T} \to \segment{0}{1}.
\end{equation*}
Uspijemo li pokazati da je $\nu$ $\sigma$-aditivna, onda Caratheodoryjeva konstrukcija daje mjeru na $\famE^T$.
Pokazuje se da za $\sigma$-aditivnost trebamo neku vrstu kompaktnosti i regularnosti.
Na primjer:

\begin{lm}  \label{lm:8.6}
    Ako je $E$ potpun, separabilan metri\v cki prostor i $\famE = \borel{E}$, tada je $\nu$ $\sigma$-aditivna.
\end{lm}

\begin{proof}{(Skica)}
    Zbog kona\v cne aditivnosti dovoljno je dokazati da za svaki niz $\niz{A_n}{n \in \nat} \subseteq \prsten{E^T}$, takav da je
    $A_1 \supseteq A_2 \supseteq \ldots$ i $\presjek{n}{} A_n = \varnothing$ vrijedi $\lim\limits_{n \to +\infty} \nu (A_n) = 0$.
    U suprotnom postojao bi $\varepsilon \geq 0$ i $\nu (A_n) \geq \varepsilon, \; \forall n \in \nat$.
    Tada se $A_n$ mogu dovoljno dobro aproksimirati odozdo kompaktima $C_n$ (po $\masP_J$, a time i po $\nu$) i dijagonalni postupak (kompaktnost je va\v zna!) po koordinatama daje to\v cku "u presjeku" $C_n$-ova, a time u $\presjek{n}{} A_n$.
\end{proof}

Iz leme \ref{lm:8.6} i prethodne diskusije slijedi:

\begin{tm}[Kolmogorov]    \label{tm:8.9}
    Za svaku konzistentnu familiju vjerojatnosti
    \begin{equation*}
        \bigNiz{\masP_J}{J \subseteq T, \; J \neq \varnothing, \; \card{J} < \infty}
    \end{equation*}
    na $\urePar{E^T}{\famE^T}$ pri \v cemu je $E$ potpun separabilan metri\v cki prostor i $\famE = \borel{E}$, postoji i jedinstvena je vjerojatnost $\masP$ na $\urePar{E^T}{\famE^T}$ takva da je za svaki $\varnothing \neq J \subseteq T$, $J$ kona\v can, $\restr{\masP}{\famE^{T, \: \pi_J}} = \masP_J$.
\end{tm}

\begin{nap} \label{nap:8.10}
    \begin{enumerate}[label=(\alph*)]
        \item Teorem \ref{tm:8.9} garantira da za svaku zadanu distribuciju stohasti\v ckog procesa na $E$ postoji realizacija tog procesa na kanonskom prostru $\urePar{E^T}{\famE^T}$ uz $X_t = \pi_t$.
        \item U nezavisnom slu\v caju teorem \ref{tm:8.9} garantira da za svaku familiju jednodimenzionalnih distribucija na $\urePar{E}{\borel{E}}$ postojih stohasti\v cki proces $(X_t)$ s takvim distribucijama i familija $(X_t)$ je nezavisna.
        Naime, produktne mjere \' ce sigurno biti konzistentne.
        Ovaj zadnji rezultat se mo\v ze u slu\v caju prebrojivog $T$ dobiti i direktnom vjerojatnosnom konstrukcijom.
    \end{enumerate}
\end{nap}

\begin{tm}  \label{tm:8.11}
    Neka je $\niz{\masP_n}{n \in \nat}$ niz vjerojatnosnih mjera na $\urePar{E}{\borel{E}}$, pri \v cemu je $E$ potpun, separabilan metri\v cki prostor.
    Tada na vjerojatnosnom prostoru $(\segment{0}{1}, \; \borel{\segment{0}{1}}, \; \restr{\lambda}{\borel{\segment{0}{1}}})$ postoji $E$-zna\v cni stohasti\v cki proces $\niz{X_n}{n \in \nat}$, takav da je familija $\indFamilija{X_n}{n \in \nat}$ nezavisna i vrijedi $\masP_{X_n} = \masP_n$, za svaki $n \in \nat$.
\end{tm}

\begin{proof}
    Podsjetimo se da za $E$ postoji Borelov skup $A \subseteq \segment{0}{1}$ i bijekcija $f : E \to A$ takav da su $f$ i $f^{-1}$ izmjerive u paru $\sigma$-algebri $\famE =  \borel{E}$ i $\borel{A}$.
    Dakle, bez smanjenja op\' cenitosti uzmemo $E \subseteq \segment{0}{1}$. Neka je $h: \segment{0}{1} \to \segment{0}{1}$, $h(x) = x$, to je uniformno distribuirana slu\v cajna varijabla i po zadatku \ref{zad:7.20} njene binarne znamenke tvore niz $\niz{h_n}{n \in \nat}$ nezavisnih jednako distribuiranih Bernoullijevih slu\v cajnih varijabli.
    Preuredimo niz $(h_n)$ u beskona\v cnu matricu $\niz{h_{ij}}{i, \; j \in \nat}$ (kao na primjer u bijekciji izme\dj u $\nat$ i $\nat \times \nat$).
    Za svaki $i \in \nat$ je $\niz{h_{ij}}{n \in \nat}$ niz nezavisnih Bernoulijevih slu\v cajnih varijabli.
    Definiramo
    \begin{equation*}
        g_i := \suma{j \in \nat}{} 2^{-j} h_{ij}.
    \end{equation*}
    Po korolaru \ref{kor:8.6} slijedi da je $g_i \distJed h$, to jest svaka $g_i$ je uniformno distribuirana na $\segment{0}{1}$.
    Po korolaru \ref{kor:7.9} $\indFamilija{g_i}{i \in \nat}$ \v cine nezavisnu familiju slu\v cajnih varijabli.
    Za svaki $i \in \nat$, $\masP$ odre\dj uje p.d.F. $F_i$, pa kao u propoziciji \ref{prop:7.18} stavimo
    \begin{equation*}
        f_i (x) := \sup \skup{r \in \real}{F_i (r) < x}
    \end{equation*}
    i definiramo $X_i = f_i \circ g_i$.
    Opet po korolaru \ref{kor:7.9} $\niz{X_i}{i \in \nat}$ su nezavisne, a po propozicij \ref{prop:7.18} $F_{X_i} = F_i$, to jest $\masP_{X_i} = \masP_i$.
\end{proof}

    %%%%%%%%%%%%%%%%%%%%
    %%  zakoni 0 - 1  %%
    %%%%%%%%%%%%%%%%%%%%

    % zakoni 0 - 1

\chapter{Zakoni 0-1}  \label{zakoni_01}

\begin{pr}[majmun za pisa\' cim strojem]  \label{pr:9.1}
    Zamislimo majmuna koji nasumica tipka po pisa\' cem stroju sa $M$ tipki.
    Mo\v zemo zamisliti da je to niz nezavisnih pokusa u kojima majmun (nasumice) poga\dj a pojedinu tipku sa vjerojatno\v s\' cu $\frac{1}{M}$.
    Zamislimo li da majmu tipka unedogled, koja je vjerojatnost da \' ce majmun u nekom "intervalu" natipkati "Dundo Maroje"?
    Uo\v cimo li da je "D.M." specifi\v can niz simbola duljine $T$, pa mo\v zemo niz pokusa promatrati kao niz simbola duljine $T$.
    Po korolaru \ref{kor:7.9} to su opet nezavisni slu\v cajni elementi i vjerojatnsot da u jednoj sekvenci majmun pogodi "D.M." je $p = \frac{1}{M^T}$.
    Iako je $p$ izrazito mali, va\v zno je da je $p > 0$.
    Ozna\v cimo li sekvencu sa $X_n$ dobivamo
    \begin{equation*}
        \begin{aligned}
            \vjeroj{\textnormal{bar jedna sekvenca "D.M."}}
            &= \vjeroj{\unija{n \in \nat}{} \{X_n = \textnormal{D.M.}\}}\\
            &= \vjeroj{\unija{n \in \nat}{} \{X_1 \neq \textnormal{D.M.}, \ldots, X_{n-1} \neq \textnormal{D.M.}, \: X_n = \textnormal{D.M.}\}}\\
            &= \textnormal{(disjunktna)}\\
            &= \suma{n \in \nat}{} (1 - p)^{n - 1} \cdot p = p + (1 - p) \: p + (1 - p)^2 \: p + \ldots\\
            &= \Big( \suma{n = 1}{\infty} (1 - p)^{n - 1} \Big) \cdot p = \frac{1}{1 - (1 - p)} \cdot p\\
            &= 1.
        \end{aligned}
    \end{equation*}
\end{pr}

\begin{nap} \label{nap:9.1-1}
    Prisjetimo se, neka je $\niz{A_n}{n \in \nat}$ niz skupova, tada su limes superior i limes inferior definirani sa:
    \begin{enumerate}[label=(\roman*)]
        \item $\liminf\limits_{n \to \infty} A_n := \unija{n = 1}{\infty} \Bigg( \presjek{k = n}{\infty} A_k \Bigg)$
        \item $\limsup\limits_{n \to \infty} A_n := \presjek{n = 1}{\infty} \Bigg( \unija{k = n}{\infty} A_k \Bigg)$
    \end{enumerate}
\end{nap}

Korisna je sljede\' ca karakterizacija limesa inferior i limesa superior

\begin{prop}    \label{prop:9.1-2}
    Neka je $\niz{A_n}{n \in \nat}$ niz podskupova od $\Omega$.
    Neka je za svaki $\omega \in \Omega$:
    \begin{equation*}
        K_\omega := \skup{n \in \nat}{\omega \in A_n},
    \end{equation*}
    tada vrijedi:
    \begin{enumerate}[label=(\arabic*)]
        \item $\omega \in \limsup\limits_{n \to \infty} A_n \iff \card{K_\omega} = \infty$,
        \item $\omega \in \liminf\limits_{n \to \infty} A_n \iff \card{\nat \setminus K_\omega} < +\infty$.
    \end{enumerate}
\end{prop}

Odnosno elementarni doga\dj aj se nalazi u limes superioru niza ako se nalazi u beskona\v cno mnogo doga\dj aja tog niza, dok se elementarni doga\dj aj nalazi u limes inferioru niza ako se nalazi u svima, osim eventualno kona\v cno mnogo elemenata tog niza.

\begin{nap} \label{nap:9.1-3}
    Is gore navedenog razloga koristimo kraticu $\io$ (eng. infinitely often) za limes superior niza.    
\end{nap}

\begin{lm}[Borel - Cantelli]  \label{lm:9.2}
    Neka je $\vjerojatnosniProstor$ vjerojatnosni prostor i neka je $\niz{A_n}{n \in \nat} \subseteq \famF$ niz doga\dj aja za koje vrijedi
    \begin{equation*}
        \suma{n = 1}{\infty} \vjeroj{A_n} < +\infty.
    \end{equation*}
    Tada je
    \begin{equation*}
        \vjeroj{A_n \; i.o.} := \vjeroj{\limsup\limits_{n \to +\infty} A_n} = 0.
    \end{equation*}
\end{lm}

\begin{proof}
    Pretpostavimo da vrijedi
    $\suma{n = 1}{\infty} \vjeroj{A_n} < +\infty$.
    \begin{equation*}
        \begin{aligned}
            \suma{n = 1}{\infty} \masE [ \karaktFja_{A_n} ] &= \suma{n = 1}{\infty} \vjeroj{A_n}\\
            &= (\textnormal{prema LTMK-u})\\
            &= \masE \Big[ \suma{n = 1}{\infty} \karaktFja_{A_n} \Big] < +\infty.
        \end{aligned}
    \end{equation*}
    Tada $(\exists C \in \famF)$ takav da je $\vjeroj{C} = 1$ i da je $\suma{n = 1}{\infty} \karaktFja_{A_n}(\omega) < +\infty$, $\forall \omega \in C$
    \begin{equation*}
        \implies \vjeroj{A_n \; i.o.} \leq \vjeroj{C^c} = 0. 
    \end{equation*}
\end{proof}

\begin{tm}[Borelov zakon 0 - 1]   \label{tm:9.3}
    Neka je $\vjerojatnosniProstor$ vjerojatnosni prostor i neka je $\niz{A_n}{n \in \nat} \subseteq \famF$ niz nezavisnih doga\dj aja.
    Tada vrijedi:
    \begin{equation*}
        \begin{aligned}
            \vjeroj{A_n \; i.o.} = 0 &\iff \suma{n = 1}{\infty} \vjeroj{A_n} < +\infty\\
            \vjeroj{A_n \; i.o.} = 1 &\iff \suma{n = 1}{\infty} \vjeroj{A_n} = +\infty.
        \end{aligned}
    \end{equation*}
\end{tm}

\begin{proof}
    Prema lemi \ref{lm:9.2} dovoljno je pokazati tvrdnju
    \begin{equation*}
        \suma{n = 1}{\infty} \vjeroj{A_n} = + \infty \implies \vjeroj{A_n \; \io} = 1.
    \end{equation*}
    Diskusijom toka funkcije $h(x) = e^{-x} + x - 1$ mo\v ze se pokazati nejednakost
    \begin{equation*}
        1 - x \leq e^{-x}.
    \end{equation*}
    Sada imamo:
    \begin{equation*}
        \begin{aligned}
            \masP \Big(\unija{k \geq n}{} A_k \Big) &= 1 - \masP \Big( \presjek{k \geq n}{} A_k^c \Big) = (\textnormal{prema zadatku \ref{zad:6.14}})\\
            &= 1 - \produkt{k \geq n}{} (1 - \vjeroj{A_k}) \geq 1 - \produkt{k \geq n}{} \Big( e^{- \masP (A_k)} \Big)\\
            &= 1 - e^{- \suma{k \geq n}{} \vjeroj{A_k}}\\
            &= 1
        \end{aligned}
    \end{equation*}
    To povla\v ci:
    \begin{equation*}
        \begin{aligned}
            \masP ( A_n \; \io ) &= \masP \Big( \presjek{n \in \nat}{} \unija{k \geq n}{} A_k \Big) = \lim\limits_{n \to \infty} \masP \Big( \unija{k \geq n}{} A_k \Big)\\
            &= 1.
        \end{aligned}
    \end{equation*}
\end{proof}

\begin{nap} \label{nap:9.4}
    U primjeru \ref{pr:9.1}. $A_n = \{ X_n = \textnormal{D.M.} \}$ su nezavisni i $\vjeroj{A_n} = p = \frac{1}{M^T} > 0 \implies \sum \vjeroj{A_n} = +\infty \implies \vjeroj{A_n \; i.o.} = 1$ (!).
    Pa ipak prosje\v cno vrijeme (ili prvi $n$) da se dogodi $A_n$ je $\frac{1}{p} = M^T = 10^{10^7}$ (ako treba 1 sekunda po znaku, prije \' ce se sunce ohladiti).
\end{nap}

\begin{defn}    \label{defn:9.5}
    Ka\v zemo da je $\sigma$-algebra na vjerojatnosnom prostoru $\vjerojatnosniProstor$ \emph{trivijalna} ako vrijedi
    \begin{equation*}
        A \in \famG \implies \vjeroj{A} \in \{0, \; 1\}
    \end{equation*}
\end{defn}

Uo\v cimo da je $\famG$ trivijalna ako i samo ako je nezavisna sa samom sobom.
Dodavanje ili oduzimanje proizvoljnog broja trivijalnih $\sigma$-algebri ne\' ce mjenjati status nezavisnosti te familije.

\begin{lm}  \label{lm:9.6}
    Neka je $\vjerojatnosniProstor$ vjerojatnosni prostor, $\famG \subseteq \famF$ $\sigma$-algebra, $E$ potpun, separabilan metri\v cki prostor i $X : \Omega \to E$ $\famG$-izmjeriv slu\v cajni element.
    Ako je $\famG$ trivijalna, tada je $X$ degenerirana.
\end{lm}

\begin{proof}
    Za svaki $n \in \nat$, $E$ mo\v zemo prikazati kao
    \begin{equation*}
        E = \unija{j \in \nat}{} B_j^n,
    \end{equation*}
    pri \v cemu je ovo disjunktna unija Borelovih skupova i za svaki $j \in \nat$ vrijedi
    \begin{equation*}
        \diam{B_j^n} < \frac{1}{n}.
    \end{equation*}
    Tada su $\{ X \in B_j^n \}$ u $\sigma$-algebri $\famG$, pa su im vjerojatnosti $0$ ili $1$.
    \begin{equation*}
        \begin{aligned}
            1 &= \vjeroj{\Omega} = \suma{j \in \nat}{} \vjeroj{X \in B_{j(n)}^n} \implies\\
            \masP \Big( X \in \presjek{n \in \nat}{} B_{j(n)}^n \Big) &= 1 \implies\\
            \presjek{n \in \nat}{} B_{j(n)}^n &\neq \varnothing
        \end{aligned}
    \end{equation*}
    S druge strane
    \begin{equation*}
        \bigDiam{\presjek{n \in \nat}{}B_{j(n)}^n} = 0 \implies (\exists x \in E) \; \presjek{n \in \nat}{} B_{j(n)}^n = \{ x \}.
    \end{equation*}
\end{proof}

\begin{defn}    \label{defn:9.7}
    Neka je $\indFamilija{\famF_n}{n \in \nat}$ niz $\sigma$-algebri na nepraznom skupu $\Omega$.
    Definiramo $\sigma$-algebru $\famT$ sa
    \begin{equation*}
        \famT := \presjek{n \in \nat}{} \indSigAlg{\famF_k}{k > n},
    \end{equation*}
    nazivamo \emph{repnom $\sigma$-algebrom} (u odnosu na $\indFamilija{\famF_n}{n \in \nat}$).

    Funkciju
    \begin{equation*}
        X : \Omega \to \urePar{E}{\famE}
    \end{equation*}
    koja je $\urePar{\famT}{\famE}$-izmjeriva nazivamo \emph{repnom funkcijom}.

    Ako je $\vjerojatnosniProstor$ vjerojatnosni prostor i $\famF_n \subseteq \famF$ za svaki $n \in \nat$, elemente $\sigma$-algebre $\famT$ nazivamo \emph{repnim doga\dj ajima}.
\end{defn}

\begin{pr}  \label{pr:9.8}
    Neka je $\niz{X_n}{n \in \nat}$ niz slu\v cajnih varijabli na vjerojatnosnom prostoru $\vjerojatnosniProstor$ i neka je $\famF_n = \sigAlg{X_n}$ za svaki $n \in \nat$.
    Tipi\v cni repni doga\dj aju su, za $a \in \extReal$,
    \begin{itemize}
        \item[] $\displaystyle \bigSkup{\omega \in \Omega}{(X_n(\omega))_{n \in \nat} \; \textnormal{ konvergira}}$
        \item[] $\displaystyle \bigSkup{\omega \in \Omega}{\lim\limits_{n \to \infty} X_n(\omega) = a}$
        \item[] $\displaystyle \bigSkup{\omega \in \Omega}{\suma{n = 1}{\infty} X_n(\omega) \; \textnormal{ konvergira}}$
    \end{itemize}
    dok
    \begin{itemize}
        \item[] $\displaystyle \bigSkup{\omega \in \Omega}{\suma{n = 1}{\infty} X_n (\omega) = a}$
    \end{itemize}
    nije repni doga\dj aj.
\end{pr}

\begin{tm}[Kolmogorovljev zakon 0-1]  \label{tm:9.9}
    Ako su $\niz{\famF_n}{n \in \nat}$, $\famF_n \subseteq \famF$, nezavisne $\sigma$-algebre na vjerojatnosnom prostrou $\vjerojatnosniProstor$, tada je repna $\sigma$-algebra $\famT$ trivijalna.
\end{tm}

\begin{proof}
    Za $n \in \nat$ definiramo $\famT_n := \indSigAlg{\famF_k}{k > n}$.
    Po korolaru \ref{kor:6.11} $\sigma$-algebre $\famF_1, \ldots, \famF_n, \; \famT_n$ su nezavisne.
    Budu\' ci da je $\famT \subseteq \famT_n$, po napomeni \ref{nap:6.9} \ref{nap:6.9d} slijedi da su $\famF_1, \ldots, \famF_n, \; \famT$ nezavisne za svaki $n \in \nat$.
    To, prema napomeni \ref{nap:6.9} \ref{nap:6.9c}, zna\v ci da su $\famT, \; \famF_1, \; \famF_2, \ldots$ nezavisne $\sigma$-algebre.
    Po korolaru \ref{kor:6.11} $\famT$ i $\indSigAlg{\famF_n}{n \in \nat}$ su nezavisne, a po napomeni \ref{nap:6.9} \ref{nap:6.9d} i $\famT \subseteq \indSigAlg{\famF_n}{n \in \nat}$ slijedi da su $\famT$ i $\famT$ nezavisne, to jest $\famT$ je trivijalna.
\end{proof}

Usporedimo li primjer \ref{pr:9.8} s teoremom \ref{tm:9.9}, jasno je da \' ce teorem \ref{tm:9.9} igrati va\v znu ulogu u analizi grani\v cnog pona\v sanja nezavisnih slu\v cajnih nizova.
Time \' cemo se detaljno bavit kasnije, a sad obratimo pa\v znju na jo\v s jedan klasi\v cni 0-1 zakon.

\begin{defn}    \label{defn:9.9-1}
    Neka je $p: \nat \to \nat$ bijekcija takva da postoji $m \in \nat$ sa svojstvom:
    \begin{equation*}
        n \geq m \implies p_n := p(n) = n.
    \end{equation*}
    Tada ka\v zemo da je $p$ \emph{kona\v cna permutacija} na $\nat$.
\end{defn}

Ako je $S$ neprazan skup i $p$ kona\v cna permutacija na $\nat$, tada $p$ inducira permutaciju $f_p$ na $S^\infty$ definiranu s
\begin{equation*}
    f_p (s_1, \; s_2, \ldots) = (s_{p_1}, \; s_{p_2}, \ldots).
\end{equation*}

\begin{defn}    \label{defn:9.9-2}
    Skup $A \subseteq S^\infty$ je \emph{simetri\v can u odnosu na kona\v cne permutacije} ako vrijedi:
    \begin{equation*}
        f_p^{-1} (A) = A,
    \end{equation*}
    za svaku kona\v cnu permutaciju $p$ na $\nat$.
\end{defn}

Uo\v cimo da na izmjerivom prostoru $\urePar{S}{\famS}$ familija simetri\v cnih skupova tvori $\sigma$-algebru, ozna\v cenu s $\famS_{\simetr}^\infty$, koja je sadr\v zana u $\sigma$-algebri $\famS^\infty$.
% dokaži tu tvrdnju!
Tu $\sigma$-algebru $\famS_{\simetr}^{\infty}$ obi\v cno zovemo \emph{permutacijski invarijantnom} na $S^\infty$.

\begin{nap} %interna napomena
    Doka\v zi gornju tvrdnju.
\end{nap}

\begin{zad} \label{zad:9.10}
    Neka su $\famF_1 \subseteq \famF_2 \subseteq \ldots \subseteq \famF$ $\sigma$-algebre na vjerojatnosnom prostoru $\vjerojatnosniProstor$.
    Za svaki $A \in \indSigAlg{\famF_n}{n \in \nat}$ postoji niz $\niz{A_n}{n \in \nat} \subseteq \unija{n \in \nat}{} \famF_n$, takav da je
    \begin{equation*}
        \lim\limits_{n \to \infty} \vjeroj{A_n \triangle A} = 0.
    \end{equation*}
\end{zad}

\begin{rj}[\ref{zad:9.10}]
    U dokazu \' cemo koristiti dvije leme.

    \begin{lm}  \label{lm:9.10-1}
        Neka su $\niz{A_n}{n \in \nat}$ i $\niz{B_n}{n \in \nat}$ nizovi skupova, te neka je $m \in \nat$, tada vrijedi
        \begin{equation*}
            \Big( \unija{n = 1}{\infty} A_n \Big) \triangle \Big( \unija{n = 1}{m} B_n \Big) \subseteq \unija{n = 1}{m} (A_n \triangle B_n) \cup \unija{n = m + 1}{\infty} A_n.
        \end{equation*}
    \end{lm}

    \begin{proof}
        Neka je
        \begin{equation*}
            x \in \Big( \unija{n = 1}{\infty} A_n \Big) \triangle \Big( \unija{n = 1}{m} B_n \Big),
        \end{equation*}
        \v sto je ekvivalentno sa
        \begin{equation*}
            x \in \Big[ \Big( \unija{n = 1}{\infty} A_n \Big) \setminus \Big( \unija{n = 1}{m} B_n \Big) \Big] \cup \Big[ \Big( \unija{n = 1}{m} B_n \Big) \setminus \Big( \unija{n = 1}{\infty} A_n \Big) \Big].
        \end{equation*}
        Promatrajmo dva slu\v caja.
        \begin{enumerate}[label=(\roman*)]
            \item
            \begin{equation*}
                x \in \Big( \unija{n = 1}{\infty} A_n \Big) \setminus \Big( \unija{n = 1}{m} B_n \Big),
            \end{equation*}
            dakle postoji $n_0 \in \nat$ takav da vrijedi
            \begin{equation*}
                x \in A_{n_0} \land x \notin \unija{n = 1}{m} B_n.
            \end{equation*}
            Ponovo promatramo dva slu\v caja, ovisno o tome je li $n_0 \leq m$ ili $n_0 > m$.
            \begin{enumerate}[label=(\alph*)]
                \item $n_0 \leq m$
                \begin{equation*}
                    \begin{aligned}
                        x \in A_{n_0} \land x \notin B_{n_0} &\implies x \in A_{n_0} \setminus B_{n_0} \implies x \in A_{n_0} \triangle B_{n_0}\\
                        &\implies x \in \unija{n = 1}{m} (A_n \triangle B_n) \cup \unija{n = m + 1}{\infty} A_n. 
                    \end{aligned}
                \end{equation*}
                \item $n_0 > m$
                \begin{equation*}
                    x \in \unija{n = m + 1}{\infty} A_n \implies x \in \unija{n = 1}{m} (A_n \triangle B_n) \cup \unija{n = m + 1}{\infty} A_n.
                \end{equation*}
            \end{enumerate}
            \item
            Neka je sada 
            \begin{equation*}
                \Big( \unija{n = 1}{m} B_n \Big) \setminus \Big( \unija{n = 1}{\infty} A_n \Big),
            \end{equation*}
            primjetimo da sada postoji $n_0 \leq m$ takav da vrijedi
            \begin{equation*}
                \begin{aligned}
                    x \in B_{n_0} \land x \notin \unija{n = 1}{\infty} A_n &\implies x \in B_{n_0} \setminus A_{n_0} \implies x \in A_{n_0} \triangle B_{n_0}\\
                    &\implies x \in \unija{n = 1}{m} (A_n \triangle B_n) \cup \unija{n = m + 1}{\infty} A_n.
                \end{aligned}
            \end{equation*}
        \end{enumerate}
        Dakle tvrdnja vrijedi.
    \end{proof}

    \begin{lm}  \label{lm:9.10-2}
        Neka je $\vjerojatnosniProstor$ vjerojatnosni prostor te neka je $\famA$ algebra na $\Omega$ takva da je $\sigAlg{\famA} = \famF$, tada za svaki $A \in \famF$ te za svaki $\varepsilon > 0$ postoji $B \in \famA$ takav da vrijedi
        \begin{equation*}
            \masP (A \triangle B) < \varepsilon.
        \end{equation*}
    \end{lm}

    \begin{proof}
        Definirajmo familiju
        \begin{equation*}
            \famH := \bigIndFamilija{A \in \famF}{\forall \varepsilon > 0, \; \exists B \in \famA, \; \td \; \masP (A \triangle B) < \varepsilon},
        \end{equation*}
        tvrdimo da je $\famH$ $\sigma$-algebra.
        \begin{enumerate}[label=(\roman*)]
            \item
            Primjetimo da je $\famA \subseteq \famH$, odakle slijedi da je $\Omega \in \famH$.
            \item
            Neka je $A \in \famH$ te neka je $\varepsilon > 0$, tada postoji $B \in \famA$ takav da je $\masP (A \triangle B) < \varepsilon$.
            Primjetimo
            \begin{equation*}
                \begin{aligned}
                    A^c \triangle B^c &= (A^c \cup B^c) \cap (A^c \cap B^c)^c = (A \cap B)^c \cap \big( (A \cup B)^c \big)^c\\
                    &= (A \cup B) \cap (A \cap B)^c\\
                    &= A \triangle B.
                \end{aligned}
            \end{equation*}
            Budu\' ci je $\famA$ algebra i $B \in \famA$, tada je i $B^c \in \famA$, te imamo
            \begin{equation*}
                \masP (A^c \triangle B^c) = \masP (A \triangle B) < \varepsilon,
            \end{equation*}
            stoga je $A^c \in \famH$.

            \item
            Prvo pokazujemo da je $\famH$ zatvorena na kona\v cne unije, to jest da je $\famH$ algebra.
            Neka su $A_1, A_2 \in \famH$ i neka je $\varepsilon > 0$, postoje $B_1, B_2 \in \famA$ takvi da
            \begin{equation*}
                \begin{aligned}
                    \masP (A_1 \triangle B_1) &< \frac{\varepsilon}{2},\\
                    \masP (A_2 \triangle B_2) &< \frac{\varepsilon}{2}.
                \end{aligned}
            \end{equation*}
            Iz leme \ref{lm:9.10-1} vidimo da vrijedi
            \begin{equation*}
                (A_1 \cup A_2) \triangle (B_1 \cup B_2) \subseteq (A_1 \triangle B_1) \cup (A_2 \triangle B_2). 
            \end{equation*}
            Sada imamo
            \begin{equation*}
                \begin{aligned}
                    \masP ((A_1 \cup A_2) \triangle (B_1 \cup B_2)) &\leq \masP (A_1 \triangle B_1) + \masP (A_2 \triangle B_2)\\
                    &< \frac{\varepsilon}{2} + \frac{\varepsilon}{2} = \varepsilon.
                \end{aligned}
            \end{equation*}
            Budu\' ci je $\famA$ algebra, slijedi da je $B_1 \cup B_2 \in \famA$, dakle $A_1 \cup A_2 \in \famH$, pa je $\famH$ algebra.

            Kako bi pokazali da je $\famH$ zatvorena na prebrojive unije, dovoljno je pokazati da je zatvorena na prebrojive unije disjunktnih doga\dj aja.
            Neka je $\niz{A_n}{n \in \nat}$ niz disjuntnih doga\dj aja u $\famH$, neka je $\varepsilon > 0$, tada postoji niz $\niz{B_n}{n \in \nat} \subseteq \famA$ takav da vrijedi
            \begin{equation*}
                \masP (A_n \triangle B_n) < \frac{\varepsilon}{2^{n + 1}}.
            \end{equation*}
            Primjetimo $A := \unija{n = 1}{\infty} A_n \in \famF$, te vrijedi
            \begin{equation*}
                \masP \Big( \unija{n = 1}{\infty} A_n \Big) = \suma{n = 1}{\infty} \masP (A_n).
            \end{equation*}
            Kako $n$-ti ostatak konvergentonog reda mora te\v ziti u $0$, postoji $m_0 \in \nat$ takav da vrijedi
            \begin{equation*}
                \masP \Big( \unija{n = m_0 + 1}{\infty} A_n \Big) = \suma{n = m_0 + 1}{\infty} \masP (A_n) < \frac{\varepsilon}{2}.
            \end{equation*}
            Sada, koriste\' ci lemu \ref{lm:9.10-1}, imamo
            \begin{equation*}
                \begin{aligned}
                    \masP \Big( \Big( \unija{n = 1}{\infty} A_n \Big) \triangle \Big( \unija{n = 1}{m_0} B_n \Big) \Big) &\leq \masP \Big( \unija{n = 1}{m_0} (A_n \triangle B_n) \cup \unija{n = m_0 + 1}{\infty} A_n \Big)\\
                    &\leq \suma{n = 1}{m_0} \masP (A_n \triangle B_n) + \masP \Big( \unija{n = m_0 + 1}{\infty} A_n \Big) < \suma{n = 1}{m_0} \frac{\varepsilon}{2^{n + 1}} + \frac{\varepsilon}{2}\\
                    &< \varepsilon.
                \end{aligned}
            \end{equation*}
            Dakle $\unija{n = 1}{\infty} A_n \in \famH$, dakle $\famH$ je $\sigma$-algebra.
        \end{enumerate}
        Primjetimo sada $\famA \subseteq \famH \subseteq \sigAlg{\famA} = \famF$, dakle, budu\' ci je $\famH$ $\sigma$-algebra koja sadr\v zi $\famA$ mora vrijediti $\famH = \famF$, zbog minimalnosti od $\famF$.
    \end{proof}

    Preostaje primjetiti da je $\unija{n = 1}{\infty} \famF_n$ algebra.
    Prva dva svojstva se jednostavno vide, dok tre\' ce slijedi iz monotonosti niza $\sigma$-algebri $\famF_n$.

    Sada su zadovoljeni uvijeti iz leme \ref{lm:9.10-2} te za $A \in \bigSigAlg{\unija{n = 1}{\infty} \famF_n} = \indSigAlg{\famF_n}{n \in \nat}$ mo\v zemo konstruirati niz $\niz{A_n}{n \in \nat} \subseteq \unija{n = 1}{\infty} \famF$, sa svojstvom
    \begin{equation*}
        \masP (A \triangle A_n) < \frac{1}{n}.
    \end{equation*}
    Odavde vidimo da za niz $\niz{A_n}{n \in \nat}$ vrijedi
    \begin{equation*}
        \lim\limits_{n \to \infty} \masP (A \triangle A_n) = 0.
    \end{equation*}
\end{rj}

Podsjetimo se da za niz $E$-zna\v cnih slu\v cajnih elemenata $(X_n)$, mo\v zemo promatrati $X:=(X_1, \; X_2, \ldots)$ kao preslikavanje
\begin{equation*}
    X: \urePar{\Omega}{\famF} \to \urePar{E^\infty}{\famE^\infty}.
\end{equation*}

\begin{tm}[Hewitt-Savagev zakon 0-1]    \label{tm:9.11}
    Neka su $X_1, \; X_2, \ldots$ nezavisni jednako distribuirani slu\v cajni elementi s vjerojatnostima u izmjerivom prostoru $\urePar{E}{\famE}$.
    Tada je
    \begin{equation*}
        \praslika{X}(\famE_{\simetr}^\infty) \subseteq \famF    
    \end{equation*}
    trivijalna $\sigma$-algebra.
\end{tm}

\begin{proof}
    $\praslika{X} (\famE^\infty) \subseteq \famF$, pa je $\praslika{X} (\famE_{\simetr}^\infty) \subseteq \famF$ i $\sigma$-algebra je.
    Treba dokazati da je trivijalna.
    Uo\v cimo da je $\masP_X$ vjerojatnost na $\urePar{E^\infty}{\famE^\infty}$, pa treba dokazati da je $\famE_{\simetr}^\infty$ trivijalno u odnosu na $\masP_X$.

    Neka je
    \begin{equation*}
        \famE_n := \sigAlg{\pi_1, \ldots, \pi_n}, \quad n \in \nat    
    \end{equation*}
    pri \v cemu su $\pi_i : E^\infty \to E$ projekcije.
    Uo\v cimo da vrijedi
    \begin{equation*}
        \begin{aligned}
            \famE^\infty &= \indSigAlg{\famE_n}{n \in \nat}\\
            \famE_{\simetr}^\infty &\subseteq \famE^\infty,
        \end{aligned}
    \end{equation*}
    pa za $A \in \famE_{\simetr}^\infty$ primijenimo zadatak \ref{zad:9.10}, znamo da mo\v zemo prona\' ci niz
    \begin{equation*}
        \niz{A_n}{n \in \nat} \subseteq \unija{n \in \nat}{} \famE_n
    \end{equation*}
    takav da je
    \begin{equation*}
        \lim\limits_{n \to \infty} \masP_X (A \triangle A_n) = 0.
    \end{equation*}
    Bez smanjenja op\' cenitosti mo\v zemo posti\' ci da je $A_n \in \famE_n$, za svaki $n \in \nat$.
    Uzmimo permutaciju takvu da
    \begin{equation*}
        \begin{aligned}
            \{1, \ldots, n\} &\to \{n+1, \ldots, 2n\}\\
            \{n+1, \ldots, 2n\} &\to \{1, \ldots, n\}.
        \end{aligned}
    \end{equation*}
    Tada je $f_p(A) = A$ (uo\v cimo i $f_p^{-1} = f_p$) i neka je
    \begin{equation*}
        B_n := f_p (A_n).
    \end{equation*}
    Budu\' ci da su $(X_n)$ nezavisni i jednako distribuirani slijedi fa je
    \begin{equation*}
        (\masP_X)_{f_p} = \masP_X    
    \end{equation*}
    pa je
    \begin{equation*}
        \begin{gathered}
            \masP_X (B_n) = \masP_X (A_n) \xrightarrow[n \to \infty]{} \masP_X (A)\\
            \masP_X (B_n \triangle A) \xrightarrow[n \to \infty]{} 0.
        \end{gathered}
    \end{equation*}
    Slijedi da je
    \begin{equation*}
        \masP_X (A \triangle (A_n \cap B_n)) \leq \masP_X (A \triangle A_n) + \masP_X (A \triangle B_n) \xrightarrow[n \to \infty]{} 0.
    \end{equation*}
    Uo\v cimo da su zbog nezavisnosti $(X_n)$ skupovi $A_n$ i $B_n$ nezavisni u odnosu na $\masP_X$, pa dobivamo:
    \begin{equation*}
        \begin{aligned}
            \masP_X (A)
            &= \lim\limits_{n \to \infty} \masP (A_n \cap B_n) = \lim\limits_{n \to \infty} \masP_X (A_n) \cdot \masP_X (B_n)\\
            &= \masP_X (A) \cdot \masP_X(A) \implies \masP(A) \in \{0, \; 1\}.
        \end{aligned}
    \end{equation*}
\end{proof}

\begin{nap} \label{nap:9.12}
    Slu\v cajna varijabla modelira numeri\v cke ishode nekog slu\v cajnog pokusa.
    
    Kona\v cno ponavljanje pokusa modeliramo slu\v cajnim vektorima.
    U oba slu\v caja je vjerojatnosna informacija sadr\v zana u funkciji distribucije koja je
    \begin{equation*}
        F: \real^d \to \segment{0}{1}
    \end{equation*}
    i za koju imamo razvijen analiti\v cki (ili kombinatorni) aparat.
    Vrlo \v cesto nam je (i s teorijskog i s primjenjenog stajali\v sta) va\v zno, barem konceptualno, razmi\v sljati o beskona\v cnom ponavaljanju pokusa.
    Prva najjednostavnija prepostavka je da se ti pokusi provode \emph{nezavisno}.
    Stoga osnovni objekt na\v se teorije postaje niz nezavisnih slu\v cajnih varijabli $\niz{X_n}{n \in \nat}$ na vjerojatnosnom prostoru $\vjerojatnosniProstor$.
    Uo\v cimo da rezultati iz poglavlja \ref{poglavlje8} pokazuju da se takav niz uvijek mo\v ze realizirati (i to \v cak na $\segment{0}{1}$).
    Dakle, za svaki izbor p.d.F. $F_1, \; F_2, \ldots$ postoji niz nezavisnih slu\v cajnih varijabli, na istom vjerojatnosnom prostoru, koje imaju to\v cno zadane distribucije.
    Jedno od najva\v znijih pitanja postaje asimptotsko pona\v sanje niza $(X_n)$.
    Osobito je va\v zan poseban slu\v caj u kojem postoji p.d.F. $F$ takav da je $F_{X_n} = F$, za svaki $n \in \nat$.
    To je slu\v caj niza \emph{nezavisnih i jednako distribuiranih slu\v cajnih varijabli}, koji \' cemo kra\' ce ozna\v cavati sa $\iid$
\end{nap}

    %%%%%%%%%%%%%%%%%%%%%%%%%%%%%%%%%%%%%%%%%
    %%  bernoullijeva razdioba i primjene  %%
    %%%%%%%%%%%%%%%%%%%%%%%%%%%%%%%%%%%%%%%%%

    % poglavlje 2.5 - bernoullijeva razdioba i primjene -> 10
\chapter{Bernoullijeva razdioba i primjene} \label{pog:2.5}

Najjednostavnija slu\v cajna varijabla je degenerirana.
Budu\' ci je suma degeneriranih slu\v cajnih varijabli ponovo degenerirana slu\v cajna varijabla, lako je opisati \v sto se doga\dj a s degeneriranim slu\v cajem.
Prvi slu\v caj u kojem se javalja "prava slu\v cajnost" je razdijoba s dvije mogu\' ce vjerojatnosti

\begin{equation*}
    \begin{pmatrix}
        a & b\\
        q & p
    \end{pmatrix}
    \quad \quad
    \begin{matrix}
        a, \; b \in \real, \; a \neq b,\\
        0 < p < 1, \; p + q = 1.
    \end{matrix}
\end{equation*}
Budu\' ci je lako $a$ i $b$ transformirati u neke druge vrijednosti $c$, $d$, promatrati \' cemo slu\v caj kada je $a = 0$, $b = 1$ ili $a = -1$, $b = 1$.
Takva distribucija naziva se Bernoullijeva (\v svicarska matemati\v cka obitelj s preko deset matemati\v cara, nije uvijek jasno tko je \v sto radio); uglavnom u \v cast Jakoba Bernoullija (1654 - 1705).
Podsjetimo se da, ako su $X_1, \ldots, X_n$ nezavisne i za svaki $1 \leq i \leq n$,
\begin{equation*}
    X_i \sim
    \begin{pmatrix}
        0 & 1\\
        q & p
    \end{pmatrix},   
\end{equation*}
tada je
\begin{equation*}
    S_n := X_1 + \ldots + X_n \quad \sim B \urePar{n}{p}
\end{equation*}
binomna slu\v cajna varijabla i za $k \in \{0, \; 1, \ldots, n \}$
\begin{equation*}
    \vjeroj{S_n = k} = {n \choose k} p^k q^{n - k}.
\end{equation*}
Za ovakve ra\v cune \v cesto koristimo Stirlingovu formulu.

\begin{zad} \label{zad:10.1}
    Doka\v zite (koriste\' ci literaturu) da vrijedi formula koju je 1730. dokazao James Stirling:
    Za svaki $n \in \nat$,
    \begin{equation*}
        n! = \sqrt{2 \pi} \: n^{n + \frac{1}{2}} e^{-n + \varepsilon_n}
    \end{equation*}
    pri \v cemu je $\frac{1}{12 n + 1} < \varepsilon_n < \frac{1}{12 n}$.
\end{zad}

Zato za aproksimaciju uzmemo:
    \begin{equation*}
        n! \approx \sqrt{2 \pi} \: n^{n + \frac{1}{2}} e^{-n}
    \end{equation*}
i kako je $n$ ve\' ci, aproksimacija je sve bolje (na primjer za $n = 10$, gre\v ska je $0.8\%$, a za $n = 100$ iznosi $0.08\%$).

\begin{pr}  \label{pr:10.2}
    Pozivi pristi\v zu u telefonsku centralu na slu\v cajan na\v cin (ili automobili kroz kontrolnu to\v cku na autoputu, ili kupci u du\' can; op\' cenito "klijenti" nekog "servisa").
    Razumno je pretpostaviti da nema "dogovora" me\dj u korisnicima, to jest da se radi o \emph{nezavisnim} slu\v cajnim pojavama.
    Obi\v cno imamo ve\' ci broj podataka, pa se mo\v ze odrediti "vjerojatnost pojave barem jednog klijenta u jedinici vremena" $= p \in \obInt{0}{1}$.
    Sli\v cno, mo\v zemo izmjeriti "prosje\v cno vrijeme koje protekne izme\dj u pojave dva uzastopna klijenta" $= \mu \in \obInt{0}{+\infty}$.
    Uo\v cimo da $\lambda := \frac{1}{\mu}$ mo\v zemo interpretirati kao frekvenciju $=$ "prosje\v can broj dospjelih poziva u jedinici vremena".
    Promatramo vremenski interval jedini\v cne duljine i stavimo $I_0^1 = 1$ (uspjeh) ako se u intervalu javi bar jedan korisnik, a $I_0^1 = 0$ (neuspjeh), u suprotnom, dakle:
    \begin{equation*}
    I_0^1 \sim
    \begin{pmatrix} 0 & 1\\
        q & p \end{pmatrix},
    \end{equation*}
    sa $q:= 1 - p$.
    Promatrajmo sada vremensku skalu $t \in \lijInt{0}{+\infty}$, tako da za svaki interval jedini\v cne duljine oblika $\lijInt{n - 1}{n}$ imamo slu\v cajnu varijablu $I_0^n$ koja mjeri uspjeh ili neuspjeh ovisno o tome je li u $\lijInt{n - 1}{n}$ do\v slo do barem jednog poziva ili ne.
    Tada $\niz{I_0^n}{n \in \nat}$ tvore niz nezavisnih, jednako distribuiranih slu\v cajnih varijabli
    \begin{equation*}
        I_0^n \sim
        \begin{pmatrix}
            0 & 1\\
            q & p
        \end{pmatrix}.
    \end{equation*}
    Ozna\v cimo li s $T_0^n =$ "vrijeme (interval) $n$-tog poziva", intuitivno je prihvatljivo da su $\niz{T_0^n}{n \in \nat}$ nezavisne i jednako distribuirane.
    Dakle $T_0^1 \distJed T_0^n$ i vrijedi
    \begin{equation*}
        \begin{aligned}
        \vjeroj{T_0^n = m} &= \vjeroj{I_0^1 = 0, \ldots, I_0^{m - 1} = 0,  \; I_0^m = 1}\\
        &= p \cdot q^{m - 1}, \quad m \in \nat.
        \end{aligned} 
    \end{equation*}
    Ovakva razdioba naziva se \emph{geometrijskom}.

    Vr\v simo sada induktivno korake "profinjenja" tako da u svakom koraku podjelimo postoje\' ce intervale na pola.
    Dakle, u $l$-tom koraku dobijemo intervale duljine $\frac{1}{2^l}$ te varijable $\niz{I_l^n}{n \in \nat}$ koje su nezavisne i jednako distribuirane
    \begin{equation*}
        I_l^n \sim
        \begin{pmatrix}
            0 & 1 \\
            q_l & p_l
        \end{pmatrix}
    \end{equation*}
    (uspjeh ili neuspjeh u $\lijInt{\frac{n - 1}{2^l}}{\frac{n}{2^l}}$) i $\niz{T_l^n}{n \in \nat}$ vremena $n$-tog poziva u "mre\v zi fino\' ce $\frac{1}{2^l}$".
    Dakle, $\vjeroj{T_l^n = m} = p_l \: q_l^{m - 1}$, a stvarno vrijeme $n$-tog poziva je $\frac{m}{2^l}$.
    Koja je veza izme\dj u $p_l$ i $p$?
    Uo\v cimo
    \begin{equation*}
        \begin{aligned}
            q &= \vjeroj{I_0^1 = 0} = \vjeroj{I_l^1 = 0, \ldots, I_l^{2^l} = 0} = (q_l)^{2^l} \implies\\
            p &= 1 - (1 - p_l)^{2^l}.
        \end{aligned}
    \end{equation*}
    Ozna\v cimo li s $V_l =$ "pravo vrijeme prvog poziva u mre\v zi fino\' ce $\frac{1}{2^l}$", dobivamo
    \begin{equation*}
        \begin{aligned}
            \masE (V_l) &= \masE \Big(\frac{T_l^1}{2^l} \Big) = \frac{p_l}{2^l} \suma{m = 1}{\infty} m \: q_l^{m - 1} = \frac{p_l}{2^l} \cdot \frac{1}{q_l} \: \suma{m = 1}{\infty} m \: q_l^m\\
            &= \frac{p_l}{2^l \: q_l} \cdot \suma{m = 1}{\infty} \big( \suma{k = 1}{m} 1 \cdot q_l^m \big) \overset{Fubini}{=} \frac{p_l}{2^l \: q_l} \suma{k = 1}{\infty} \: \suma{m = k}{\infty} q_l^m\\
            &= \frac{p_l}{2^l \: q_l} \suma{k = 1}{\infty} q_l^k \: \underbrace{\suma{m = k}{\infty} q_l^{m - k}}_{= \frac{1}{1-q_l} = \frac{1}{p_l}} = \frac{1}{2^l \: q_l} \cdot \suma{k = 1}{\infty} q_l^k = \frac{1}{2^l \: \cancel{q_l}} \cdot \frac{\cancel{q_l}}{1 - q_l}\\
            &= \frac{1}{p_l \: 2^l}.
        \end{aligned}
    \end{equation*}
    Neka su sada $\niz{T_n}{n \in \nat}$ vremena $n$-tog poziva u kontinuiranoj skali.
    Tada je $\mu = \masE (T^n) = \masE (T^1)$ i ima smisla
    \begin{equation*}
        \begin{aligned}
            & \lim\limits_{l \to \infty} \masE (V_l) = \masE (T^1) \implies \\
            \lambda = & \lim\limits_{l \to \infty} p_l \: 2^l = \lim\limits_{l \to \infty} \: (-1) \cdot \frac{(1-p)^{\frac{1}{2^l}}}{\frac{1}{2^l}}\\
            = & - \ln (1 - p) \implies\\
            p =& 1 - e^{-\lambda}.
        \end{aligned}
    \end{equation*}
    Potpuno analogno mogli bi gledati svako vrijeme $t>0$ i dobili bismo
    \begin{equation*}
        p(t) = \vjeroj{T^1 \leq t} = 1 - e^{- \lambda \: t},
    \end{equation*}
    to jest, svaka $T^n \sim Exp(\lambda)$.

    Promatramo li broj poziva ostvarenih do momenta $t = 1$ i ozna\v cimo sa $N_1$, slijedi da je za $k \in \nat_0$
    \begin{equation*}
        \begin{aligned}
            \vjeroj{N_1 = k} &= \lim\limits_{l \to \infty} \masP \Big(\suma{m = 1}{2^l} I_l^m = k \Big) = \lim\limits_{l \to \infty} {2^l \choose k} p_l^k \: q_l^{2^l - k}\\
            &= \lim\limits_{l \to \infty} \frac{1}{k!} \cdot (q_l)^{2^l} \cdot (2^l) \: (2^l - 1) \ldots (2^l - k + 1) \cdot \big( \frac{p_l}{q_l} \big)^k\\
            &= \frac{e^{-\lambda}}{k!} \cdot 1^k \cdot \Big[ \lim\limits_{l \to \infty} \frac{p_l \: 2^l}{q_l} \Big]^k\\
            &= \frac{e^{-\lambda}}{k!} \: \lambda^k.
        \end{aligned}
    \end{equation*}
    Sli\v cno za $t > 0$ dobivamo
    \begin{equation*}
        \vjeroj{N_t = k} = \frac{(\lambda \: t)^k}{k!} \: e^{-\lambda}.
    \end{equation*}
    Dakle imamo:
    \begin{equation*}
        \textnormal{geometrijska} \leadsto \textnormal{eksponencijalna, binomna} \leadsto \textnormal{Poissonova}.
    \end{equation*}
    Informacija o $T^n$ se mo\v ze \v citati iz $N_t$ i obratno jer 
    \begin{equation*}
        \Big\{ \suma{k = 1}{n} T^k \leq t \Big\} = \{ N_t \geq n \}.
    \end{equation*}
    Stohasti\v cki proces $\niz{N_t}{t \geq 0}$ je \emph{Poissonov proces s parametrom $\lambda > 0$}.
\end{pr}

U primjeru \ref{pr:10.2} promatrali smo Bernoullijeve slu\v cajne varijable kod kojih se vjerojatnost uspjeha mjenja na svakom nivou i bila je uskla\dj ena s veli\v cinom $l$ tako da
\begin{equation*}
    \frac{p_l}{\frac{1}{2^l}} \xrightarrow[l \to \infty]{} \lambda.
\end{equation*}
Pogledajmo tipi\v can slu\v caj primjera u koje se $p$ ne mjenja.

\begin{pr}  \label{pr:10.3}
    Neka je $\niz{X_n}{n \in \nat} \; \iid$ za koji vrijedi
    \begin{equation*}
        X_n \sim
        \begin{pmatrix}
            -1 & 1\\
            q & p
        \end{pmatrix},
    \end{equation*}
    $0 < p < 1$ i $q = 1 - p$. Vidjeli smo u poglavlju \ref{zakoni_01} da se takav niz uvijek mo\v ze realizirati (i to \v cak na $\segment{0}{1}$).
    Ovaj niz je model "beskona\v cne igre" u kojoj u svakom koraku dobijemo ili izgubimo nov\v canu jedinicu (primjene ovog modela su puno \v sire od "hazardnih igara").
    Neka je
    \begin{equation*}
        S_n = \textnormal{"na\v se dokumentirano bogatstvo u trenutku $n$"},
    \end{equation*}
    to jest bez smanjenja op\' cenitosti uzimamo
    \begin{equation*}
        \begin{aligned}
            S_0 &= 0,\\
            S_n &= X_1 + \ldots + X_n, \quad n \in \nat;
        \end{aligned}
    \end{equation*}
    uo\v cimo da je lako promatrati neku drugu po\v cetnu razdiobu $Y$ - samo uzmemo $V_n = Y + S_n$.
    O\v cito, ako je $n$ paran, $S_n$ mo\v ze posti\' ci samo parne vrijednosti izme\dj u $-n$ i $n$, te za $k = 0, \: 1, \ldots, n$ imamo
    \begin{equation*}
        \vjeroj{S_n = -n + 2K} = {n \choose k} p^k \: q^{n - k},
    \end{equation*}
    i sli\v cno dobijemo za neparne $n$.
    \v Sto se doga\dj a s trajektorijama $n \mapsto S_n(\omega)$ za velike $n$?
    Uo\v cimo da je funkcija
    \begin{equation*}
        f(\omega) := \limsup\limits_{n \to \infty} S_n(\omega)    
    \end{equation*}
    pro\v sirena slu\v cajna varijabla.
    Promatrajmo $\overline{\real}^{\infty}$ i skup $A_c \subseteq \overline{\real}^\infty$, za $c \in \real$, definiran sa
    \begin{equation*}
        A_c := \bigSkup{(r_n) \in \overline{\real}^\infty}{\limsup\limits_{n \to \infty} r_n \leq c}.
    \end{equation*}
    Ako je $p$ kona\v cna permutacija od $\nat$, tada poredak prvih kona\v cno brojeva ne\' ce utjecati na limes superior, pa je $\praslika{f} (A_c) = A_c$, to jest $A_c \in (\borel{\real}^\infty)_\simetr$.
    Budu\' ci su $(X_n) \; \iid$ po Hewitt-Savageovom zakon slijedi da je
    \begin{equation*}
        \{f \leq c\} \in \praslika{X} (\borel{\real}^\infty)_\simetr.
    \end{equation*}
    To jest $f$ je izmjeriva s obzirom na trivijalnu $\sigma$-algebru.
    Po lemi \ref{lm:9.6} postoji to\v cno jedan $b \in \extReal$, takav da je
    \begin{equation*}
        \limsup\limits_{n \to \infty} S_n = b \; (g.s.).
    \end{equation*}
    Na isti na\v cin dobijemo
    \begin{equation*}
        \liminf\limits_{n \to \infty} S_n = a \; (g.s.),
    \end{equation*}
    i vrijedi $a, \; b \in \extReal$, $a \leq b$.
    Sli\v cno zaklju\v cujemo u slu\v caju skupa
    \begin{equation*}
        A_B := \bigSkup{(r_n) \in \extReal^\infty}{(r_n) \textnormal{ u $B$ beskona\v cno puta}},
    \end{equation*}
    pri \v cemu je $B \in \borel{\extReal}$.
    Opet po Hewitt-Savage-ovom zakonu vrijedi:
    \begin{equation*}
        \vjeroj{S_n \in B \; \io} \in \{0, \; 1\}.
    \end{equation*}
    % \limsup\limits_{n \to \infty} \{ S_n \in B\} ?= S_n \in B \; \io

    Mo\v ze li biti $b \in \real$?

    Uo\v cimo prvo da zbog nezavisnosti i jednake distribuiranosti vrijedi
    \begin{equation*}
        (X_1, \; X_2, \ldots) \distJed (X_2, \; X_3, \ldots),
    \end{equation*}
    \v sto daje 
    \begin{equation*}
        \limsup\limits_{n \to \infty} (S_{n + 1} - X_1) = b \; (g.s.).
    \end{equation*}
    Stoga dobivamo:
    \begin{equation*}
        \begin{gathered}
            \begin{aligned}
                b &= \limsup\limits_{n \to \infty} S_{n + 1}\\
                &= \limsup\limits_{n \to \infty} \big[ (S_{n + 1} - X_1) + X_1 \big]\\
                &= b + X_1 \; (g.s.)
            \end{aligned}\\
            \implies X_1 = 0 \; (g.s.),
        \end{gathered}
    \end{equation*}
    \v sto je kontradikcija.
    Dakle, $a, b \in \{-\infty, \; +\infty\}$.

    Posebno, ako je $b = -\infty$ i $a = + \infty$, tada
    \begin{equation*}
        \begin{aligned}
            \lim\limits_{n  \to \infty} S_n &= -\infty \; (g.s.)\\
            \lim\limits_{n \to \infty} S_n &= +\infty \; (g.s.).
        \end{aligned}
    \end{equation*}
    Ako je $a = -\infty$, $b = +\infty$, tada gotovo sigurno trajektorije osciliraju sve vi\v se i vi\v se.

    Ako je $p = q = \frac{1}{2}$, tada je $(-X_1, \; -X_2, \ldots) \distJed (X_1, \; X_2, \ldots)$, pa je
    \begin{equation*}
        \limsup\limits_{n \to \infty} (-S_n) = \limsup\limits_{n \to \infty} S_n = b,
    \end{equation*}
    a s druge strane je
    \begin{equation*}
        \limsup\limits_{n \to \infty} (-S_n) = - \liminf\limits_{n \to \infty} S_n = -a,
    \end{equation*}
    a kako ne mo\v ze biti $b = -a$ i $b = a$ istovremeno, mora biti $a = -\infty$, $b = +\infty$.
    %istraži malo ove limsup i liminf

    Ako je $p \neq q$, onda promatramo $\varepsilon > 0$ i vrijedi
    \begin{equation*}
        \begin{aligned}
            \masP \Bigg( \Bigg|\frac{S_n}{n} - (p - q) \Bigg| \geq \varepsilon \Bigg) =& \suma{|k - n \: (p-q)| \geq n \varepsilon}{} \vjeroj{S_n = k} \leq \suma{k \in R_{S_n}}{} \frac{(k - n(p-q))^4}{n^4 \: \varepsilon^4} \: \vjeroj{S_n = k}\\
            %
            %   dokaži jednakost
            %
            =& \frac{k n + c}{n^3 \: \varepsilon^4} \implies\\
            \implies & \suma{n = 1}{\infty} \masP \Bigg( \Bigg| \frac{S_n}{n} - (p - q) \Bigg| \geq \varepsilon \Bigg) < + \infty, \quad \forall \varepsilon > 0.\\
            \overset{\ref{lm:9.2}}{\implies} & \masP \Bigg( \Bigg| \frac{S_n}{n} - (p - q) \Bigg| \geq \varepsilon, \; \io \Bigg) = 0\\
            \overset{\textnormal{Uz }\; \varepsilon = \frac{|p - q|}{2}}{\implies} & \vjeroj{\sign (S_n) \neq \sign (p - q) \; \io} = 0\\
            \implies & \lim\limits_{n \to \infty} S_n
            =
            \begin{cases}
                + \infty, &p > q\\
                - \infty, &q < p
            \end{cases}
            \quad (g.s.)
        \end{aligned}
    \end{equation*}
\end{pr}

Vidjet \' cemo uskoro da mo\v zemo dobiti i preciznije odgovore.
Uo\v cimo da u nekim od ovih dokaza nije bila va\v zna precizna distribucija; to \' cemo precizirati u sljede\' cem zadatku.

\begin{zad} \label{zad:10.4}
    Neka je $(X_n) \; \iid$ niz, $S_0 = 0$, $S_n = X_1 + \ldots + X_n$.
    Doka\v zi:
    \begin{enumerate}[label=(\alph*)]
        \item $(\exists c \in \extReal)\; \limsup\limits_{n \to \infty} S_n = c \; (g.s.)$
        \item Ako $X_n$ nije degeneriran u $0$, tada je $c \in \{-\infty, \; +\infty\}$.
        \item Ako $X_n$ nije degeneriran u $0$ i simetri\v can je (tj. $X_n \distJed - X_n$), tada vrijedi
        \begin{equation*}
            \begin{aligned}
                \liminf\limits_{n \to \infty} S_n &= - \infty \; (g.s.)\\
                \limsup\limits_{n \to \infty} S_n &= +\infty \; (g.s.)
            \end{aligned}
        \end{equation*}
    \end{enumerate}
\end{zad}


    \part{Zakoni velikih brojeva}


    %%%%%%%%%%%%%%%%%%%%%%%%%%%%%%%%%%%%%%%
    %%  konvergencija nezavisnih nizova  %%
    %%%%%%%%%%%%%%%%%%%%%%%%%%%%%%%%%%%%%%%

    % poglavlje 11 - konvergencija nezavisnih nizova

\chapter{Konvergencija nezavisnih nizova}

U ovom poglavlju koristit \' cemo stalne oznake: $\vjerojatnosniProstor$, \' ce ozna\v cavati vjerojatnosni prostor, a $\niz{X_n}{n \in \nat}$, niz slu\v cajnih varijabli na $\Omega$.

Uo\v cimo da je skup $\conv{(X_n)} := \{ \lim\limits_{n} X_n \in \real \} = \skup{\omega \in \Omega}{\lim\limits_{n \in \infty} X_n (\omega) \in \real}$\\
$ = \presjek{k = 1}{\infty} \unija{n = 1}{\infty} \presjek{i = 1}{\infty} \skup{\omega}{|X_{n + i} (\omega) - X_n (\omega)| < \frac{1}{k}}$ uvijek doga\dj aj, te da je $X:= \limsup\limits_{n \to \infty} X_n$ pro\v sirena slu\v cajna varijabla.
Stoga
\begin{equation}    \label{jed:11.1}
    \vjeroj{\conv{(X_n)}} = 1 \implies \vjeroj{X = \lim\limits_n X_n \in \real} = 1
\end{equation}
i u tom slu\v caju ka\v zemo da $(X_n)$ \emph{konvergira gotovo sigurno}.
Ako je $Y$ slu\v cajna varijabla i postoji $D \in \famF$ takav da je $\vjeroj{D} = 0$ i za svaki $\omega \in D^c$ je $Y(\omega) = \lim\limits_{n \to \infty} X_n (\omega)$, tada ka\v zemo da $(X_n)$ \emph{konvergira gotovo sigurno prema} $Y$ i pi\v semo $X \xrightarrow[n \to \infty]{g.s.} Y$ ili $Y = (g.s.) \; \lim\limits_{n \to \infty} X_n$.
O\v cito vrijedi
\begin{equation}    \label{jed:11.2}
    (\exists Y) \; X_n \xrightarrow[n \to \infty]{g.s.} Y \iff \vjeroj{\conv{(X_n)} = 1}
\end{equation}
i u tom slu\v caju $X_n \xrightarrow[n \to \infty]{g.s.} X$ i $X = Y \; (g.s.)$.
Nadalje o\v cito vrijedi:
\begin{equation}    \label{jed:11.3}
    X_n \xrightarrow[n \to \infty]{g.s.} Y \iff X_n - Y \xrightarrow[n \to \infty]{g.s.} 0,
\end{equation}
i
\begin{equation}    \label{jed:11.4}
    (X_n) \textnormal{ konvergira } g.s.
    \iff
    \begin{matrix}
        \niz{X_n(\omega)}{n \in \nat} \textnormal{ Cauchyjev}\\
        \textnormal{za gotovo sve } \omega \in \Omega.
    \end{matrix}
\end{equation}

\begin{nap} \label{nap:11.4-1}
    Prisjetimo se, neka je $\niz{A_n}{n \in \nat}$ niz skupova, tada su limes superior i limes inferior definirani sa:
    \begin{enumerate}[label=(\roman*)]
        \item $\liminf\limits_{n \to \infty} A_n := \unija{n = 1}{\infty} \Bigg( \presjek{k = n}{\infty} A_k \Bigg)$
        \item $\limsup\limits_{n \to \infty} A_n := \presjek{n = 1}{\infty} \Bigg( \unija{k = n}{\infty} A_k \Bigg)$
    \end{enumerate}
\end{nap}

Va\v znu ulogu u opisu ovog pojma imaju skupovi $A_k^n := \skup{\omega \in \Omega}{|X_n(\omega) - X(\omega)| > \frac{1}{k}}$, $A_k := \limsup\limits_{n \to \infty} A_k^n$ i $D := \unija{k = 1}{\infty} A_k$.
Uo\v cimo $l \leq k \implies A_l \subseteq A_k$ i dobivamo
\begin{equation*}
    \begin{aligned}
        D &= \skup{\omega \in \Omega}{X_n(\omega) \cancel{\xrightarrow{}}X(\omega)}\\
        \implies  \vjeroj{D} &= \lim\limits_{k \to \infty} \vjeroj{A_k}\\
        \vjeroj{D} = 0 &\iff \vjeroj{A_k} = 0, \; \forall k \in \nat.
    \end{aligned}
\end{equation*}
Sada vrijedi
\begin{equation}    \label{jed:11.5}
    \begin{aligned}
        &X_n \xrightarrow[n \to \infty]{g.s.} X\\
        \iff &\vjeroj{D} = 0\\
        \iff &\lim\limits_{m \to \infty} \masP \Bigg( \unija{l = m}{\infty} A_k^l \Bigg) = 0, \quad \forall k \in \nat\\
        \iff &\lim\limits_{m \to \infty} \masP \Bigg( \unija{l = m}{\infty} \{ |X_l - X| > \varepsilon \} \Bigg) = 0, \quad \forall \varepsilon > 0.
    \end{aligned}
\end{equation}

Iz \eqref{jed:11.4} i \eqref{jed:11.5} direktno dobijemo:

\begin{lm}  \label{lm:11.6}
    Niz $\niz{X_n}{n \in \nat}$ je konvergentan $g.s.$ ako i samo ako vrijedi
    \begin{equation*}
        \lim\limits \masP \Bigg( \unija{l = m}{\infty} \{ |X_l - X| > \varepsilon \} \Bigg) = 0, \quad \forall \varepsilon > 0
    \end{equation*}
\end{lm}

Pretpostavimo sada da su $(X_n)$ nezavisne slu\v cajne varijable.
Uo\v cimo da je $\conv{(X_n)}$ repni doga\dj aj, pa prema Kolmogorovljevom zakonu 0-1 mora biti
\begin{equation}    \label{jed:11.7}
    \vjeroj{\conv{(X_n)}} \in \{ 0, \; 1 \}.
\end{equation}
Nadalje, $\liminf\limits_{n \to \infty} X_n$ i $\limsup\limits_{n \to \infty} X_n$ su repne funkcije, pa postoje $- \infty \leq a \leq A \leq +\infty$ takvi da vrijedi:
\begin{equation}    \label{jed:11.8}
    \begin{aligned}
        \liminf\limits_{n \to \infty} X_n &= a \quad (g.s.)\\
        \limsup\limits_{n \to \infty} X_n &= A \quad (g.s.)
    \end{aligned}
\end{equation}

\begin{tm}  \label{tm:11.9}
    Neka je $\niz{X_n}{n \in \nat}$ niz nezavisnih slu\v cajnih varijabli.
    Sljede\' ce tvrdnje su ekvivalentne:
    \begin{enumerate}[label=(\roman*)]
        \item   \label{tm:11.9.1}
        $(X_n)$ je konvergentan $g.s.$;
        \item   \label{tm:11.9.2}
        $-\infty < a = A < +\infty$;
        \item   \label{tm:11.9.3}
        $a \in \real$ i $X_n \xrightarrow[n \to \infty]{g.s.} a$;
        \item   \label{tm:11.9.4}
        $\suma{n = 1}{\infty} \vjeroj{|X_n - a| > \varepsilon} < +\infty, \quad \forall \varepsilon > 0$.
    \end{enumerate}
\end{tm}

\begin{proof}
    Iz \eqref{jed:11.7} i \eqref{jed:11.8} slijedi \ref{tm:11.9.1} $\iff$ \ref{tm:11.9.2}, dok je \ref{tm:11.9.2} $\iff$ \ref{tm:11.9.3} o\v cito.
    Doka\v zimo \ref{tm:11.9.3} $\implies$ \ref{tm:11.9.4}.
    Ozna\v cimo sa $C_n := \{ |X_n - a| > \varepsilon\}$, za zadani $\varepsilon > 0$.
    Po pretpostavci $\niz{C_n}{n \in \nat}$ su nezavisni doga\dj aji, a zbog $X_n \xrightarrow[n \to \infty]{g.s.} a$ dobivamo $\vjeroj{\limsup\limits_{n \to \infty} C_n} = 0$.
    Po Borelovom zakonu 0-1 $\suma{n = 1}{\infty} \vjeroj{C_n} < +\infty$.
    Doka\v zimo \ref{tm:11.9.4} $\implies$ \ref{tm:11.9.3}.
    Zbog $\suma{n = 1}{\infty} \vjeroj{C_n}$, Borel 0-1 daje $\vjeroj{\limsup\limits{n \to \infty} C_n} = 0$.
    Uo\v cimo, ako $X_n \cancel{\to} a$, tada je $\omega \in \limsup\limits_{n \to \infty} C_n$, za neki $\varnothing = \frac{1}{k}$.
    Dokaz slijedi direktno.
\end{proof}

Dakle, vrlo su rjetki slu\v cajevi, kak niz nezavisnih slu\v cajnih varijabli kongergira gotovo sigurno.
Slu\v caj $\iid$ je jo\v s posebniji i nema potrebe za daljnjim prou\v cavanjem.

\begin{kor} \label{kor:11.10}
    Neka je $\niz{X_n}{n \in \nat} \; \iid$ niz.
    Tada niz $(X_n)$ konverigra gotovo sigurno ako i samo ako je $X_n = a \; (g.s.)$, za svaki $n \in \nat$.
\end{kor}

\begin{proof}
    Dovoljnost je o\v cita, a nu\v znost slijedi iz \ref{tm:11.9} \ref{tm:11.9.4} zbog
    \begin{equation*}
        \begin{gathered}
            +\infty > \suma{n = 1}{\infty} \vjeroj{|X_n - a|>\varnothing} = \iid = \suma{n = 1}{\infty} \vjeroj{|X_1 - a|> \varepsilon}\\
            \implies \vjeroj{|X_1 - a| > \varepsilon} = 0, \quad \varepsilon > 0.
        \end{gathered}
    \end{equation*}
\end{proof}

\begin{zad} \label{zad:11.11}
    Neka je $\niz{X_n}{n \in \nat}$ $\iid$ niz.
    Doka\v zite:
    \begin{equation*}
        \vjeroj{\limsup\limits_{n \to \infty} X_n  = +\infty} = 1 \iff \vjeroj{X_1 < c} < 1,
    \end{equation*}
    za svaki $0 < c < +\infty$.
\end{zad}

\begin{nap} \label{nap:11.12}
    Time smo uglavnom opisali \v sto se mo\v ze desti s $n \mapsto X_n(\omega)$ za velike $n$.
    Intuitivno mo\v zemo re\' ci da \' cemo za nezavisne $(X_n)$ "rjetko" imati konvergenciju.
    U $\iid$ slu\v caju samo degenerirana razdioba daje konvergenciju.
    Na primjer ako je $X_1 \sim \begin{pmatrix} 0 & 1\\ q & p \end{pmatrix}$ i $0 < p < 1$, trajektorija $n \mapsto X_n(\omega)$ \' ce za gotovo sve $\omega$ oscilirati "bez pravila" od nule do jedinice $a = 0$ $A = 1$.
    Dakle o samom nizu $(X_n (\omega))$ u nezavisnom slu\v caju nemamo ni\v sta osobito za dodati.
    \v Sto se mo\v ze re\' ci o $\suma{n = 1}{\infty} X_n (\omega)$?
    Ili duga\v cije pitanje, postoji li nizovi konstanati $(a_n)$, $(b_n)$, takvi da $\frac{S_n(\omega) - b_n}{a_n}$ konvergira?
    Tipi\v cno $a_n \nearrow +\infty$, a $S_n := X_1 + \ldots + X_n$.
    U ovom slu\v caju teorija je bogatija i takve rezultate nazivamo jakim zakonima velikih brojeva.
\end{nap}

Osim "konvergencije po to\v ckama" $(g.s.)$ teorija vjerojatnosti bavi se i drugim tipovima konvergencije.

\begin{pr}  \label{pr:11.13}
    Neka je $\vjerojatnosniProstor = (\segment{0}{1}, \; \borel{\segment{0}{1}}, \; \lambda)$ i
    \begin{equation*}
        Z_{n, m} (\omega) :=
        \begin{cases}
            1, &\frac{m-1}{n} < \omega \leq \frac{m}{n}\\
            0, &\textnormal{ina\v ce}
        \end{cases}
        \quad
        \begin{matrix}
            n \in \nat,\\
            m \in \{ 1, \ldots, n \}.
        \end{matrix}
    \end{equation*}
    Poredamo $(Z_{n, m})$ u niz $(X_n)$ kao $(Z_{1,1}, \; Z_{2,1}, \; Z_{2, 2}, \; Z_{3,1}, \; Z_{3,2}, \; Z_{3,3}, \ldots)$.
    Uo\v cimo da za $\omega \in \lijInt{0}{1}$ $(X_k (\omega))$ ima beskona\v cno nula i beskona\v cno jedinica, pa niz $\niz{X_k (\omega)}{k \in \nat}$ ne konvergira.
    Posebno $X_k \cancel{\xrightarrow[n \to \infty]{g.s.}} 0$.
    S druge strane niz je "po vjerojatnosti" sve bli\v ze nuli, jer za svaki $\varepsilon > 0$ je $\vjeroj{|Z_{n, m}| > \varepsilon} = \frac{1}{n} \xrightarrow[n \to \infty]{} 0$. 
\end{pr}

\begin{defn}    \label{defn:11.13-1}
    Re\' ci \' cemo da niz $\niz{X_n}{n \in \nat}$ \emph{konvergira po vjerojatnosti} prema slu\v cajnoj varijabli $Y$, ako za svaki $\varepsilon > 0$ vrijedi
    \begin{equation*}
        \lim\limits_{n \to \infty} \vjeroj{|X_n - y| > \varepsilon} = 0.
    \end{equation*}
    To ozna\v cavamo sa $X_n \xrightarrow[n \to \infty]{\masP} Y$, ili $(\masP) \lim\limits_{n \to \infty} X_n = Y$. 
\end{defn}

\begin{zad} \label{zad:11.14}
    Ako je $X_n \xrightarrow[n \to \infty]{\masP} Y$ i $X_n \xrightarrow[n \to \infty]{\masP} Z$, tada $Y =  Z \; (g.s.)$
\end{zad}

Uo\v cimo da je
\begin{equation}    \label{jed:11.15}
    X_n \xrightarrow[n \to \infty]{\masP} Y \iff \vjeroj{|X_n - Y| > \frac{1}{k}} \to 0, \quad \forall k \in \nat.
\end{equation}
Koriste\' ci \eqref{jed:11.5} direktno dobivamo:

\begin{lm}  \label{lm:11.16}
    Sljede\' ce tvrdnje su ekvivalentne:
    \begin{enumerate}[label=(\roman*)]
        \item $X_n \xrightarrow[n \to \infty]{g.s.} Y$;
        \item $(\forall \varepsilon > 0)(\exists \delta > 0) (\exists N \urePar{\varepsilon}{\delta} \in \nat)\\
        (n \in \nat, \; n \geq N \urePar{\varepsilon}{\delta} \implies \masP \big( \presjek{j = n}{\infty} \{ |X_j - Y| \leq \varepsilon \} \big) \geq 1 - \delta)$;
        \item   \label{lm:11.16.3}
        $\sup\limits_{j \geq n} |X_j - Y| \xrightarrow[n  \to \infty]{\masP} 0$.
    \end{enumerate}
\end{lm}
 Iz leme \ref{lm:11.16} \ref{lm:11.16.3} direktno slijedi:

\begin{kor}    \label{kor:11.17}
    Konvergencija gotovo sigurno povla\v ci konvergenciju po vjerojatnosti k istom limesu, odnosno:
    \begin{equation*}
        X_n \xrightarrow[n \to \infty]{g.s.} Y \implies X_n \xrightarrow[n \to \infty]{\masP} Y.
    \end{equation*}
\end{kor}

Obrat op\' cenito ne vrijedi (vidi primjer \ref{pr:11.13}).
Ipak, veza izme\dj u ovih konvergencija je direktna.
 
\begin{lm} \label{lm:11.18}
    Niz $\niz{X_n}{n \in \nat}$ konvergira po vjerojatnosti prema nekoj slu\v cajnoj varijabli ako i samo ako za svaki $\varepsilon > 0$ vrijedi
    \begin{equation*}
        \lim\limits_{n \to \infty} \Bigg( \sup\limits_{
            \begin{smallmatrix}
               m \in \nat\\
               m > n
            \end{smallmatrix}
        }  \vjeroj{ |X_m - X_n| > \varepsilon }\Bigg) = 0.
    \end{equation*}
\end{lm}

\begin{proof}
    Nu\v znost se lako vidi, poka\v zimo dovoljnost.
    \begin{equation*}
        (\forall k \in \nat)(\exists m_k \in \nat)\Big(n > m \geq m_k \implies \masP \Big( |X_n - X_m|>\frac{1}{2^k} \Big) < \frac{1}{2^k} \Big).
    \end{equation*}
    Formirajmo podniz: $n_1 := m_1$, $n_{i + 1} := \max \urePar{n_i + 1}{m_{i + 1}}$, te $X_k' := X_{n_k}$.
    Tada je $\suma{k = 1}{\infty} \masP \Big( |X_{k + 1}' - X_k'| > \frac{1}{2^k} \big) \leq \suma{k = 1}{\infty} \frac{1}{2^k}$, pa prema Borel-Cantellijevoj lemi $(\exists A \in \famF) (\vjeroj{A} = 0)$ za $\omega \in A^c$ $(\exists k_0 (\omega) \in \nat) \Big(k \geq k_0 \implies |X_{k + 1}'(\omega) - X_k' (\omega)| \leq \frac{1}{2^k} \Big)$, to jest za svaki $\omega \in A^c$ vrijedi $\sup\limits_{m > n} |X_m' - X_n'| \leq \suma{k = n}{\infty} |X_{k + 1}' - X_k'| \leq \suma{k = n}{\infty} \frac{1}{2^k} = \frac{1}{2^{n - 1}} \xrightarrow[n \to \infty]{} 0$.
    Posebno za $\omega \in A^c$ je $(X_n' (\omega))$ Cauchyjev niz.
    Po jednad\v zbi \eqref{jed:11.4} postoji slu\v cajna varijabla $Y$ takva da $X_k' \xrightarrow[n \to \infty]{g.s.} Y$.
    Prema korolaru \ref{kor:11.17} vrijedi $X_{n_k} \xrightarrow[n \to \infty]{\masP} Y$.
    Budu\' ci je $\vjeroj{|X_n - Y| > \varepsilon} \leq \masP \Big(  |X_n - X_{n_k}| > \frac{\varepsilon}{2} \Big) + \masP \Big( |X_{n_k} - Y| > \frac{\varepsilon}{2} \Big)$, tvrdnja slijedi. 
\end{proof}

\begin{tm} \label{tm:11.19}
    Niz $(X_n)$ konvergira po vjerojatnosti prema $Y$ ako i samo ako za svaki podniz od $(X_n)$ postoji podniz podniza koji konvergira gotovo sigurno prema $Y$.
\end{tm}

\begin{proof}
    $X_n \xrightarrow[n \to \infty]{\masP} Y \implies Y_k = X_{n_k} \xrightarrow[n \to \infty]{\masP} Y$.
    U dokazu leme \ref{lm:11.18} konstruiran je podniz $Y_k'$ niza $(Y_k)$ koji konvergira gotovo sigurno prema nekoj $Z$.
    Zbog $Y_k' \xrightarrow[n \to \infty]{\masP} Y$ slijedi $Y=Z \; (g.s.)$.
    Obratno, kada $X_n \cancel{\xrightarrow[n \to \infty]{\masP}} Y$ postojao bi podniz $(X_{n_k})$ te $\varepsilon > 0$ i $\delta > 0$ takvi da je $\vjeroj{|X_{n_k} - Y| > \varepsilon} > \delta > 0$.
    Tada niti jedan podniz od $(X_{n_k})$ ne mo\v ze konvergirati prema $Y$ po vjerojatnosti, pa onda niti gotovo sigurno.
\end{proof}

\begin{zad} \label{zad:11.20}
    Niz $(X_n)$ je gotovo sigurno konvergentan ako i samo ako vrijedi
    \begin{equation*}
        \sup\limits_{
            \begin{smallmatrix}
                m \in \nat\\
                m > n
            \end{smallmatrix}
        } |X_m - X_n| \xrightarrow[n \to \infty]{\masP} 0.
    \end{equation*}
\end{zad}

\v Sto mo\v zemo re\' ci o nezavisnim nizovima?

\begin{kor} \label{kor:11.21}
    Neka je $(X_n)$ niz nezavisnih slu\v cajnih varijabli.
    Ako $X_n \xrightarrow[n \to \infty]{\masP} Y$, tada je $Y$ degenerirana.
\end{kor}

\begin{proof}
    Ako je $X_n \xrightarrow[n \to \infty]{\masP} Y$, tada po toeremu \ref{tm:11.19} postoji podniz $(X_{n_k})$ takav da $X_{n_k} \xrightarrow[n \to \infty]{g.s.} Y$.
    Uo\v cimo da je i $(X_{n_k})$ niz nezavisnih slu\v cajnih varijabli, pa je po teoremu \ref{tm:11.9} $Y$ degenerirana.
\end{proof}

\begin{nap} \label{nap:11.22}
    Zna\v ce li korolar \ref{kor:11.21} i teorem \ref{tm:11.9} da za niz nezavisnih slu\v cajnih varijabli $(X_n)$ vrijedi
    \begin{equation*}
        X_n \xrightarrow[n \to \infty]{g.s.} Y \iff X_n \xrightarrow[n \to \infty]{\masP} Y?
    \end{equation*}
    Odgovor je ne, kao \v sto pokazuje sljede\' ci primjer.
    Vrijedi li barem u slu\v caju $X_n \xrightarrow[n \to \infty]{\masP} b \in \real$, mora li biti $a = b$ ili $A = b$?
    Ponovo ne. U sljede\' cem primjeru $b = 0$, dok je $a = -\infty$, $A = +\infty$.
\end{nap}

\begin{pr}  \label{pr:11.23}
    Neka je $Z \sim N \urePar{0}{1}$. Za svaki $0<t<1$ postoji to\v cno jedan $\eta_t > 0$ takav da je $\vjeroj{|Z| > \eta_t} = t$.
    Uo\v cimo $t \searrow 0 \implies \eta_t \nearrow +\infty$.
    Za $n \in \nat$ definiramo $c_n := \frac{1}{\eta_\frac{1}{n}}\implies c_n > c_{n + 1} \searrow 0$.
    Neka je $(Z_n)$ $\iid$ niz takav da je $Z_n \distJed Z$ (u poglavlju \ref{zakoni_01} smo vidjeli da takav uvijek mo\v zemo konstruirati).
    Tada je $(X_n)$, pri \v cemu je $X_n := c_n \: Z_n$, niz slu\v cajnih varijabli.
    Budu\' ci da je
    \begin{equation*}
        \suma{n = 1}{\infty} \vjeroj{|X_n| > 1} = \suma{n = 1}{\infty} \masP \Big( |Z| > \frac{1}{\eta_\frac{1}{n}} \Big) = \suma{n = 1}{\infty} \frac{1}{n} = +\infty,
    \end{equation*}
    po teoremu \ref{tm:11.9} \ref{tm:11.9.4} slijedi $X_n \cancel{\xrightarrow[n \to \infty]{g.s.}} 0$.
    S druge strane, za $\varepsilon > 0$,
    \begin{equation*}
        \vjeroj{|X_n| > \varepsilon} = \masP \Big( |Z| > \varepsilon \: \eta_\frac{1}{n} \Big) \xrightarrow[n \to \infty]{} 0, \quad \eta_\frac{1}{n} \nearrow +\infty.
    \end{equation*}
    Stoga $X_n \xrightarrow[n \to \infty]{\masP} 0$.
\end{pr}

U $\iid$ slu\v caju pona\v sanje je isto kao i kod gotovo sigurne konvergencije.

\begin{kor} \label{kor:11.24}
    Ako je $(X_n) \; \iid$ niz tada $(X_n)$ konvergira po vjerojatnosti prema nekoj slu\v cajnoj varijabli ako i samo ako postoji $a \in \real$ takav da je $X_n = a \; (g.s.)$ $\forall n \in \nat$.
\end{kor}

\begin{proof}
    Dovoljnost je o\v cita, a nu\v znost slijedi iz teorema \ref{tm:11.19} i korolar \ref{kor:11.10}
\end{proof}

Pitanja iz napomene \ref{nap:11.12} postavljamo i za $(\masP)$-konvergenciju.
Teoremi o $(\masP)$-konvergenciju $\frac{S_n - b_n}{a_n}$ zovu se \emph{slabi zakoni velikih brojeva}.

    %%%%%%%%%%%%%%%%%%%%%%%%%%%%%%%%%%%%
    %%  slabi zakoni velikih brojeva  %%
    %%%%%%%%%%%%%%%%%%%%%%%%%%%%%%%%%%%%

    % p12 - slabi zakoni velikih brojeva

\chapter{Slabi zakoni velikih brojeva}

U ovom poglavlju $\niz{X_n}{n \in \nat}$ je niz slu\v cajnih varijabli na vjerojatnosnom prostoru $\vjerojatnosniProstor$ i $S_n := X_1 + \ldots + X_n$, za svaki $n \in \nat$.

Prvo promatramo \v sto se doga\dj aako postoje drugi momenti (vidi poglavlje \ref{dist_sl_elem}).
Osnovna tehnika u tom slu\v caju je poznata \emph{\v Cebi\v sevljeva nejednakost}:

\begin{prop}    \label{prop:12.1}
    \begin{enumerate}[label=(\roman*)]
        \item   \label{prop:12.1.1}
        Ako je $X \geq 0$ slu\v cajna varijabla takva da je $0 < \masE X < +\infty$, tada, za svaki $r > 0$ vrijedi
        \begin{equation*}
            \vjeroj{X > r \: \masE X} \leq \frac{1}{r}.
        \end{equation*}
        \item   \label{prop:12.1.2}
        Ako je $X$ slu\v cajna varijabla i $\masE [X^2] < +\infty$, tada, za svaki $\varepsilon > 0$ vrijedi:
        \begin{equation*}
            \vjeroj{|X - \masE X| > \varepsilon} \leq \frac{\Var X}{\varepsilon^2}.
        \end{equation*}
    \end{enumerate}
\end{prop}

\begin{proof}
    \begin{enumerate}[label=(\roman*)]
        \item Bez smanjenja op\' cenitosti mo\v zemo pretpostaviti da je $\masE X = 1$.
        Uo\v cimo da je $r \cdot \karaktFja_{\{X > r\}} \leq X$.
        \item Iz \ref{prop:12.1.1} za $Y:= |X - \masE X|^2$ i za $r:= \frac{\varepsilon^2}{\Var X}$, jer vrijedi:
        \begin{equation*}
            \begin{aligned}
                \vjeroj{|X - \masE X| > \varepsilon} &= \vjeroj{Y > \varepsilon^2} = \vjeroj{Y > r \: \masE Y}\\
                & \leq \frac{1}{r} = \frac{\Var X}{\varepsilon^2}.
            \end{aligned}
        \end{equation*}
    \end{enumerate}
\end{proof}

Podsjetimo se da ako $X$, $Y$ imaju kona\v cne druge momente, tada imamo sljede\' cu defninciju.

\begin{defn}    \label{defn:12.1-1}
    Neka su $X, \; Y \in L^2(\masP)$ slu\v cajne varijable, tada definiramo \emph{kovarijancu} od $X$ i $Y$ kao:
    \begin{equation}    \label{jed:12.2}
        \Cov \urePar{X}{Y} := \frac{\masE [X \cdot Y] - \masE X \cdot \masE Y}{\sqrt{\Var X} \cdot \sqrt{\Var Y}}.
    \end{equation}
    Ako je $\Cov \urePar{X}{Y} = 0$, ka\v zemo da su $X$ i $Y$ \emph{nekorelirane}.
\end{defn}

\begin{zad} \label{zad:12.3}
    Doka\v zi da vrijedi:
    \begin{equation*}
        X, \; Y \in L^2(\masP), \textnormal{ nezavisne} \implies \Cov \urePar{X}{Y} = 0.
    \end{equation*}
    Poka\v zi da obrat op\' cenito ne vrijedi.
\end{zad}

\begin{zad} \label{zad:12.4}
    Neka su $X_1, \ldots, X_n \in L^2(\masP)$ i neka vrijedi $\Cov \urePar{X_i}{X_j} = 0$, za svaki $i \neq j$, tada vrijedi:
    \begin{equation*}
        \Var (X_1 + \ldots + X_n) = \suma{k = 1}{n} \Var X_k.
    \end{equation*}
\end{zad}

\begin{tm}  \label{tm:12.5}
    Ako je $X_n \in L^2 (\masP)$, za svaki $n \in \nat$, i ako vrijedi $\Cov \urePar{X_n}{X_m} = 0$, za svaki $m, \; n \in \nat$, te ako postoji $0 < c < + \infty$ takva da je $\Var (X_n) \leq c$, za svaki $n \in \nat$, tada
    \begin{equation*}
        \frac{S_n - \masE S_n}{n} \xrightarrow[n \to \infty]{\masP} 0.
    \end{equation*}
\end{tm}

\begin{proof}
    Neka je $Z_n := \frac{S_n}{n}$.
    Tada je $\frac{S_n - \masE S_n}{n} = Z_n - \masE Z_n$ i vrijedi $\Var Z_n = \frac{1}{n^2} \suma{k = 1}{n} \Var X_k \leq \frac{n \: c}{n^2} = \frac{C}{n}$.
    Prema propoziciji \ref{prop:12.1} \ref{prop:12.1.2} imamo
    \begin{equation*}
        \begin{aligned}
            \vjeroj{|Z_n - \masE Z_n| > \varepsilon} & \leq \frac{\Var Z_n}{\varepsilon^2} \leq \frac{c}{\varepsilon^2} \cdot \frac{1}{n}\\
            & \xrightarrow[n \to \infty]{} 0.
        \end{aligned}
    \end{equation*}
\end{proof}

\begin{zad} \label{zad:12.6}
    Neka je $\niz{X_n}{n \in \nat}$ niz nezavisnih slu\v cajnih varijabli i $X_n \in L^2(\masP)$, za svaki $n \in \nat$.
    \begin{enumerate}[label=(\alph*)]
        \item Ako je $\lim\limits_{n \to \infty} \suma{k = 1}{n} \Var X_k = 0$, tada vrijedi
        \begin{equation*}
            \frac{S_n - \masE S_n}{n} \xrightarrow[n \to \infty]{\masP} 0.
        \end{equation*}
        \item Ako je $\masE X_n = \mu$ i ako je $\Var X_n = \sigma^2$, za svaki $n \in \nat$, tada je
        \begin{equation*}
            \frac{S_n}{n} \xrightarrow[n \to \infty]{\masP} \mu.
        \end{equation*}
        \item Ako je $Y_n \sim B \urePar{n}{p}$, za svaki $n \in \nat$, tada je
        \begin{equation*}
            \frac{Y_n}{n} \xrightarrow[n \to \infty]{\masP} p.
        \end{equation*}
    \end{enumerate}
\end{zad}

Vidimo da iz teorema \ref{tm:12.5} slijedi:

\begin{kor}[\v Cebi\v sevljev slabi zakon] \label{kor:12.6}
    Ako su $\niz{X_n}{n \in \nat}$ nezavisne slu\v cajne varijable u $L^2(\masP)$ i ako postoji $0 < c < +\infty$, takav da je $\Var X_n \leq c$, za svaki $n qin \nat$, tada je
    \begin{equation*}
        \frac{1}{n} (S_n - \masE S_n) \xrightarrow[n \to \infty]{\masP} 0.
    \end{equation*}
\end{kor}

Mo\v zemo li u nezavisnom slu\v caju ispustiti pretpostavku o postojanju 2. momenta?
Pogledat \' cemo $\iid$ slu\v caj, ali nam prvo trebaju neki pomo\' cni rezultati.
Biti \' ce nam korisna sljede\' ca definicija.

\begin{defn}    \label{defn:12.6-1}
    Neka su $(a_n)$, $(b_n)$ nizovi realnih brojeva, ka\v zemo da je niz $a_n$ \emph{malo o} od niza $b_n$, u oznaci $a_n = o(b_n)$, ako vrijedi
    \begin{equation}    \label{jed:12.7}
        \lim\limits_{n \to \infty} \frac{a_n}{b_n} = 0.
    \end{equation}
\end{defn}

Posebno, vidimo da je
\begin{equation*}
    a_n = o(1) \iff \lim\limits_{n \to \infty} a_n = 0.
\end{equation*}

\begin{defn}    \label{defn:12.7-1}
    Za slu\v cajnu varijablu $X$ realni broj $m(X)$ zovemo \emph{medijanom} ako vrijedi
    \begin{equation*}
        \vjeroj{X \leq m(X)} \geq \frac{1}{2} \quad \textnormal{i} \quad \vjeroj{X \geq m(X)} \geq \frac{1}{2}.
    \end{equation*}
\end{defn}

\begin{lm}[L\' evyjeva nejednakost]  \label{lm:12.8}
    Ako su $\niz{X_n}{n \in \nat}$ nezavisne slu\v cajne varijable, tada, za svaki $\varepsilon > 0$ vrijedi:
    \begin{equation*}
        \begin{gathered}
            \vjeroj{\max\limits_{1 \leq j \leq n} [S_j - m(S_j - S_n)] \geq \varepsilon } \leq 2 \: \vjeroj{S_n \geq \varepsilon}\\
            \vjeroj{\max\limits_{1 \leq j \leq n} |S_j - m(S_j - S_n)| \geq \varepsilon } \leq 2 \: \vjeroj{|S_n| \geq \varepsilon}.
        \end{gathered}
    \end{equation*}
\end{lm}

\begin{proof}
    Koriste\' ci prvu nejednakost na $(X_n)$ i na $(-X_n)$, te koriste\' ci $m(-X) = -m(X)$, dobivamo drugu nejednakost.
    Doka\v zimo prvu nejednakost.
    Stavimo
    \begin{equation*}
        T:= \min ( \skup{j \in \{1, \ldots, n\}}{S_j - m(S_j - S_n) \geq \varepsilon} ),
    \end{equation*}
    ako takav minimum postoji, odnosno $T:= n + 1$, ako ne postoji.
    Nadalje, stavimo
    \begin{equation*}
        B_j := \{ m(S_j - S_n) \geq S_j - S_n \}, \quad 1 \leq j \leq n.
    \end{equation*}
    Stoga je $\vjeroj{B_j} \geq \frac{1}{2}$, za svaki $1 \leq j \leq n$.
    Uo\v cimo da je $\{ T = j \} \in \sigAlg{X_1, \ldots X_j}$, te $B_j \in \sigAlg{X_{j + 1}, \ldots, X_n}$, \v sto daje da su $\{T = j\}$ i $B_j$ nezavisni.
    Budu\' ci da su $\{T = j\}$ disjunktni, iz $\{ S_n \geq \varepsilon \} \supseteq \unija{j = 1}{n} (B_j \cap \{ T = j \})$, dobivamo
    \begin{equation*}
        \begin{aligned}
            \vjeroj{S_n \geq \varepsilon} &\geq \suma{j = 1}{n} \vjeroj{B_j \cap \{ T = j \}} = \suma{j = 1}{n} \vjeroj{B_j} \cdot \vjeroj{T = j}\\
            &\geq \frac{1}{2} \suma{j = 1}{n} \vjeroj{T = j} = \frac{1}{2} \vjeroj{1 \leq T \leq n}.
        \end{aligned}
    \end{equation*}
\end{proof}

Sljede\' ci teorem daju potpuni odgovor u slu\v caju $\iid$ slu\v cajnog niza.

\begin{tm}[W. Feller]   \label{tm:12.9}
    Neka je $\niz{X_n}{n \in \nat}$ $\iid$ niz.
    Tada postoji niz brojeva $\niz{b_n}{n \in \nat}$, takav da vrijedi
    \begin{equation*}
        \frac{S_n}{n} - b_n \xrightarrow[n \to \infty]{\masP} 0 \iff \lim\limits_{n \to \infty} n \: \vjeroj{|X_1| > n} = 0.
    \end{equation*}
    U tom slu\v caju je
    \begin{equation*}
        \lim\limits_{n \to \infty} (b_n - \masE [ X_1 \cdot \karaktFja_{\{ |X_1| \leq n \}}]) = 0.
    \end{equation*}
\end{tm}

\begin{zad} \label{zad:12.10}
    Ako za slu\v cajne varijable $\niz{Z_n}{n \in \nat}$ postoji niz brojeva $0 < b_n \nearrow + \infty$ takav da vrijedi
    \begin{equation*}
        \frac{Z_n}{b_n} \xrightarrow[n \to \infty]{\masP} 0,
    \end{equation*}
    tada vrijedi
    \begin{equation*}
        \max\limits_{1 \leq j \leq n} |m(Z_j - Z_n)| = o(b_n).
    \end{equation*}
\end{zad}

\begin{proof}{(teorema \ref{tm:12.9})}
    \begin{itemize}
        \item[$\implies$] Pretpostavimo da postoji niz $(c_n)$ takav da vrijedi $\frac{S_n - c_n}{n} \xrightarrow[n \to \infty]{\masP} 0$.
        Stavimo $c_0 = 0$ i stavimo $d_n := c_n - c_{n - 1}, \; n \in \nat$.
        Tada je
        \begin{equation*}
            \frac{X_n -d_n}{n} = \frac{S_n - c_n}{n} - \frac{n - 1}{n} \cdot \Big( \frac{S_{n - 1} - c_{n - 1}}{n - 1} \Big) \xrightarrow[n \to \infty]{\masP} 0,
        \end{equation*}
        pa je zbog $\iid$
        \begin{equation*}
            \frac{X_1 - d_n}{n} \xrightarrow[n \to \infty]{\masP} 0 \implies d_n = o(n).
        \end{equation*}
        Po L\' evyjevoj nejednakosti (lema \ref{lm:12.8}) slijedi da za svaki $\varepsilon > 0$
        \begin{equation*}
            \masP \Big( \max\limits_{1 \leq j \leq n} | S_j - c_j -m( S_j - c_j - S_n + c_n) | \geq \frac{n \: \varepsilon}{2}  \Big)\leq 2 \: \masP \Big( |S_n - c_n| \geq \frac{n \: \varepsilon}{2} \Big) \xrightarrow[n \to \infty]{} 0.
        \end{equation*}
    \end{itemize}
\end{proof}

    %%%%%%%%%%%%%%%%%%%%%%%%%%%%
    %%  konvergencija redova  %%
    %%%%%%%%%%%%%%%%%%%%%%%%%%%%

    % poglavlje 13 - konvergencija redova

\chapter{Konvergencija redova}

U ovom poglavlju $\niz{X_n}{n \in \nat}$ je niz nezavisnih slu\v cajnih varijabli na vjerojatnosnom prostoru $\vjerojatnosniProstor$ i
\begin{equation*}
    S_n := X_1 + \ldots + X_n, \quad n \in \nat.
\end{equation*}
\v Sto mo\v zemo re\' ci o konvergenciji reda $\suma{n \in \nat}{} X_n$ gotovo sigurno i po vjerojatnosti?
Iz Kolmogorovljevog zakona 0-1 slijedi (teorem \ref{tm:9.9}):
\begin{equation}    \label{jed:13.1}
    \suma{n = 1}{\infty} X_n \quad \textnormal{ konvergira } (g.s.) \textnormal{ ili divergira } (g.s.)
\end{equation}
Nas zanima slu\v caj konvergencije.
Ako $\suma{n = 1}{\infty} X_n$ konvergira $(g.s.)$, tada $X_n \xrightarrow[n \to \infty]{g.s.} 0$, pa zbog nezavisnosti i po teoremu \ref{tm:11.9} slijedi:
\begin{equation}    \label{jed:13.2}
    \suma{n = 1}{\infty} X_n \; \textnormal{ konvergira } (g.s.) \implies \suma{n = 1}{\infty} \vjeroj{|X_n| > 1} < +\infty.
\end{equation}

Odavde slijedi da $\iid$ slu\v caj nije previ\v se zanimljiv.
Preciznije:
\begin{equation}    \label{jed:13.3}
    (X_n) \; \iid \quad \suma{n = 1}{\infty} X_n \; \textnormal{ konvergira } (g.s.) \implies X_n = 0 \; (g.s.), \quad \forall n \in \nat. 
\end{equation}

\begin{nap} \label{nap:13.4}
    Ako izbacimo jednaku distribuiranost, a zadr\v zimo nezavisnost, mo\v zemo dobiti bilo kakav limes reda.
    Na primjer, neka je $X_1 = Y$, pri \v cemu je $Y$ proizvoljna, zadana slu\v cajna varijabla i $X_n \equiv 0$, za $n \geq 2$.
    Tada je $\niz{X_n}{n \in \nat}$ nezavisan niz i
    \begin{equation*}
        \suma{n = 1}{\infty} X_n = Y.
    \end{equation*}
\end{nap}

\begin{nap} \label{nap:13.5}
    Ideja "rezanja" slu\v cajnih varijabli mo\v ze pomo\' ci da se ostvari nu\v zan uvijet konvergencije iz \eqref{jed:13.2}.
    Uz to je vezan va\v zan pojam ekvivalentnih nizova.
\end{nap}

\begin{defn}    \label{defn:13.6}
    Ka\v zemo da su nizovi slu\v cajnih varijabli $\niz{X_n}{n \in \nat}$ i $\niz{Y_n}{n \in \nat}$ na vjerojatnosnom prostoru $\vjerojatnosniProstor$ \emph{ekvivalentni} ako vrijedi:
    \begin{equation*}
        \suma{n = 1}{\infty} \vjeroj{X_n \neq Y_n} < +\infty.
    \end{equation*}
\end{defn}

Ako su nizovi ekvivalentni, onda iz Borel-Cantellijeve leme (lema \ref{lm:9.2}) slijedi
\begin{equation*}
    \vjeroj{X_n \neq Y_n \; \io} < +\infty,
\end{equation*}
\v sto daje da za ekvivalentne nizove $(X_n)$ i $(Y_n)$ vrijedi:
\begin{equation}    \label{jed:13.7}
    \suma{n = 1}{\infty} X_n \quad \textnormal{ konvergira } (g.s.) \quad \iff \quad \suma{n = 1}{\infty} Y_n \quad \textnormal{ konvergira } (g.s).
\end{equation}

Nadalje, uo\v cimo da vrijedi:
\begin{equation}    \label{jed:13.8}
    \begin{matrix}
        (X_n) \textnormal{ i } \big(X_n \cdot \karaktFja_{\{ |X_n| \leq 1 \}}\big)\\
        \textnormal{su ekvivalentni}
    \end{matrix}
    \quad \iff \quad
    \suma{n = 1}{\infty} \vjeroj{|X_n|>1} < +\infty.
\end{equation}
Dakle, za konvergenciju $(g.s.)$ osnovno pitanje postaje koji su nu\v zni i dovoljni uvijeti da $\suma{n = 1}{\infty} X_n$ konvergira $(g.s.)$?
\v Sto mo\v zemo re\' ci o konvergenciji po vjerojatnosti?
Sljede\' ci teorem pokazuje da su kod redova nezavisnih slu\v cajnih varijabli ove konvergencije jednake.

\begin{tm}[P. L\' evy]  \label{tm:13.9}
    Red $\suma{n = 1}{\infty} X_n$ konvergira gotovo sigurno ako i samo ako konverigra po vjerojatnosti.
\end{tm}

\begin{proof}
    O\v cito je dovoljno dokazati da ako $(S_n)$ konvergira po vjerojatnosti, tada konvergira i gotovo sigurno.
    Uvedimo oznaku
    \begin{equation*}
        S_{h, n} := S_n - S_h, \quad \forall h, n \in \nat.
    \end{equation*}
    Koriste\' ci lemu \ref{lm:11.18} dobivamo da za svaki $0 < \varepsilon < \frac{1}{4}$, postoji $h_0 \in \nat$, takav da vrijedi
    \begin{equation*}
        n, h \in \nat, \quad n > h \geq h_0 \implies \vjeroj{|S_{h, n}| Y \varepsilon} < \varepsilon.
    \end{equation*}
    Odavde slijedi $|m(S_{h, n})| \leq \varepsilon$, pa koriste\' ci L\' evyjeve nejednakosti (lema \ref{lm:12.8}) i nezavisnost od $(X_n)$ dobivamo
    \begin{equation*}
        % možda je m(S_{h, k})
        \begin{aligned}
            \masP \Big( \max\limits_{h < n \leq k} |S_{h, n}| > 2 \varepsilon \Big) &= \masP \Big( \max\limits_{h < n \leq k} |S_{h, n}| > 2 \varepsilon, \; \max\limits_{h < n \leq k} |m(S_{k, h})| \leq \varepsilon \Big)\\
            &\leq \masP \Big( \max\limits_{h < n \leq k} |S_{h, n} - m(|S_{h, n} - S_{h, k}|)| > \varepsilon \Big)\\
            &\leq 2 \: \vjeroj{|S_{h, n}| > 2 \varepsilon}\\
            &< 2 \: \varepsilon. 
        \end{aligned}
    \end{equation*}
    Neka je $h_0$ fiksan i neka $h \nearrow +\infty$.
    Sada dobivamo
    \begin{equation*}
        \masP \Big( \sup\limits_{n > h} |S_{h, n}| > 2 \: \varepsilon \Big) \leq 2 \: \varepsilon.
    \end{equation*}
    Pa tvrdnja slijedi iz zadatka \ref{zad:11.20}.
\end{proof}

Sljede\' ci tehni\v cki rezultat je izrazito koristan.

\begin{lm}[Abelova lema]    \label{lm:13.10}
    Neka su $\niz{a_n}{n \in \nat}$ i $\niz{b_n}{n \in \nat}$ nizovi realnih brojeva.
    Za $n \in \nat_0$, neka su
    \begin{equation*}
        \begin{aligned}
            A_n &:= \suma{j = 0}{n} a_j \quad (n + 1) \textnormal{-va parcijana suma}\\
            A_n^* &:= \suma{j = n + 1}{\infty} a_j \quad \textnormal{ostatak reda, ako red konvergira}.
        \end{aligned}
    \end{equation*}
    Tada vrijedi:
    \begin{enumerate}[label=(\roman*)]
        \item Za svaki $n \in \nat$, vrijedi
        \begin{equation*}
            \suma{j = 1}{n} a_j b_j = A_n b_n - A_0 b_1 - \suma{j = 1}{n - 1} A_j (B_{j + 1} - b_j).
        \end{equation*}
        \item ako $\suma{n}{} a_n$ konvergira, tada za svaki $n \in \nat$ vrijedi:
        \begin{equation*}
            \suma{j = 1}{n} a_j b_j = A_0^* b_j - A_n^* b_n + \suma{j = 1}{n - 1} A_j^* (b_{j + 1} - b_j). 
        \end{equation*}
    \end{enumerate}
\end{lm}

\begin{proof}
    \begin{equation*}
        \begin{aligned}
            \suma{j = 1}{n} a_j b_j &= \suma{j = 1}{n} (A_j - A_{j - 1}) b_j = \suma{j = 1}{n} A_j b_j - \suma{k = 1}{n - 1} A_k b_{k + 1}\\
            &= A_n b_n - A_0 b_1 - \suma{j = 1}{n - 1} (b_{j + 1} - b_j).
        \end{aligned}
    \end{equation*}
    Ako je $\suma{j = 0}{\infty} a_j = A \in \real$, tada je za svaki $n \in \nat$ $A = A_n + A_n^*$. Slijedi:
    \begin{equation*}
        \begin{aligned}
            \suma{j = 1}{n} a_j b_j &= (A - A_n^*)b_n - (A - A_0^*) b_1 - \suma{j = 1}{n - 1} (A - A_j^*) (b_{j + 1} - b_j)\\
            &= A_0^* b_1 - A_n^* b_n + \suma{j = 1}{n - 1} A_j^* (b_{j + 1} - b_j) + A b_n - A b_1 - A \cdot \underbrace{\suma{j = 1}{n - 1} (b_{j + 1} - b_j)}_{= b_n - b_1}.
        \end{aligned}
    \end{equation*}
\end{proof}

\begin{lm}[Kroneckerova lema]  \label{lm:13.10-1}
    Ako je $\suma{j = 0}{\infty} a_j = A \in \real$ i $0 < b_n \nearrow +\infty$, tada je
    \begin{equation*}
        \suma{j = 1}{n} a_j b_j = o(b_n).
    \end{equation*}
\end{lm}

\begin{zad} \label{zad:13.11}
    Doka\v zi lemu \ref{lm:13.10-1}.
\end{zad}

Po\v cnimo s pretpostavkom o drugim momentima.

\begin{tm}  \label{tm:13.12}
    Ako je $X_n \in L^2 (\masP)$, za svaki $n \in \nat$ i $\suma{n = 1}{\infty} \Var (X_n) < +\infty$, tada
    \begin{equation*}
        \suma{n = 1}{\infty} (X_n - \masE X_n) \quad \textnormal{ konvergira } (g.s.).
    \end{equation*}
\end{tm}

\begin{proof}
    Bez smanjenja op\' cenitosti mo\v zemo pretpostaviti da je $\masE X_n = 0$, $\forall n \in \nat$, ina\' ce gledamo varijable $Y_n = X_n - \masE X_n$.
    Po teoremu \ref{tm:13.9} dovoljno je dokazati da $(S_n)$ konvergira po vjerojatnosti.
    Ponovo koristimo lemu \ref{lm:11.18}.
    Po \v Cebi\v sevljevoj nejednakosti slijedi
    \begin{equation*}
        \begin{aligned}
            \vjeroj{|S_m - S_n| > \varepsilon} &\leq \frac{1}{\varepsilon^2} \Var (S_m - S_n) = \frac{1}{\varepsilon^2} \Var \Big( \suma{j = n + 1}{m} X_j \Big) = (\textnormal{nezavisnost})\\
            &= \frac{1}{\varepsilon^2} \suma{j = n + 1}{m} \Var (X_j) \leq \frac{1}{\varepsilon^2} \suma{j = n + 1}{\infty} \Var (X_j).
        \end{aligned}
    \end{equation*}
    A sada zbog $\suma{n = 1}{\infty} X_n < +\infty$ slijedi:
    \begin{equation*}
        \sup\limits_{m > n} \vjeroj{|S_m - S_n| > \varepsilon} \leq \frac{1}{\varepsilon^2} \suma{j = n + 1}{\infty} \Var (X_j) \xrightarrow[n \to \infty]{} 0.
    \end{equation*}
\end{proof}

\begin{prop}    \label{prop:13.13}
    Ako je $\masE \Big[ \sup\limits_{n \in \nat} X_n^2 \Big] < + \infty$ (posebno to zna\v ci da je $X_n \in L^2 (\masP), \; \forall n$) i $\masE X_n = 0$, za svaki $n \in \nat$ i $\masP \Big( \sup\limits_{n \in \nat} |S_n| < + \infty \Big) > 0$, tada
    \begin{equation*}
        \begin{aligned}
            & \suma{n = 1}{\infty} X_n \quad \textnormal{ konvergira } (g.s.)\\
            & \suma{n = 1}{\infty} \masE [ X_n^2 ] < + \infty
        \end{aligned}
    \end{equation*}
\end{prop}

\begin{proof}
    Prema teoremu \ref{tm:13.12} dovoljno je pokazati da je $\suma{n = 1}{\infty} \masE [ X_n^2 ] < +\infty$.
    Stavimo
    \begin{equation*}
        V := \sup\limits_{n \in \nat} X_n^2.
    \end{equation*}
    Vidimo da je $V \geq 0$ pro\v sirena slu\v cajna varijabla i prema pretpostavci je $\masE V < +\infty$, pa je $V$ slu\v cajna varijabla.
    Zbog neprekidnosti vjerojatnosti s obzirom na rastu\' ce nizove doga\dj aja, slijedi da postoji $K > 0$ dovoljno velik, takav da vrijedi $\masP \Big( \sup\limits_{n \in \nat} |S_n| < K \Big) > 0$.
    Definiramo
    \begin{equation*}
        \begin{gathered}
            T: \Omega \to \nat_0 \cup \{+\infty\}\\
            T:= \inf \bigSkup{n \in \nat}{|S_n| \geq K}.
        \end{gathered}
    \end{equation*}
    Dakle $T$ je pro\v sirena slu\v cajna varijabla.
    Iz izbora $K$ slijedi $\masP (T = +\infty) > 0$.
    Nadalje, primjetimo da je
    \begin{equation*}
        \{T \geq n\} = \{ |S_1| < K \} \cap \ldots \cap \{|S_{n-1}| < K\} \in \sigAlg{X_1, \ldots, X_{n-1}}.
    \end{equation*}
    Dakle $X_n$ i $\karaktFja_{\{T \geq n\}}$ su nezavisne slu\v cajne varijable.
    Stavimo
    \begin{equation*}
        U_n := \suma{j = 1}{n} X_j \karaktFja_{\{T \geq j\}}.
    \end{equation*}
    Vidimo da je $U_n$ $\sigAlg{X_1, \ldots, X_n}$-izmjeriva slu\v cajna varijabla i vrijedi
    \begin{equation*}
        U_n = S_{\min (T, n)} \quad (g.s.).
    \end{equation*}
    Odavde slijedi
    \begin{equation*}
        \begin{aligned}
            U_n^2 &= \big| S_{\min (T - 1, n - 1)} + X_{\min (T, n)} \big|^2\\
            &\leq 2 (K^2 + V),
        \end{aligned}
    \end{equation*}
    pa vidimo
    \begin{equation*}
        \masE (U_n^2) \leq 2 (K^2 + \masE V) < +\infty.
    \end{equation*}
    Stavimo li $U_0 := 0$, dobivamo da za svaki $j \in \nat$ vrijedi
    \begin{equation*}
        U_j^2 = U_{j - 1}^2 + 2 U_{j - 1} X_j \karaktFja_{\{ T \geq j \}} + X_j^2 \karaktFja_{\{ T \geq j \}}.
    \end{equation*}
    Uo\v cimo da je $\masE [ U_{j - 1}^2 ] < + \infty$, stoga se smije oduzeti,
    \begin{equation*}
        \begin{aligned}
            \masE [U_j^2] - \masE [U_{j - 1}^2] &= (\textnormal{nezavisnost})\\
            &= \masE [2 U_{j - 1} \karaktFja_{\{ T \geq j \}}] \cdot \underbrace{\masE X_j}_{= 0} + \masE [X_j^2] \masP (T \geq j)\\
            &= \masE [X_j^2] \masP (T \geq j).
        \end{aligned}
    \end{equation*}
    Dakle za svaki $n \in \nat$
    \begin{equation*}
        \begin{gathered}
            \begin{aligned}
                \underbrace{\masP ( T = +\infty )}_{> 0} \cdot \suma{j = 1}{n} \masE [X_j^2] &\leq \suma{j = 1}{n} \masP (T \geq j) \masE [X_j^2] = \suma{j = 1}{n} (\masE [U_j^2] - \masE [U_{j - 1}^2]) = \masE [U_n^2]\\
                &\leq 2(K^2 + \masE V ) < +\infty \implies 
            \end{aligned}\\
            (\forall n \in \nat) \quad \suma{j = 1}{n} \masE [X_j^2] \leq \frac{2 (K^2 + \masE V)}{\masP (T = +\infty)} < + \infty \implies\\
            \suma{j = 1}{\infty} \masE [X_j^2] < +\infty.
        \end{gathered}
    \end{equation*}
\end{proof}

\begin{prop}    \label{prop:13.14}
    Ako je $\masE \Big[ \sum\limits_{n \in \nat} |X_n| \Big] < +\infty$ i $\suma{n = 1}{\infty} X_n$ konvergira $(g.s.)$, tada i $\suma{n = 1}{\infty} \masE X_n$ konvergira.
\end{prop}

\begin{proof}
    Budu\' ci da je $\masE \Big[\sup\limits_{n \in \nat} |X_n| \Big] < +\infty$, vrijedi $X_n \in L^1 (\masP)$, to jest $\masE X_n \in \real$, $n \in \nat$.
    Budu\' ci da $(S_n)$ konvergira $(g.s.)$ ispunjen je uvjet
    \begin{equation*}
        \masP \Big( \sup\limits_{n \in \nat} |S_n| < +\infty \Big) > 0
    \end{equation*}
    iz propozicije \ref{prop:13.13}.
    Provedemo li to\v cno postupke dokaza iz propozicije \ref{prop:13.13}, te na isti na\v cin definiramo sve slu\v cajne varijable; $K, t, U_n, \; n \in \nat$, tada za svaki $n \in \nat$ imamo
    \begin{equation*}
        |U_n| = |S_{\min (T - 1, n - 1)} + X_{\min (T, n)}| \leq K + \sup\limits_{n \in \nat} |X_n| =: H,
    \end{equation*}
    \v sto daje $H \geq 0$, slu\v cajna varijabla i $\masE H < + \infty$.
    Dakle niz $(U_n)$ je dominiran funkcijom $H \in L^1 (\masP)$.

    Sada definiramo
    \begin{equation*}
        \begin{aligned}
            S_{\infty} &:=\lim\limits_{n \to \infty} S_n = \suma{n = 1}{\infty} X_n\\
            S_T (\omega) &:= S_{T (\omega)} (\omega),
        \end{aligned}
    \end{equation*}
    uo\v cimo na $\{T = +\infty\}$ imamo $S_T = S_\infty$.
    Sada za gotovo svaki $\omega \in \Omega$ vrijedi
    \begin{equation*}
        U_n (\omega) = S_{\min (T(\omega), n)} (\omega) \xrightarrow[n \to \infty]{} S_T (\omega).
    \end{equation*}
    Po teoremu o dominiranoj konvergenciji vrijedi:
    \begin{equation*}
        \masE [U_n] \xrightarrow[n \to \infty]{} \masE [S_T],
    \end{equation*}
    odakle slijedi
    \begin{equation*}
        | \masE [S_T]| \leq \masE H < +\infty.
    \end{equation*}
    Kao i u propoziciji \ref{prop:13.13} dobijemo
    \begin{equation*}
        \masE [U_n] = \suma{j = 1}{n} \vjeroj{T \geq j} \masE [X_j].
    \end{equation*}
    Primjetimo da ovdje $\masE [X_j]$ nisu nu\v zno pozitivni, stoga ne mo\v zemo iskoristiti "trik" djeljenja sa $\vjeroj{T = +\infty}$.
    Me\dj utim budu\' ci vrijedi
    \begin{equation*}
        \masP (T \geq n) \geq \vjeroj{T = \infty} > 0,
    \end{equation*}
    pa za svaki $n \in \nat$ dobijemo
    \begin{equation*}
        \masE [ X_n ] = \frac{\masE [ U_n ] - \masE [U_{n - 1}]}{\vjeroj{T \geq n}}.
    \end{equation*}
    Definiramo nizove $\niz{a_n}{n \in \nat}$ te $\niz{b_n}{n \in \nat}$ sa
    \begin{equation*}
        \begin{gathered}
            a_0 := 0, \; a_j := \masE [U_j] - \masE [U_{j - 1}], \quad j \in \nat\\
            b_j := \frac{1}{\vjeroj{T \geq j}}, \quad j \in \nat.
        \end{gathered}
    \end{equation*}
    Uo\v cimo da je $b_{j + 1} \geq b_{j} > 0$ te $b_j \to \frac{1}{\vjeroj{T = \infty}}$, dok $\suma{j = 0}{\infty}$ konverigra prema $\masE [S_T]$.
    Po Abelovoj lemi (lema \ref{lm:13.10})
    \begin{equation*}
        \begin{aligned}
            \suma{j = 1}{\infty} \masE [X_j] &= \suma{j = 1}{n} a_j b_j = \frac{\masE [U_n]}{\vjeroj{T \geq n}} - \suma{j = 1}{n - 1} \Big( \frac{1}{\vjeroj{T > 1}} - \frac{1}{\vjeroj{T \geq j}} \Big) \cdot \masE [U_j] \to \suma{j = 1}{\infty} a_j b_j\\
            &= A_0^* b_1 + \suma{j = 1}{\infty} A_j^* (b_{j + 1} - b_j)
        \end{aligned}
    \end{equation*}
    \v sto je kona\v cno.
\end{proof}

Direktno iz propozicije \ref{prop:13.13} i propozicije \ref{prop:13.14} slijedi.

\begin{kor} \label{kor:13.15}
    Ako $\suma{n = 1}{\infty} X_n$ konvergiraju $(g.s.)$ i postoji $0 < c < +\infty$ takav da je
    \begin{equation*}
        \vjeroj{|X_n| \leq c, \; \forall n \in \nat} = 1,
    \end{equation*}
    tada obje sume $\suma{n = 1}{\infty} \masE X_n$ te $\suma{n = 1}{\infty} \Var (X_n)$ konvergiraju.
\end{kor}

% sredi pretpostavke - vjerojatno nezavisni, ne nužno jednako distribuirani
\begin{tm}[Kolmogorovljev teorem o tri reda]    \label{tm:13.16}
    \begin{equation*}
        \suma{n = 1}{\infty} X_n \textnormal{ konvergira } (g.s.)
        \iff
        \begin{aligned}
            &\suma{n = 1}{\infty} \masP (|X_n| > 1) < +\infty\\
            &\suma{n = 1}{\infty} \masE [X_n \cdot \karaktFja_{\{ |X_n| \leq 1 \}}] < +\infty\\
            &\suma{n = 1}{\infty} \Var (X_n \cdot \karaktFja_{\{ |X_n| \leq 1 \}}) < + \infty.
        \end{aligned}
    \end{equation*}
\end{tm}

\begin{proof}
    Definirajmo niz $\niz{Y_n}{n \in \nat}$ sa
    \begin{equation*}
        Y_n := X_n \cdot \karaktFja_{\{ |X_n| \leq 1 \}}, \quad \forall n \in \nat.
    \end{equation*}
    Tada je $(Y_n)$ niz slu\v cajnih varijabli i vrijedi $|Y_n| \leq 1$, $\forall n \in \nat$. Posebno postoje $\masE Y_n$ i $\Var Y_n$.
    \begin{itemize}
        \item[$\implies$]
        Konvergencija reda $\suma{n = 1}{\infty} X_n$ gotovo sigurno povla\v ci konvergenciju prvog reda prema relaciji \eqref{jed:13.2}.
        Tada prema relaciji \eqref{jed:13.8} su nizovi $(X_n)$ i $(Y_n)$ ekvivalentni.
        Po relaciji \eqref{jed:13.7} znamo da red $\suma{n = 1}{\infty} Y_n$ konvergira gotovo sigurno, pa konvergencija drugog i tre\' ceg reda slijedi iz korolara \ref{kor:13.15}
        \item[$\impliedby$]
        Zbog tre\' ceg reda i teorema \ref{tm:13.12} red $\suma{n = 1}{\infty} (Y_n - \masE Y_n)$ konvergira gotovo sigurno.
        Zbog drugog reda $\suma{n = 1}{\infty} Y_n$ konvergira gotovo sigurno. Zbog prvog reda $(X_n)$ i $(Y_n)$ su ekvivalentni, pa po \eqref{jed:13.7} slijedi tvrdnja. 
    \end{itemize}
\end{proof}

    %%%%%%%%%%%%%%%%%%%%%%%%%%%%%%%%%%
    %%  jaki zakon velikih brojeva  %%
    %%%%%%%%%%%%%%%%%%%%%%%%%%%%%%%%%%

    % jaki zakon velikih brojeva poglavlje 14

\chapter{Jaki zakoni velikih brojeva}

U ovom poglavlju $\niz{X_n}{n \in \nat}$ je niz nezavisnih jednako distribuiranih slu\v cajnih varijabli na vjerojatnosnom prostoru $\vjerojatnosniProstor$ i neka je
\begin{equation*}
    S_n := X_1 + \ldots + X_n, \quad n \in \nat.
\end{equation*}
Nadalje za $0 < p < 2$ i za $n \in \nat$, ozna\v cavamo
\begin{equation*}
    Y_{n, p} := n^{-\frac{1}{p}} X_n \cdot \karaktFja_{\Big\{|X_n| \leq n^\frac{1}{p} \Big\}}.
\end{equation*}
To zna\v ci da je za svaki $0 < p < 2$, $\niz{Y_{n, p}}{n \in \nat}$ niz nezavisnih slu\v cajnih varijabli.
Nadalje, budu\' ci da su $Y_{n, p}$ ome\dj ene slu\v cajne varijable, one posjeduju momente svakog reda.

\begin{tm}  \label{tm:14.1}
    Ako postoji $0 < p < 2$ takav da je $\masE \Big[ |X_1|^p \Big] < +\infty$, tada
    \begin{equation*}
        \suma{n = 1}{\infty} \Big( \frac{X_n}{n^\frac{1}{p}} - \masE Y_{n, p} \Big)
    \end{equation*}
    konvergira gotovo sigurno.
    
    Ako uz to vrijedi da je $0 < p < 1$ (ili $1 < p <2$ i $\masE X_1 = 0$), tada
    \begin{equation*}
        \suma{n = 1}{\infty} \frac{X_n}{n^\frac{1}{p}}
    \end{equation*}
    konvergira gotovo sigurno.
\end{tm}

\begin{proof}
    Neka je $\alpha \in \{1, 2\}$, te neka je $p < \alpha$.
    Za $j \in \nat$,
    \begin{equation*}
        A_j := \bigSkup{\omega \in \Omega}{(j - 1)^\frac{1}{p} < |X_1 (\omega)| \leq j^\frac{1}{p}}.
    \end{equation*}
    Tada je
    \begin{equation*}
        \begin{aligned}
            \suma{n = 1}{\infty} \masE \Big[ |Y_{n, p}|^\alpha \Big] &= \suma{n = 1}{\infty} \suma{j = 1}{n} \frac{1}{n^\frac{\alpha}{p}} \int\limits_{A_j} |X_1|^\alpha \: d \masP =
            \begin{psmallmatrix}
                \textnormal{Fubini za}\\
                \textnormal{sume } \geq 0
            \end{psmallmatrix}
            = \suma{j = 1}{\infty} \suma{n = j}{\infty} n^{- \frac{\alpha}{p}} \int\limits_{A_j} |X_1|^\alpha \: d \masP\\
            &= \suma{j = 1}{\infty} \int\limits_{A_j} |X_1|^\alpha \Big( \suma{n = j}{\infty} n^{-\frac{\alpha}{p}} \Big) \: d \masP \leq
            \begin{psmallmatrix}
                \textnormal{integralni}\\
                \textnormal{test za redove}
            \end{psmallmatrix}\\
            &\leq \suma{j = 1}{\infty} \int\limits_{A_j} |X_1|^\alpha \: d \masP \Big( \frac{1}{j^\frac{\alpha}{p}} + \int\limits_j^\infty \frac{du}{u^\frac{\alpha}{p}}\Big) = \suma{j = 1}{\infty} \int\limits_{A_j} |X_1|^\alpha \: d \masP \cdot \Big( \frac{1}{j^\frac{\alpha}{p}} + \frac{1}{j^{\frac{\alpha}{p} - 1}} \cdot \frac{p}{\alpha - p} \Big)\\
            &= \suma{j = 1}{\infty} \int\limits_{A_j} \Bigg[ \frac{|X_j|}{ \underbrace{j^\frac{1}{p}}_{\leq 1}} \Bigg]^{\alpha - p} \cdot |X_1|^p \: d \masP \cdot \Big( \underbrace{\frac{1}{j}}_{\leq 1} + \frac{p}{\alpha - p} \Big) \leq \suma{j = 1}{\infty} \int\limits_{A_j} |X_1|^p \: d \masP \cdot \frac{\alpha}{\alpha - p}\\
            &= \frac{\alpha}{\alpha - p} \cdot \masE \big[ |X_1|^p \big] < +\infty.
        \end{aligned}
    \end{equation*}
    Uzmemo prvo $\alpha = 2$ i po teoremu \ref{tm:13.12} primjenjenom na $\niz{Y_{n, p}}{n \in \nat}$ dobivamo
    \begin{equation*}
        \suma{n = 1}{\infty} (Y_{n, p} - \masE Y_{n, p})
    \end{equation*}
    konvergira gotovo sigurno.
    Uo\v cimo
    \begin{equation*}
        \suma{n = 1}{\infty} \masP \Big( \frac{X_n}{n^\frac{1}{p}} \neq Y_{n, p} \Big) = \suma{n = 1}{\infty} \masP (|X_1| > n^\frac{1}{p}) \leq \masE [ |X_1|^p ] < +\infty,
    \end{equation*}
    dakle nizovi $\bigNiz{\frac{X_n}{n^\frac{1}{p}}}{n \in \nat}$ i $\niz{Y_{n, p}}{n \in \nat}$ su ekvivalentni.
    Stoga
    \begin{equation*}
        \suma{n = 1}{\infty} \Big( \frac{X_n}{n^\frac{1}{p}} - \masE Y_{n, p} \Big)
    \end{equation*}
    konvergira gotovo sigurno.

    Neka je sada $1 < p < 2$ i $\masE X_1 = 0$.
    Dobivamo
    \begin{equation*}
        \begin{aligned}
            \suma{n = 1}{\infty} |\masE Y_{n, p}| &\leq
            \begin{psmallmatrix}
                \textnormal{jer } \masE X_n = 0
            \end{psmallmatrix}
            \leq \suma{n = 1}{\infty} n^{-\frac{1}{p}}
            \int\limits_{\Big\{ |X_n| > n^\frac{1}{p} \Big\}} |X_n| \: d \masP =
            \begin{psmallmatrix}
                \textnormal{isti izvod kao ranije}\\
                \textnormal{uz } X_1 \distJed X_n
            \end{psmallmatrix}\\
            &= \suma{n = 1}{\infty} \suma{i = n + 1}{\infty} n^\frac{1}{p} \int\limits_{A_j} |X_1| \: d \masP = \suma{j = 2}{\infty} \suma{n = 1}{j - 1} n^{-\frac{1}{p}} \int\limits_{A_j} |X_1| \: d \masP\\
            &\leq \suma{j = 2}{\infty} \int\limits_{A_j} |X_1| \: d \masP \Big( \underbrace{\int\limits_0^{j-1} \frac{du}{u^\frac{1}{p}} }_{\frac{p}{p - 1} (j - 1)\frac{p - 1}{p}} \Big) =
            \begin{psmallmatrix}
                \textnormal{dodamo } j = 1 \textnormal{, jer}\\
                \textnormal{je } j - 1 = 0
            \end{psmallmatrix}\\
            &= \frac{p}{p - 1} \suma{j = 1}{\infty} (j - 1)^\frac{p - 1}{p} \int\limits_{A_j} |X_1| \: d \masP \leq
            \begin{psmallmatrix}
                \textnormal{kao u gornjem}\\
                \textnormal{izvodu}
            \end{psmallmatrix}
            \leq \frac{p}{p - 1} \suma{j = 1}{\infty} \int\limits_{A_j} |X_1| \: d \masP\\
            &= \frac{p}{p - 1} \masE [ |X_1|^p ] < + \infty.
        \end{aligned}
    \end{equation*}
\end{proof}

\begin{tm}[Marcinkiewicz-Zygmundov jaki zakon]  \label{tm:14.2}
    Postoji konstanta $c \in \real$, takva da
    \begin{equation*}
        \frac{S_n - nc}{n^\frac{1}{p}} \xrightarrow[n \to \infty]{g.s.} 0,
    \end{equation*}
    ako i samo ako je
    \begin{equation*}
        \masE [ |X_1|^p ] < + \infty.
    \end{equation*}
    Ako je jo\v s uz to $1 \leq p < 2$, tada je $c = \masE X_1$.
    Ako je uz gornju tvrdnju jo\v s $0 < p <1$, tvrdnja vrijedi za svaki $c \in \real$ (posebno i za $c = 0$).
\end{tm}

\begin{proof}
    \quad \\
    \begin{enumerate}
        \item[$\implies$]
        Ako za neki $c \in \real$, vrijedi
        \begin{equation*}
            \frac{S_n - nc}{n^\frac{1}{p}} \xrightarrow[n \to \infty]{g.s.} 0,
        \end{equation*}
        tada je
        \begin{equation*}
            \frac{X_n}{n^\frac{1}{p}} = \frac{S_n - S_{n - 1}}{n^\frac{1}{p}} = \frac{S_n - nc}{n^\frac{1}{p}} - \Big( \frac{n - 1}{n} \Big)^\frac{1}{p} \cdot \frac{S_{n - 1} - nc}{(n - 1)^\frac{1}{p}} \xrightarrow[n \to \infty]{g.s.} 0.
        \end{equation*}
        Tada sigurno vrijedi
        \begin{equation*}
            \masP \Big( \liminf\limits_{n \to \infty} \Big\{ \Big| \frac{X_n}{n^\frac{1}{p}} \Big| < 1 \Big\} \Big) = 1,
        \end{equation*}
        zbog konvergencije gornjeg niz u $0$, pa onda iz svojstva
        \begin{equation*}
            \Big( \liminf\limits_{n \to \infty} A_n \Big)^c = \limsup\limits_{n \to \infty} A_n^c,
        \end{equation*}
        dobivamo
        \begin{equation*}
            \masP \Big( \Big| \frac{X_n}{n^\frac{1}{p}} \Big| \geq 1 \; \io \Big) = 0.
        \end{equation*}
        Po Borelovom zakonu 0-1 imamo
        \begin{equation*}
            \begin{gathered}
                \suma{n = 1}{\infty} \masP ( |X_1|^p \geq n ) \overset{\iid}{=} \suma{n = 1}{\infty} \masP (|X_1| \geq n^\frac{1}{p}) < +\infty\\
                \implies \quad \masE [ |X_1|^p ] < +\infty.
            \end{gathered}
        \end{equation*}

    \item[$\impliedby$]
        Ako je $1 < p <2$ i $\masE |X_1|^p < +\infty$, onda po teoremu \ref{tm:14.1} red
        \begin{equation*}
            \suma{n = 1}{\infty} \frac{X_n - \masE X_n}{n^\frac{1}{p}}
        \end{equation*}
        konvergira gotovo sigurno.

        Ako je $p = 1$ i $\masE |X_1|^p < +\infty$, sada po teoremu \ref{tm:14.1} dobivamo da red
        \begin{equation*}
            \suma{n = 1}{\infty} \frac{X_n - \masE \Big[ X_n \cdot \karaktFja_{\{ |X_n| \leq n \}} \Big]}{n}
        \end{equation*}
        konverigra gotovo sigurno, dakle za $0 < p <1$ i $\masE |X_1|^p < \infty$ dobivamo konvergenciju gotovo sigurno
        \begin{equation*}
            \suma{n = 1}{\infty} \frac{X_n}{n^\frac{1}{p}}.
        \end{equation*}
        
        Za $p \neq 1$ rezultat slijedi direktno iz Kroneckerove leme (lema \ref{lm:13.10-1}) za $c = 0$ ili $c = \masE X_1$.

        Za $p = 1$, zbog $\iid$ imamo
        \begin{equation*}
            \masE \Big[ X_n \karaktFja_{\{ |X_n| \leq n \}} \Big] = \masE \Big[ X_1 \karaktFja_{\{ |X_1| \leq n \}} \Big] \xrightarrow[n \to \infty]{} \masE X_1. 
        \end{equation*}
        Po Kronekerovoj lemi (\ref{lm:13.10-1})
        \begin{equation*}
            \lim\limits_{n \to \infty} \frac{1}{n} \Big( S_n - \suma{j = 1}{n} \masE \big[ X_j \karaktFja_{\{ |X_j| \leq j \}} \big] \Big) = 0 \; (g.s.)
        \end{equation*}
        Uo\v cimo da za $0 < p < 1$ i za $c \in \real$ imamo
        \begin{equation*}
            \frac{n \cdot c}{n^\frac{1}{p}} \xrightarrow[n \to \infty]{} 0,
        \end{equation*}
    \end{enumerate}
    pa smo u potpunosti dokazali teorem.
\end{proof}

Primjenimo li gornji teorem na slu\v caj $p = 1$ dobivamo:

\begin{kor}[Kolmogorovljev jaki zakon] \label{kor:14.3}
    \begin{equation*}
        \frac{S_n}{n} \; \textnormal{ konvergira } (g.s.) \quad \iff \quad \masE [|X_1|] < + \infty.
    \end{equation*}
    U tom slu\v caju mora biti
    \begin{equation*}
        \frac{S_n}{n} \xrightarrow[n \to \infty]{g.s.} \masE X_1.
    \end{equation*}
\end{kor}

\begin{nap} \label{nap:14.4}
    Kolmogorovljev jaki zakon je fundamentalni teorem u ovom pristupu vjerojatnosti.
    Njegova interpretacija je u skladu s na\v som intuitivnom predod\v zbom o vjerojatnosti kao "relativnoj frekvenciji".
    Pretpostavimo da vr\v simo eksperiment i zanima nas doga\dj aj $A$ koji \' ce se dogoditi s nekom vjerojatno\v s\' cu $\vjeroj{A} = p$.
    Vr\v simo nezavisna ponavljanja pokusa i u $n$-tom ponavljanju stavimo $X_n := \karaktFja_A$.
    Tada je $\niz{X_n}{n \in \nat}$ nezavisan i jednako distribuiran niz slu\v cajnih varijabli i
    \begin{equation*}
        \masE [|X_1|] = \masE X_1 = p < + \infty.
    \end{equation*}
    Po Kolmogorovljevom jakom zakonu velikih brojeva (korolar \ref{kor:14.3}) vrijedi
    \begin{equation*}
        \begin{matrix}
            \textnormal{"relativna frekvencija"}\\
            \textnormal{doga\dj aja } A
        \end{matrix}
        = \frac{S_n}{n} \xrightarrow[n \to \infty]{g.s.} \masE X_1 = p = \vjeroj{A}.
    \end{equation*}
\end{nap}

\begin{nap} \label{nap:14.5}
    Zapravo vrijedi i ja\v ci teorem, ali ne\' cemo ga dokazivati ovdje.
    Ako su $\niz{X_n}{n \in \nat}$ jednako distribuirane slu\v cajne varijable koje su u parovima nezavisne i ako je
    \begin{equation*}
        \masE [ |X_1| ] < +\infty,
    \end{equation*}
    tada
    \begin{equation*}
        \frac{S_n}{n} \xrightarrow[n \to \infty]{g.s.} \masE X_1.
    \end{equation*}
    % za dokaz vidjeti Zeit. Warsh. Ver. Geb. (1981), 119-122
\end{nap}

Stavimo li u korolar \ref{kor:14.3} da su
\begin{equation*}
    X_n \sim
    \begin{pmatrix}
        0 & 1\\
        \tilde{q} & \tilde{p}
    \end{pmatrix}
\end{equation*}
dobivamo

\begin{kor}[Borelov jaki zakon] \label{kor:14.6}
    Ako su $\niz{Y_n}{n \in \nat}$ dobiven iz Bernoullijeve sheme i $Y_n \sim B(n, \tilde{p})$, tada
    \begin{equation*}
        \frac{Y_n}{n} \xrightarrow[n \to \infty]{g.s.} \tilde{p}.
    \end{equation*}
\end{kor}

Koriste\' ci sli\v cne tehnike kao gore (Kronecker) mo\v ze dobiti sljede\' ci rezultat:

\begin{tm}[Kai-Lai Chungov jaki zakon]  \label{tm:14.6-1}
    Neka je $\niz{Z_n}{n \in \nat}$ niz nezavisnih slu\v cajnih varijabli i $\masE Z_n = 0$, za svaki $n \in \nat$.
    Ako postoje funkcije
    \begin{equation*}
        \varphi_n : \real_+ \to \real_+,
    \end{equation*}
    takve da su
    \begin{equation*}
        \frac{\varphi_n (t)}{t}, \quad \frac{t^2}{\varphi_n (t)}
    \end{equation*}
    neopadaju\' ce funkcije i postoje brojevi $c_n \in \real \setminus \{0\}$ takvi da je
    \begin{equation*}
        \suma{n = 1}{\infty} \frac{\masE [ \varphi_n (|Z_n|) ]}{\varphi_n (|c_n|)} < +\infty,
    \end{equation*}
    tada red
    \begin{equation*}
        \suma{n = 1}{\infty} \frac{Z_n}{c_n}
    \end{equation*}
    konvergira gotovo sigurno.
\end{tm}

Uo\v cimo u prethodnom teoremu nema pretpostavke jednake distribuiransoti.

\begin{zad} \label{zad:14.7}
    Doka\v zi teorem \ref{tm:14.6-1}
\end{zad}

\begin{zad} \label{zad:14.8}
    Neka su $\niz{Z_n}{n \in \nat}$ nezavisne slu\v cajne varijable.
    Ako vrijedi:
    \begin{equation*}
        \begin{gathered}
            \begin{gathered}
                0 < \alpha_n < 1\\
                1 \leq \alpha_n \leq 2, \; \masE Z_n = 0
            \end{gathered}
            \quad n \in \nat.
        \end{gathered}
    \end{equation*}
    te ako je
    \begin{equation*}
        \suma{n = 1}{\infty} \frac{\masE [|Z_n|^{\alpha_n}]}{n^{\alpha_n}} < +\infty,
    \end{equation*}
    tada vrijedi
    \begin{equation*}
        \frac{1}{n} \suma{j = 1}{n} Z_j \xrightarrow[n \to \infty]{g.s.} 0.
    \end{equation*}
\end{zad}
(Obrati pa\v znju na slu\v caj $\alpha_n = 2$, $\forall n \in \nat$)

Promotrimo sada situaciju s nizom nezavisnih jednako distribuiranih slu\v cajnih varijabli malo detaljnije.
Dakle, neka je $X_1 \in L^2 (\masP)$ (to jest svi $X_n$ imaju druge momente).
Oduzimanjem $\mu := \masE X_1$ lako dolazimo do situacije u kojoj bez smanjanja op\' cenitosti mo\v zemo pretpostaviti da je $\masE X_n = 0$, za svaki $n \in \nat$ (time je $\masE S_n = 0$).
Ozna\v cimo $\masE [X_1^2] = \masE [X_n^2]$ sa $\sigma^2$ i bez smanjenja op\' cenitosti $0 < \sigma^2 < +\infty$.

Tada iz teorema \ref{tm:14.2} slijedi da je za svaki $1 \leq p < 2$
\begin{equation}    \label{jed:14.9}
    \frac{S_n}{n^\frac{1}{p}} \xrightarrow[n \to \infty]{g.s.} 0.
\end{equation}
To zna\v ci da za gotovo sve $\omega \in \Omega$, trajektorija
\begin{equation*}
    n \mapsto S_n (\omega),
\end{equation*}
za velike $n$, osciliraju manje nego
\begin{equation*}
    n \mapsto n^\frac{1}{p}.
\end{equation*}

%
%   slika
%

\begin{zad} \label{zad:14.10}
    Doka\v zi da uz navedene pretpostavke vrijedi
    \begin{equation*}
        \begin{gathered}
            \limsup\limits_{n \to \infty} \frac{S_n}{\sqrt{n}} = +\infty \; (g.s.)\\
            \liminf\limits_{n \to \infty} \frac{S_n}{\sqrt{n}} = -\infty \; (g.s.)
        \end{gathered}
    \end{equation*}
\end{zad}

\begin{prop}    \label{prop:14.11}
    Uz navedene pretpostavke za svaki $\varepsilon > 0$ vrijedi
    \begin{equation*}
        \lim\limits_{n \to \infty} \frac{S_n}{\sqrt{n} \cdot (\ln n)^{\frac{1}{2} + \varepsilon}} = 0 \; (g.s.)
    \end{equation*}
\end{prop}

\begin{proof}
    Neka je
    \begin{equation*}
        \begin{aligned}
            a_1 &> 0,\\
            a_n &= \sqrt{n} \cdot (\ln n)^{\frac{1}{2} + \varepsilon}, \quad n \geq 2.
        \end{aligned}
    \end{equation*}
    Tada je
    \begin{equation*}
        \suma{n = 1}{\infty} \Var \Big( \frac{X_n}{a_n} \Big) = \sigma^2 \Big( \frac{1}{a_1^2} + \suma{n = 2}{\infty} \frac{1}{n \: (\ln n)^{1 + 2 \varepsilon}} \Big) < +\infty,
    \end{equation*}
    pa prema teoremu \ref{tm:13.12}
    \begin{equation*}
        \suma{n = 1}{\infty} \frac{X_n}{a_n}
    \end{equation*}
    konvergira gotovo sigurno, pa po Kroneckerovoj lemi (lema \ref{lm:13.10-1})
    \begin{equation*}
        \frac{S_n}{a_n} \xrightarrow[n \to \infty]{g.s.} 0.
    \end{equation*}
\end{proof}

Uo\v cimo da je
\begin{equation*}
    \limsup\limits_{n \to \infty} \frac{|S_n|}{h_n} = \const \; (g.s.)
\end{equation*}
za svaki $h_n \nearrow +\infty$.

Pitanje, postoji li $h_n$ tako da je $\const = 1$ je odgovoreno u "zakonu ponovljenog logaritma" uz
\begin{equation*}
    h_n = \sigma \cdot \sqrt{2n \: \ln (\ln n)}.
\end{equation*}
Pitanje postoji li neka distribucija prema kojoj "konvergira" $\frac{S_n}{h_n}$ tra\v zi detaljniji odgovor.

    %%%%%%%%%%%%%%%%%%%%%%%%%%%%%%
    %%  uvod u slucajne setnje  %%
    %%%%%%%%%%%%%%%%%%%%%%%%%%%%%%

    % poglavlje 3.5 -> predavanje 15 - uvod u slučajne šetnje

\chapter{Uvod u slu\v cajne \v setnje}

U ovom poglavlju uzimamo da je $\niz{X_n}{n \in \nat}$ je niz nezavisnih jednako distribuiranih slu\v cajnih varijabli na vjerojatnosnom prostoru $\vjerojatnosniProstor$.
Neka je
\begin{equation*}
    \begin{aligned}
        S_0 &:= 0,\\
        S_n &:= X_1 + \ldots + X_n, \quad n \in \nat.
    \end{aligned}
\end{equation*}

Uo\v cimo da je $\niz{S_n}{n \in \nat}$ mo\v zemo promatrati i kao stohasti\v cki proces, koji nazivamo \emph{slu\v cajnom \v setnjom} (s vrijednostima) u $\real$.
Parametar $n \in \nat_0$ obi\v cno interpretiramo kao vrijeme.
Uz tu interpretaciju, zakoni velikih brojeva govore o asimptotskom pona\v sanju \v slu\v cajne \v setnje ("u beskona\v cnosti").
Iz rezultata u poglavlju \ref{pog:2.5} (posebno primjer \ref{pr:10.3} i zadatak \ref{zad:10.4}) \v citamo da su mogu\' ca \v cetiri osnovna tipa pona\v sanja (trivijalni niz, konvergencija u jednu od beskona\v cnosti i "sve ve\' ce oscilacije").
Preciznije, dokazali smo teorem.

\begin{tm}  \label{tm:15.1}
    Ako je $\niz{S_n}{n \in \nat}$ slu\v cajna \v setnja u $\real$, tada vrijedi to\v cno jedna od sljede\' cih mogu\v cnosti (i to je ispunjeno s vjerojatno\v scu 1):
    \begin{enumerate}[label=(\roman*)]
        \item $S_n \xrightarrow[n \to \infty]{} 0 \; g.s.,$
        \item $S_n \xrightarrow[n \to \infty]{} +\infty \; g.s.,$
        \item $S_n \xrightarrow[n \to \infty]{} -\infty \; g.s.,$
        \item $\liminf\limits_{n \to \infty} S_n = - \infty \; g.s. \quad$ i $\quad \limsup\limits_{n \to \infty} S_n = + \infty \; g.s.$
    \end{enumerate}
\end{tm}

    \part{Centralni grani\v cni teoremi}

    %%%%%%%%%%%%%%%%%%%%%%%%%%%
    %%  slaba konvergencija  %%
    %%%%%%%%%%%%%%%%%%%%%%%%%%%

    % poglavalje 5.1 predavanje 16 - slaba konvergencija

\chapter{Slaba konvergencija}

Neka je
\begin{equation*}
    \inc := \bigSkup{f: \real \to \real}{x \leq y \implies f (x) \leq f(y)}.
\end{equation*}
Ako je $F$ d.f., tada je $F \in \inc$.
Ako je $F$ p.f.F., tada je $\restr{F}{\real} \in \inc$, pa u tom smislu kra\' ce pi\v semo da je $F \in \inc$.
O\v cito za svaku $G \in \inc$ i za svaki $x \in \real$ vrijedi:
\begin{equation}    \label{jed:16.1}
    \lim\limits_{y \nearrow x} G(y) := G(x -) \leq G (x) \leq G(x +) := \lim\limits_{y \searrow x} G(y).
\end{equation}
Za funkciju $G \in \inc$ ozna\v cimo sa $G^R$ funkciju definiranu s $G^R (x) := G(x+)$.
Tada je $G^R \in \inc$ te $G^R$ je neprekidna zdesna, pa je $G^R$ d.f.
Ako je $F$ p.d.F. tada je $F^R = F$.

\begin{defn}  \label{defn:16.1-1}
    Neka je $G \in \inc$.
    Ako za $x \in \real$ vrijedi $G(x-) = G(x) = G(x+)$, ka\v zemo da $x$ pripada skupu $C(G)$ \emph{svih to\v caka neprekidnosti} od $G$.     
\end{defn}

Uo\v cimo da je $\real \setminus C(G)$ najvi\v se prebrojiv, pa je $C(G)$ gust u $\real$.
Nadalje, vrijedi
\begin{equation}    \label{jed:16.2}
    \begin{gathered}
        G(G) \subseteq C(G^R)\\
        \restr{G^R}{C(G)} = \restr{G}{C(G)},
    \end{gathered}
\end{equation}
razmislite o jednakosti ovih skupova.
Lako se vidi da za $G \in \inc$ sljede\' ci limesi postoje u $\extReal$:
\begin{equation}    \label{jed:16.3}
    \begin{aligned}
        G(+\infty) &:= \lim\limits_{x \nearrow +\infty} G(x)\\
        G(-\infty) &:= \lim\limits_{x \searrow -\infty} G(x)\\
        G^R(+\infty) &= G(+\infty)\\
        G^R(-\infty) &= G(-\infty).
    \end{aligned}
\end{equation}

Za $0 < a < +\infty$ uvedimo oznake:
\begin{equation*}
        \begin{aligned}
            \Delta G(a) &:= G(a) - G(-a),\\
            \Delta G &:= G(+\infty) - G(-\infty),
        \end{aligned}
\end{equation*}
te je $\Delta G(a) \in \real$, $\Delta G \in \extReal$, tako\dj er vrijedi:
\begin{equation}    \label{jed:16.4}
    0 \leq \Delta G(a) \leq \Delta G = \lim\limits_{a \nearrow +\infty} \Delta G (a) = \lim\limits_{
        \begin{smallmatrix}
            a \nearrow +\infty\\
            a \in C(G)
        \end{smallmatrix}
    } \Delta G (a) \leq +\infty.
\end{equation}
    
\end{document}

