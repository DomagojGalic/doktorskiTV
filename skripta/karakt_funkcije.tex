% 5. dio poglavlje 2, 17. predavanje karakteristične funkcije

\chapter{Karakteristi\v cne funkcije}

Fourierova transforamcija je vrlo korisna i u vjerojatnosti.
U ovom poglavlju $X$ uvijek ozna\v cava slu\v cajnu varijablu na vjerojatnostnom prostoru $\vjerojatnosniProstor$.

\begin{defn}    \label{def:17.1}
    Funkcija $\varphi_X :  \real \to \masC$, definirana sa
    \begin{equation*}
        \varphi_X (t) := \masE [e^{itX}] = \masE[\cos (tX)] + i \cdot \masE [ \sin (tX) ], \quad t \in \real,
    \end{equation*}
    naziva se \emph{karakteristi\v cnom funkcijom} slu\v cajne varijable $X$.
\end{defn}

Lako se vidi da je
\begin{equation*}
    \varphi_X (t) = \int_{-\infty}^{+\infty} e^{itX} \: d F_X (x) = \int_{-\infty}^{+ \infty} \cos (t X)  \: d F_X (x) + i \cdot \int_{-\infty}^{+\infty} \sin (t X) \: d F_X (x),
\end{equation*}
te da je $\varphi_X (t)$ uvijek definiran.
Ako je $F \in \inc$, $F^R = F$ i $\Delta F < +\infty$, ista definicija daje funkciju $\varphi_F : \real \to \masC$, koja \' ce imati svojstva analogna onima od $\varphi_X$.

Lako se vidi da za slu\v cajnu varijablu $Y$ i $t \in \real$ vrijedi:
\begin{equation}    \label{jed:17.2}
    X \distJed Y \implies \varphi_X = \varphi_Y,
\end{equation}
te
\begin{equation}    \label{jed:17.3}
    \varphi_X(-t) = \overline{\varphi_X (t)}, \quad \quad |\varphi_X (t)| \leq \varphi_X(0) = 1.
\end{equation}

Uo\v cimo da je
\begin{equation*}
    |\varphi_X (t+h) - \varphi_X(t)| \leq \int_{-\infty}^{+\infty} |e^{i h X} - 1| \: d F_X (x).    
\end{equation*}
Budu\' ci da $|e^{ihX} -1| \xrightarrow[h \to 0]{} 0$ i uniformno je ome\dj ena s $2$ (\v sto je integrabilna funkcija u odnosu na $F_X$) iz LTDK slijedi
\begin{equation}    \label{jed:17.4}
    \varphi_X \textnormal{ je uniformno neprekidna na } \real.
\end{equation}

Za $z \in \masC$ vrijedi
\begin{equation*}
    e^z = \suma{n = 0}{\infty} \frac{z^n}{n !},
\end{equation*}
pa \' ce svojstva $\varphi_X$ ovisiti o postojanju momenata od $X$.

\begin{lm}  \label{lm:17.5}
    Za svaki $x \in \real$ i $n \in \nat \cup \{ 0 \}$ vrijedi
    \begin{equation*}
        \Big| e^{ix} - \suma{j = 0}{n} \frac{(ix)^j}{j !} \Big| \leq \min \Big\{ \frac{|x|^{n + 1}}{(n + 1)!}, \frac{2 \cdot |x|^n}{n !} \Big\}.
    \end{equation*}
\end{lm}

\begin{proof}
    Po Taylorovom razvoju
    \begin{equation*}
        \begin{aligned}
            e^{ix} - \suma{j = 0}{n} \frac{(i x)^j}{j !} &= \frac{(i x)^{n + 1}}{n !} \int_0^1 e^{i x u} (1 - u)^n \: du =
            \begin{pmatrix}
                \textnormal{pracijalna}\\
                \textnormal{integracija}
            \end{pmatrix}\\
            &= - \frac{(ix)^n}{n !} + \frac{(i x)^n}{(n - 1) !} \int_0^1 e^{i x u} (1 - u)^{n - 1} \: d u.
        \end{aligned}
    \end{equation*}
    Uo\v cimo
    \begin{equation*}
        \int_0^1 (1 - u)^k \: d u = \frac{(1 - u)^{k + 1}}{k + 1} (-1) \Bigg|_0^1 = \frac{1}{(k + 1)}.
    \end{equation*}
    Zbog
    \begin{equation*}
        |e^{ixu}| = 1,
    \end{equation*}
    prva jednakost daje prvu ocjenu, a druga jednakost drugu ocjenu.
\end{proof}

\begin{lm}  \label{lm:17.6}
    Ako je $X \in L^n (\masP)$, za neki $n \in \nat$, tada je $\varphi$ $n$-puta diferencijabilno i
    \begin{equation*}
        \varphi_X^{(n)} (t) = \int_{-\infty}^{+\infty} (i x)^n e^{i t x} \: d F_X (x).
    \end{equation*}
\end{lm}

\begin{proof}
    Indukcijom po $n$, pri \v cemu korak ide kao i baza,
    \begin{equation*}
        \frac{\varphi_X (t + s) - \varphi_X (t)}{s} = \masE \Big[ e^{i t X} \cdot \frac{e^{s X} - 1}{s} \Big].
    \end{equation*}
    Uo\v cimo da je
    \begin{equation*}
        \frac{e^{i s X} - 1}{s} \xrightarrow[s \to 0]{} i X,
    \end{equation*}
    i ome\dj en je sa $|X|$ (po lemi \ref{lm:17.5}).
    Primjenom LTDK slijedi
    \begin{equation*}
        \lim\limits_{s \to 0} \frac{\varphi_X (t + s) - \varphi_X (t)}{s} = \masE [i X \cdot e^{i t X}] = \int_{-\infty}^{+\infty} i x e^{i t x} \: d F_X (x).
    \end{equation*}
\end{proof}

Iz leme \ref{lm:17.5} i leme \ref{lm:17.6} direktno dobijemo

\begin{kor} \label{kor:17.7}
    Neka je $n \in \nat \cup \{0\}$.
    Tada vrijedi:
    \begin{enumerate}[label=(\roman*)]
        \item \label{kor:17.7.1}
        \begin{equation*}
            \masE \big[ |X|^2 \big] < +\infty \implies \varphi_X^{(n)}(0) = \masE \big[ (i \cdot X)^n \big],
        \end{equation*}
        \item \label{kor:17.7.2}
        \begin{equation*}
            \begin{aligned}
                \masE \big[|X|^{n + 1} \big] < +\infty \implies \Big| \varphi_X (t) - \suma{j = 0}{n} \frac{(it)^j}{j !} \cdot \masE \big[ X^j \big]  \Big| \leq
                \min \Big\{ &\frac{|t|^{n+1}}{(n + 1)!} \cdot \masE \big[ |X|^{n + 1} \big],\\
                &\frac{2 \cdot |t|^n}{n !} \cdot \masE \big[ |X|^n \big] \Big\},   
            \end{aligned}
        \end{equation*}
        \item \label{kor:17.7.3}
        \begin{equation*}
            \begin{aligned}
                \masE \big[ |X|^n \big] < +\infty \implies \varphi_X (t) &= \suma{j = 0}{n} \frac{(i t)^j}{j!} \cdot \masE \big[ X^j \big] + o (t^n)\\
                &= \suma{j = 0}{n} \frac{t^j}{j!} \cdot \varphi_X^{(j) } (0) + o(t^n).
            \end{aligned}
        \end{equation*}
    \end{enumerate}
\end{kor}

Uo\v cimo da za svaka \v cetiri kompleksna broja vrijedi
\begin{equation}    \label{jed:17.8}
    |\alpha_1 \alpha_2 - \beta_1 \beta_2| \leq |\alpha_1| \cdot | \alpha_2 - \beta_2| + |\beta_2| \cdot | \alpha_1 - \beta_1|.
\end{equation}

\begin{lm}  \label{lm:17.9}
    Neka je $n \in \nat$, $z_1, \ldots, z_n \in \masC$, $w_1, \ldots, w_n \in \masC$ takvi da je
    \begin{equation*}
        \max \big\{ |z_1|, \ldots, |z_n|, |w_1|, \ldots, |w_n| \big\} \leq \alpha.
    \end{equation*}
    Tada je
    \begin{equation*}
        \Big| \produkt{j = 1}{n} z_j - \produkt{j = 1}{n} w_j \Big| \leq \alpha^{n - 1} \cdot \suma{j = 1}{n} |z_j - w_j|.
    \end{equation*}
\end{lm}

\begin{proof}
    Indukcijom po $n$.
    Slu\v caj $n = 1$ je trivijalan.
    
    Koriste\' ci nejednakost \eqref{jed:17.8}, imamo
    \begin{equation*}
        \begin{aligned}
            \Big| \Big( \produkt{j = 1}{n} z_j \Big) \cdot z_{n + 1} - \Big( \produkt{j = 1}{n} w_j \Big) \cdot w_{n + 1} \Big| &\leq \alpha^n | z_{n+1} - w_{n+1} | + \alpha \cdot \Big|  \produkt{j = 1}{n} z_j - \produkt{j = 1}{n} w_j \Big|\\
            &\leq
            \begin{pmatrix}
                \textnormal{pretpostavka}\\
                \textnormal{indukcije}
            \end{pmatrix}\\
            &\leq \alpha^n \cdot \suma{j = 1}{n + 1} |z_j - w_j|.
        \end{aligned}
    \end{equation*}
\end{proof}

\begin{lm}  \label{lm:17.10}
    Ako je $\niz{\widetilde{z_n}}{n \in \nat} \subseteq \masC$ i $\widetilde{z_n} \to z \in \masC$, tada je
    \begin{equation*}
        \lim\limits_{n \to \infty} \Big( 1 + \frac{\widetilde{z_n}}{n} \Big)^n = e^z.
    \end{equation*}
\end{lm}

\begin{proof}
    U lemi \ref{lm:17.9} $w_1 = \ldots = w_n = e^\frac{\widetilde{z_n}}{n}$, $z_1 = \ldots = z_n = \Big[ \frac{\widetilde{z_n}}{n} + 1 \Big]$

    Za $\gamma > |z|$ imamo $|\widetilde{z_n}| < \gamma$ i $\Big| \frac{\widetilde{z_n}}{n} \Big| \leq 1$, za $n$ dovoljno velik.
    Slijedi
    \begin{equation*}
        \Big| \Big( 1 + \frac{\widetilde{z_n}}{n} \Big)^n - e^{\widetilde{z_n}} \Big| \leq \Big(1 + \frac{\gamma}{n} \Big)^{n-1} \suma{j = 1}{n} \Big| e^{\frac{\widetilde{z_n}}{n}} - \Big(1 + \frac{\widetilde{z_n}}{n}\Big) \Big|.
    \end{equation*}
    Po lemi \ref{lm:17.5} imamo
    \begin{equation*}
        \begin{gathered}
            \Big| e^{\frac{\widetilde{z_n}}{n}} - \Big( 1 + \frac{\widetilde{z_n}}{n} \Big) \Big| \leq 2 \cdot \frac{\Big| \frac{\gamma}{n}\Big|^2}{2!} = \frac{\gamma^2}{n^2}\\
            \implies \Big| \Big( 1 + \frac{\widetilde{z_n}}{n} \Big)^n - e^{\widetilde{z_n}} \Big| \leq \underbrace{\Big( 1 + \frac{\gamma}{n} \Big)^{n - 1}}_{\leq e^\gamma} \cdot \gamma^2 \cdot n \cdot \frac{1}{n^2} \xrightarrow[n \to \infty]{} 0.
        \end{gathered}
    \end{equation*}
\end{proof}

Prvo od fundamentalnih pitanja koje nas zanima je vrijedi li obrat u \eqref{jed:17.2}.
Budu\' ci je $C(F_X) \cap C(F_Y)$ gust u $\real$, sljede\' ci teorem daje potvrdan odgovor na to pitanje.

\begin{tm}[L\' eviyjev teorem inverzije]  \label{tm:17.11}
    Za svaki $a, b \in C(F_X)$, $a < b$, vrijedi
    \begin{equation*}
        F_X (b) - F_X (a) = \lim\limits_{T \to \infty} \frac{1}{2 \pi} \int_{-T}^T \frac{e^{ita} - e^{itb}}{it} \cdot \varphi_X (t) \: d t.
    \end{equation*}
\end{tm}