% 5. dio poglavlje 2, 17. predavanje karakteristične funkcije

\chapter{Karakteristi\v cne funkcije}

Fourierova transforamcija je vrlo korisna i u vjerojatnosti.
U ovom poglavlju $X$ uvijek ozna\v cava slu\v cajnu varijablu na vjerojatnostnom prostoru $\vjerojatnosniProstor$.

\begin{defn}    \label{def:17.1}
    Funkcija $\varphi_X :  \real \to \masC$, definirana sa
    \begin{equation*}
        \varphi_X (t) := \masE [e^{itX}] = \masE[\cos (tX)] + i \cdot \masE [ \sin (tX) ], \quad t \in \real,
    \end{equation*}
    naziva se \emph{karakteristi\v cnom funkcijom} slu\v cajne varijable $X$.
\end{defn}

Lako se vidi da je
\begin{equation*}
    \varphi_X (t) = \int_{-\infty}^{+\infty} e^{itX} \: d F_X (x) = \int_{-\infty}^{+ \infty} \cos (t X)  \: d F_X (x) + i \cdot \int_{-\infty}^{+\infty} \sin (t X) \: d F_X (x),
\end{equation*}
te da je $\varphi_X (t)$ uvijek definiran.
Ako je $F \in \inc$, $F^R = F$ i $\Delta F < +\infty$, ista definicija daje funkciju $\varphi_F : \real \to \masC$, koja \' ce imati svojstva analogna onima od $\varphi_X$.

Lako se vidi da za slu\v cajnu varijablu $Y$ i $t \in \real$ vrijedi:
\begin{equation}    \label{jed:17.2}
    X \distJed Y \implies \varphi_X = \varphi_Y,
\end{equation}
te
\begin{equation}    \label{jed:17.3}
    \varphi_X(-t) = \overline{\varphi_X (t)}, \quad \quad |\varphi_X (t)| \leq \varphi_X(0) = 1.
\end{equation}

Uo\v cimo da je
\begin{equation*}
    |\varphi_X (t+h) - \varphi_X(t)| \leq \int_{-\infty}^{+\infty} |e^{i h X} - 1| \: d F_X (x).    
\end{equation*}
Budu\' ci da $|e^{ihX} -1| \xrightarrow[h \to 0]{} 0$ i uniformno je ome\dj ena s $2$ (\v sto je integrabilna funkcija u odnosu na $F_X$) iz LTDK slijedi
\begin{equation}    \label{jed:17.4}
    \varphi_X \textnormal{ je uniformno neprekidna na } \real.
\end{equation}

Za $z \in \masC$ vrijedi
\begin{equation*}
    e^z = \suma{n = 0}{\infty} \frac{z^n}{n !},
\end{equation*}
pa \' ce svojstva $\varphi_X$ ovisiti o postojanju momenata od $X$.

\begin{lm}  \label{lm:17.5}
    Za svaki $x \in \real$ i $n \in \nat \cup \{ 0 \}$ vrijedi
    \begin{equation*}
        \Big| e^{ix} - \suma{j = 0}{n} \frac{(ix)^j}{j !} \Big| \leq \min \Big\{ \frac{|x|^{n + 1}}{(n + 1)!}, \frac{2 \cdot |x|^n}{n !} \Big\}.
    \end{equation*}
\end{lm}

\begin{proof}
    Po Taylorovom razvoju
    \begin{equation*}
        \begin{aligned}
            e^{ix} - \suma{j = 0}{n} \frac{(i x)^j}{j !} &= \frac{(i x)^{n + 1}}{n !} \int_0^1 e^{i x u} (1 - u)^n \: du =
            \begin{pmatrix}
                \textnormal{pracijalna}\\
                \textnormal{integracija}
            \end{pmatrix}\\
            &= - \frac{(ix)^n}{n !} + \frac{(i x)^n}{(n - 1) !} \int_0^1 e^{i x u} (1 - u)^{n - 1} \: d u.
        \end{aligned}
    \end{equation*}
    Uo\v cimo
    \begin{equation*}
        \int_0^1 (1 - u)^k \: d u = \frac{(1 - u)^{k + 1}}{k + 1} (-1) \Bigg|_0^1 = \frac{1}{(k + 1)}.
    \end{equation*}
    Zbog
    \begin{equation*}
        |e^{ixu}| = 1,
    \end{equation*}
    prva jednakost daje prvu ocjenu, a druga jednakost drugu ocjenu.
\end{proof}

\begin{lm}  \label{lm:17.6}
    Ako je $X \in L^n (\masP)$, za neki $n \in \nat$, tada je $\varphi$ $n$-puta diferencijabilno i
    \begin{equation*}
        \varphi_X^{(n)} (t) = \int_{-\infty}^{+\infty} (i x)^n e^{i t x} \: d F_X (x).
    \end{equation*}
\end{lm}

\begin{proof}
    Indukcijom po $n$, pri \v cemu korak ide kao i baza,
    \begin{equation*}
        \frac{\varphi_X (t + s) - \varphi_X (t)}{s} = \masE \Big[ e^{i t X} \cdot \frac{e^{s X} - 1}{s} \Big].
    \end{equation*}
    Uo\v cimo da je
    \begin{equation*}
        \frac{e^{i s X} - 1}{s} \xrightarrow[s \to 0]{} i X,
    \end{equation*}
    i ome\dj en je sa $|X|$ (po lemi \ref{lm:17.5}).
    Primjenom LTDK slijedi
    \begin{equation*}
        \lim\limits_{s \to 0} \frac{\varphi_X (t + s) - \varphi_X (t)}{s} = \masE [i X \cdot e^{i t X}] = \int_{-\infty}^{+\infty} i x e^{i t x} \: d F_X (x).
    \end{equation*}
\end{proof}

Iz leme \ref{lm:17.5} i leme \ref{lm:17.6} direktno dobijemo

\begin{kor} \label{kor:17.7}
    Neka je $n \in \nat \cup \{0\}$.
    Tada vrijedi:
    \begin{enumerate}[label=(\roman*)]
        \item \label{kor:17.7.1}
        \begin{equation*}
            \masE \big[ |X|^2 \big] < +\infty \implies \varphi_X^{(n)}(0) = \masE \big[ (i \cdot X)^n \big],
        \end{equation*}
        \item \label{kor:17.7.2}
        \begin{equation*}
            \begin{aligned}
                \masE \big[|X|^{n + 1} \big] < +\infty \implies \Big| \varphi_X (t) - \suma{j = 0}{n} \frac{(it)^j}{j !} \cdot \masE \big[ X^j \big]  \Big| \leq
                \min \Big\{ &\frac{|t|^{n+1}}{(n + 1)!} \cdot \masE \big[ |X|^{n + 1} \big],\\
                &\frac{2 \cdot |t|^n}{n !} \cdot \masE \big[ |X|^n \big] \Big\},   
            \end{aligned}
        \end{equation*}
        \item \label{kor:17.7.3}
        \begin{equation*}
            \begin{aligned}
                \masE \big[ |X|^n \big] < +\infty \implies \varphi_X (t) &= \suma{j = 0}{n} \frac{(i t)^j}{j!} \cdot \masE \big[ X^j \big] + o (t^n)\\
                &= \suma{j = 0}{n} \frac{t^j}{j!} \cdot \varphi_X^{(j) } (0) + o(t^n).
            \end{aligned}
        \end{equation*}
    \end{enumerate}
\end{kor}

Uo\v cimo da za svaka \v cetiri kompleksna broja vrijedi
\begin{equation}    \label{jed:17.8}
    |\alpha_1 \alpha_2 - \beta_1 \beta_2| \leq |\alpha_1| \cdot | \alpha_2 - \beta_2| + |\beta_2| \cdot | \alpha_1 - \beta_1|.
\end{equation}

\begin{lm}  \label{lm:17.9}
    Neka je $n \in \nat$, $z_1, \ldots, z_n \in \masC$, $w_1, \ldots, w_n \in \masC$ takvi da je
    \begin{equation*}
        \max \big\{ |z_1|, \ldots, |z_n|, |w_1|, \ldots, |w_n| \big\} \leq \alpha.
    \end{equation*}
    Tada je
    \begin{equation*}
        \Big| \produkt{j = 1}{n} z_j - \produkt{j = 1}{n} w_j \Big| \leq \alpha^{n - 1} \cdot \suma{j = 1}{n} |z_j - w_j|.
    \end{equation*}
\end{lm}

\begin{proof}
    Indukcijom po $n$.
    Slu\v caj $n = 1$ je trivijalan.
    
    Koriste\' ci nejednakost \eqref{jed:17.8}, imamo
    \begin{equation*}
        \begin{aligned}
            \Big| \Big( \produkt{j = 1}{n} z_j \Big) \cdot z_{n + 1} - \Big( \produkt{j = 1}{n} w_j \Big) \cdot w_{n + 1} \Big| &\leq \alpha^n | z_{n+1} - w_{n+1} | + \alpha \cdot \Big|  \produkt{j = 1}{n} z_j - \produkt{j = 1}{n} w_j \Big|\\
            &\leq
            \begin{pmatrix}
                \textnormal{pretpostavka}\\
                \textnormal{indukcije}
            \end{pmatrix}\\
            &\leq \alpha^n \cdot \suma{j = 1}{n + 1} |z_j - w_j|.
        \end{aligned}
    \end{equation*}
\end{proof}

\begin{lm}  \label{lm:17.10}
    Ako je $\niz{\widetilde{z_n}}{n \in \nat} \subseteq \masC$ i $\widetilde{z_n} \to z \in \masC$, tada je
    \begin{equation*}
        \lim\limits_{n \to \infty} \Big( 1 + \frac{\widetilde{z_n}}{n} \Big)^n = e^z.
    \end{equation*}
\end{lm}

\begin{proof}
    U lemi \ref{lm:17.9} $w_1 = \ldots = w_n = e^\frac{\widetilde{z_n}}{n}$, $z_1 = \ldots = z_n = \Big[ \frac{\widetilde{z_n}}{n} + 1 \Big]$

    Za $\gamma > |z|$ imamo $|\widetilde{z_n}| < \gamma$ i $\Big| \frac{\widetilde{z_n}}{n} \Big| \leq 1$, za $n$ dovoljno velik.
    Slijedi
    \begin{equation*}
        \Big| \Big( 1 + \frac{\widetilde{z_n}}{n} \Big)^n - e^{\widetilde{z_n}} \Big| \leq \Big(1 + \frac{\gamma}{n} \Big)^{n-1} \suma{j = 1}{n} \Big| e^{\frac{\widetilde{z_n}}{n}} - \Big(1 + \frac{\widetilde{z_n}}{n}\Big) \Big|.
    \end{equation*}
    Po lemi \ref{lm:17.5} imamo
    \begin{equation*}
        \begin{gathered}
            \Big| e^{\frac{\widetilde{z_n}}{n}} - \Big( 1 + \frac{\widetilde{z_n}}{n} \Big) \Big| \leq 2 \cdot \frac{\Big| \frac{\gamma}{n}\Big|^2}{2!} = \frac{\gamma^2}{n^2}\\
            \implies \Big| \Big( 1 + \frac{\widetilde{z_n}}{n} \Big)^n - e^{\widetilde{z_n}} \Big| \leq \underbrace{\Big( 1 + \frac{\gamma}{n} \Big)^{n - 1}}_{\leq e^\gamma} \cdot \gamma^2 \cdot n \cdot \frac{1}{n^2} \xrightarrow[n \to \infty]{} 0.
        \end{gathered}
    \end{equation*}
\end{proof}

Prvo od fundamentalnih pitanja koje nas zanima je vrijedi li obrat u \eqref{jed:17.2}.
Budu\' ci je $C(F_X) \cap C(F_Y)$ gust u $\real$, sljede\' ci teorem daje potvrdan odgovor na to pitanje.

\begin{tm}[L\' eviyjev teorem inverzije]  \label{tm:17.11}
    Za svaki $a, b \in C(F_X)$, $a < b$, vrijedi
    \begin{equation*}
        F_X (b) - F_X (a) = \lim\limits_{T \to \infty} \frac{1}{2 \pi} \int_{-T}^T \frac{e^{ita} - e^{itb}}{it} \cdot \varphi_X (t) \: d t.
    \end{equation*}
\end{tm}

\begin{proof}
    Uo\v cimo najprije da je $\varphi_X$ neprekidna funkcije.
    Ozna\v cimo
    \begin{equation*}
        I_T := \int_{-T}^T \frac{e^{ita} - e^{itb}}{it} \cdot \varphi_X (t) \: d t.
    \end{equation*}
    Primjetimo da vrijedi
    \begin{equation*}
        \Big| \frac{1}{i t} (e^{- i t a} - e^{- i t b}) \cdot e^{i t X} \Big| \leq \Big| \int_a^b e^{-ity} \: d y \Big| \leq b - a,
    \end{equation*}
    stoga mo\v zemo iskoristiti Fubinijev teorem:
    \begin{equation*}
        \begin{aligned}
            T_T &= \frac{1}{2 \pi} \masE \Big[ \int_{-T}^T \frac{e^{it(X - a)} - e^{it(X - b)}}{it} \cdot \varphi_X (t) \: d t \Big]\\
            &= \frac{1}{\pi} \masE \Big[ \int_0^T \frac{1}{t} \big( \sin t(X-a) - \sin t(X-b) \big) \: d t \Big]\\
            &= \frac{1}{\pi} \masE \Big[ \int_0^{T(X-a)} \frac{\sin t}{t} \: d t - \int_0^{T (X-b)} \frac{\sin t}{t} \: d t \Big].
        \end{aligned}
    \end{equation*}
    Budu\' ci je
    \begin{equation*}
        \lim\limits_{u \to \infty} \int_0^u \frac{\sin x}{x} \: d x = \frac{\pi}{2},
    \end{equation*}
    dobivamo
    \begin{equation*}
        \frac{1}{\pi} \int_{T (u - b)}^{T (u - a)} \frac{\sin t}{t} \: dt \xrightarrow[T \to + \infty]{}
        \begin{cases}
            1,    &a < u < b\\
            \frac{1}{2}, &u \in \{a, b\}\\
            0,    &u < a \textnormal{ ili } u > b
        \end{cases}.
    \end{equation*}
    Uo\v cimo,
    \begin{equation*}
        \Big| \frac{1}{\pi} \int_{T (u- b)}^{T (u - a)} \frac{\sin t}{t} \: d t \Big| \leq 2,
    \end{equation*}
    pa prema LTDK vrijedi
    \begin{equation*}
        \begin{aligned}
            \lim\limits_{T \to \infty} I_T &= \masE \Big[ \lim\limits_{T \to \infty} \Big( \frac{1}{\pi} \int_0^{T (u - a)} \frac{\sin t}{t} \: dt - \frac{1}{\pi} \int_0^{T (u - b)} \frac{\sin t}{t} \: d t \Big) \Big]\\
            &= \masE \Big[ \frac{1}{2} \karaktFja_{\big\{ X \in \{a, b\} \big\}} + \karaktFja_{\{a < X < b\}} \Big]\\
            &= \frac{1}{2} (\masP (X = a) + \masP (X = b) ) + \masP (a < X < b) =
            \begin{psmallmatrix}
                a, b \in C(F_X)\\
                \Downarrow\\
                \masP_X (\{ a \}) = \masP_X (\{ b \}) = 0
            \end{psmallmatrix}\\
            &= \masP ( a < X \leq b) = F_X (b) - F_X (a).
        \end{aligned}
    \end{equation*}
\end{proof}

Osnovni ra\v cun s karakteristi\v cnim funkcijama je jednostavan.
Uo\v cimo da je
\begin{equation*}
    \masE \big[ e^{it (aX+b)} \big] = e^{itb} \cdot \masE \big[ e^{i (ta) X} \big],
\end{equation*}
pa za $a, b \in \real$, slijedi
\begin{equation}    \label{jed:17.12}
    \varphi_{aX+b} (t) = e^{itb} \cdot \varphi_X (a t).
\end{equation}
Ako su $X, Y$ nezavisne, tada su i $e^{itX}, e^{itY}$ nezavisne:
\begin{equation}    \label{jed:17.13}
    \begin{aligned}
        \varphi_{X+Y} (t) &= \masE \big[ e^{it(X+Y)} \big] = \masE [e^{itX} \cdot e^{itY}] = (\textnormal{nezavisnost})\\
        &= \masE \big[ e^{itX} \big] \cdot \masE \big[ e^{itY} \big] = \varphi_X (t) \cdot \varphi_Y (t).
    \end{aligned}
\end{equation}

\begin{zad} \label{zad:17.14}
    Razvijte pojam karakteristi\v cne funckcije slu\v cajnog vektora.
    Iska\v zite i doka\v zite analogone teorema \ref{tm:17.11}, \eqref{jed:17.12} i \eqref{jed:17.13}.
\end{zad}

\begin{tm}  \label{tm:17.15}
    Slu\v cajne varijable $X_1, \ldots, X_n$ su nezavisne ako i samo ako vrijedi, za svake $(t_1, \ldots, t_n) \in \real^n$,
    \begin{equation*}
        \varphi_{(X_1, \ldots, X_n)} (t_1 , \ldots, t_n) = \produkt{k = 1}{n} \varphi_{X_k} (t_k).
    \end{equation*}
\end{tm}

\begin{proof}
    \begin{itemize}
        \item[$\implies$]
        Slijedi po zadatku \ref{zad:17.14} i nezavisnosti, jer
        \begin{equation*}
            \begin{aligned}
                \varphi_{(X_1, \ldots, X_n)} (t_1, \ldots, t_n) &= \masE \Big[ e^{i \suma{k = 1}{n} t_k X_k} \Big] = \masE \Big[ \produkt{k = 1}{n} e^{i t_k X_k} \Big]\\
                &= \produkt{k = 1}{n} \masE \Big[ e^{i t_k X_k} \Big] = \produkt{k = 1}{n} \varphi_{X_k} (t_k).
            \end{aligned}
        \end{equation*}
        \item[$\impliedby$]
        Neka je $\masX = (X_1, \ldots, X_n)$ i
        \begin{equation*}
            H (x_1, \ldots, x_n) := \produkt{k = 1}{n} F_{X_k} (x_k),
        \end{equation*}
        postoji vjerojatnosna mjera $\masP_H$ na $\real^n$ \v cija funkcija distribucije je $H$.
        Po pretpostavci slijedi
        \begin{equation*}
            \varphi_{(X_1, \ldots, X_n)} = \Big( = \varphi_{F_\masX} \Big) = \varphi_H.
        \end{equation*}
        Pa po teoremu inverzije (teorem \ref{tm:17.11}) i zadatku \ref{zad:17.14} dobivamo da je $F_\masX = H$, \v sto povla\v ci nezavisnost.
    \end{itemize}
\end{proof}

\begin{zad} \label{zad:17.16}
    Izra\v cunajmo karakteristi\v cne funkcije osnovnih distribucija prema prilo\v zenoj tablici.
\end{zad}

\begin{nap} \label{nap:17.17}
    Prirodno pitanje koje se javlja vezano uz karakteristi\v cne funkcije je koji su nu\v zni i dovoljni uvjeti da $\varphi : \real \to \masC$ bude karakteristi\v cna funkcija neke d.f.F.
    Postoje teorijski rezultati koje ovdje ne\' cemo dokazivati niti dalje koristiti (vidi N. Sarapa str. 490).
    Nu\v zan i dovoljan uvijet je da je $\varphi$ neprekidna u nuli i \emph{pozitivno semidefinitna}, to jest, da za svaki $n \in \nat$, $t_1, \ldots, t_n \in \real$, $\alpha_1, \ldots, \alpha_n \in \masC$ vrijedi
    \begin{equation}    \label{jed:17.18}
        \suma{j = 1}{n} \suma{k = 1}{n} \varphi(t_j - t_k) \alpha_j \overline{\alpha_k} \geq 0.
    \end{equation} 
\end{nap}

Uo\v cimo da su Poissonova, normalna te Cauchyjeva primjeri razdioba \v cije karakteristi\v cne funkcije su eksponencijalnog tipa (posebno, nije nikad nula).
Kako bismo mogli bolje opisati razdijobe ovakvog tipa, dobro je poznavati pojam kompleksnog logaritma.
Op\' cenito imamo razne mogu\' cnosti za izbor logaritma, ali mi \' cemo uzeti onaj koji se naziva \emph{glavnom vrijedno\v s\' cu}.
Preciznije, ako je $\varphi : \real \to \masC \setminus \{0\}$ neprekidna funkcija i $\varphi (0) \in \obInt{0}{+ \infty}$, tada postoji to\v cno jedna neprekidna funkcija $\alpha : \real \to \real$ takva da je $\alpha (0) = 0$ i vrijedi $\varphi (t) = | \varphi (t) | e^{i \alpha (t)}$.
Tada definiramo
\begin{equation}    \label{jed:17.19}
    \ln \varphi (t) := \ln | \varphi (t) | + i \alpha (t).
\end{equation}
Uo\v cimo da vrijedi
\begin{equation}    \label{jed:17.20}
    \varphi (t) = e^{\ln \varphi (t)}, \quad \forall t \in \real.
\end{equation}
Koriste\' ci neprekidnost mo\v ze se pokazati da za niz karatkeristi\v cnih funkcija $(\varphi_n)$ koje nisu nula i $\varphi$ karakteristi\v cna funkcija koja nije nula vrijedi
\begin{equation}    \label{jed:17.21}
    \varphi_n (t) \xrightarrow[n \to \infty]{} \varphi (t), \; \forall t \in \real \iff \ln \varphi_n (t) \xrightarrow[n \to \infty]{} \ln \varphi (t), \; \forall t \in \real.
\end{equation}

\begin{defn}    \label{defn:17.22}
    Re\' ci \' cemo da je karakteristi\v cna funkcija $\varphi$ \emph{beskona\v cno djeljiva} ako za svaki $n \in \nat$ postoji karakteristi\v cna funkcija $\varphi_n$ takva da je
    \begin{equation*}
        \varphi (t) = [\varphi_n (t)]^n \quad \forall t \in \real.
    \end{equation*}
\end{defn}

\begin{zad} \label{zad:17.23}
    Doka\v zite da su karatkeristi\v cne funkcije sljede\' cih varijabli beskona\v cno djeljive:
    \begin{itemize}
        \item degenerirana
        \item Poissonova
        \item normalna
        \item Cauchyjeva
        \item Gama-razdioba.
    \end{itemize}
\end{zad}

Ako je $\varphi$ beskona\v cno djeljiva karaktersiti\v cna funkcija i $n \in \nat$, onda mo\v zemo uzeti nezavisne, jednako distribuirane slu\v cjane varijable $X_1, \ldots, X_n$, takve da je $\varphi_{X_k} = \varphi_n$, $k = 1, \ldots, n$.
Neka je $X = X_1 + \ldots + X_n$.
Tada je
\begin{equation}    \label{jed:17.24}
    \varphi = \varphi_X.
\end{equation}

Ako su $\varphi$ i $\psi$ beskona\v cno djeljive karaktersiti\v cne funkcije, onda mo\v zemo uzeti nezavisne $X$ i $Y$ takve da je $\varphi = \varphi_X$ te $\psi = \varphi_Y$.
Neka je $Z = X + Y$.
Tada je $\varphi \cdot \psi = \varphi_Z$, a iz \eqref{jed:17.24} se lako vidi da je $\varphi_Z$ i beskona\v cno djeljiva.
Dakle
\begin{equation}    \label{jed:17.25}
    \varphi, \psi \textnormal{ beskona\v cno djeljive k.f.} \implies \varphi \cdot \psi \textnormal{ beskona\v cno djeljiva k.f.}
\end{equation}

\begin{zad} \label{zad:17.26}
    Ako je $\varphi$ beskona\v cno djeljiva karakteristi\v cna funkcija, tada je i $|\varphi|$ beskona\v cno djeljiva karakteristi\v cna funkcija.
\end{zad}