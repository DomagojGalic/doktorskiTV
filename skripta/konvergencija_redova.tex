% poglavlje 13 - konvergencija redova

\chapter{Konvergencija redova}

U ovom poglavlju $\niz{X_n}{n \in \nat}$ je niz nezavisnih slu\v cajnih varijabli na vjerojatnosnom prostoru $\vjerojatnosniProstor$ i
\begin{equation*}
    S_n := X_1 + \ldots + X_n, \quad n \in \nat.
\end{equation*}
\v Sto mo\v zemo re\' ci o konvergenciji reda $\suma{n \in \nat}{} X_n \; (g.s.)$ i po vjerojatnosti?
Iz Kolmogorovljevog zakona 0-1 slijedi (teorem \ref{tm:9.9}):
\begin{equation}    \label{jed:13.1}
    \suma{n = 1}{\infty} X_n \quad \textnormal{ konvergira } (g.s.) \textnormal{ ili divergira } (g.s.)
\end{equation}
Nas zanima slu\v caj konvergencije.
Ako $\suma{n = 1}{\infty} X_n$ konvergira $(g.s.)$, tada $X_n \xrightarrow[n \to \infty]{g.s.} 0$, pa zbog nezavisnosti i po teoremu \ref{tm:11.9} slijedi:
\begin{equation}    \label{jed:13.2}
    \suma{n = 1}{\infty} X_n \; \textnormal{ konvergira } (g.s.) \implies \suma{n = 1}{\infty} \vjeroj{|X_n| > 1} < +\infty.
\end{equation}

Odavde slijedi da $\iid$ slu\v caj nije previ\v se zanimljiv.
Preciznije:
\begin{equation}    \label{jed:13.3}
    (X_n) \; \iid \quad \suma{n = 1}{\infty} X_n \; \textnormal{ konvergira } (g.s.) \implies X_n = 0 \; (g.s.), \quad \forall n \in \nat. 
\end{equation}

\begin{nap} \label{nap:13.4}
    Ako izbacimo jednaku distribuiranost, a zadr\v imo nezavisnost, mo\v zemo dobiti bilo kakav limes reda.
    Na primjer, neka je $X_1 = Y$, pri \v cemu je $Y$ proizvoljna, zadana slu\v cajna varijabla i $X_n \equiv 0$, za $n \geq 2$.
    Tada je $\niz{X_n}{n \in \nat}$ nezavisan niz i
    \begin{equation*}
        \suma{n = 1}{\infty} X_n = Y.
    \end{equation*}
\end{nap}

\begin{nap} \label{nap:13.5}
    Ideja "rezanja" slu\v cajnih varijabli mo\v ze pomo\' ci da se ostvari nu\v zan uvijet konvergencije iz \eqref{jed:13.2}.
    Uz to je vezan va\v zan pojam ekvivalentnih nizova.
\end{nap}

\begin{defn}    \label{defn:13.6}
    Ka\v zemo da su nizovi slu\v cajnih varijabli $\niz{X_n}{n \in \nat}$ i $\niz{Y_n}{n \in \nat}$ na vjerojatnosnom prostoru $\vjerojatnosniProstor$ su \emph{ekvivalentni} ako vrijedi:
    \begin{equation*}
        \suma{n = 1}{\infty} \vjeroj{X_n \neq Y_n} < +\infty.
    \end{equation*}
\end{defn}

Ako su nizovi ekvivalentni, onda iz Borel-Cantellijeve leme (lema \ref{lm:9.2}) slijedi
\begin{equation*}
    \vjeroj{X_n \neq Y_n, \; \io} < +\infty,
\end{equation*}
\v sto daje da za ekvivalentne nizove $(X_n)$ i $(Y_n)$ vrijedi:
\begin{equation}    \label{jed:13.7}
    \suma{n = 1}{\infty} X_n \quad \textnormal{ konvergira } (g.s.) \quad \iff \quad \suma{n = 1}{\infty} Y_n \quad \textnormal{ konvergira } (g.s).
\end{equation}

Nadalje, uo\v cimo da vrijedi:
\begin{equation}    \label{jed:13.8}
    \begin{matrix}
        (X_n) \textnormal{ i } \big(X_n \cdot \karaktFja_{\{ |X_n| \leq 1 \}}\big)\\
        \textnormal{su ekvivalentni}
    \end{matrix}
    \quad \iff \quad
    \suma{n = 1}{\infty} \vjeroj{|X_n|>1} < +\infty.
\end{equation}
Dakle, za konvergenciju $(g.s.)$ osnovno pitanje postaje koji su nu\v zni i dovoljni uvijeti da $\suma{n = 1}{\infty} X_n$ konvergira $(g.s.)$?
\v Sto mo\v zemo re\' ci o konvergenciji po vjerojatnosti?
Sljede\' ci teorem pokazuje da su kod redova nezavisnih slu\v cajnih varijabli ove konvergencije jednake.

\begin{tm}[P. L\' evy]  \label{tm:13.9}
    Red $\suma{n = 1}{\infty} X_n$ konvergira gotovo sigurno ako i samo ako konverigra po vjerojatnosti.
\end{tm}

\begin{proof}
    O\v cito je dovoljno dokazati da ako $(S_n)$ konvergira po vjerojatnosti, tada konvergira i gotovo sigurno.
    Uvedimo oznaku
    \begin{equation*}
        S_{h, n} := S_n - S_h, \quad \forall h, n \in \nat.
    \end{equation*}
    Koriste\' ci lemu \ref{lm:11.18} dobivamo da za svaki $0 < \varepsilon < \frac{1}{4}$, postoji $h_0 \in \nat$, takav da vrijedi
    \begin{equation*}
        n, h \in \nat, \quad n > h \geq h_0 \implies \vjeroj{|S_{h, n}| Y \varepsilon} < \varepsilon.
    \end{equation*}
    Odavde slijedi $|m(S_{h, n})| \leq \varepsilon$, pa koriste\' ci L\' evyjeve nejednakosti (lema \ref{lm:12.8}) i nezavisnost od $(X_n)$ dobivamo
    \begin{equation*}
        % možda je m(S_{h, k})
        \begin{aligned}
            \masP \Big( \max\limits_{h < n \leq k} |S_{h, n}| > 2 \varepsilon \Big) &= \masP \Big( \max\limits_{h < n \leq k} |S_{h, n}| > 2 \varepsilon, \; \max\limits_{h < n \leq k} |m(S_{k, h})| \leq \varepsilon \Big)\\
            &\leq \masP \Big( \max\limits_{h < n \leq k} |S_{h, n} - m(|S_{h, n} - S_{h, k}|)| > \varepsilon \Big)\\
            &\leq 2 \: \vjeroj{|S_{h, n}| > 2 \varepsilon}\\
            &< 2 \: \varepsilon. 
        \end{aligned}
    \end{equation*}
    Neka je $h_0$ fiksan i neka $h \nearrow +\infty$.
    Sada dobivamo
    \begin{equation*}
        \masP \Big( \sup\limits_{n > h} |S_{h, n}| > 2 \: \varepsilon \Big) \leq 2 \: \varepsilon.
    \end{equation*}
    Pa tvrdnja slijedi iz zadatka \ref{zad:11.20}.
\end{proof}

Sljede\' ci tehni\v cki rezultat je izrazito koristan.

\begin{lm}[Abelova lema]    \label{lm:13.10}
    Neka su $\niz{a_n}{n \in \nat}$ i $\niz{b_n}{n \in \nat}$ nizovi realnih brojeva.
    Za $n \in \nat_0$, neka su
    \begin{equation*}
        \begin{aligned}
            A_n &:= \suma{j = 0}{n} a_j \quad (n + 1) \textnormal{-va parcijana suma}\\
            A_n^* &:= \suma{j = n + 1}{\infty} a_j \quad \textnormal{ostatak reda, ako red konvergira}.
        \end{aligned}
    \end{equation*}
    Tada vrijedi:
    \begin{enumerate}[label=(\roman*)]
        \item Za svaki $n \in \nat$, vrijedi
        \begin{equation*}
            \suma{j = 1}{n} a_j b_j = A_n b_n - A_0 b_1 - \suma{j = 1}{n - 1} A_j (B_{j + 1} - b_j).
        \end{equation*}
        \item ako $\suma{n}{} a_n$ konvergira, tada za svaki $n \in \nat$ vrijedi:
        \begin{equation*}
            \suma{j = 1}{n} a_j b_j = A_0^* b_j - A_n^* b_n + \suma{j = 1}{n - 1} A_j^* (b_{j + 1} - b_j). 
        \end{equation*}
    \end{enumerate}
\end{lm}

\begin{proof}
    \begin{equation*}
        \begin{aligned}
            \suma{j = 1}{n} a_j b_j &= \suma{j = 1}{n} (A_j - A_{j - 1}) b_j = \suma{j = 1}{n} A_j b_j - \suma{k = 1}{n - 1} A_k b_{k + 1}\\
            &= A_n b_n - A_0 b_1 - \suma{j = 1}{n - 1} (b_{j + 1} - b_j).
        \end{aligned}
    \end{equation*}
    Ako je $\suma{j = 0}{\infty} a_j = A \in \real$, tada je za svaki $n \in \nat$ $A = A_n + A_n^*$. Slijedi:
    \begin{equation*}
        \begin{aligned}
            \suma{j = 1}{n} a_j b_j &= (A - A_n^*)b_n - (A - A_0^*) b_1 - \suma{j = 1}{n - 1} (A - A_j^*) (b_{j + 1} - b_j)\\
            &= A_0^* b_1 - A_n^* b_n + \suma{j = 1}{n - 1} A_j^* (b_{j + 1} - b_j) + A b_n - A b_1 - A \cdot \underbrace{\suma{j = 1}{n - 1} (b_{j + 1} - b_j)}_{= b_n - b_1}.
        \end{aligned}
    \end{equation*}
\end{proof}

\begin{lm}[Kroneckerova lema]  \label{lm:13.10-1}
    Ako je $\suma{j = 0}{\infty} a_j = A \in \real$ i $0 < b_n \nearrow +\infty$, tada je
    \begin{equation*}
        \suma{j = 1}{n} a_j b_j = o(b_n).
    \end{equation*}
\end{lm}

\begin{zad} \label{zad:13.11}
    Doka\v zi lemu \ref{lm:13.10-1}.
\end{zad}

Po\v cnimo s pretpostavkom o drugim momentima.

\begin{tm}  \label{tm:13.12}
    Ako je $X_n \in L^2 (\masP)$, za svaki $n \in \nat$ i $\suma{n = 1}{\infty} \Var (X_n) < +\infty$, tada
    \begin{equation*}
        \suma{n = 1}{\infty} (X_n - \masE X_n) \quad \textnormal{ konvergira } (g.s.).
    \end{equation*}
\end{tm}

\begin{proof}
    Bez smanjenja op\' cenitosti mo\v zemo pretpostaviti da je $\masE X_n = 0$, $\forall n \in \nat$, ina\' ce gledamo varijable $Y_n = X_n - \masE X_n$.
    Po teoremu \ref{tm:13.9} dovoljno je dokazati da $(S_n)$ konvergira po vjerojatnosti.
    Ponovo korsitimo lemu \ref{lm:11.18}.
    Po \v Cebi\v sevljevoj nejednakosti slijedi
    \begin{equation*}
        \begin{aligned}
            \vjeroj{|S_m - S_n| > \varepsilon} &\leq \frac{1}{\varepsilon^2} \Var (S_m - S_n) = \frac{1}{\varepsilon^2} \Var \Big( \suma{j = n + 1}{m} X_j \Big) = (\textnormal{nezavisnost})\\
            &= \frac{1}{\varepsilon^2} \suma{j = n + 1}{m} \Var (X_j) \leq \frac{1}{\varepsilon^2} \suma{j = n + 1}{\infty} \Var (X_j).
        \end{aligned}
    \end{equation*}
    A sada zbog $\suma{n = 1}{\infty} X_n < +\infty$ slijedi:
    \begin{equation*}
        \sup\limits_{m > n} \vjeroj{|S_m - S_n| > \varepsilon} \leq \frac{1}{\varepsilon^2} \suma{j = n + 1}{\infty} \Var (X_j) \xrightarrow[n \to \infty]{} 0.
    \end{equation*}
\end{proof}