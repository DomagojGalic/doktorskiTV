% o vjerojatnosnom prostoru


\chapter{Vjerojatnosni prostor}

Sustavno razmi\v sljanje o vjerojatnosnim pojmovima po\v cinje u 16.
stolje\' cu nove ere u Italiji (Cardano, Tartaglia). Ve\' c do 19.
stolje\' ca razvijaju se slo\v zenije ideje, kao uvijetna vjerojatnost,
Laplace-ov model, te prvi grani\v cni teoremi. Odnos intuitivnog
poimanja vjerojatnost i pripadnog matemati\v ckog pojma suptilno
je pitanje. Ilustrirajmo jedan aspekt tog odnosa uspore\dj uju\' ci
pitanje tipa "Koja je vjerojatnost da na Marsu postoji \v zivot?",
s pitanjem "Koja je vjerojatnost da subotom izme\dj u 12 i 13 sati
autoputom Zagreb - Karlovac pro\dj e barem tisu\' cu automobila?".
U prvom slu\' caju htjeli bismo pridru\v ziti odre\dj enu "mjeru"
(dakle broj) stupnju vjerovanja da na Marsu postoji \v zivot.
To vodi na ideju tako zvane \emph{subjektivne vjerojatnosti}
(Keynes 1921.). Uo\v cimo da je uz prvo pitanje vrlo te\v sko
vezati neki pokus, a ne mogu\' ce ideju "ponavljanja pokusa".
S druge strane, u drugom slu\v caju mo\v zemo jednostavno provoditi
mjerenja svake subote (ponavljanje pokusa) i formirati "distribuciju"
vezanu uz taj slu\v cajni pokus. Ovo nas vodi na takozvani
\emph{Objektivni pristup} koji se tako\dj er razvija u prvoj polovici
20. stolje\' ca (von Mises 1928. i Kolmogorov 1933.) Iako ovaj
pristup konceptualno ograni\v cavaju\' ce djeluje na teoriju,
matemati\v cnost ovog pristupa postaje klju\v cnim razlogom njegove
op\' ce prihva\' cenosti.
Osobito je za nas va\v zan Kolmogorovljev pristup teoriji
vjerojatnosti, koji ne razmi\v slja o tome kako pojedinom stvarnom
doga\dj aju dati odre\dj enu dati odre\dj enu vjerojatnost
$p \in [0, \: 1]$, ve\' c uz pretpostavku da takvi brojevi postoje,
razvija pravila o njihovim odnosima. Time sama priroda doga\dj aja
postaje sekundarna u toj teoriji (jasno, u primjenama je i dalje od
prvorazredne va\v znosti), a bitna postaje analiza "distribucije"
i njenih pravila.
S druge strane, za bolje razumjevanje koncepata teorije, a osobito
u njenim primjenama, \v cest uzimamo u obzir konkretne primjere.
Primjeri iz hazardnih igara (od arapskog al-zahr za igra\v cu
kocku) \v cesto donose boljem razumijevanju same teorije. Na
primjer, jako je jednostavno modelirati jedno bacanje simetri\v cne
kovanice ($50\%$ \v sanse za ishod glave i $50\%$ za ishod pisma).
Dvije kovanice ve\' c mogu predstavljati problem.

\begin{pr}[D'Alambert 1754] \label{primjer1}
    Kolika je vjerojatnost da prilikom jednog bacanja dvije
    simetri\v cne kovanice bar jednom "padne pismo"?
\end{pr}

\begin{rj}
    D'Alambert ka\v ze da imamo 3 mogu\' cnosti (dvije glave,
    dva pisma i po jedno od svakoga), od kojih su za nas povoljne
    dvije, stoga je vjerojatnost $^2 / _3$!?
    
    Ovaj primjer pokazuje koliko je va\v zno precizno odrediti
    \emph{osnovne (elementarne) ishode}. Uzmemo li 2G, 2P i 1G1P kao
    osnovne ishode, onda moramo dobro razmisliti koje su njihove
    vjerojatnosti (nisu $^1 / _3$). Zamislimo da smo jednu
    kovanicu obojili. Jesmo li time promjenili pokus? Nismo.
    Sada ishode pokusa mo\v zemo prikazati kao ure\dj ene parove,
    ako obojanu kovanicu stavio na prvo mjesto, vidimo da su
    mogu\' ci ishodi:
    (G, G), (G, P), (P, G), (P, P)
    
    Odakle vidimo da je tra\v zena vjerojatnost zapravo $^3 / _4$.
\end{rj}

Prvi va\v zan objekt je takozvani \emph{prostor elementarnih
doga\dj aja} (sample space). U jeziku teorije skupova dovoljno je
zahtjevati da to bude neprazan skup $\Omega \neq \varnothing$.

U primjeru \ref{primjer1} imamo $ \Omega = \{(G, \: G), (G, \: P),
(P, \: G), (P, \: P)\}$. \v Sto su doga\dj aji? Na primjer doga\dj aj
iz primjera \ref{primjer1} je opisan sa \emph{"palo je barem
jedno pismo"}, to jest, to je podskup $A = \{(G, \: P),(P, \: G),
(P, \: P)\} \subseteq \Omega$.

Dakle, doga\dj aje je prirodno promatrati kao elemente neke familije
$\F \subseteq \partitive{\Omega}$. Zapravo, najbolje bi bilo uzeti
sve podskupove, dakle familiju $\partitive{\Omega}$, kao doga\dj aje.
\v Sto \' ce biti vjerojatnosti? Ve\' c primjeri sugeriraju da su
vjerojatnosti brojevi pridru\v zeni doga\dj ajima. Dosta je o\v cito
da od tih brojeva zahtjevamo bar sljede\' ca svojstva: ako su $A$,
$B$ doga\dj aji tada za vjerojatnosti $\vjeroj{A}$, $\vjeroj{B}$
vrijedi:

\begin{equation} \label{vjerPrav1}
    0 \leq \vjeroj{A} \leq 1
\end{equation}

\begin{align} \label{vjerPrav2}
    \begin{split}
        \vjeroj{\varnothing} &= 0 \\
        \vjeroj{\Omega} &= 1
    \end{split}
\end{align}

\begin{equation} \label{vjerPrav3}
    A \cap B = \varnothing \implies \vjeroj{A \cup B} = \vjeroj{A}
        + \vjeroj{B}
\end{equation}

\begin{pr}[Laplace-ov model] \label{primjer5}
    Neka je $\Omega \neq \varnothing$ kona\v can skup. Ideja je
    modelirati slu\v caj u kojem su svi osnovni ishodi \emph{jednako
    vjerojatni}. Lako se vidi da taj zahtjev zajedno sa pravilima
    \eqref{vjerPrav1}, \eqref{vjerPrav2} i \eqref{vjerPrav3},
    nu\v zno vodi na to\v cno jedan model, onaj u kojem je svaki
    $A \subseteq \Omega$ doga\dj aj i vrijedi:
    \begin{equation} \label{lapVjeroj}
        \vjeroj{A} = \frac{\card{A}}{\card{\Omega}}
    \end{equation}
\end{pr}

U ovom primjeru imamo zadane vjerojatnosti $\vjeroj{\{ \omega\}}
= 1 / {\card{\Omega}}$, za svaki $\omega \in \Omega$ i onda nas
pravilo \eqref{vjerPrav3} vodi do formule \eqref{lapVjeroj}.

To se mo\v ze i poop\' citi, te je svaki model nad kon\v cnim
$\Omega$ mogu\' ce opisati na sljede\' ci na\v cin.

\begin{pr}
    Neka je $\card{\Omega} = n \in \N$. Neka su $p_1, \dots p_n$
    realni brojevi za koje vrijedi:
    \begin{equation}
        p_1, \dots, p_n \in [0, \: 1]
    \end{equation}

    \begin{equation}
        \suma{i = 1}{n} p_i = 1
    \end{equation}

    Tada je sa

    \begin{equation} \label{konacVP}
        \vjeroj{A} = \suma{\omega_i \in A}{} p_i
    \end{equation}
    dana vjerojatnost na $\partitive{\Omega}$ koja zadovoljava
    pravila \eqref{vjerPrav1}, \eqref{vjerPrav2} i \eqref{vjerPrav3}
    i svaka vjerojatnost na $\partitive{\Omega}$ koja zadovoljava
    \eqref{vjerPrav1}, \eqref{vjerPrav2} i \eqref{vjerPrav3} mo\v ze
    se prikazati na ovaj na\v cin.
\end{pr}

Primjetimo da se ovaj pristup direktno prenosi i na prebrojive
$\Omega$ ako dodamo poop\' cenje pravila \eqref{vjerPrav3}:

\begin{equation} \label{vjerPrav4}
    (A_n, \: n \in \N) \subseteq \partitive{\Omega}, \; A_i
        \cap A_j = \varnothing, \; i \neq j \implies
        \vjeroj{\unija{n}{} A_n} = \suma{n}{} \vjeroj{A_n}
\end{equation}

\begin{zad} \label{zadatak12}
    Doka\v zi sljede\' ce tvrdnje.
    \begin{enumerate}[label=(\alph*)]
        \item Doka\v zi da \eqref{vjerPrav1}, \eqref{vjerPrav2}
            i \eqref{vjerPrav4} $\implies$ \eqref{vjerPrav3}
        \item Doka\v zi da \eqref{vjerPrav1}, \eqref{vjerPrav2},
            \eqref{vjerPrav3} i pravilo:
            \begin{align} \label{neprPrazSkup}
                (A_n, \: n \in \N) \subseteq \partitive{\Omega},
                    \; A_1 \supseteq A_2 \supset \dots, \quad
                    \presjek{n}{} A_n
                    = \varnothing \implies \lim_{n \to \infty}
                    \vjeroj{A_n} = 0
             \end{align}
             impliciraju \eqref{vjerPrav4}
    \end{enumerate}
\end{zad}

%%
%% rješenje zadatka 12
%%

\begin{rj}
    \begin{enumerate}[label=(\alph*)]
        \item Definiramo niz $(C_n \: | \: n \in \N)$, sa $C_1 = A,
            \; C_2 = B, \; C_{n + 2} = \varnothing, \; n \in \N$.
            Vidimo da je to niz disjunktnih doga\dj aja.
            Sada vrijedi
            \begin{equation*}
                \vjeroj{\unija{n \in \N}{} C_n}
                \overset{\eqref{vjerPrav4}}{=} \suma{n \in \N}{}
                \vjeroj{C_n} = \vjeroj{C_1} + \vjeroj{C_2}
                + \underbrace{\suma{n = 3}{\infty} \vjeroj{C_n}}_{=0}
            \end{equation*}
            Odakle je lako vidjeti:
            \begin{align*}
                \vjeroj{A \cup B \cup \unija{n = 3}{\infty}
                \varnothing} &= \vjeroj{A \cup B} \\
                &= \vjeroj{A} + \vjeroj{B}
            \end{align*}
        \item Neka je $\niz{A_n}{n \in \N} \subseteq \F$ niz
            disjunktnih doga\dj aja. Definiramo $B_n :=
            \unija{k = n}{\infty} A_k$, te primjetimo da je
            $B_1 \supseteq B_2 \supseteq \dots$, tako\dj er
            primjetimo da je $\presjek{n = 1}{\infty} B_n =
            \varnothing$. Naime prepostavimo da je $x \in
            \presjek{k = n_0}{\infty} B_n$, tada je $x \in B_n
            \; \forall n \in \N$, to jest, $ \forall n \in \N \;
            x \in \unija{k = n}{\infty} A_k$.
            Odabreimo neki $n_0 \in \N$, sada $x \in
            \unija{k = n_0}{\infty} A_k \implies \exists ! \: k_0
            \geq n_0$ takav da $x \in A_{k_0}$, jer je
            $\niz{A_n}{n \in \N}$ familija disjunktnih doga\dj aja,
            ali tada $x \in A_k$, za $k > k_0 \implies x  \in
            \unija{k = n_1}{\infty} A_k, \; n_1 > k_0$ stoga
            $x \in B_n$ za $n > k_0$, \v sto je kontradikcija pa je
            $\presjek{n = 1}{\infty} B_n = \varnothing$.\\
            Sada imamo
            \begin{align*}
                \vjeroj{\unija{k = 1}{\infty} A_k}
                &= \vjeroj{\unija{k = 1}{n} A_k}
                + \vjeroj{\unija{k = n + 1}{\infty} A_k} \\
                &= \suma{k = 1}{n} \vjeroj{A_k}
                + \vjeroj{\unija{k = n + 1}{\infty} A_k}. 
            \end{align*}
            Pu\v stanjem limesa dobijemo:
            \begin{align*}
                \lim_n \vjeroj{\unija{k = n + 1}{\infty} A_k}
                \overset{\eqref{neprPrazSkup}}{=}& \lim_n
                \vjeroj{\presjek{n + 1}{\infty}(
                    \unija{k = n + 1}{\infty} A_k)}\\
                =& 0.
            \end{align*}
            I prelaskom na limes dobijemo tra\v zenu tvrdnju.
    \end{enumerate}
\end{rj}

\begin{pr} \label{primjer14}
    Neka je $\card{\Omega} = \aleph_0$ i poredajmo $\Omega$ u niz
    $\{ \omega_1, \: \omega_2, \dots \}$. Svaki model koji
    zadovoljava \eqref{vjerPrav1}, \eqref{vjerPrav2} i
    \eqref{vjerPrav4} na $\partitive{\Omega}$ mo\v ze se dobiti
    preko niza realnih brojeva $\{p_n, \: n \in \N \}$ tako da
    vrijedi:
    \begin{equation}   % možda promjeniti format ovoga? 
        p_n \in [0, \: 1], \quad \forall n \in \N
    \end{equation}

    \begin{equation}
        \suma{n \in \N}{} p_n = 1
    \end{equation}

    \begin{equation} \label{prebVP}
        \vjeroj{A} = \suma{\omega_n \in A}{} p_n, \quad A
            \subseteq \Omega
    \end{equation}
\end{pr}

To zna\v ci da prakti\v cki sve situacije u kojima je $\Omega$ najvi\v se
prebrojiv mo\v zemo svesti na \emph{vjerojatnosti prostor} $(\Omega,
\: \partitive{\Omega}, \: \Pp)$, pri \v cemu je $\Pp$ opisan
sa \eqref{konacVP} ili \eqref{prebVP}

\v Sto ako je $\Omega$ neprebrojiv?

\begin{pr}
    Neka je $\Omega = \segment{a}{b} \subseteq \R$ i opi\v simo pokus
    "slu\v cajnog odabira to\v cke u $\Omega$". Prirodno je
    o\v cekivati da za $\segment{c}{d} \subseteq \segment{a}{b}$ vrijedi:
    \begin{equation} \label{mjeraDilat}
        \vjeroj{\segment{c}{d}} = \frac{d - c}{b - a}
    \end{equation}
    Te za doga\dj aj $A \subseteq \Omega$ i $x \in \R$, takve da je
    $x + A \subseteq \Omega$, vrijedi
    \begin{equation} \label{mjeraTransl}
        \vjeroj{x + A} = \vjeroj{A}.
    \end{equation}
    Postoji li takav model na $\partitive{\Omega}$
\end{pr}

\begin{tm}
    Ne postoji funkcija $\Pp: \partitive{\segment{a}{b}} \to \R$
    koja zadovoljava \eqref{vjerPrav1}, \eqref{vjerPrav2},
    \eqref{vjerPrav4}, \eqref{mjeraDilat} i \eqref{mjeraTransl}.
\end{tm}

%% možda dodati kao zadatak teorem o osnovnim svojstvima mjere?
%% iz njega slijede sve ove stvari + pokazat da iz normiranosti
%% vjerojatnosti slijedi da je vjerojatnost praznog skupa 0

\begin{proof}
    Bez smanjanja op\' cenitosti uzmemo $\segment{a}{b}
    = \segment{-3}{3}$ i pretpostavimo suprotno, to jest da postoji
    $\Pp$ s tra\v zenim svojstvima. Iz \eqref{vjerPrav1},
    \eqref{vjerPrav2}, \eqref{vjerPrav4} po zadatku \ref{zadatak12}
    slijedi \eqref{vjerPrav3}, te se lako poka\v ze i
    \begin{equation} \label{monotonstVjeroj}
        A \subseteq B \implies \vjeroj{A} \leq \vjeroj{B}
    \end{equation}
    Na $\segment{-1}{1}$ definiramo relaciju ekvivalencije $\sim$
    sa $x \sim y \iff x - y \in \Q$. Po \emph{aksiomu izbora} postoji
    $A \subseteq \segment{-1}{1}$, takav da je za svaki $x \in
    \segment{-1}{1}$ (i pripadni $[x] \in \segment{-1}{1} \big/
    _{\sim}$), $\card{A \cap [x]} = 1$ (u $A$ smo stavili samo jedan
    element iz svake klase ekvivalencije).
    Skup $\Q \cap \segment{-2}{2}$ je prebrojiv pa ga mo\v zemo
    poredati u niz $\{ q_n \: | \: n \in \N \}$.
    Definirajmo $A_n = q_n + A \subseteq \segment{-3}{3}, \; n \in
    \N$. Neka je $H := \unija{n=1}{\infty} A_n \subseteq
    \segment{-3}{3}$. 
    Iz definicije relacije $\sim$ slijedi da su $A_n$ me\dj usobno
    disjunktni, pa \eqref{vjerPrav4} daje:
    \begin{equation*}
        \vjeroj{H} = \suma{n=1}{\infty} \vjeroj{A_n}
            \overset{\eqref{mjeraTransl}}{=} \suma{n=1}{\infty}
            \vjeroj{A}.
    \end{equation*}
    Kada bi $\vjeroj{A} > 0$, onda bi vrijedilo $\suma{n=1}{\infty}
    \vjeroj{A} = + \infty \implies \vjeroj{H} = +\infty$, \v sto je
    kontradikcija sa \v cinjenicom da je $\vjeroj{H} \in
    \segment{0}{1}$, stoga nu\v zno vrijedi $\vjeroj{A} = 0$, pa onda
    i $\vjeroj{H} = 0$.
    Za svaki $x \in \segment{-1}{1}$, vrijedi $A \cap [x] = \{y\},
    \; y \in \segment{-1}{1}, \; x - y \in \Q \cap \segment{-2}{2}$.
    Dakle $x \in q_n + A \subseteq H$, za neki $n$. Pa mora vrijediti
    $\segment{-1}{1} \subseteq H$, pa po \eqref{mjeraDilat} i
    \eqref{monotonstVjeroj} slijedi
    \begin{equation*}
        \frac{1}{3} = \vjeroj{\segment{-1}{1}} \leq \vjeroj{H} = 0,
    \end{equation*}
    \v sto je kontradikcija, stoga funkcija $\Pp$ sa tra\v zenim
    svojstvima ne postoji.
\end{proof}

Priroda problema je takva da nemo\v zemo odustati niti od jednog od
zahtjeva \eqref{vjerPrav1}, \eqref{vjerPrav2}, \eqref{vjerPrav4},
\eqref{mjeraDilat}, \eqref{mjeraTransl}, stoga moramo odustati od
zahtjeva da definiramo funkciju $\Pp$ na $\partitive{\Omega}$.

\v Zelimo sa\v cuvati osnovne operacije i to nas vodi na sljede\' ce
definicije.

\begin{defn}
    Vjerojatnosni prostor je ure\dj ena trojka
    $\vjerojatnosniProstor$ koja se sastoji od nepraznog skupa
    $\Omega$, $\sigma$-algebre $\F$ doga\dj aja na $\Omega$ te
    funkcije $\Pp: \F \to \R$ koja zadovoljava \eqref{vjerPrav1},
    \eqref{vjerPrav2} i \eqref{vjerPrav4}.
    Funkciju $\Pp$ nazivamo \emph{vjerojatnosnom mjerom}.
\end{defn}

\begin{nap}
    \begin{enumerate}[label=(\alph*)]
        \item Podsjetimo se da je $\F \subseteq \partitive{\Omega}$
            $\sigma$-algebra, ako je $\varnothing \in \F$, $\F$ je
            zatvoren na komplemente i prebrojive unije. Posljedica
            toga je da je $\F$ zatvorena na kona\v cne i prebrojive
            upotrebe uobi\v cajenih skupovnih operacija.
            \emph{Doga\dj aji}  su samo oni podskupovi od $\Omega$
            koji su elementi $\sigma$-algebre $\F$.
        \item Ure\dj en par $\izmjerivProstor$ koji se sastoji od
            nepraznog skupa $\Omega$ i $\sigma$-algebre $\F$ na
            $\Omega$ nazivamo \emph{izmjerivim prostorom}.
            Ako na $\F$ imamo funkciju $\mu : \F \to
            \segment{0}{+ \infty}$ koja zadovoljava \eqref{vjerPrav4}
            i $\mjera{\varnothing} = 0$, onda ka\v zemo da je $\mu$
            \emph{(pozitivna) mjera}.
            Ako postoje $(E_n, \: n \in \N) \subseteq \F$, takvi da
            je $\unija{n \in \N}{} E_n = \Omega$ i $\mjera{E_n} <
            +\infty, \; \forall n \in \N$, tada ka\v zemo da je $\mu$
            \emph{$\sigma$-kona\v cna (pozitivna) mjera}.
            Ako je $\mjera{\Omega} < +\infty$, tada ka\v zemo da je
            $\mu$ kona\v cna (pozitivna) mjera.
            Dakle Kolmogorov je definirao vjerojatnosni prostor kao
            poseban slu\v caj prostora kona\v cne mjere.
    \end{enumerate}
\end{nap}