% o vjerojatnosnom prostoru

\part{Uvod}

\chapter{Vjerojatnosni prostor}

Sustavno razmi\v sljanje o vjerojatnosnim pojmovima po\v cinje u 16.
stolje\' cu nove ere u Italiji (Cardano, Tartaglia). Ve\' c do 19.
stolje\' ca razvijaju se slo\v zenije ideje, kao uvijetna vjerojatnost,
Laplace-ov model, te prvi grani\v cni teoremi. Odnos intuitivnog
poimanja vjerojatnost i pripadnog matemati\v ckog pojma suptilno
je pitanje. Ilustrirajmo jedan aspekt tog odnosa uspore\dj uju\' ci
pitanje tipa "Koja je vjerojatnost da na Marsu postoji \v zivot?",
s pitanjem "Koja je vjerojatnost da subotom izme\dj u 12 i 13 sati
autoputom Zagreb - Karlovac pro\dj e barem tisu\' cu automobila?".
U prvom slu\' caju htjeli bismo pridru\v ziti odre\dj enu "mjeru"
(dakle broj) stupnju vjerovanja da na Marsu postoji \v zivot.
To vodi na ideju tako zvane \emph{subjektivne vjerojatnosti}
(Keynes 1921.). Uo\v cimo da je uz prvo pitanje vrlo te\v sko
vezati neki pokus, a ne mogu\' ce ideju "ponavljanja pokusa".
S druge strane, u drugom slu\v caju mo\v zemo jednostavno provoditi
mjerenja svake subote (ponavljanje pokusa) i formirati "distribuciju"
vezanu uz taj slu\v cajni pokus. Ovo nas vodi na takozvani
\emph{Objektivni pristup} koji se tako\dj er razvija u prvoj polovici
20. stolje\' ca (von Mises 1928. i Kolmogorov 1933.) Iako ovaj
pristup konceptualno ograni\v cavaju\' ce djeluje na teoriju,
matemati\v cnost ovog pristupa postaje klju\v cnim razlogom njegove
op\' ce prihva\' cenosti.