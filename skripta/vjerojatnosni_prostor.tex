% o vjerojatnosnom prostoru


\chapter{Vjerojatnosni prostor}

Sustavno razmi\v sljanje o vjerojatnosnim pojmovima po\v cinje u 16. stolje\' cu nove ere u Italiji (Cardano, Tartaglia). Ve\' c do 19. stolje\' ca razvijaju se slo\v zenije ideje, kao uvijetna vjerojatnost, Laplace-ov model, te prvi grani\v cni teoremi. Odnos intuitivnog poimanja vjerojatnost i pripadnog matemati\v ckog pojma suptilno je pitanje.
Ilustrirajmo jedan aspekt tog odnosa uspore\dj uju\' ci pitanje tipa "Koja je vjerojatnost da na Marsu postoji \v zivot?", s pitanjem "Koja je vjerojatnost da subotom izme\dj u 12 i 13 sati autoputom Zagreb - Karlovac pro\dj e barem tisu\' cu automobila?".
U prvom slu\' caju htjeli bismo pridru\v ziti odre\dj enu "mjeru" (dakle broj) stupnju vjerovanja da na Marsu postoji \v zivot.
To vodi na ideju tako zvane \emph{subjektivne vjerojatnosti} (Keynes 1921.).
Uo\v cimo da je uz prvo pitanje vrlo te\v sko vezati neki pokus, a nemogu\' ce ideju "ponavljanja pokusa".
S druge strane, u drugom slu\v caju mo\v zemo jednostavno provoditi mjerenja svake subote (ponavljanje pokusa) i formirati "distribuciju" vezanu uz taj slu\v cajni pokus.
Ovo nas vodi na takozvani \emph{Objektivni pristup} koji se tako\dj er razvija u prvoj polovici 20. stolje\' ca (von Mises 1928. i Kolmogorov 1933.)
Iako ovaj pristup konceptualno ograni\v cavaju\' ce djeluje na teoriju, matemati\v cnost ovog pristupa postaje klju\v cnim razlogom njegove op\' ce prihva\' cenosti.
Osobito je za nas va\v zan Kolmogorovljev pristup teoriji vjerojatnosti, koji ne razmi\v slja o tome kako pojedinom stvarnom doga\dj aju dati odre\dj enu dati odre\dj enu vjerojatnost $p \in [0, \: 1]$, ve\' c uz pretpostavku da takvi brojevi postoje, razvija pravila o njihovim odnosima.
Time sama priroda doga\dj aja postaje sekundarna u toj teoriji (jasno, u primjenama je i dalje od prvorazredne va\v znosti), a bitna postaje analiza "distribucije" i njenih pravila.
S druge strane, za bolje razumjevanje koncepata teorije, a osobito u njenim primjenama, \v cest uzimamo u obzir konkretne primjere.
Primjeri iz hazardnih igara (od arapskog al-zahr za igra\v cu kocku) \v cesto donose boljem razumijevanju same teorije.
Na primjer, jako je jednostavno modelirati jedno bacanje simetri\v cne kovanice ($50\%$ \v sanse za ishod glave i $50\%$ za ishod pisma).
Dvije kovanice ve\' c mogu predstavljati problem.

\begin{pr}[D'Alambert 1754] \label{pr:1.1}
    Kolika je vjerojatnost da prilikom jednog bacanja dvije simetri\v cne kovanice bar jednom "padne pismo"?

    D'Alambert ka\v ze da imamo 3 mogu\' cnosti (dvije glave,
    dva pisma i po jedno od svakoga), od kojih su za nas povoljne
    dvije, stoga je vjerojatnost $\frac{2}{3}$!?
    
    Ovaj primjer pokazuje koliko je va\v zno precizno odrediti \emph{osnovne (elementarne) ishode}.
    Uzmemo li 2G, 2P i 1G1P kao osnovne ishode, onda moramo dobro razmisliti koje su njihove vjerojatnosti (nisu $\frac{1}{3}$). Zamislimo da smo jednu kovanicu obojili.
    Jesmo li time promjenili pokus?
    Nismo.
    Sada ishode pokusa mo\v zemo prikazati kao ure\dj ene parove, ako obojanu kovanicu stavio na prvo mjesto, vidimo da su mogu\' ci ishodi:
    (G, G), (G, P), (P, G), (P, P)
    
    Odakle vidimo da je tra\v zena vjerojatnost zapravo $\frac{3}{4}$.
\end{pr}

Prvi va\v zan objekt je takozvani \emph{prostor elementarni doga\dj aja} (sample space).
U jeziku teorije skupova dovoljno je zahtjevati da to bude neprazan skup $\Omega \neq \varnothing$.

U primjeru \ref{pr:1.1} imamo $ \Omega = \{(G, \: G), (G, \: P), (P, \: G), (P, \: P)\}$.
\v Sto su doga\dj aji?
Na primjer doga\dj aj iz primjera \ref{pr:1.1} je opisan sa \emph{"palo je barem jedno pismo"}, to jest, to je podskup $A = \{(G, \: P),(P, \: G), (P, \: P)\} \subseteq \Omega$.

Dakle, doga\dj aje je prirodno promatrati kao elemente neke familije $\F \subseteq \partitive{\Omega}$.
Zapravo, najbolje bi bilo uzeti sve podskupove, dakle familiju $\partitive{\Omega}$, kao doga\dj aje.
\v Sto \' ce biti vjerojatnosti?
Ve\' c primjeri sugeriraju da su vjerojatnosti brojevi pridru\v zeni doga\dj ajima.
Dosta je o\v cito da od tih brojeva zahtjevamo bar sljede\' ca svojstva: ako su $A$, $B$ doga\dj aji tada za vjerojatnosti $\vjeroj{A}$, $\vjeroj{B}$ vrijedi:

\begin{equation} \label{jed:1.2}
    0 \leq \vjeroj{A} \leq 1
\end{equation}

\begin{align} \label{jed:1.3}
    \begin{split}
        \vjeroj{\varnothing} &= 0 \\
        \vjeroj{\Omega} &= 1
    \end{split}
\end{align}

\begin{equation} \label{jed:1.4}
    A \cap B = \varnothing \implies \vjeroj{A \cup B} = \vjeroj{A} + \vjeroj{B}
\end{equation}

\begin{pr}[Laplace-ov model] \label{pr:1.5}
    Neka je $\Omega \neq \varnothing$ kona\v can skup.
    Ideja je modelirati slu\v caj u kojem su svi osnovni ishodi \emph{jednako vjerojatni}.
    Lako se vidi da taj zahtjev zajedno sa pravilima \eqref{jed:1.2}, \eqref{jed:1.3} i \eqref{jed:1.4}, nu\v zno vodi na to\v cno jedan model, onaj u kojem je svaki $A \subseteq \Omega$ doga\dj aj i vrijedi:

    \begin{equation} \label{jed:1.6}
        \vjeroj{A} = \frac{\card{A}}{\card{\Omega}}
    \end{equation}
\end{pr}

U ovom primjeru imamo zadane vjerojatnosti $\vjeroj{\{ \omega\}} = \frac{1}{\card{\Omega}}$, za svaki $\omega \in \Omega$ i onda nas pravilo \eqref{jed:1.4} vodi do formule \eqref{jed:1.6}.

To se mo\v ze i poop\' citi, te je svaki model nad kon\v cnim $\Omega$ mogu\' ce opisati na sljede\' ci na\v cin.

\begin{pr} \label{pr:1.7}
    Neka je $\card{\Omega} = n \in \N$.
    Neka su $p_1, \dots p_n$ realni brojevi za koje vrijedi:
    \begin{equation}    \label{jed:1.8}
        p_1, \dots, p_n \in [0, \: 1]
    \end{equation}

    \begin{equation}    \label{jed:1.9}
        \suma{i = 1}{n} p_i = 1
    \end{equation}

    Tada je sa

    \begin{equation} \label{jed:1.10}
        \vjeroj{A} = \suma{\omega_i \in A}{} p_i
    \end{equation}
    dana vjerojatnost na $\partitive{\Omega}$ koja zadovoljava pravila \eqref{jed:1.2}, \eqref{jed:1.3} i \eqref{jed:1.4} i svaka vjerojatnost na $\partitive{\Omega}$ koja zadovoljava \eqref{jed:1.2}, \eqref{jed:1.3} i \eqref{jed:1.4} mo\v ze se prikazati na ovaj na\v cin.
\end{pr}

Primjetimo da se ovaj pristup direktno prenosi i na prebrojive $\Omega$ ako dodamo poop\' cenje pravila \eqref{jed:1.4}:

\begin{equation} \label{jed:1.11}
    (A_n, \: n \in \N) \subseteq \partitive{\Omega}, \; A_i \cap A_j = \varnothing, \; i \neq j \implies \vjeroj{\unija{n}{} A_n} = \suma{n}{} \vjeroj{A_n}
\end{equation}

\begin{zad} \label{zad:1.12}
    Doka\v zi sljede\' ce tvrdnje.
    \begin{enumerate}[label=(\alph*)]
        \item Doka\v zi da \eqref{jed:1.2}, \eqref{jed:1.3} i \eqref{jed:1.11} $\implies$ \eqref{jed:1.4}
        \item Doka\v zi da \eqref{jed:1.2}, \eqref{jed:1.3}, \eqref{jed:1.4} i pravilo:
        \begin{equation} \label{jed:1.13}
            (A_n, \: n \in \N) \subseteq \partitive{\Omega}, \; A_1 \supseteq A_2 \supset \dots, \quad \presjek{n}{} A_n = \varnothing \implies \lim_{n \to \infty} \vjeroj{A_n} = 0
        \end{equation}
        impliciraju \eqref{jed:1.11}
    \end{enumerate}
\end{zad}

%%
%% rješenje zadatka 12
%%

\begin{rj}[\ref{zad:1.12}]
    \begin{enumerate}[label=(\alph*)]
        \item Definiramo niz $\niz{C_n}{n \in \N}$, sa $C_1 = A, \; C_2 = B, \; C_{n + 2} = \varnothing, \; n \in \N$.
            Vidimo da je to niz disjunktnih doga\dj aja.
            Sada vrijedi
            \begin{equation*}
                \vjeroj{\unija{n \in \N}{} C_n} \overset{\eqref{jed:1.11}}{=} \suma{n \in \N}{} \vjeroj{C_n} = \vjeroj{C_1} + \vjeroj{C_2} + \underbrace{\suma{n = 3}{\infty} \vjeroj{C_n}}_{=0}
            \end{equation*}
            Odakle je lako vidjeti:
            \begin{align*}
                \vjeroj{A \cup B \cup \unija{n = 3}{\infty} \varnothing} &= \vjeroj{A \cup B} \\
                &= \vjeroj{A} + \vjeroj{B}
            \end{align*}
        \item Neka je $\niz{A_n}{n \in \N} \subseteq \F$ niz disjunktnih doga\dj aja.
        Definiramo niz $\niz{B_n}{n \in \nat}$ sa
        \begin{equation*}
            B_n := \unija{k = n}{\infty} A_k.
        \end{equation*}
        Primjetimo da je $B_n \supseteq B_{n + 1}$, dakle $(B_n)$ je rastu\' ci niz doga\dj aja.
        \v Zelimo pokazati da vrijedi
        \begin{equation*}
            \presjek{n = 1}{\infty} B_n = \varnothing.
        \end{equation*}
        Prepostavimo da je $x \in \presjek{k = n_0}{\infty} B_n$, tada vrijedi
        \begin{equation*}
            x \in B_n \; \forall n \in \nat \iff x \in \unija{k = n}{\infty} A_k \; \forall n \in \nat.
        \end{equation*}
        Odaberimo proizvoljni $n_0 \in \N$.
        Sada, budu\' ci je $\niz{A_n}{n \in \N}$ familija disjunktnih doga\dj aja, vrijedi
        \begin{equation*}
            x \in \unija{k = n_0}{\infty} A_k \implies \exists ! \: k_0 \geq n_0 \; \td \; x \in A_{k_0}.
        \end{equation*}
        Ali tada $x \notin A_k$, za $k > k_0$, \v sto je kontradikcija s $x \in B_n \; \forall n \in \nat$, pa je $\presjek{n = 1}{\infty} B_n = \varnothing$.\\
        Sada imamo
        \begin{align*}
            \vjeroj{\unija{k = 1}{\infty} A_k}
            &= \vjeroj{\unija{k = 1}{n} A_k}
            + \vjeroj{\unija{k = n + 1}{\infty} A_k} \\
            &= \suma{k = 1}{n} \vjeroj{A_k}
            + \vjeroj{\unija{k = n + 1}{\infty} A_k}. 
        \end{align*}
        Pu\v stanjem limesa dobijemo:
        \begin{align*}
            \lim_n \vjeroj{\unija{k = n + 1}{\infty} A_k}
            \overset{\eqref{jed:1.13}}{=}& \lim_n
            \vjeroj{\presjek{n + 1}{\infty}\unija{k = n + 1}{\infty} A_k}\\
            =& 0.
            \end{align*}
        I prelaskom na limes dobijemo tra\v zenu tvrdnju.
    \end{enumerate}
\end{rj}

\begin{pr} \label{pr:1.14}
    Neka je $\card{\Omega} = \aleph_0$ i poredajmo $\Omega$ u niz $\{ \omega_1, \: \omega_2, \dots \}$.
    Svaki model koji zadovoljava \eqref{jed:1.2}, \eqref{jed:1.3} i \eqref{jed:1.11} na $\partitive{\Omega}$ mo\v ze se dobiti preko niza realnih brojeva $\{p_n, \: n \in \N \}$ tako da vrijedi:
    \begin{equation}   % možda promjeniti format ovoga? 
        p_n \in [0, \: 1], \quad \forall n \in \N
    \end{equation}

    \begin{equation}
        \suma{n \in \N}{} p_n = 1
    \end{equation}

    \begin{equation} \label{jed:1.17}
        \vjeroj{A} = \suma{\omega_n \in A}{} p_n, \quad A
            \subseteq \Omega
    \end{equation}
\end{pr}

To zna\v ci da prakti\v cki sve situacije u kojima je $\Omega$ najvi\v se prebrojiv mo\v zemo svesti na \emph{vjerojatnosti prostor} $(\Omega, \: \partitive{\Omega}, \: \Pp)$, pri \v cemu je $\Pp$ opisan sa \eqref{jed:1.10} ili \eqref{jed:1.17}

\v Sto ako je $\Omega$ neprebrojiv?

\begin{pr}  \label{pr:1.18}
    Neka je $\Omega = \segment{a}{b} \subseteq \R$ i opi\v simo pokus "slu\v cajnog odabira to\v cke u $\Omega$".
    Prirodno je o\v cekivati da za $\segment{c}{d} \subseteq \segment{a}{b}$ vrijedi:
    \begin{equation} \label{jed:1.19}
        \vjeroj{\segment{c}{d}} = \frac{d - c}{b - a}
    \end{equation}
    Te za doga\dj aj $A \subseteq \Omega$ i $x \in \R$, takve da je
    $x + A \subseteq \Omega$, vrijedi
    \begin{equation} \label{jed:1.20}
        \vjeroj{x + A} = \vjeroj{A}.
    \end{equation}
    Postoji li takav model na $\partitive{\Omega}$
\end{pr}

\begin{tm}
    Ne postoji funkcija $\Pp: \partitive{\segment{a}{b}} \to \R$ koja zadovoljava \eqref{jed:1.2}, \eqref{jed:1.3}, \eqref{jed:1.11}, \eqref{jed:1.19} i \eqref{jed:1.20}.
\end{tm}

%% možda dodati kao zadatak teorem o osnovnim svojstvima mjere?
%% iz njega slijede sve ove stvari + pokazat da iz normiranosti
%% vjerojatnosti slijedi da je vjerojatnost praznog skupa 0

\begin{proof}
    Bez smanjanja op\' cenitosti uzmemo $\segment{a}{b} = \segment{-3}{3}$ i pretpostavimo suprotno, to jest da postoji $\Pp$ s tra\v zenim svojstvima.
    Iz \eqref{jed:1.2}, \eqref{jed:1.3}, \eqref{jed:1.11} po zadatku \ref{zad:1.12} slijedi \eqref{jed:1.4}, te se lako poka\v ze i
    \begin{equation} \label{monotonstVjeroj}
        A \subseteq B \implies \vjeroj{A} \leq \vjeroj{B}
    \end{equation}
    Na $\segment{-1}{1}$ definiramo relaciju ekvivalencije $\sim$ sa $x \sim y \iff x - y \in \Q$.
    Po \emph{aksiomu izbora} postoji $A \subseteq \segment{-1}{1}$, takav da je za svaki $x \in \segment{-1}{1}$ (i pripadni $[x] \in \segment{-1}{1} \big/_{\sim}$), $\card{A \cap [x]} = 1$ (u $A$ smo stavili samo jedan element iz svake klase ekvivalencije).
    Skup $\Q \cap \segment{-2}{2}$ je prebrojiv pa ga mo\v zemo poredati u niz $\niz{q_n}{n \in \N}$. 
    Definirajmo $A_n = q_n + A \subseteq \segment{-3}{3}, \; n \in
    \N$.
    Neka je $H := \unija{n=1}{\infty} A_n \subseteq \segment{-3}{3}$. 
    Iz definicije relacije $\sim$ slijedi da su $A_n$ me\dj usobno disjunktni, pa \eqref{jed:1.11} daje:
    \begin{equation*}
        \vjeroj{H} = \suma{n=1}{\infty} \vjeroj{A_n}
            \overset{\eqref{jed:1.20}}{=} \suma{n=1}{\infty}
            \vjeroj{A}.
    \end{equation*}
    Kada bi $\vjeroj{A} > 0$, onda bi vrijedilo $\suma{n=1}{\infty} \vjeroj{A} = + \infty \implies \vjeroj{H} = +\infty$, \v sto je kontradikcija sa \v cinjenicom da je $\vjeroj{H} \in \segment{0}{1}$, stoga nu\v zno vrijedi $\vjeroj{A} = 0$, pa onda i $\vjeroj{H} = 0$.
    Za svaki $x \in \segment{-1}{1}$, vrijedi $A \cap [x] = \{y\}, \; y \in \segment{-1}{1}, \; x - y \in \Q \cap \segment{-2}{2}$.
    Dakle $x \in q_n + A \subseteq H$, za neki $n$.
    Pa mora vrijediti $\segment{-1}{1} \subseteq H$, pa po \eqref{jed:1.19} i \eqref{monotonstVjeroj} slijedi
    \begin{equation*}
        \frac{1}{3} = \vjeroj{\segment{-1}{1}} \leq \vjeroj{H} = 0,
    \end{equation*}
    \v sto je kontradikcija, stoga funkcija $\Pp$ sa tra\v zenim svojstvima ne postoji.
\end{proof}

Priroda problema je takva da nemo\v zemo odustati niti od jednog od zahtjeva \eqref{jed:1.2}, \eqref{jed:1.3}, \eqref{jed:1.11}, \eqref{jed:1.19}, \eqref{jed:1.20}, stoga moramo odustati od zahtjeva definicije funkcije $\Pp$ na $\partitive{\Omega}$.

\v Zelimo sa\v cuvati osnovne operacije i to nas vodi na sljede\' cu definiciju.

\begin{defn}
    Vjerojatnosni prostor je ure\dj ena trojka $\vjerojatnosniProstor$ koja se sastoji od nepraznog skupa $\Omega$, $\sigma$-algebre $\F$ doga\dj aja na $\Omega$ te funkcije $\Pp: \F \to \R$ koja zadovoljava \eqref{jed:1.2}, \eqref{jed:1.3} i \eqref{jed:1.11}.
    Funkciju $\Pp$ nazivamo \emph{vjerojatnosnom mjerom}.
\end{defn}

\begin{nap} \label{nap:1.24}
    \begin{enumerate}[label=(\alph*)]
        \item Podsjetimo se da je $\F \subseteq \partitive{\Omega}$ $\sigma$-algebra, ako je $\varnothing \in \F$, $\F$ je zatvoren na komplemente i prebrojive unije. Posljedica toga je da je $\F$ zatvorena na kona\v cne i prebrojive upotrebe uobi\v cajenih skupovnih operacija.
        \emph{Doga\dj aji}  su samo oni podskupovi od $\Omega$ koji su elementi $\sigma$-algebre $\F$.
        \item Za $\sigma$-algebru $\famF$ na $\Omega \neq  \varnothing$, dovoljno je tra\v ziti:
        \begin{enumerate}[label=(\roman*)]
            \item $\varnothing \in \famF$
            \item $A \in \famF \implies A^c \in \famF$
            \item $\niz{A_n}{n \in \nat} \subseteq \famF \implies \unija{n \in \nat}{} A_n \in \famF$.
        \end{enumerate}
        \item Ure\dj en par $\izmjerivProstor$ koji se sastoji od nepraznog skupa $\Omega$ i $\sigma$-algebre $\F$ na $\Omega$ nazivamo \emph{izmjerivim prostorom}.
        Ako na $\F$ imamo funkciju $\mu : \F \to \segment{0}{+ \infty}$ koja zadovoljava \eqref{jed:1.11} i $\mjera{\varnothing} = 0$, onda ka\v zemo da je $\mu$ \emph{(pozitivna) mjera}.
        Ako postoje $\niz{E_n}{n \in \N} \subseteq \F$, takvi da je $\unija{n \in \N}{} E_n = \Omega$ i $\mjera{E_n} < +\infty, \; \forall n \in \N$, tada ka\v zemo da je $\mu$ \emph{$\sigma$-kona\v cna (pozitivna) mjera}.
        Ako je $\mjera{\Omega} < +\infty$, tada ka\v zemo da je $\mu$ kona\v cna (pozitivna) mjera.
        Dakle Kolmogorov je definirao vjerojatnosni prostor kao poseban slu\v caj prostora kona\v cne mjere.
        \item Za vjerojatnosnu mjeru nije potrebno zahtjevati $\vjeroj{\varnothing} = 0$.
        Minimalni zahtjevi da bi na izmjerivom prostoru $\izmjerivProstor$ funkcija
        \begin{equation*}
            \masP: \famF \to \real
        \end{equation*}
        bila vjerojatnost su:
        \begin{enumerate}[label=(\roman*)]
            \item $\vjeroj{A} \geq 0, \quad \forall A \in \famF$
            \item $\vjeroj{\Omega} = 1$
            \item $\niz{A_n}{n \in \nat}$ niz disjunktnih doga\dj aja, tada:
            \begin{equation*}
                \vjeroj{\unija{n \in \nat}{} A_n} = \suma{n \in \nat}{} \vjeroj{A_n}.
            \end{equation*}
        \end{enumerate}
    \end{enumerate}
\end{nap}

\begin{tm}[O osnovnim svojstvima mjere] \label{tm:1.24-1}
    Neka je $\prostorMjere$ prostor mjere, tada vrijedi:
    \begin{enumerate}[label={(\roman*)}]
        \item \label{tm:1.24-1.1}
        \emph{Monotonost}: Ako je $A \subseteq B$, vrijedi 
            $\mjera{A} \leq \mjera{B}$
        \item \emph{Subaditivnost}: Ako je $A \subseteq \unija{n=1}{\infty} A_n$, tada
            vrijedi $\mjera{A} \leq \suma{n=1}{\infty} \mjera{A_n}$
        \item \label{tm:1.24-1.3}
        \emph{Neprekidnost odozdo}: Ako $A_n \nearrow A$, 
            tj. $(A_n)_{n \in \N}$ td. $A_1 \subseteq A_2 \subseteq \dots$ i vrijedi
            $\unija{n \in \N}{} A_n = A$, tada vrijedi $\mjera{A} =
            \lim\limits_{n} \mjera{A_n}$.
        \item \emph{Neprekidnost odozgo}: Ako $A_n \searrow A$, tj. $(A_n)_{n \in \N}$ td.
            $A_1 \supseteq A_2 \supseteq \dots$, ako postoji $n_0 \in \N$ tako da
            $\mjera{A_{n_0}} < \infty$ i vrijedi $\presjek{n \in \N}{} A_n = A$,
            tada vrijedi $\mjera{A} = \lim\limits_{n} \mjera{A_n}$.
    \end{enumerate}
\end{tm}

\begin{proof}
    \begin{enumerate}[label={(\roman*)}]
        \item Neka je $A \subseteq B$, tada je $B = A \dot{\cup} (B \setminus A)$, a sada
            vrijedi $\mjera{B} = \mjera{A} + \mjera{B \setminus A}$, budu\' ci je
            $\mjera{B \setminus A} \geq 0$, vrijedi da je $\mjera{A} \leq \mjera{B}$.

        \item Definiramo $B_1 := A_1, \: B_k := A_k \setminus (\unija{i=1}{k} A_i),
            \: k \geq 1$. Tvrdim da je $(B_n)_{n \in \N}$ niz disjunktnih skupova.
            Neka su $m, \: n \in \N$, takvi da $m \neq n$, bez smanjenja op\' cenitosti
            mo\v zemo pretpostaviti da je $m < n$. Neka je $x \in B_n \implies x \in A_n 
            \land x \notin A_k, \: k < n \implies x \notin B_m$.
            Dakle $B_m \cap B_n = \emptyset.$ Tako\dj er tvrdimo da je $\unija{k=1}{n} A_k
            = \unija{k=1}{n} B_k, \: \forall n n \in \N$.
            Dokaz vr\v simo matemati\v ckom indukcijom po $n \in \N$.
            \begin{enumerate}
                \item[(B)] $A_1 = B_1$.
                \item[(P)] Neka je $n \in \N$, pretpostavimo da vrijedi $\unija{k=1}{n} A_k
                = \unija{k=1}{n} B_k$.
                \item[(K)] $\unija{k=1}{n+1} B_k = (\unija{k=1}{n} B_k) \cup B_{n+1}
                    = (\unija{k=1}{n} A_k) \cup \underbrace{(A_n \setminus
                    \unija{k=1}{n} A_k)}_{\textnormal{po definiciji}}
                    = \unija{k=1}{n+1} A_k$.
            \end{enumerate}
            Po principu matemati\v cke indukcije tvrdnja vrijedi za svaki $n \in \N$.
            Sada vrijedi $\unija{n \in \N}{} A_n = \unija{n \in \N}{} B_n$, primjetimo
            $x \in \unija{n \in \N}{} A_n \implies \exists n_0 \in \N$, takav da
            $x \in A_{n_0} \implies x \in \unija{k=1}{n_0} A_k = \unija{k=1}{n_0} B_k
            \implies x \in \unija{n \in \N}{} B_n$. Obratna inkluzija se dokazuje identi\v
            cno.
            Po pretpostavci je $A \subset \unija{n \in \N}{} A_n = \unija{n \in \N}{} B_n$,
            sada po \ref{tm:1.24-1.1} vrijedi $\mjera{A} \leq
            \mjera{\unija{n \in \N}{} B_n} = \suma{n=1}{\infty} B_n$. Tako\dj er budu\' ci
            vrijedi $B_k \subseteq A_k$, vrijedi i $\mjera{B_k} \leq \mjera{A_k}$, stoga
            imamo $\suma{n=1}{\infty} \mjera{B_n} \leq \suma{n=1}{\infty} \mjera{A_n}$.
            Odavde vidimo da vrijedi:
            \begin{equation*}
                \mjera{A} \leq \suma{n=1}{\infty} A_n.
            \end{equation*}
        \item Definiramo niz $(B_n)_{n \in \N}$ sa $B_1 := A_1, \: B_n = A_n \setminus
            A_{n-1}, \: n > 1$. Tvrdimo da je $(B_n)_{n \in \N}$ niz disjunktnih skupova
            te vrijedi $\unija{k=1}{n} A_k = \unija{k=1}{n} B_k$. Neka $m, \: n \in \N
            \: m \neq n$, bez smanjenja op\' cenitosti mo\v zemo pretpostaviti $m < n$.
            Neka je $x \in B_m \cap B_n, \: x \in B_n \implies x \notin A_{n-1}
            \implies x \in A_k, \: k \leq n-1$, jer je $(A_n)_{n \in \N}$ rastu\' c,
            a kako je $B_m \subseteq A_{n-1}$, \v sto je kontradikcija, dakle vrijedi
            $B_n \cap B_m = \emptyset$. Drugu tvrdnju dokazujemo po indukciji.
            \begin{enumerate}
                \item[(B)] $B_1 = A_1$.
                \item[(P)] Neka je $n \in \N$, tada vrijedi $\unija{k=1}{n} A_k
                    = \unija{k=1}{n} B_k$.
                \item[(K)] $\unija{k=1}{n+1} B_k = (\unija{k=1}{n} A_k) \cup B_{n+1}
                    = (\unija{k=1}{n} A_k) \cap (A_{n+1} \setminus A_n)
                    = \unija{k=1}{n+1} A_k$.
            \end{enumerate}
            Po Principu matemati\v cke indukcije tvrdnja vrijedi za svaki $n \in \N$.
            Vrijedi $A_n = \unija{k=1}{n} A_k$, jer je rije\v c o rastu\' cim nizu,
            dakle vrijedi $\unija{k=1}{n} A_k = \unija{k = 1}{n} B_k = A_n$.
            Sada imamo $\mjera{A} = \mjera{\unija{n \in \N}{} A_n} = \mjera{
                \unija{n \in \N}{} B_n} = \suma{n=1}{\infty} \mjera{B_n}
                = \lim\limits_{n} \suma{k=1}{n} \mjera{B_k} = \lim\limits_{n}
                \mjera{\unija{k=1}{n} B_k} = \lim\limits_{n} \mjera{A_n}$.
        \item Definirajmo niz $\niz{B_n}{n \in \nat}$ sa
        \begin{equation*}
            B_n := A_{n_0} \setminus A_{n_0 + n},
        \end{equation*}
        tvrdimo da je  $(B_n)$ rastu\' ci niz.
        Vidimo da vrijedi
        \begin{equation*}
            \begin{gathered}
                \begin{aligned}
                    A_{n_0} \setminus \presjek{n=1}{\infty} A_n &= A_{n_0} \cap \Big( \presjek{n=1}{\infty} A_n \Big)^c = A_{n_0} \cap \Big( \unija{n = 1}{\infty} A_n^c \Big) = \unija{n = 1}{\infty} (A_{n_0} \cap A_n^c)\\
                    &= \unija{n = 1}{\infty} A_{n_0} \setminus A_n \implies
                \end{aligned}\\
                \unija{n=1}{\infty} B_n = A_{n_0} \setminus \presjek{n=1}{\infty} A_n.
            \end{gathered}
        \end{equation*}
            Neka je $x \in B_m$, dakle
            \begin{equation*}
                    x \in A_{n_0} \setminus A_{n_0+m} \implies x \in A_{n_0}
                    \land x \notin A_{n_0+m} \implies x \in A_{n_0} \land x \in (A_{n_0+m})^c.
            \end{equation*}
            Budu\' ci vrijedi
            \begin{equation*}
                A_{n_0+m} \supseteq A_{n_0+m+k} \implies (A_{n_0+m})^c
            \subseteq (A_{n_0+m+k})^c,
            \end{equation*}
            onda je
            \begin{equation*}
                \begin{gathered}
                    x \in A_{n_0} \land x \in (A_{n_0+m})^c \implies x \in A_{n_0} \land x \in (A_{n_0+m+k})^c \implies\\
                    x \in A_{n_0} \setminus A_{n_0+(m+k)} = B_{m+k}.
                \end{gathered}
            \end{equation*}
            Posebno za $k=1$ tvrdnja slijedi to jest $B_n \subseteq B_{n+1}$.
            Nadalje vrijedi
            \begin{equation*}
                \unija{n=1}{\infty}
            B_n = A_{n_0} \setminus \presjek{n=1}{\infty} A_{n_0+n}.
            \end{equation*}
            Budu\' ci je $(A_n)$ padaju\' ci niz, onda vrijedi
            \begin{equation*}
                \presjek{n=1}{\infty} A_n = \presjek{n=1}{\infty} A_{n_0+n} \implies \unija{n=1}{\infty} B_n = A_{n_0} \setminus \presjek{n=1}{\infty} A_n.
            \end{equation*}
            Sada iz tvrdnje \ref{tm:1.24-1.3} vrijedi
            \begin{equation*}
                \mjera{A_{n_0} \setminus
            \presjek{n=1}{\infty} A_n} = \mjera{\unija{n=1}{\infty} B_n}
            = \lim\limits_{n} \mjera{B_n} = \lim\limits_{n} \mjera{A_{n_0} \setminus
            A_{n_0+n}}.
            \end{equation*}
            Sada, zbog $\mjera{A_{n_0}} < \infty$, vrijedi
            \begin{equation*}
                \begin{aligned}
                    \mjera{A_{n_0}} - \mjera{\presjek{n=1}{\infty} A_n} &= \lim\limits_{n} (\mjera{A_{n_0}} - \mjera{A_{n_0+n}}) = \mjera{A_{n_0}} - \lim\limits_{n} \mjera{A_{n_0+n}}\\
                    &= \mjera{A_{n_0}} - \lim\limits_{n} \mjera{A_n},
                \end{aligned}
            \end{equation*}
            dakle vrijedi
            \begin{equation*}
                \mjera{\presjek{n=1}{\infty} A_n} = \lim\limits_{n} \mjera{A_n}.
            \end{equation*}
    \end{enumerate}
\end{proof}