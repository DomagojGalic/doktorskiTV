% o vjerojatnosnom prostoru

\part{Uvod}

\chapter{Vjerojatnosni prostor}

Sustavno razmi\v sljanje o vjerojatnosnim pojmovima po\v cinje u 16.
stolje\' cu nove ere u Italiji (Cardano, Tartaglia). Ve\' c do 19.
stolje\' ca razvijaju se slo\v zenije ideje, kao uvijetna vjerojatnost,
Laplace-ov model, te prvi grani\v cni teoremi. Odnos intuitivnog
poimanja vjerojatnost i pripadnog matemati\v ckog pojma suptilno
je pitanje. Ilustrirajmo jedan aspekt tog odnosa uspore\dj uju\' ci
pitanje tipa "Koja je vjerojatnost da na Marsu postoji \v zivot?",
s pitanjem "Koja je vjerojatnost da subotom izme\dj u 12 i 13 sati
autoputom Zagreb - Karlovac pro\dj e barem tisu\' cu automobila?".
U prvom slu\' caju htjeli bismo pridru\v ziti odre\dj enu "mjeru"
(dakle broj) stupnju vjerovanja da na Marsu postoji \v zivot.
To vodi na ideju tako zvane \emph{subjektivne vjerojatnosti}
(Keynes 1921.). Uo\v cimo da je uz prvo pitanje vrlo te\v sko
vezati neki pokus, a ne mogu\' ce ideju "ponavljanja pokusa".
S druge strane, u drugom slu\v caju mo\v zemo jednostavno provoditi
mjerenja svake subote (ponavljanje pokusa) i formirati "distribuciju"
vezanu uz taj slu\v cajni pokus. Ovo nas vodi na takozvani
\emph{Objektivni pristup} koji se tako\dj er razvija u prvoj polovici
20. stolje\' ca (von Mises 1928. i Kolmogorov 1933.) Iako ovaj
pristup konceptualno ograni\v cavaju\' ce djeluje na teoriju,
matemati\v cnost ovog pristupa postaje klju\v cnim razlogom njegove
op\' ce prihva\' cenosti.
Osobito je za nas va\v zan Kolmogorovljev pristup teoriji
vjerojatnosti, koji ne razmi\v slja o tome kako pojedinom stvarnom
doga\dj aju dati odre\dj enu dati odre\dj enu vjerojatnost
$p \in [0, \: 1]$, ve\' c uz pretpostavku da takvi brojevi postoje,
razvija pravila o njihovim odnosima. Time sama priroda doga\dj aja
postaje sekundarna u toj teoriji (jasno, u primjenama je i dalje od
prvorazredne va\v znosti), a bitna postaje analiza "distribucije"
i njenih pravila.
S druge strane, za bolje razumjevanje koncepata teorije, a osobito
u njenim primjenama, \v cest uzimamo u obzir konkretne primjere.
Primjeri iz hazardnih igara (od arapskog al-zahr za igra\v cu
kocku) \v cesto donose boljem razumijevanju same teorije. Na
primjer, jako je jednostavno modelirati jedno bacanje simetri\v cne
kovanice ($50\%$ \v sanse za ishod glave i $50\%$ za ishod pisma).
Dvije kovanice ve\' c mogu predstavljati problem.

\begin{pr}[D'Alambert 1754]
    Kolika je vjerojatnost da prilikom jednog bacanja dvije
    simetri\v cne kovanice bar jednom "padne pismo"?
\end{pr}

\begin{rj}
    D'Alambert ka\v ze da imamo 3 mogu\' cnosti (dvije glave,
    dva pisma i po jedno od svakoga), od kojih su za nas povoljne
    dvije, stoga je vjerojatnost $^2 / _3$!?
    
    Ovaj primjer pokazuje koliko je va\v zno precizno odrediti
    \emph{osnovne (elementarne) ishode}. Uzmemo li 2G, 2P i 1G1P kao
    osnovne ishode, onda moramo dobro razmisliti koje su njihove
    vjerojatnosti (nisu $^1 / _3$). Zamislimo da smo jednu
    kovanicu obojili. Jesmo li time promjenili pokus? Nismo.
    Sada ishode pokusa mo\v zemo prikazati kao ure\dj ene parove,
    ako obojanu kovanicu stavio na prvo mjesto, vidimo da su
    mogu\' ci ishodi:
    (G, G), (G, P), (P, G), (P, P)
    
    Odakle vidimo da je tra\v zena vjerojatnost zapravo $^3 / _4$.
\end{rj}

Prvi va\v zan objekt je takozvani \emph{prostor elementarnih
doga\dj aja} (sample space). U jeziku teorije skupova dovoljno je
zahtjevati da to bude neprazan skup $\Omega \neq \varnothing$
