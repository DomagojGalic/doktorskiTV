% cjelina 5.3 Definicija martingala -> predavanje 23

\chapter{Definicija martingala}

\begin{pr}  \label{pr:23.1}
    Opi\v simo model jednostavne slu\v cajne igre koju igraju Ante (A) i Branimir (B).
    Igra se odvija u nepreciziranom broju uzastopnih kola.
    U svakom kolu mogu\' ca su samo dva ishoda:
    \begin{itemize}
        \item[] "Ante pla\' ca Branimiru $1$ kunu" (vjerojatnost je $p$)
        \item[] "Branimir pla\' ca Anti $1$ kunu" (vjerojatnost je $q = 1 - p$) 
    \end{itemize}
    i vrijedi $0 < p < 1$.
    Dakle, na primjer s pozicije igra\v ca B, imamo u svakom kolu rezdiobu Bernoulijevog tipa
    \begin{equation*}
        \begin{pmatrix}
            -1& 1\\
            q& p
        \end{pmatrix}.
    \end{equation*}
    Neka je $\niz{X_n}{n \in \nat}$ niz nezavisnih slu\v cajnih varijabli s gornjom razdiobom i neka je $\niz{S_n}{n \in \nat \cup \{0\}}$ slu\v cajna \v setnja s tim da je $S_0 \equiv K \in \nat$.
    Uo\v cimo da je $S_n$ svota novca kojom raspola\v ze $B$ u trenutku $n$, s tim da dopu\v stamo i $S_n < 0$ (u smislu da se B zadu\v zio).
    Kada je igra po\v stena?
    Kada favorizira igra\v ca B, a kada A?
    Neka je $\famF_0 := \{ \varnothing, \Omega \}$ i neka je $\famF_n := \indSigAlg{X_k}{k = 1, \ldots, n}$.
    U trenutku $n$ $\sigma$-algebra $\famF_n$ sadr\v zi svu informaciju o igri do tog trenutka.
    Igra \' ce biti po\v stena je o\v cekivani prirast "bogatstva" za ova igra\v ca isti, uz uvjet poznavanja "pro\v slosti":
    \begin{equation*}
        \masE \big[ S_{n + 1} - S_n \big| \famF_n \big] = 0,
    \end{equation*}
    zbog \v cinjenice da je $S_n$ $\famF_n$-izmjeriva, to je ekvivalentno zahtjevu:
    \begin{equation}    \label{jed:23.2}
        \masE [S_{n + 1} | \famF_n] = S_n.
    \end{equation}
    Igra favorizira igra\v ca B (igra\v ca A) ako je u \eqref{jed:23.2} umjesto znaka jednakosti znak $\geq$ ($\leq$).
    Uo\v cimo da je
    \begin{equation*}
        S_{n + 1} = X_{n + 1} + S_n
    \end{equation*}
    i $X_{n + 1}$ je neovisna od $\famF$, \v sto daje
    \begin{equation*}
        \masE [S_{n + 1} | \famF_n] = \masE [X_{n + 1}] + S_n.
    \end{equation*}
    O\v cito vrijedi
    \begin{equation*}
        \masE [X_{n + 1}] = p - q.
    \end{equation*}
    Dakle
    \begin{equation}    \label{jed:23.3}
        \begin{aligned}
            p = q = \frac{1}{2} \quad &\iff \quad \textnormal{igra je po\v stena}\\
            p > q \quad &\iff \quad \textnormal{igra favorizira B}\\
            p < q \quad &\iff \quad \textnormal{igra favorizira A}.
        \end{aligned}
    \end{equation}
\end{pr}

\begin{defn}    \label{defn:23.4}
    Neka je $\masT$ jedan od sljede\' cih skupova $\{ 0, \ldots, n \}$, $\nat \cup \{0\}$, $\nat \cup \{0, +\infty\}$, $\{ \ldots, -n, \ldots, -1, 0\}$, $\segment{0}{a}$, $\desInt{0}{+\infty}$, $\segment{0}{+\infty}$.
    Neka je $\vjerojatnosniProstor$ vjerojatnsni prostor s filtracijom $\indFamilija{\famF_t}{t \in \masT}$.
    Stohasti\v cki proces $\niz{X_t}{t \in \masT}$ je \emph{martingal} (\emph{submartingal, supermartingal}), u odnosu na $(\famF_t)$, ako vrijedi:
    \begin{enumerate}[label=(\alph*)]
        \item   \label{defn:23.4.1}
        svaka $X_t$ je slu\v cajna varijabla u $L^1 (\masP)$
        \item   \label{defn:23.4.2}
        $(X_t)$ je $\{\famF_t\}$-adaptiran, to jest $X_t$ je $\famF_t$-izmjeriv za svaki $t \in \masT$
        \item   \label{defn:23.4.3}
        za svaki $s \leq t$, $s, t \in \masT$, vrijedi
        \begin{equation*}
            \masE [X_t | \famF_s] = (\geq, \leq ) X_s.
        \end{equation*}
    \end{enumerate}
\end{defn}

Uo\v cimo nekoliko svojstava koja direktno slijedi iz definicije.
Stohasti\v cki proces je martingal ako i samo ako je supermartingal i submartingal.
Stohasti\v cki proces $(X_t)$ je supermartingal ako i samo ako je $(-X_t)$ submartingal (stoga se rezultati za jednu klasu procesa direktno prenose na drugu).
Koriste\' ci linearnost uvjetnog o\v cekivanja direktno dobijemo

\begin{prop}    \label{prop:23.5}
    Neka su $\niz{X_t}{t \in \masT}$ i $\niz{Y_t}{t \in \masT}$ martingali u odnosu na istu filtraciju $\indFamilija{\famF_t}{t \in \masT}$, te neka su $\alpha, \beta \in \real$, tada je i $\niz{\alpha X_t + \beta Y_t}{t \in \masT}$ martingal s obzirom na filtraciju $\indFamilija{\famF_t}{t \in \masT}$.
\end{prop}

\begin{prop}    \label{prop:23.6}
    Neka su $\niz{X_t}{t \in \masT}$ i $\niz{Y_t}{t \in \masT}$ supermartingali u odnosu na istu filtraciju $\indFamilija{\famF_t}{t \in \masT}$, te neka su $\alpha \geq 0, \beta \geq 0$, tada je i $\niz{\alpha X_t + \beta Y_t}{t \in \masT}$ supermartingal s obzirom na filtraciju $\indFamilija{\famF_t}{t \in \masT}$.
\end{prop}

\begin{zad} \label{zad:23.7}
    Ako su $\niz{X_t}{t \in \masT}$, $\niz{Y_t}{t \in \masT}$ supermartingali u odnosu na filtraciju $\indFamilija{\famF_t}{t \in \masT}$, tada je $\niz{X_t \land Y_t}{t \in \masT}$ supermartingal s obzirom na istu filtraciju.
\end{zad}

\begin{rj}[\ref{zad:23.7}]
    Neka su $s, t \in \masT$ takvi da je $s \leq t$.
    Slu\v cajna varijabla $X_t \land Y_t$ je $\famF_t$-izmjeriva, kao minimum izmjerivih funkcija, tako\dj er vrijedi
    \begin{equation*}
        \masE [X_t \land Y_t] \leq \masE X_t < +\infty,
    \end{equation*}
    dakle $X_t \land Y_t \in L^1 (\masP)$.

    Primjetimo
    \begin{equation*}
        X_t \land Y_t \leq X_t
    \end{equation*}
    sada po monotonosti uvjetnog o\v cekivanja imamo
    \begin{equation*}
        \masE [X_t \land Y_t | \famF_s] \leq \masE [X_t | \famF_s] \leq X_s,
    \end{equation*}
    gdje zadnja nejednakost slijedi iz \v cinjenice da je $\niz{X_t}{t \in \masT}$ supermartingal.
    Analogno dobijemo izraz
    \begin{equation*}
        \masE [X_t \land Y_t | \famF_s] \leq Y_s,
    \end{equation*}
    odakle zaklju\v cujemo da je
    \begin{equation*}
        \masE [X_t \land Y_t | \famF_s] \leq X_s \land Y_s,
    \end{equation*}
    \v sto je i trebalo pokazati.
\end{rj}

\begin{pr}[$1$-dimenzionalne su\v cajne \v setnje]  \label{pr:23.8}
    \quad \\
    Primjer \ref{pr:23.1} se da lako poop\' citi.
    Neka su $\niz{X_n}{n \in \nat}$ nezavisne, jednako distribuirane slu\v cajne varijable, takve da je $X_n \in L^1 (\masP)$, za svaki $n \in \nat$.
    Neka je $S_0 := 0, S_n := X_1 + \ldots + X_n$, pripadna slu\v cajna \v setnja i $\famF_n := \sigAlg{X_1, \ldots, X_n}$.
    Tada je $\niz{S_n}{n \in \nat_0}$ stohasti\v cki proces koji je $\{\famF_n\}$-adaptiran i $S_n \in L^1 (\masP)$.
    Zbog
    \begin{equation*}
        \masE [S_{n+1} | \famF_n] = \masE X_{n + 1} + \masE S_n = \masE X_1 + S_n,
    \end{equation*}
    slijedi
    \begin{equation*}
        \niz{S_n}{n \in \nat_0} \; \textnormal{ je } \quad
        \begin{cases}
            \textnormal{supermartingal ako je } &\masE X_1 < 0\\
            \textnormal{martingal ako je } &\masE X_1 = 0\\
            \textnormal{submartingal ako je } &\masE X_1 > 0
        \end{cases}.
    \end{equation*}
\end{pr}

\begin{pr}[L\' evyjevi procesi] \label{pr:23.9}
    Ideje iz primjera \ref{pr:23.8} se lako prenose i na ovaj slu\v caj.
    Neka je $\niz{X_t}{t \in \desInt{0}{+\infty}}$ L\' evyjev proces sa svojstvom $X_t \in L^1 (\masP)$, za svaki $t \geq 0$.
    Neka je $\famF_t := \indSigAlg{X_s}{0 \leq s \leq t}$, za svaki $t \geq 0$.
    O\v cito je $\niz{X_t}{t \geq 0}$ $\{\famF_t\}$-adaptiran.
    Neka je $0 \leq s < t$, imamo
    \begin{equation*}
        \begin{aligned}
            \masE [X_t | \famF_s] &= \masE [X_t - X_s + X_s | \famF_s] =
            \begin{psmallmatrix}
                \textnormal{adaptiranost}
            \end{psmallmatrix}\\
            &= \masE [X_t - X_s | \famF_s] + X_s =
            \begin{psmallmatrix}
                \textnormal{napomena \ref{nap:21.16}}\\
                \textnormal{propozicija \ref{prop:22.16}}
            \end{psmallmatrix}
            = \masE [X_t - X_s] + X_s\\
            &= \masE [X_{t - s}] + X_s.
        \end{aligned}
    \end{equation*}
    Zbog propozicije \ref{prop:23.5} i korolara \ref{kor:17.7} \ref{kor:17.7.1} slijedi
    \begin{equation*}
        \masE [X_{t - s}] = (t - s) \cdot \masE X_1.
    \end{equation*}
    Dakle imamo
    \begin{equation*}
        \niz{X_t}{ t \geq 0} \; \textnormal{ je} \quad
        \begin{cases}
            \textnormal{supermartingal ako je } &\masE X_1 < 0\\
            \textnormal{martingal ako je } &\masE X_1 = 0\\
            \textnormal{submartingal ako je } &\masE X_1 > 0
        \end{cases}.
    \end{equation*}
    Posebno, primjetimo da je Brownovo gibanje $(B_t)$ martingal, dok je Poissonov proces $(N_t)$ submartingal.
    Ako je $\lambda > 0$ parametar Poissonovog procesa, uo\v cimo da gornji izvod pokazuje da je $\niz{N_t - \lambda \cdot t}{t \geq 0}$ martingal.
\end{pr}