% poglavlje 2.5 - bernoullijeva razdioba i primjene -> 10
\chapter{Bernoullijeva razdioba i primjene} \label{pog:2.5}

Najjednostavnija slu\v cajna varijabla je degenerirana.
Budu\' ci je suma degeneriranih slu\v cajnih varijabli ponovo degenerirana slu\v cajna varijabla, lako je opisati \v sto se doga\dj a s degeneriranim slu\v cajem.
Prvi slu\v caj u kojem se javalja "prava slu\v cajnost" je razdijoba s dvije mogu\' ce vjerojatnosti

\begin{equation*}
    \begin{pmatrix}
        a & b\\
        q & p
    \end{pmatrix}
    \quad \quad
    \begin{matrix}
        a, \; b \in \real, \; a \neq b,\\
        0 < p < 1, \; p + q = 1.
    \end{matrix}
\end{equation*}
Budu\' ci je lako $a$ i $b$ transformirati u neke druge vrijednosti $c$, $d$, promatrati \' cemo slu\v caj kada je $a = 0$, $b = 1$ ili $a = -1$, $b = 1$.
Takva distribucija naziva se Bernoullijeva (\v svicarska matemati\v cka obitelj s preko deset matemati\v cara, nije uvijek jasno tko je \v sto radio); uglavnom u \v cast Jakoba Bernoullija (1654 - 1705).
Podsjetimo se da, ako su $X_1, \ldots, X_n$ nezavisne i za svaki $1 \leq i \leq n$,
\begin{equation*}
    X_i \sim
    \begin{pmatrix}
        0 & 1\\
        q & p
    \end{pmatrix},   
\end{equation*}
tada je
\begin{equation*}
    S_n := X_1 + \ldots + X_n \quad \sim B \urePar{n}{p}
\end{equation*}
binomna slu\v cajna varijabla i za $k \in \{0, \; 1, \ldots, n \}$
\begin{equation*}
    \vjeroj{S_n = k} = {n \choose k} p^k q^{n - k}.
\end{equation*}
Za ovakve ra\v cune \v cesto koristimo Stirlingovu formulu.

\begin{zad} \label{zad:10.1}
    Doka\v zite (koriste\' ci literaturu) da vrijedi formula koju je 1730. dokazao James Stirling:
    Za svaki $n \in \nat$,
    \begin{equation*}
        n! = \sqrt{2 \pi} \: n^{n + \frac{1}{2}} e^{-n + \varepsilon_n}
    \end{equation*}
    pri \v cemu je $\frac{1}{12 n + 1} < \varepsilon_n < \frac{1}{12 n}$.
\end{zad}

Zato za aproksimaciju uzmemo:
    \begin{equation*}
        n! \approx \sqrt{2 \pi} \: n^{n + \frac{1}{2}} e^{-n}
    \end{equation*}
i kako je $n$ ve\' ci, aproksimacija je sve bolje (na primjer za $n = 10$, gre\v ska je $0.8\%$, a za $n = 100$ iznosi $0.08\%$).

\begin{pr}  \label{pr:10.2}
    Pozivi pristi\v zu u telefonsku centralu na slu\v cajan na\v cin (ili automobili kroz kontrolnu to\v cku na autoputu, ili kupci u du\' can; op\' cenito "klijenti" nekog "servisa").
    Razumno je pretpostaviti da nema "dogovora" me\dj u korisnicima, to jest da se radi o \emph{nezavisnim} slu\v cajnim pojavama.
    Obi\v cno imamo ve\' ci broj podataka, pa se mo\v ze odrediti "vjerojatnost pojave barem jednog klijenta u jedinici vremena" $= p \in \obInt{0}{1}$.
    Sli\v cno, mo\v zemo izmjeriti "prosje\v cno vrijeme koje protekne izme\dj u pojave dva uzastopna klijenta" $= \mu \in \obInt{0}{+\infty}$.
    Uo\v cimo da $\lambda := \frac{1}{\mu}$ mo\v zemo interpretirati kao frekvenciju $=$ "prosje\v can broj dospjelih poziva u jedinici vremena".
    Promatramo vremenski interval jedini\v cne duljine i stavimo $I_0^1 = 1$ (uspjeh) ako se u intervalu javi bar jedan korisnik, a $I_0^1 = 0$ (neuspjeh), u suprotnom, dakle:
    \begin{equation*}
    I_0^1 \sim
    \begin{pmatrix} 0 & 1\\
        q & p \end{pmatrix},
    \end{equation*}
    sa $q:= 1 - p$.
    Promatrajmo sada vremensku skalu $t \in \lijInt{0}{+\infty}$, tako da za svaki interval jedini\v cne duljine oblika $\lijInt{n - 1}{n}$ imamo slu\v cajnu varijablu $I_0^n$ koja mjeri uspjeh ili neuspjeh ovisno o tome je li u $\lijInt{n - 1}{n}$ do\v slo do barem jednog poziva ili ne.
    Tada $\niz{I_0^n}{n \in \nat}$ tvore niz nezavisnih, jednako distribuiranih slu\v cajnih varijabli
    \begin{equation*}
        I_0^n \sim
        \begin{pmatrix}
            0 & 1\\
            q & p
        \end{pmatrix}.
    \end{equation*}
    Ozna\v cimo li s $T_0^n =$ "vrijeme (interval) $n$-tog poziva", intuitivno je prihvatljivo da su $\niz{T_0^n}{n \in \nat}$ nezavisne i jednako distribuirane.
    Dakle $T_0^1 \distJed T_0^n$ i vrijedi
    \begin{equation*}
        \begin{aligned}
        \vjeroj{T_0^n = m} &= \vjeroj{I_0^1 = 0, \ldots, I_0^{m - 1} = 0,  \; I_0^m = 1}\\
        &= p \cdot q^{m - 1}, \quad m \in \nat.
        \end{aligned} 
    \end{equation*}
    Ovakva razdioba naziva se \emph{geometrijskom}.

    Vr\v simo sada induktivno korake "profinjenja" tako da u svakom koraku podjelimo postoje\' ce intervale na pola.
    Dakle, u $l$-tom koraku dobijemo intervale duljine $\frac{1}{2^l}$ te varijable $\niz{I_l^n}{n \in \nat}$ koje su nezavisne i jednako distribuirane
    \begin{equation*}
        I_l^n \sim
        \begin{pmatrix}
            0 & 1 \\
            q_l & p_l
        \end{pmatrix}
    \end{equation*}
    (uspjeh ili neuspjeh u $\lijInt{\frac{n - 1}{2^l}}{\frac{n}{2^l}}$) i $\niz{T_l^n}{n \in \nat}$ vremena $n$-tog poziva u "mre\v zi fino\' ce $\frac{1}{2^l}$".
    Dakle, $\vjeroj{T_l^n = m} = p_l \: q_l^{m - 1}$, a stvarno vrijeme $n$-tog poziva je $\frac{m}{2^l}$.
    Koja je veza izme\dj u $p_l$ i $p$?
    Uo\v cimo
    \begin{equation*}
        \begin{aligned}
            q &= \vjeroj{I_0^1 = 0} = \vjeroj{I_l^1 = 0, \ldots, I_l^{2^l} = 0} = (q_l)^{2^l} \implies\\
            p &= 1 - (1 - p_l)^{2^l}.
        \end{aligned}
    \end{equation*}
    Ozna\v cimo li s $V_l =$ "pravo vrijeme prvog poziva u mre\v zi fino\' ce $\frac{1}{2^l}$", dobivamo
    \begin{equation*}
        \begin{aligned}
            \masE (V_l) &= \masE \Big(\frac{T_l^1}{2^l} \Big) = \frac{p_l}{2^l} \suma{m = 1}{\infty} m \: q_l^{m - 1} = \frac{p_l}{2^l} \cdot \frac{1}{q_l} \: \suma{m = 1}{\infty} m \: q_l^m\\
            &= \frac{p_l}{2^l \: q_l} \cdot \suma{m = 1}{\infty} \big( \suma{k = 1}{m} 1 \cdot q_l^m \big) \overset{Fubini}{=} \frac{p_l}{2^l \: q_l} \suma{k = 1}{\infty} \: \suma{m = k}{\infty} q_l^m\\
            &= \frac{p_l}{2^l \: q_l} \suma{k = 1}{\infty} q_l^k \: \underbrace{\suma{m = k}{\infty} q_l^{m - k}}_{= \frac{1}{1-q_l} = \frac{1}{p_l}} = \frac{1}{2^l \: q_l} \cdot \suma{k = 1}{\infty} q_l^k = \frac{1}{2^l \: \cancel{q_l}} \cdot \frac{\cancel{q_l}}{1 - q_l}\\
            &= \frac{1}{p_l \: 2^l}.
        \end{aligned}
    \end{equation*}
    Neka su sada $\niz{T_n}{n \in \nat}$ vremena $n$-tog poziva u kontinuiranoj skali.
    Tada je $\mu = \masE (T^n) = \masE (T^1)$ i ima smisla
    \begin{equation*}
        \begin{aligned}
            & \lim\limits_{l \to \infty} \masE (V_l) = \masE (T^1) \implies \\
            \lambda = & \lim\limits_{l \to \infty} p_l \: 2^l = \lim\limits_{l \to \infty} \: (-1) \cdot \frac{(1-p)^{\frac{1}{2^l}}}{\frac{1}{2^l}}\\
            = & - \ln (1 - p) \implies\\
            p =& 1 - e^{-\lambda}.
        \end{aligned}
    \end{equation*}
    Potpuno analogno mogli bi gledati svako vrijeme $t>0$ i dobili bismo
    \begin{equation*}
        p(t) = \vjeroj{T^1 \leq t} = 1 - e^{- \lambda \: t},
    \end{equation*}
    to jest, svaka $T^n \sim Exp(\lambda)$.

    Promatramo li broj poziva ostvarenih do momenta $t = 1$ i ozna\v cimo sa $N_1$, slijedi da je za $k \in \nat_0$
    \begin{equation*}
        \begin{aligned}
            \vjeroj{N_1 = k} &= \lim\limits_{l \to \infty} \masP \Big(\suma{m = 1}{2^l} I_l^m = k \Big) = \lim\limits_{l \to \infty} {2^l \choose k} p_l^k \: q_l^{2^l - k}\\
            &= \lim\limits_{l \to \infty} \frac{1}{k!} \cdot (q_l)^{2^l} \cdot (2^l) \: (2^l - 1) \ldots (2^l - k + 1) \cdot \big( \frac{p_l}{q_l} \big)^k\\
            &= \frac{e^{-\lambda}}{k!} \cdot 1^k \cdot \Big[ \lim\limits_{l \to \infty} \frac{p_l \: 2^l}{q_l} \Big]^k\\
            &= \frac{e^{-\lambda}}{k!} \: \lambda^k.
        \end{aligned}
    \end{equation*}
    Sli\v cno za $t > 0$ dobivamo
    \begin{equation*}
        \vjeroj{N_t = k} = \frac{(\lambda \: t)^k}{k!} \: e^{-\lambda}.
    \end{equation*}
    Dakle imamo:
    \begin{equation*}
        \textnormal{geometrijska} \leadsto \textnormal{eksponencijalna, binomna} \leadsto \textnormal{Poissonova}.
    \end{equation*}
    Informacija o $T^n$ se mo\v ze \v citati iz $N_t$ i obratno jer 
    \begin{equation*}
        \Big\{ \suma{k = 1}{n} T^k \leq t \Big\} = \{ N_t \geq n \}.
    \end{equation*}
    Stohasti\v cki proces $\niz{N_t}{t \geq 0}$ je \emph{Poissonov proces s parametrom $\lambda > 0$}.
\end{pr}

U primjeru \ref{pr:10.2} promatrali smo Bernoullijeve slu\v cajne varijable kod kojih se vjerojatnost uspjeha mjenja na svakom nivou i bila je uskla\dj ena s veli\v cinom $l$ tako da
\begin{equation*}
    \frac{p_l}{\frac{1}{2^l}} \xrightarrow[l \to \infty]{} \lambda.
\end{equation*}
Pogledajmo tipi\v can slu\v caj primjera u koje se $p$ ne mjenja.

\begin{pr}  \label{pr:10.3}
    Neka je $\niz{X_n}{n \in \nat} \; \iid$ za koji vrijedi
    \begin{equation*}
        X_n \sim
        \begin{pmatrix}
            -1 & 1\\
            q & p
        \end{pmatrix},
    \end{equation*}
    $0 < p < 1$ i $q = 1 - p$. Vidjeli smo u poglavlju \ref{zakoni_01} da se takav niz uvijek mo\v ze realizirati (i to \v cak na $\segment{0}{1}$).
    Ovaj niz je model "beskona\v cne igre" u kojoj u svakom koraku dobijemo ili izgubimo nov\v canu jedinicu (primjene ovog modela su puno \v sire od "hazardnih igara").
    Neka je
    \begin{equation*}
        S_n = \textnormal{"na\v se dokumentirano bogatstvo u trenutku $n$"},
    \end{equation*}
    to jest bez smanjenja op\' cenitosti uzimamo
    \begin{equation*}
        \begin{aligned}
            S_0 &= 0,\\
            S_n &= X_1 + \ldots + X_n, \quad n \in \nat;
        \end{aligned}
    \end{equation*}
    uo\v cimo da je lako promatrati neku drugu po\v cetnu razdiobu $Y$ - samo uzmemo $V_n = Y + S_n$.
    O\v cito, ako je $n$ paran, $S_n$ mo\v ze posti\' ci samo parne vrijednosti izme\dj u $-n$ i $n$, te za $k = 0, \: 1, \ldots, n$ imamo
    \begin{equation*}
        \vjeroj{S_n = -n + 2K} = {n \choose k} p^k \: q^{n - k},
    \end{equation*}
    i sli\v cno dobijemo za neparne $n$.
    \v Sto se doga\dj a s trajektorijama $n \mapsto S_n(\omega)$ za velike $n$?
    Uo\v cimo da je funkcija
    \begin{equation*}
        f(\omega) := \limsup\limits_{n \to \infty} S_n(\omega)    
    \end{equation*}
    pro\v sirena slu\v cajna varijabla.
    Promatrajmo $\overline{\real}^{\infty}$ i skup $A_c \subseteq \overline{\real}^\infty$, za $c \in \real$, definiran sa
    \begin{equation*}
        A_c := \bigSkup{(r_n) \in \overline{\real}^\infty}{\limsup\limits_{n \to \infty} r_n \leq c}.
    \end{equation*}
    Ako je $p$ kona\v cna permutacija od $\nat$, tada poredak prvih kona\v cno brojeva ne\' ce utjecati na limes superior, pa je $\praslika{f} (A_c) = A_c$, to jest $A_c \in (\borel{\real}^\infty)_\simetr$.
    Budu\' ci su $(X_n) \; \iid$ po Hewitt-Savageovom zakon slijedi da je
    \begin{equation*}
        \{f \leq c\} \in \praslika{X} (\borel{\real}^\infty)_\simetr.
    \end{equation*}
    To jest $f$ je izmjeriva s obzirom na trivijalnu $\sigma$-algebru.
    Po lemi \ref{lm:9.6} postoji to\v cno jedan $b \in \extReal$, takav da je
    \begin{equation*}
        \limsup\limits_{n \to \infty} S_n = b \; (g.s.).
    \end{equation*}
    Na isti na\v cin dobijemo
    \begin{equation*}
        \liminf\limits_{n \to \infty} S_n = a \; (g.s.),
    \end{equation*}
    i vrijedi $a, \; b \in \extReal$, $a \leq b$.
    Sli\v cno zaklju\v cujemo u slu\v caju skupa
    \begin{equation*}
        A_B := \bigSkup{(r_n) \in \extReal^\infty}{(r_n) \textnormal{ u $B$ beskona\v cno puta}},
    \end{equation*}
    pri \v cemu je $B \in \borel{\extReal}$.
    Opet po Hewitt-Savage-ovom zakonu vrijedi:
    \begin{equation*}
        \vjeroj{S_n \in B \; \io} \in \{0, \; 1\}.
    \end{equation*}
    % \limsup\limits_{n \to \infty} \{ S_n \in B\} ?= S_n \in B \; \io

    Mo\v ze li biti $b \in \real$?

    Uo\v cimo prvo da zbog nezavisnosti i jednake distribuiranosti vrijedi
    \begin{equation*}
        (X_1, \; X_2, \ldots) \distJed (X_2, \; X_3, \ldots),
    \end{equation*}
    \v sto daje 
    \begin{equation*}
        \limsup\limits_{n \to \infty} (S_{n + 1} - X_1) = b \; (g.s.).
    \end{equation*}
    Stoga dobivamo:
    \begin{equation*}
        \begin{gathered}
            \begin{aligned}
                b &= \limsup\limits_{n \to \infty} S_{n + 1}\\
                &= \limsup\limits_{n \to \infty} \big[ (S_{n + 1} - X_1) + X_1 \big]\\
                &= b + X_1 \; (g.s.)
            \end{aligned}\\
            \implies X_1 = 0 \; (g.s.),
        \end{gathered}
    \end{equation*}
    \v sto je kontradikcija.
    Dakle, $a, b \in \{-\infty, \; +\infty\}$.

    Posebno, ako je $b = -\infty$ i $a = + \infty$, tada
    \begin{equation*}
        \begin{aligned}
            \lim\limits_{n  \to \infty} S_n &= -\infty \; (g.s.)\\
            \lim\limits_{n \to \infty} S_n &= +\infty \; (g.s.).
        \end{aligned}
    \end{equation*}
    Ako je $a = -\infty$, $b = +\infty$, tada gotovo sigurno trajektorije osciliraju sve vi\v se i vi\v se.

    Ako je $p = q = \frac{1}{2}$, tada je $(-X_1, \; -X_2, \ldots) \distJed (X_1, \; X_2, \ldots)$, pa je
    \begin{equation*}
        \limsup\limits_{n \to \infty} (-S_n) = \limsup\limits_{n \to \infty} S_n = b,
    \end{equation*}
    a s druge strane je
    \begin{equation*}
        \limsup\limits_{n \to \infty} (-S_n) = - \liminf\limits_{n \to \infty} S_n = -a,
    \end{equation*}
    a kako ne mo\v ze biti $b = -a$ i $b = a$ istovremeno, mora biti $a = -\infty$, $b = +\infty$.
    %istraži malo ove limsup i liminf

    Ako je $p \neq q$, onda promatramo $\varepsilon > 0$ i vrijedi
    \begin{equation*}
        \begin{aligned}
            \masP \Bigg( \Bigg|\frac{S_n}{n} - (p - q) \Bigg| \geq \varepsilon \Bigg) =& \suma{|k - n \: (p-q)| \geq n \varepsilon}{} \vjeroj{S_n = k} \leq \suma{k \in R_{S_n}}{} \frac{(k - n(p-q))^4}{n^4 \: \varepsilon^4} \: \vjeroj{S_n = k}\\
            %
            %   dokaži jednakost
            %
            =& \frac{k n + c}{n^3 \: \varepsilon^4} \implies\\
            \implies & \suma{n = 1}{\infty} \masP \Bigg( \Bigg| \frac{S_n}{n} - (p - q) \Bigg| \geq \varepsilon \Bigg) < + \infty, \quad \forall \varepsilon > 0.\\
            \overset{\ref{lm:9.2}}{\implies} & \masP \Bigg( \Bigg| \frac{S_n}{n} - (p - q) \Bigg| \geq \varepsilon, \; \io \Bigg) = 0\\
            \overset{\textnormal{Uz }\; \varepsilon = \frac{|p - q|}{2}}{\implies} & \vjeroj{\sign (S_n) \neq \sign (p - q) \; \io} = 0\\
            \implies & \lim\limits_{n \to \infty} S_n
            =
            \begin{cases}
                + \infty, &p > q\\
                - \infty, &q < p
            \end{cases}
            \quad (g.s.)
        \end{aligned}
    \end{equation*}
\end{pr}

Vidjet \' cemo uskoro da mo\v zemo dobiti i preciznije odgovore.
Uo\v cimo da u nekim od ovih dokaza nije bila va\v zna precizna distribucija; to \' cemo precizirati u sljede\' cem zadatku.

\begin{zad} \label{zad:10.4}
    Neka je $(X_n) \; \iid$ niz, $S_0 = 0$, $S_n = X_1 + \ldots + X_n$.
    Doka\v zi:
    \begin{enumerate}[label=(\alph*)]
        \item $(\exists c \in \extReal)\; \limsup\limits_{n \to \infty} S_n = c \; (g.s.)$
        \item Ako $X_n$ nije degeneriran u $0$, tada je $c \in \{-\infty, \; +\infty\}$.
        \item Ako $X_n$ nije degeneriran u $0$ i simetri\v can je (tj. $X_n \distJed - X_n$), tada vrijedi
        \begin{equation*}
            \begin{aligned}
                \liminf\limits_{n \to \infty} S_n &= - \infty \; (g.s.)\\
                \limsup\limits_{n \to \infty} S_n &= +\infty \; (g.s.)
            \end{aligned}
        \end{equation*}
    \end{enumerate}
\end{zad}