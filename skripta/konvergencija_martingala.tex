% konvergencija martingala, cjelina 5.4 -> predavanje 24

\chapter{Konvergencija martingala}

Razmislimo o jednostavnoj strategiji prilikom igranja neke igre opisane submartingalom ili martingalom.
Na\v sa strategija \'ce biti ili "propu\v stamo igru na temelju postoje\' cih informacija" ili "ulazimo u igru ne temelju postoje\' cih informacija".
Slijede\' ci teorem ka\v ze da kako god izabrali takvu "strategiju propu\v stanja igre", ne\' cemo promijeniti prirodu igre.

\begin{tm}[Halmos, "optimal skipping theorem"]  \label{tm:24.1}
    \quad \\
    Neka je $\niz{X_n}{n \in \nat}$ submartingal (martingal) u odnosu na filtraciju $\indFamilija{\famF_n}{n \in \nat}$.
    Neka je $\indFamilija{B_n}{n \in \nat}$ niz Borelovih skupova, tako da je $B_n \in \borel{\real^n}$, za svaki $n \in \nat$.
    Neka su slu\v cajne varijable $\varepsilon_1, \varepsilon_2, \ldots$ definirane sa
    \begin{equation*}
        \varepsilon_k :=
        \begin{cases}
            1 &, (X_1, \ldots, X_k) \in B_k\\
            0 &, (X_1, \ldots, X_k) \notin B_k
        \end{cases},
    \end{equation*}
    i $\niz{Y_n}{n \in \nat}$ niz slu\v cajnih varijabli definiranih sa
    \begin{equation*}
        \begin{gathered}
            Y_1 := X_1,\\
            Y_{k + 1} := Y_k + \varepsilon_k \cdot (X_{k + 1} - X_k).
        \end{gathered}
    \end{equation*}
    Tada je $(Y_n)$ submartingal (martingal) u odnosu na filtraciju $\indFamilija{\famF_n}{n \in \nat}$ i vrijedi $\masE Y_n \leq \masE X_n$ ($\masE Y_n = \masE X_n$), za svaki $n \in \nat$.
\end{tm}

\begin{proof}
    O\v cito je $\varepsilon_n$ $\famF_n$-izmjeriva, pa je $(Y_n)$ $\{\famF_n\}$-adaptiran niz iz $L^1(\masP)$.
    Uo\v cimo da je $\masE Y_1 = \masE X_1$ i promotrimo
    \begin{equation*}
        \begin{gathered}
            \masE [Y_{n + 1} | \famF_n] = Y_n + \varepsilon_n \cdot \masE [X_{n + 1} - X_n | \famF_n]\\
            \begin{aligned}
                \textnormal{submartingal} &\implies \masE [X_{n + 1} - X_n | \famF_n] \geq 0\\
                \textnormal{martingal} &\implies \masE [X_{n + 1} - X_n | \famF_n] = 0.
            \end{aligned}
        \end{gathered}
    \end{equation*}
    Primjetimo
    \begin{equation*}
        X_{n + 1} - Y_{n + 1} = (1 - \varepsilon_n) (X_{n + 1} - X_n) + (X_n - Y_n),
    \end{equation*}
    slijedi
    \begin{equation*}
        \masE [X_{n + 1} - Y_{n + 1} | \famF]
        \begin{smallmatrix}
            \geq\\
            (=)
        \end{smallmatrix}
        \masE [X_n - Y_n | \famF_n] = X_n - Y_n,
    \end{equation*}
    odakle vidimo da vrijedi
    \begin{equation*}
        \masE [X_{n + 1} - Y_{n + 1}]
        \begin{smallmatrix}
            \geq\\
            (=)
        \end{smallmatrix}
        \masE [X_n - Y_n]
    \end{equation*}
    \v sto daje tvrdnju.
\end{proof}

Ako je $x_1, \ldots, X_n$ niz brojeva i $\obInt{a}{b}$ otvoreni interval.
Mo\v zemo brojati koliko puta ovaj niz "presko\v ci" $\obInt{a}{b}$ prema gore.
Stavimo
\begin{equation*}
    \begin{aligned}
        T_1 &:=
        \begin{cases}
            \min \indFamilija{k}{1 \leq k \leq n, \; x_k \leq a}\\
            +\infty, \quad \indFamilija{k}{1 \leq k \leq n, \; x_k \leq a} = \varnothing
        \end{cases},\\
        T_2 &:=
        \begin{cases}
            \min \indFamilija{k}{T_1 < k \leq n, \; x_k \geq b}\\
            +\infty, \quad \indFamilija{k}{T_1 < k \leq n, \; x_k \geq b} = \varnothing
        \end{cases}.
    \end{aligned}
\end{equation*}
Neka je
\begin{equation*}
    N := \mathrm{card} \indFamilija{1 \leq k \leq n}{T_k < +\infty},
\end{equation*}
uz konvenciju $\mathrm{card} \varnothing = +\infty$.
Stavimo
\begin{equation*}
    U_{a b} :=
    \begin{cases}
        \frac{N}{2},& N \textnormal{ paran}\\
        \frac{N + 1}{2},& N \textnormal{ neparan}\\
        0,& N = +\infty
    \end{cases}
\end{equation*}

\begin{tm}[J. L. Doob, "upcrossing theorem"]    \label{tm:24.2}
    \quad \\
    Neka su $a, b \in \real$, $a < b$ i neka je $\niz{X_k}{k = 1, \leq, n}$ submartingal u odnosu na filtraciju $\indFamilija{\famF_k}{k = 1, \ldots, n}$.
    Neka je $U_{a b}$ slu\v cajna varijabla definirana pomo\' cu $X_1, \ldots, X_n$.
    Tada je
    \begin{equation*}
        \masE [U_{a b}] \leq \frac{1}{b - a} \masE [(X_n - a)^+].
    \end{equation*}
\end{tm}