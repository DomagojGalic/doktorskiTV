% konvergencija martingala, cjelina 5.4 -> predavanje 24

\chapter{Konvergencija martingala}

Razmislimo o jednostavnoj strategiji prilikom igranja neke igre opisane submartingalom ili martingalom.
Na\v sa strategija \'ce biti ili "propu\v stamo igru na temelju postoje\' cih informacija" ili "ulazimo u igru ne temelju postoje\' cih informacija".
Slijede\' ci teorem ka\v ze da kako god izabrali takvu "strategiju propu\v stanja igre", ne\' cemo promijeniti prirodu igre.

\begin{tm}[Halmos, "optimal skipping theorem"]  \label{tm:24.1}
    \quad \\
    Neka je $\niz{X_n}{n \in \nat}$ submartingal (martingal) u odnosu na filtraciju $\indFamilija{\famF_n}{n \in \nat}$.
    Neka je $\indFamilija{B_n}{n \in \nat}$ niz Borelovih skupova, tako da je $B_n \in \borel{\real^n}$, za svaki $n \in \nat$.
    Neka su slu\v cajne varijable $\varepsilon_1, \varepsilon_2, \ldots$ definirane sa
    \begin{equation*}
        \begin{aligned}
            \varepsilon_k :&=
        \begin{cases}
            1 ,& (X_1, \ldots, X_k) \in B_k\\
            0 ,& (X_1, \ldots, X_k) \notin B_k
        \end{cases}\\
        &= \karaktFja_{B_k} (X_1, \ldots, X_k),
        \end{aligned}
    \end{equation*}
    i $\niz{Y_n}{n \in \nat}$ niz slu\v cajnih varijabli definiranih sa
    \begin{equation*}
        \begin{gathered}
            Y_1 := X_1,\\
            Y_{k + 1} := Y_k + \varepsilon_k \cdot (X_{k + 1} - X_k).
        \end{gathered}
    \end{equation*}
    Tada je $(Y_n)$ submartingal (martingal) u odnosu na filtraciju $\indFamilija{\famF_n}{n \in \nat}$ i vrijedi $\masE Y_n \leq \masE X_n$ ($\masE Y_n = \masE X_n$), za svaki $n \in \nat$.
\end{tm}

\begin{proof}
    O\v cito je $\varepsilon_n$ $\famF_n$-izmjeriva, pa je $(Y_n)$ $\{\famF_n\}$-adaptiran niz iz $L^1(\masP)$.
    Uo\v cimo da je $\masE Y_1 = \masE X_1$ i promotrimo
    \begin{equation*}
        \begin{gathered}
            \masE [Y_{n + 1} | \famF_n] = Y_n + \varepsilon_n \cdot \masE [X_{n + 1} - X_n | \famF_n]\\
            \begin{aligned}
                \textnormal{submartingal} &\implies \masE [X_{n + 1} - X_n | \famF_n] \geq 0\\
                \textnormal{martingal} &\implies \masE [X_{n + 1} - X_n | \famF_n] = 0.
            \end{aligned}
        \end{gathered}
    \end{equation*}
    Primjetimo
    \begin{equation*}
        X_{n + 1} - Y_{n + 1} = (1 - \varepsilon_n) (X_{n + 1} - X_n) + (X_n - Y_n),
    \end{equation*}
    slijedi
    \begin{equation*}
        \masE [X_{n + 1} - Y_{n + 1} | \famF]
        \begin{smallmatrix}
            \geq\\
            (=)
        \end{smallmatrix}
        \masE [X_n - Y_n | \famF_n] = X_n - Y_n,
    \end{equation*}
    odakle vidimo da vrijedi
    \begin{equation*}
        \masE [X_{n + 1} - Y_{n + 1}]
        \begin{smallmatrix}
            \geq\\
            (=)
        \end{smallmatrix}
        \masE [X_n - Y_n]
    \end{equation*}
    \v sto daje tvrdnju.
\end{proof}

Ako je $x_1, \ldots, x_n$ niz brojeva i $\obInt{a}{b}$ otvoreni interval.
Mo\v zemo brojati koliko puta ovaj niz "presko\v ci" $\obInt{a}{b}$ prema gore.
Stavimo
\begin{equation*}
    \begin{aligned}
        T_1 &:=
        \begin{cases}
            \min \: \bigIndFamilija{k \in \{1, \ldots, n\}}{ x_k \leq a}\\
            +\infty, \quad \bigIndFamilija{k \in \{1, \ldots, n\}}{ x_k \leq a} = \varnothing
        \end{cases},\\
        T_2 &:=
        \begin{cases}
            \min \indFamilija{ T_1 < k \leq n}{ x_k \geq b}\\
            +\infty, \quad \indFamilija{T_1 < k \leq n}{ x_k \geq b} = \varnothing
        \end{cases}.
    \end{aligned}
\end{equation*}
Neka je
\begin{equation*}
    N := \mathrm{card} \indFamilija{1 \leq k \leq n}{T_k < +\infty},
\end{equation*}
uz konvenciju $\mathrm{card} \varnothing = +\infty$.
Stavimo
\begin{equation*}
    U_{a b} :=
    \begin{cases}
        \frac{N}{2},& N \textnormal{ paran}\\
        \frac{N + 1}{2},& N \textnormal{ neparan}\\
        0,& N = +\infty
    \end{cases}
\end{equation*}

\begin{tm}[J. L. Doob, "upcrossing theorem"]    \label{tm:24.2}
    \quad \\
    Neka su $a, b \in \real$, $a < b$ i neka je $\niz{X_k}{k = 1, \ldots, n}$ submartingal u odnosu na filtraciju $\indFamilija{\famF_k}{k = 1, \ldots, n}$.
    Neka je $U_{a b}$ slu\v cajna varijabla definirana pomo\' cu $X_1, \ldots, X_n$.
    Tada je
    \begin{equation*}
        \masE [U_{a b}] \leq \frac{1}{b - a} \masE [(X_n - a)^+].
    \end{equation*}
\end{tm}

\begin{proof}
    Budu\' ci da je $\niz{(X_k - a)^+}{k = 1, \ldots, n}$ ponovo submartingal te pripadni $U_{0, b - a}$ je isti kao i $U_{a b}$ za polazni submartingal $(X_k)$, bez smanjenja op\' cenitosti mo\v zemo pretpostaviti da je $a = 0$ i $X_k \geq 0$, za svaki $k$ ura\v cunajmo samo preskoke prema gore, to jest u teoremu \ref{tm:24.1} biramo $\varepsilon_j$ tako da je $\varepsilon_j = 0$ za $j < T_1$, $\varepsilon_j = 1$, za $T_1 \leq j < T_2$, $\varepsilon_j = 0$, za $T_2 \leq j < T_3$, to jest ako je $i \in \nat$, tada
    \begin{equation*}
        \varepsilon_j =
        \begin{cases}
            1 ,& T_{2 i - 1} \leq j < T_{2 i}\\
            0 ,& \textnormal{ina\v ce}
        \end{cases}.
    \end{equation*}
    Slijedi da je
    \begin{equation*}
        Y_n \geq b U_{0, b},
    \end{equation*}
    odakle dobijemo
    \begin{equation*}
        \masE [U_{0, b}] \leq \frac{1}{b} \masE Y_n \leq \frac{1}{b} \masE X_n,
    \end{equation*}
    gdje zadnja nejednakost slijedi iz teorema \ref{tm:24.1}.
\end{proof}

Za sljede\' ci teorem uo\v cimo da za niz $\niz{x_n}{n \in \nat} \subseteq \real$ vrijedi
\begin{equation}    \label{jed:24.3}
    \begin{matrix}
        (X_n) \textnormal{ ne konvergira}\\
        \textnormal{u } \extReal
    \end{matrix}
    \quad \iff \quad
    \begin{matrix}
        \textnormal{postoje } a, b \in \masQ, \; a < b\\
        \liminf\limits_{n \to \infty} x_n \leq a < b \leq \limsup\limits_{n \to \infty} x_n
    \end{matrix}.
\end{equation}

\begin{tm}[O konvergenciji submartingala]   \label{tm:24.4}
    \quad \\
    Neka je $\niz{X_n}{n \in \nat}$ submartingal u odnosu na filtraciju $\indFamilija{\famF_n}{n \in \nat}$.
    Ako je
    \begin{equation*}
        \sup \: \bigIndFamilija{\masE [ X_n^+]}{n \in \nat} < +\infty,
    \end{equation*}
    tada postoji $X_\infty \in L^1(\masP)$, takav da je
    \begin{equation*}
        \lim\limits_{n \to \infty} X_n = X_\infty \; (g.s.).
    \end{equation*}
\end{tm}

\begin{proof}
    Ako za neke $a < b$, vrijedi
    \begin{equation*}
        \masP \bigNiz{\omega \in \Omega}{\liminf\limits_{n \to \infty} X_n (\omega) \leq a < b \leq \limsup\limits_{n \to \infty} X_n (\omega) } > 0,
    \end{equation*}
    tada vrijedi $U_{a b} = +\infty$, za cijeli niz.
    Uo\v cimo da je
    \begin{equation*}
        U_{a b} = \lim\limits_{n \to \infty} U_{a b ; n}.
    \end{equation*}
    Budu\' ci je niz $\niz{U_{a b; n}}{n \in \nat}$ rastu\' c vrijedi
    \begin{equation*}
        \masE [U_{a b}] = \lim\limits_{n \to \infty} \masE [U_{a b; n}].
    \end{equation*}
    No, imamo
    \begin{equation*}
        \masE [U_{a b; n}] \leq \frac{1}{b - a} \masE [(X_n - a)^+] \leq \frac{1}{b - a} \Big( \sup\limits_{n \in \nat} \: \masE [X_n^+] + a^- \Big) < +\infty,
    \end{equation*}
    % ovaj dio mi nije jasan :/
    \v sto je kontradikcija.

    Po \eqref{jed:24.3} postoji limes gotovo sigurno i to je $X_\infty$.
    Uo\v cimo
    \begin{equation*}
        |X_n| = X_n^+ + X_n^- = 2 \cdot X_n^+ - X_n .
    \end{equation*}
    Budu\' ci je $(X_n)$ submartingal, vrijedi $\masE X_n \geq \masE X_1$, odakle imamo
    \begin{equation*}
        \masE [|X_n|] \leq 2 \sup\limits_{n \in \nat} \: \masE [X_n^+] - \masE X_1 < +\infty,
    \end{equation*}
    pa prema Fatuovoj lemi imamo
    \begin{equation*}
        \masE [|X_\infty|] \leq \liminf\limits_{n \to \infty} \masE [|X_n|] \leq 2 \sup\limits_{n \in \nat} \: \masE [X_n^+] - \masE X_1 < +\infty.
    \end{equation*}
\end{proof}

\begin{zad} \label{zad:24.5}
    \begin{enumerate}[label=(\alph*)]
        \item \label{zad:24.5.1}
        Formulirajte i doka\v zite teorem konvergencije za supermartingale.
        \item \label{zad:24.5.2}
        Formulirajte i doka\v zite teorem konvergencije za submartingale u slu\v caju
        \begin{equation*}
            T = \{ \ldots, -n, \ldots, -1, 0 \}.
        \end{equation*}
        Opi\v site posebno slu\v caj martingala.
    \end{enumerate}
\end{zad}

\begin{nap} \label{nap:24.6}
    Uo\v cite da je u dokazu teorema \ref{tm:24.4} pokazano i da je familija $\indFamilija{X_n}{n \in \nat}$ $L^1 (\masP)$-ome\dj eno, to jest vrijedi
    \begin{equation*}
        \sup\limits_{n \in \nat} \: \masE [|X_n|] < +\infty.
    \end{equation*}
    Dakle za submartingale vrijedi
    \begin{equation*}
        \sup\limits_{n \in \nat} \: \masE [X_n^+] < +\infty \quad \implies \quad
        \begin{matrix}
            \textnormal{konvergencija i}\\
            L^1 (\masP) \textnormal{-ome\dj enost.}
        \end{matrix}
    \end{equation*}
    Submartingal mo\v ze konvergirati bez da bude $L^1 (\masP)$-ome\dj en.
\end{nap}

\begin{kor} \label{kor:24.7}
    Ako je $\niz{X_n}{n \in \nat}$ supermartingal i $X_n \geq 0$, za svaki $n \in \nat$, tada postoji $X \in L^1 (\masP)$ takav da vrijedi
    \begin{equation*}
        \lim\limits_{n \to \infty} X_n = X \; (g.s.)
    \end{equation*}
    te je $\masE X \leq \masE X_1$.
\end{kor}

\begin{proof}
    Stavimo
    \begin{equation*}
        Y_n := - X_n \leq 0,
    \end{equation*}
    daje submartingal za koji vrijedi
    \begin{equation*}
        \masE [Y_n^+] = 0,
    \end{equation*}
    pa iskoristimo teorem \ref{tm:24.4}.
    Zbog $\masE X_1 \geq \masE X_n$, zadnja tvrdnja slijedi iz Fatuove leme.
\end{proof}

Podsjetimo se da, u skladu s definicijom za pripadni $F_{X_n}$, je familija slu\v cajnih varijabli $\indFamilija{Y_n}{n \in \nat}$ na vjerojatnosnom prostoru $\vjerojatnosniProstor$ \emph{uniformno integrabilna} ako je
\begin{equation}    \label{jed:24.8}
    \lim\limits_{c \to \infty} \Big( \sup\limits_{n \in \nat} \: \int\limits_{\{ |Y_n| > c \}} |Y_n| \: d \masP \Big) = 0.
\end{equation}

Uo\v cimo da to posebno zna\v ci da je za svaku $Y_n \in L^1 (\masP)$.
Ako imamo kona\v cnu familiju $\indFamilija{Y_k}{k = 1, \ldots, n} \subseteq L^1 (\masP)$, lako se vidi da da je ta familija uniformno integrabilna.

Koriste\' ci Fatuovu lemu, doka\v zite ovo pro\v sirenje.

\begin{zad}  \label{zad:24.9}
    Neka je $\indFamilija{Y_n}{n \in \nat}$ uniformno integrabilna familija slu\v cajnih varijabli.
    Tada vrijedi:
    \begin{equation*}
        \int\limits_\Omega \Big( \liminf\limits_{n \to \infty} Y_n \Big) \: d \masP \leq \liminf\limits_{n \to \infty} \int\limits_\Omega Y_n \: d \masP \leq \limsup\limits_{n \to \infty} \int\limits_\Omega Y_n \: d \masP \leq \int\limits_\Omega \Big( \limsup\limits_{n \to \infty} Y_n \Big) \: d \masP.
    \end{equation*}
\end{zad}

\begin{tm}  \label{tm:24.10}
    Neka je $\indFamilija{Y_n}{n \in \nat}$ uniformno integrabilna funkcija slu\v cajnih varijabli.
    Ako je
    \begin{equation*}
        Y_n \xrightarrow[n \to \infty]{g.s.} Y,
    \end{equation*}
    tada je $Y \in L^1 (\masP)$ i vrijedi
    \begin{equation*}
        \masE Y_n \xrightarrow[n \to \infty]{} \masE Y.
    \end{equation*}
\end{tm}

\begin{proof}
    Uo\v cimo da vrijedi
    \begin{equation*}
        |Y| = (g.s.) \: \lim\limits_{n \to \infty} |Y_n| = \liminf\limits_{n \to \infty} |Y_n|
    \end{equation*}
    pa prema zadatku \ref{zad:24.9} vrijedi
    \begin{equation*}
        \int\limits_\Omega |Y| \: d \masP \leq \liminf\limits_{n \to \infty} \int\limits_\Omega |Y_n| \: d \masP.
    \end{equation*}
    No, zbog uniformne integrabilnosti, za $\varepsilon = 1 > 0$, postoji $c$ dovoljno velik, tako da vrijedi
    \begin{equation}    \label{jed:24.11}
        \begin{aligned}
            \int\limits_\Omega |Y_n| \: d \masP = \int\limits_{\{ |Y_n| > c \}} |Y_n| \: d \masP + \int\limits_{\{ |Y_n| \leq c \}} |Y_n| \: d \masP &\leq 1 + c \cdot \masP (|Y_n| \leq c)\\
            &\leq 1 + c < +\infty.
        \end{aligned}
    \end{equation}
    Odavde slijedi da je $Y \in L^1 (\masP)$.
    Koriste\' ci zadatak \ref{zad:24.9} dobijemo
    \begin{equation*}
        \masE Y \leq \liminf\limits_{n \to \infty} \masE Y_n \leq \limsup\limits_{n \to \infty} \masE Y_n \leq \masE Y,
    \end{equation*}
    \v sto daje tvrdnju.
\end{proof}

\begin{nap} \label{nap:24.12}
    \begin{enumerate}[label=(\alph*)]
        \item \label{nap:24.12.1}
        Uo\v cimo da prelaskom na podnizove lako doka\v zemo gornju tvrdnju i za slu\v caj $Y_n \xrightarrow[n \to \infty]{\masP} Y$.
        \item \label{nap:24.12.2}
        U \eqref{jed:24.11} smo pokazali da svaka uniformno integrabilna familija mora biti $L^1 (\masP)$-ome\dj ena.
        Obrat ne mora vrijediti, ali vrijedi sljede\' ci teorem.
    \end{enumerate}
\end{nap}

\begin{tm}  \label{tm:24.13}
    Familija $\indFamilija{Y_n}{n \in \nat}$ je uniformno integrabilna ako i samo ako je $L^1 (\masP)$-ome\dj ena i uniformno neprekidna, to jest vrijedi
    \begin{equation*}
        \lim\limits_{\masP (A) \to 0} \Big( \sup\limits_{n \in \nat} \: \int\limits_A |Y_n| \: d \masP \Big) = 0.
    \end{equation*}
\end{tm}

\begin{proof}
    Po \v Cebi\v sevljevoj nejednakosti imamo
    \begin{equation*}
        \masP ( |Y_n| \geq c) \leq \frac{1}{n} \masE [|Y_n|],
    \end{equation*}
    iz \v cega slijedi dovoljnost.
    Kao i u \eqref{jed:24.11} dobijemo
    \begin{equation*}
        \int\limits_A |Y_n| \: d \masP \leq \int\limits_{\{ |Y_n| \geq c \}} |Y_n| \: d \masP + c \cdot \masP ( A ),
    \end{equation*}
    \v sto daje nu\v znost.
\end{proof}

\begin{tm}  \label{tm:24.14}
    Ako je $1 \leq p < +\infty$, $\indFamilija{|Y_n|^p}{n \in \nat}$ je uniformno integrabilna i
    \begin{equation*}
        Y_n \xrightarrow[n \to \infty]{g.s.} Y,
    \end{equation*}
    tada,
    \begin{equation*}
        Y_n \xrightarrow[n \to \infty]{L^p} Y.
    \end{equation*}
\end{tm}

\begin{proof}
    Zbog \v cinjenice da je
    \begin{equation*}
        |Y_n|^p \xrightarrow[n \to \infty]{g.s.} |Y|^p,
    \end{equation*}
    teorem \ref{tm:24.10} daje $Y \in L^p (\masP)$.
    No tada je $\indFamilija{|Y_n - Y|^p}{n \in \nat}$ uniformno integrabilna familija te te\v zi u $0$ gotovo sigurno.
    Sada tvrdnja slijedi po teoremu \ref{tm:24.10}.
\end{proof}

% uputa: koristi teorem 24.13
\begin{zad} \label{zad:24.15}
    Poka\v zi obrat teorema \ref{tm:24.14}.
\end{zad}

\begin{zad} \label{zad:24.16}
    Neka je $Y \in L^1 (\masP)$ i neka je $\indFamilija{\famF_n}{n \in \nat}$ filtracija.
    Poka\v zi da je $\indFamilija{\masE [Y | \famF_n]}{n \in \nat}$ uniformno integrabilna.
\end{zad}

\begin{tm}  \label{tm:24.17}
    Neka je $\niz{X_n}{n \in \nat}$ submartingal s obzirom na filtraciju $\indFamilija{\famF_n}{n \in \nat}$.
    Tada je $\indFamilija{X_n}{n \in \nat}$ uniformno integrabilna ako i samo postoji $X_\infty \in L^1 (\masP)$ takva da je
    \begin{equation*}
        X_n \xrightarrow[n \to \infty]{L^1} X_\infty.
    \end{equation*}
    Ako je $\niz{X_n}{n \in \nat}$ martingal, slijedi
    \begin{equation*}
        X_n = \masE [X_\infty | \famF_n], \quad n \in \nat.
    \end{equation*}
\end{tm}

\begin{proof}
    Ako je familija $\indFamilija{X_n}{n \in \nat}$ uniformno integrabilna, tada je $L^1 (\masP)$-ome\dj ena, toga po teoremu \ref{tm:24.4} postoji slu\v cajna varijabla $X_\infty \in L^1 (\masP)$ takva da vrijedi
    \begin{equation*}
        X_n \xrightarrow[n \to \infty]{g.s.} X_\infty.
    \end{equation*}
    Sada po teoremu \ref{tm:24.14} uz $p = 1$ imamo
    \begin{equation*}
        X_n \xrightarrow[n \to \infty]{L^1} X_\infty.
    \end{equation*}
    Obrat slijedi iz zadatka \ref{zad:24.15}.

    Budu\' ci da je $X_\infty \in L^1 (\masP)$, ima smisla $\masE [X_\infty | \famF_n]$.
    Za $A \in \famF_n$ i $m > n$ vrijedi
    \begin{equation*}
        \int\limits_A X_m \: d \masP = \int\limits_A X_n \: d \masP,
    \end{equation*}
    zato \v sto je $\niz{X_n}{n \in \nat}$ martingal.
    No vrijedi
    \begin{equation*}
        \karaktFja_A \cdot X_m \xrightarrow[n \to \infty]{g.s.} \karaktFja_A \cdot X_\infty,
    \end{equation*}
    a prema teoremu \ref{tm:24.14} dobijemo
    \begin{equation*}
        \int\limits_A X_m \: d \masP \xrightarrow[n \to \infty]{} \int\limits_A X_\infty \: d \masP,
    \end{equation*}
    to jest
    \begin{equation*}
        \int\limits_A X_n \: d \masP = \int\limits_A X_\infty \: d \masP.
    \end{equation*}
\end{proof}

\begin{nap} \label{nap:24.18}
    Ako je $\famF_\infty = \bigSigAlg{\unija{n \in \nat}{} \famF_n}$, uo\v cimo da je $X_\infty$ $\famF_\infty$-izmjeriva, to jest
    $\niz{X_n}{n \in \nat \cup \{+ \infty\}}$ je martingal s obzirom na filtraciju $\indFamilija{\famF_n}{n \in \nat \cup \{+ \infty\}}$.
\end{nap}