% poglavlje 4.3 -> predavanje 18 - Helly-Brayev teorem, teorem neprekidnosti i njihove posljedice

\chapter{Helly-Brayev teorem, teorem neprekidnosti i njihove posljedice}

\begin{tm}[Helly-Bray]  \label{tm:18.1}
    Neka je $\niz{F_n}{n \in \nat} \subseteq \inc$, te neka je $g: \real \to \real$.
    \begin{enumerate}[label=(\roman*)]
        \item   \label{tm:18.1.1}
        Ako $F_n \xrightarrow{c} F$ te $g$ ome\dj ena i neprekidna, tada
        \begin{equation*}
            \lim\limits_{n \to \real} \int_{\real} g \: d F_n = \int_{\real} g \: d F \in \real.
        \end{equation*}
        \item   \label{tm:18.1.2}
        Ako $F_n \xrightarrow{w} F$ te $g$ neprekidna i vrijedi
        \begin{equation*}
            \lim\limits_{x \to \pm \infty} g (x) = 0,    
        \end{equation*}
        tada vrijedi
        \begin{equation*}
            \lim\limits_{n \to \infty} \int_{\real} g \: d F_n = \int_{\real} g \: d F \in \real.
        \end{equation*}
    \end{enumerate}
\end{tm}

\begin{proof}
    \begin{enumerate}[label=(\roman*)]
        \item
        Uo\v cimo da postoji $0<M<+\infty$ takav da je $|g(x)| \leq M$, $\forall x \in \real$.
        Po korolaru \ref{kor:16.12} $(F_n)$ je napet, stoga po zadatku \ref{zad:16.11} \ref{zad:16.11.3}, $g$ je $(F_n)-\ui$.
        Rezultat slijedi po korolaru \ref{kor:16.14}.
        \item
        Po pretpostavkama na $g$ slijedi
        \begin{equation*}
            \big(\forall \varepsilon > 0\big) \big(\exists a = a (\varepsilon) \in \obInt{0}{+\infty}\big) \; \td \; \big(|x| \geq a \implies |g(x)| < \varepsilon \big).
        \end{equation*}
        Po definiciji \ref{defn:16.10} slijedi da je $g$ $(F_n)-\ui$.
        Rezultat slijedi po korolaru \ref{kor:16.14}.
    \end{enumerate}
\end{proof}

\begin{kor} \label{kor:18.2}
    Ako $F_n \xrightarrow{w} F$ i postoji $s>0$ takav da je
    \begin{equation*}
        \bigNiz{\int\limits_{\real} |x|^s \: d F_n (s)}{n \in \nat}
    \end{equation*}
    ome\dj en niz, tada vrijedi:
    \begin{enumerate}[label=(\roman*)]
        \item   \label{kor:18.2.1}
        \begin{equation*}
            F_n \xrightarrow[n \to \infty]{c} F,  
        \end{equation*}
        \item   \label{kor:18.2.2}
        za svaki $0 \leq r < s$,
        \begin{equation*}
            \int\limits_{\real} |x|^r \: d F_n (x) \to \int\limits_{\real} |x|^r \: d F (x),
        \end{equation*}
        \item   \label{kor:18.2.3}
        za svaki $k \in \{ 1, 2, \ldots, \lfloor s \rfloor \}$, $k \neq s$,
        \begin{equation*}
            \int\limits_{\real} x^k \: d F_n (x) \to \int\limits_{\real} x^k \: d F (x).
        \end{equation*}
    \end{enumerate}
\end{kor}

\begin{proof}
    Koristimo korolar \ref{kor:16.14} kao i u prethodnom teoremu, jer su $|x|^r$ $(F_n)-\ui$, zbog
    \begin{equation*}
        \int\limits_{\{ |x| \geq a \}} |x|^r \: d F_n (x) \leq \frac{1}{a^{s-r}} \int\limits_{\real} |x|^s \: d F_n (x) \leq \frac{\const}{a^{s-r}}, \quad \forall n \in \nat.
    \end{equation*}
\end{proof}

\begin{tm}[Fr\' echet-Shohat]  \label{tm:18.3}
    Neka je $(F_n)$ niz p.d.F. takav da za svaki $k \in \nat$ postoje
    \begin{equation*}
        \begin{aligned}
            \alpha_{n, k} &:= \int\limits_\real x^k \: d F_n (x)\\
            \alpha_k &= \lim\limits_{n \to \infty} \alpha_{n, k}  
        \end{aligned}
    \end{equation*}
    i neka su $\alpha_k$ momenti jedisntveno odre\dj eni p.d.F. $F$.
    Tada
    \begin{equation*}
        F_n \xrightarrow{c} F.
    \end{equation*}
\end{tm}

\begin{proof}
    Po teoremu \ref{tm:16.16} za svaki podniz niza $(F_n)$ postoji pod-podniz, recimo $(F_{n_j})$ i $G \in \inc$, takvi da
    \begin{equation*}
        F_{n_j} \xrightarrow{w} G.
    \end{equation*}
    Budu\' ci da je $\niz{\alpha_{n, k}}{n \in \nat}$ konvergentan, mora biti i ome\dj en, pa po korolaru \ref{kor:18.2} slijedi
    \begin{equation*}
        F_{n_j} \xrightarrow[j \to \infty]{c} G \quad \textnormal{i} \quad \alpha_{n_j, k} \xrightarrow[j \to \infty]{} \int\limits_{\real} x^k \: d G (x), \quad \forall k \in \nat.
    \end{equation*}
    Stoga je $G$ vjerojatnosna funkcija distribucije koja ima sve momente jednake $\alpha_k$.
    Po pretpostavci postoji samo jedna takva p.d.F, pa je $G = F$.
    Dakle,
    \begin{equation*}
        F_{n_j} \xrightarrow[j \to \infty]{c} F,
    \end{equation*}
    pa tvrdnja slijedi iz zadatka \ref{zad:16.18}.
\end{proof}

\begin{zad}  \label{zad:18.4}
    Neka $F_n \xrightarrow{c} F$.
    \begin{enumerate}[label=(\roman*)]
        \item   \label{zad:18.4.1}
        Ako za svaki $x \notin C(F)$ vrijedi $F_n (x \pm) \to F (x \pm)$, tada $(F_n)$ konvergira uniformano na $\real$ prema $F$.
        \item   \label{zad:18.4.2}
        Ako je $C(F) = \real$, tada
        \begin{equation*}
            \sup\limits_{x \in \real} \big| F_n (x) - F(x) \big| \xrightarrow[n \to \infty]{} 0.
        \end{equation*}
    \end{enumerate}
\end{zad}

\begin{tm}[Glivenko-Cantelli]   \label{tm:18.5}
    Neka je $(X_n)$ $\iid$ niz s distribucijom $F$.
    Neka je $F^\omega_n$ \emph{empirijska funkcija distribucije}, to jest
    \begin{equation*}
        F^\omega_n (x) := \frac{1}{n} \suma{j = 1}{n} \karaktFja_{\{ X_j \leq x \}}(\omega), \quad n \in \nat, \omega \in \Omega, x \in \real.
    \end{equation*}
    Tada, za gotovo sve $\omega \in \Omega$ vrijedi,
    \begin{equation*}
        \lim\limits_{n \to \infty} \Big( \sup\limits_{x \in \real} | F^\omega_n(x) - F(x) | \Big) = 0.
    \end{equation*}
\end{tm}

\begin{nap} \label{nap:18.5-1}
    Empirijsku funkciju distribucije tako\dj er mo\v zemo zapisati na sljede\' ce na\v cine:
    \begin{equation*}
        \begin{gathered}
            F_n (x) =    \frac{1}{n} \suma{k = 1}{n} \karaktFja_{\lijInt{-\infty}{x}} \circ X_k\\
            F^\omega_n (x) = \frac{1}{n} \suma{k = 1}{n} \karaktFja_{\desInt{X_k(\omega)}{+\infty}}(x)
        \end{gathered}
    \end{equation*}
\end{nap}

\begin{proof}
    Uo\v cimo da je
    \begin{equation*}
        \karaktFja_{\{X_j \leq x\}} \sim
        \begin{pmatrix}
            0 & 1\\
            1 - F (x) & F (x)
        \end{pmatrix},
    \end{equation*}
    pa je i $\omega \mapsto F^\omega_n (x)$ slu\v cajna varijabla i vrijedi
    \begin{equation*}
        n \cdot F^\omega_n (x) \sim B(n, F(x)).
    \end{equation*}
    Zbog neprekidnosti zdesna slijedi
    \begin{equation*}
        \sup\limits_{x \in \real} |F^\omega_n (x) - F (x)| = \sup\limits_{q \in \masQ} |F^\omega_n (q) - F (q)|,
    \end{equation*}
    pa je i
    \begin{equation*}
        \omega \mapsto \sup\limits_{x \in \real} |F^\omega_n (x) - F (x)|, \quad n \in \nat,
    \end{equation*}
    niz slu\v cajnih varijabli.
    Nadalje, za svaki $n \in \nat$ i za gotovo sve $\omega \in \Omega$ je $x \mapsto F^\omega_n (x)$ p.d.F..
    Dakle, za gotovo svaki $\omega \in \Omega$ je $\niz{F^\omega_n}{n \in \nat}$, niz p.d.F..
    Po jakom zakonu velikih brojeva
    \begin{equation*}
        F^\omega_n (x) \xrightarrow{g.s.} F(x).
    \end{equation*}
    Sli\v cno dobijemo i da su $\omega \mapsto F^\omega_n (x-)$ slu\v cajne varijable, te da je
    \begin{equation*}
        n \cdot F^\omega_n (x-) \sim B(n, F(x-))
    \end{equation*}
    i kona\v cno
    \begin{equation*}
        F^\omega_n (x-) \xrightarrow{g.s.} F(x-).
    \end{equation*}
    
    Promatramo
    \begin{equation*}
        A := \bigIndFamilija{\omega \in \Omega}{(F^\omega_n) \textnormal{ niz p.d.F.}, \; \big(F^\omega_n (q) - F(q) \big) \to 0, \; \big( F^\omega_n (q-) - F(q-) \big) \to 0, \; q \in \masQ}.
    \end{equation*}
    Slijedi da je $\masP (A) = 1$.
    Budu\' ci da je $F$ p.d.F., iz napomene \ref{nap:16.6} slijdi da za svaki $\omega \in A$ vrijedi
    \begin{equation*}
        F^\omega_n \xrightarrow{c} F,
    \end{equation*}
    te da su ispunjene pretpostavke zadatka \ref{zad:18.4} \ref{zad:18.4.1}, slijedi da za $\omega \in A$ vrijedi
    \begin{equation*}
        \sup\limits_{x \in \real} |F^\omega_n (x) - F(x)| \to 0.
    \end{equation*}
\end{proof}

\begin{tm}[teorem neprekidnosti, Paul L\' evy]  \label{tm:18.6}
    \quad \newline
    Neka je $\niz{F_n}{n \in \nat}$ niz p.d.F. i $\varphi_n := \varphi_{F_n}$, $n \in \nat$.
    \begin{enumerate}[label=(\roman*)]
        \item   \label{tm:18.6.1}
        Ako $F_n \xrightarrow{c} F$, tada za svaki $T > 0$,
        \begin{equation*}
            \varphi_n (t) \to \varphi_F (t)
        \end{equation*}
        uniformno za $t \in \segment{-T}{T}$.
        \item   \label{tm:18.6.2}
        Ako $\varphi_n (t) \to \varphi (t)$, za svaki $t \in \real$ i $\varphi$ je funkcija neprekidna u $0$, tada postoji $F$ koja je p.d.F. i vrijedi
        \begin{equation*}
            \varphi = \varphi_F \quad \textnormal{ i } \quad F_n \xrightarrow{c} F.
        \end{equation*}
    \end{enumerate}
\end{tm}