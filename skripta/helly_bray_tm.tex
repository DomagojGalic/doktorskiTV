% poglavlje 4.3 -> predavanje 18 - Helly-Brayev teorem, teorem neprekidnosti i njihove posljedice

\chapter{Helly-Brayev teorem, teorem neprekidnosti i njihove posljedice}

\begin{tm}[Helly-Bray]  \label{tm:18.1}
    Neka je $\niz{F_n}{n \in \nat} \subseteq \inc$, te neka je $g: \real \to \real$.
    \begin{enumerate}[label=(\roman*)]
        \item   \label{tm:18.1.1}
        Ako $F_n \xrightarrow{c} F$ te $g$ ome\dj ena i neprekidna, tada
        \begin{equation*}
            \lim\limits_{n \to \real} \int_{\real} g \: d F_n = \int_{\real} g \: d F \in \real.
        \end{equation*}
        \item   \label{tm:18.1.2}
        Ako $F_n \xrightarrow{w} F$ te $g$ neprekidna i vrijedi
        \begin{equation*}
            \lim\limits_{x \to \pm \infty} g (x) = 0,    
        \end{equation*}
        tada vrijedi
        \begin{equation*}
            \lim\limits_{n \to \infty} \int_{\real} g \: d F_n = \int_{\real} g \: d F \in \real.
        \end{equation*}
    \end{enumerate}
\end{tm}

\begin{proof}
    \begin{enumerate}[label=(\roman*)]
        \item
        Uo\v cimo da postoji $0<M<+\infty$ takav da je $|g(x)| \leq M$, $\forall x \in \real$.
        Po korolaru \ref{kor:16.12} $(F_n)$ je napet, stoga po zadatku \ref{zad:16.11} \ref{zad:16.11.3}, $g$ je $(F_n)-\ui$.
        Rezultat slijedi po korolaru \ref{kor:16.14}.
        \item
        Po pretpostavkama na $g$ slijedi
        \begin{equation*}
            \big(\forall \varepsilon > 0\big) \big(\exists a = a (\varepsilon) \in \obInt{0}{+\infty}\big) \; \td \; \big(|x| \geq a \implies |g(x)| < \varepsilon \big).
        \end{equation*}
        Po definiciji \ref{defn:16.10} slijedi da je $g$ $(F_n)-\ui$.
        Rezultat slijedi po korolaru \ref{kor:16.14}.
    \end{enumerate}
\end{proof}

\begin{kor} \label{kor:18.2}
    Ako $F_n \xrightarrow{w} F$ i postoji $s>0$ takav da je
    \begin{equation*}
        \bigNiz{\int\limits_{\real} |x|^s \: d F_n (s)}{n \in \nat}
    \end{equation*}
    ome\dj en niz, tada vrijedi:
    \begin{enumerate}[label=(\roman*)]
        \item   \label{kor:18.2.1}
        \begin{equation*}
            F_n \xrightarrow[n \to \infty]{c} F,  
        \end{equation*}
        \item   \label{kor:18.2.2}
        za svaki $0 \leq r < s$,
        \begin{equation*}
            \int\limits_{\real} |x|^r \: d F_n (x) \to \int\limits_{\real} |x|^r \: d F (x),
        \end{equation*}
        \item   \label{kor:18.2.3}
        za svaki $k \in \{ 1, 2, \ldots, \lfloor s \rfloor \}$, $k \neq s$,
        \begin{equation*}
            \int\limits_{\real} x^k \: d F_n (x) \to \int\limits_{\real} x^k \: d F (x).
        \end{equation*}
    \end{enumerate}
\end{kor}

\begin{proof}
    Koristimo korolar \ref{kor:16.14} kao i u prethodnom teoremu, jer su $|x|^r$ $(F_n)-\ui$, zbog
    \begin{equation*}
        \int\limits_{\{ |x| \geq a \}} |x|^r \: d F_n (x) \leq \frac{1}{a^{s-r}} \int\limits_{\real} |x|^s \: d F_n (x) \leq \frac{\const}{a^{s-r}}, \quad \forall n \in \nat.
    \end{equation*}
\end{proof}

\begin{tm}[Fr\' echet-Shohat]  \label{tm:18.3}
    Neka je $(F_n)$ niz p.d.F. takav da za svaki $k \in \nat$ postoje
    \begin{equation*}
        \begin{aligned}
            \alpha_{n, k} &:= \int\limits_\real x^k \: d F_n (x)\\
            \alpha_k &= \lim\limits_{n \to \infty} \alpha_{n, k}  
        \end{aligned}
    \end{equation*}
    i neka su $\alpha_k$ momenti jedisntveno odre\dj eni p.d.F. $F$.
    Tada
    \begin{equation*}
        F_n \xrightarrow{c} F.
    \end{equation*}
\end{tm}

\begin{proof}
    Po teoremu \ref{tm:16.16} za svaki podniz niza $(F_n)$ postoji pod-podniz, recimo $(F_{n_j})$ i $G \in \inc$, takvi da
    \begin{equation*}
        F_{n_j} \xrightarrow{w} G.
    \end{equation*}
    Budu\' ci da je $\niz{\alpha_{n, k}}{n \in \nat}$ konvergentan, mora biti i ome\dj en, pa po korolaru \ref{kor:18.2} slijedi
    \begin{equation*}
        F_{n_j} \xrightarrow[j \to \infty]{c} G \quad \textnormal{i} \quad \alpha_{n_j, k} \xrightarrow[j \to \infty]{} \int\limits_{\real} x^k \: d G (x), \quad \forall k \in \nat.
    \end{equation*}
    Stoga je $G$ vjerojatnosna funkcija distribucije koja ima sve momente jednake $\alpha_k$.
    Po pretpostavci postoji samo jedna takva p.d.F, pa je $G = F$.
    Dakle,
    \begin{equation*}
        F_{n_j} \xrightarrow[j \to \infty]{c} F,
    \end{equation*}
    pa tvrdnja slijedi iz zadatka \ref{zad:16.18}.
\end{proof}

\begin{zad}  \label{zad:18.4}
    Neka $F_n \xrightarrow{c} F$.
    \begin{enumerate}[label=(\roman*)]
        \item   \label{zad:18.4.1}
        Ako za svaki $x \notin C(F)$ vrijedi $F_n (x \pm) \to F (x \pm)$, tada $(F_n)$ konvergira uniformano na $\real$ prema $F$.
        \item   \label{zad:18.4.2}
        Ako je $C(F) = \real$, tada
        \begin{equation*}
            \sup\limits_{x \in \real} \big| F_n (x) - F(x) \big| \xrightarrow[n \to \infty]{} 0.
        \end{equation*}
    \end{enumerate}
\end{zad}

\begin{tm}[Glivenko-Cantelli]   \label{tm:18.5}
    Neka je $(X_n)$ $\iid$ niz s distribucijom $F$.
    Neka je $F^\omega_n$ \emph{empirijska funkcija distribucije}, to jest
    \begin{equation*}
        F^\omega_n (x) := \frac{1}{n} \suma{j = 1}{n} \karaktFja_{\{ X_j \leq x \}}(\omega), \quad n \in \nat, \omega \in \Omega, x \in \real.
    \end{equation*}
    Tada, za gotovo sve $\omega \in \Omega$ vrijedi,
    \begin{equation*}
        \lim\limits_{n \to \infty} \Big( \sup\limits_{x \in \real} | F^\omega_n(x) - F(x) | \Big) = 0.
    \end{equation*}
\end{tm}

\begin{nap} \label{nap:18.5-1}
    Empirijsku funkciju distribucije tako\dj er mo\v zemo zapisati na sljede\' ce na\v cine:
    \begin{equation*}
        \begin{gathered}
            F_n (x) =    \frac{1}{n} \suma{k = 1}{n} \karaktFja_{\lijInt{-\infty}{x}} \circ X_k\\
            F^\omega_n (x) = \frac{1}{n} \suma{k = 1}{n} \karaktFja_{\desInt{X_k(\omega)}{+\infty}}(x)
        \end{gathered}
    \end{equation*}
\end{nap}

\begin{proof}
    Uo\v cimo da je
    \begin{equation*}
        \karaktFja_{\{X_j \leq x\}} \sim
        \begin{pmatrix}
            0 & 1\\
            1 - F (x) & F (x)
        \end{pmatrix},
    \end{equation*}
    pa je i $\omega \mapsto F^\omega_n (x)$ slu\v cajna varijabla i vrijedi
    \begin{equation*}
        n \cdot F^\omega_n (x) \sim B(n, F(x)).
    \end{equation*}
    Zbog neprekidnosti zdesna slijedi
    \begin{equation*}
        \sup\limits_{x \in \real} |F^\omega_n (x) - F (x)| = \sup\limits_{q \in \masQ} |F^\omega_n (q) - F (q)|,
    \end{equation*}
    pa je i
    \begin{equation*}
        \omega \mapsto \sup\limits_{x \in \real} |F^\omega_n (x) - F (x)|, \quad n \in \nat,
    \end{equation*}
    niz slu\v cajnih varijabli.
    Nadalje, za svaki $n \in \nat$ i za gotovo sve $\omega \in \Omega$ je $x \mapsto F^\omega_n (x)$ p.d.F..
    Dakle, za gotovo svaki $\omega \in \Omega$ je $\niz{F^\omega_n}{n \in \nat}$, niz p.d.F..
    Po jakom zakonu velikih brojeva
    \begin{equation*}
        F^\omega_n (x) \xrightarrow{g.s.} F(x).
    \end{equation*}
    Sli\v cno dobijemo i da su $\omega \mapsto F^\omega_n (x-)$ slu\v cajne varijable, te da je
    \begin{equation*}
        n \cdot F^\omega_n (x-) \sim B(n, F(x-))
    \end{equation*}
    i kona\v cno
    \begin{equation*}
        F^\omega_n (x-) \xrightarrow{g.s.} F(x-).
    \end{equation*}
    
    Promatramo
    \begin{equation*}
        A := \bigIndFamilija{\omega \in \Omega}{(F^\omega_n) \textnormal{ niz p.d.F.}, \; \big(F^\omega_n (q) - F(q) \big) \to 0, \; \big( F^\omega_n (q-) - F(q-) \big) \to 0, \; q \in \masQ}.
    \end{equation*}
    Slijedi da je $\masP (A) = 1$.
    Budu\' ci da je $F$ p.d.F., iz napomene \ref{nap:16.6} slijdi da za svaki $\omega \in A$ vrijedi
    \begin{equation*}
        F^\omega_n \xrightarrow{c} F,
    \end{equation*}
    te da su ispunjene pretpostavke zadatka \ref{zad:18.4} \ref{zad:18.4.1}, slijedi da za $\omega \in A$ vrijedi
    \begin{equation*}
        \sup\limits_{x \in \real} |F^\omega_n (x) - F(x)| \to 0.
    \end{equation*}
\end{proof}

\begin{tm}[teorem neprekidnosti, Paul L\' evy]  \label{tm:18.6}
    \quad \newline
    Neka je $\niz{F_n}{n \in \nat}$ niz p.d.F. i $\varphi_n := \varphi_{F_n}$, $n \in \nat$.
    \begin{enumerate}[label=(\roman*)]
        \item   \label{tm:18.6.1}
        Ako $F_n \xrightarrow{c} F$, tada za svaki $T > 0$,
        \begin{equation*}
            \varphi_n (t) \to \varphi_F (t)
        \end{equation*}
        uniformno za $t \in \segment{-T}{T}$.
        \item   \label{tm:18.6.2}
        Ako $\varphi_n (t) \to \varphi (t)$, za svaki $t \in \real$ i $\varphi$ je funkcija neprekidna u $0$, tada postoji $F$ koja je p.d.F. i vrijedi
        \begin{equation*}
            \varphi = \varphi_F \quad \textnormal{ i } \quad F_n \xrightarrow{c} F.
        \end{equation*}
    \end{enumerate}
\end{tm}

\begin{proof}
    Neka je $\varepsilon > 0$.
    Po korolaru \ref{kor:16.12} mo\v zemo izabrati $M > 0$ dovoljno velik, tako da vrijedi
    \begin{equation*}
        \begin{aligned}
            \pm M \in C(F) \cap \presjek{n \in \nat}{} C(F_n),\\
            F(-M) + (1 - F(M)) < \varepsilon,\\
            F_n (-M) + (1 - F_n (M)) < \varepsilon, \quad n \in \nat.
        \end{aligned}
    \end{equation*}
    Tada vrijedi
    \begin{equation*}
        \begin{aligned}
            | \varphi_n (t) - \varphi_F (t) | \leq& \Big| \underbrace{\int^{-M}_{-\infty} e^{itx} \: d F_n (x) - \int_{-\infty}^{-M} e^{itx} \: d F (x) }_{I_1} \Big|\\
            &+ \Big| \underbrace{\int_{-M}^{M} e^{itx} \: d F_n (x) - \int_{-M}^{M} e^{itx} \: d F (x) }_{I_2} \Big|\\
            &+ \Big| \underbrace{\int_{M}^{+\infty} e^{itx} \: d F_n (x) - \int_{M}^{+\infty} e^{itx} \: d F (x) }_{I_3} \Big|.
        \end{aligned}
    \end{equation*}
    Zbog $|e^{itx}| = 1$ slijedi (uniformno za sve $t \in \real$)
    \begin{equation*}
        \begin{aligned}
            |I_1| &\leq F_n (-M) + F(M) < 2 \varepsilon\\
            |I_3| &\leq (1 - F_n(M)) + (1 - F (M)) < 2 \varepsilon.
        \end{aligned}
    \end{equation*}
    Neka je $T > 0$ i uzimamo $t_0 \in \segment{-T}{T}$.
    Koriste\' ci parcijalnu integraciju
    \begin{equation*}
        I_2 = \big( F_n(x) - F(x) \big) e^{itx} \Big|_{-M}^{M} - it \int_{-M}^M e^{itx} \big( F_n (x) - F(x) \big) \: dx,
    \end{equation*}
    \v sto daje
    \begin{equation*}
        \begin{aligned}
            |I_2| &\leq |F_n(M) - F(M)| + |F_n(-M) + F(-M)| + T \int_{-M}^{M} |F_n (x) - F (x)| \: dx\\
            &\leq 4 \varepsilon + T \int_{-M}^{M} |F_n (x) - F (x)| \: dx.
        \end{aligned}
    \end{equation*}
    Uo\v cimo, $|F_n (x) - F (x)| \leq 2$, funkcija $2$ je integrabilna na $\segment{-M}{M}$, $\real \setminus C(F)$ je najvi\v se prebrojiv, dakle
    \begin{equation*}
        F_n (x) \xrightarrow[\lambda-g.s.]{n \to \infty} F(x).
    \end{equation*}
    Po LTDK sada vrijedi
    \begin{equation*}
        \int_{-M}^M |F_n (x) - F(x)| \: d x \to 0.
    \end{equation*}
    Dakle, za $n$ dovoljno velik ovaj integral je manji od $\frac{\varepsilon}{T}$.
    Zna\v ci, za $n$ dovljno velik (i ovisan samo o $\varepsilon$ i $T$) i za svaki $t \in \segment{-T}{T}$,
    \begin{equation*}
        |\varphi_n (t) - \varphi_F (t)| < 9 \varepsilon.
    \end{equation*}
    Time je dokazana tvrdnja \ref{tm:18.6.1}.
    Doka\v zimo \ref{tm:18.6.2}.

    Po teoremu \ref{tm:16.16}
    \begin{equation*}
        (\exists F \in \inc) \; \textnormal{ i podniz } \; (F_{n_k}) \; \td \; F_{n_k} \xrightarrow{w} F.
    \end{equation*}
    Za svaki $\delta > 0$ vrijedi
    \begin{equation*}
        \begin{aligned}
            \frac{1}{2 \delta} \int_{-\delta}^\delta  \varphi_n (t) \: d t &= \frac{1}{2 \delta} \int_0^\delta \int_{-\infty}^{+\infty} \big( e^{itx} + e^{-itx} \big) \: d F_n (x) \: d t\\
            &= \frac{1}{\delta} \int_0^\delta \int_{-\infty}^{+\infty} \cos tx \: d F_n (x) \: dt =
            \begin{psmallmatrix}
                \textnormal{Fubini}\\
                |\cos tx| \leq 1
            \end{psmallmatrix}\\
            &= \int_{-\infty}^{+\infty} \frac{\sin \delta x}{\delta x} \: d F_n (x).
        \end{aligned}
    \end{equation*}
    Uo\v cimo da je $x \mapsto \frac{\sin \delta x}{\delta x}$ neprekidno na $\real$ i i\v s\v cezava u $\pm \infty$.
    Po Helly-Bray teoremu (teoremu \ref{tm:18.1}) primjenjenom na $(F_{n_k})$ dobivamo
    \begin{equation*}
        \int_{-\infty}^{\infty} \frac{\sin \delta x}{\delta x} \: d F_{n_k} (x) \xrightarrow[k \to \infty]{} \int_{-\infty}^{+\infty} \frac{\sin \delta x}{\delta x} \: d F (x).
    \end{equation*}
    Po LTDK (koriste\' ci ocjenu \eqref{jed:17.3}) dobijemo
    \begin{equation*}
        \begin{gathered}
            \frac{1}{2 \delta} \int_{-\delta}^\delta \varphi_n (t) \: d t \to \frac{1}{2 \delta} \int_{-\delta}^{\delta} \varphi (t) \: d t\\
            \implies \frac{1}{2 \delta} \int_{-\delta}^{\delta} \varphi (t) \: dt = \int_{-\infty}^{+\infty} \frac{\sin \delta x}{\delta x} \: d F (x).
        \end{gathered}
    \end{equation*}
    Neka $\delta \to 0$.
    Budu\' ci da je $\varphi$ neprekidna u $0$ lijeva strana te\v zi u $\varphi (0)$.
    Budu\' ci da je $\Big| \frac{\sin n}{n} \Big| \leq 1$ i $\Delta F \leq 1$, po LTDK, desna strana te\v zi u $\Delta F$.
    Dakle imamo
    \begin{equation*}
        \Delta F = \varphi (0) = 1.
    \end{equation*}
    Odakle slijedi da je $F$ p.d.F. i
    \begin{equation*}
        F_{n_k} \xrightarrow{w} F.
    \end{equation*}
    
    Po \ref{tm:18.6.1} slijedi
    \begin{equation*}
        \varphi_{n_k} \to \varphi_F \quad \implies \quad \varphi = \varphi_F.
    \end{equation*}
    Neka je $(F_{n_p})$ bilo koji podniz od $(F_n)$.
    Po teoremu \ref{tm:16.16} postoji podniz $(F_{n_{p_l}})$ i $\widetilde{F} \in \inc$ takav da $F_{n_{p_l}} \xrightarrow{w} \widetilde{F}$.
    Isti dokaz koji smo upravo napravili pokazuje $\widetilde{F} = F$ i $F_{n_{p_l}} \xrightarrow{c} F$.
    Po zadatku \ref{zad:16.18} tvrdnja slijedi.
\end{proof}

O\v cito vrijedi:

\begin{kor} \label{kor:18.7}
    Neka je $F$ p.d.F. i $(F_n)$ niz p.d.F..
    Neka je $\varphi := \varphi_F$ i $\varphi_n := \varphi_{F_n}$, za svaki $n \in \nat$.
    Tada vrijedi
    \begin{equation*}
        F_n \xrightarrow[n \to \infty]{c} F \quad \iff \quad \varphi_n (t) \xrightarrow[n \to \infty]{} \varphi (t), \; \forall t \in \real.
    \end{equation*}
\end{kor}

Koristimo teorem neprekidnosti i u sljede\' cim tvrdnjama o beskona\v cno dijeljivim karakteristi\v cnim funkcijama.
Prvo pokazujemo da za beskona\v cno djeljivu $\varphi$ ima smisla govoriti o $\ln \varphi$ (i neprekidna je funkcija).

\begin{prop}    \label{prop:18.8}
    Ako je $\varphi$ beskona\v cno djeljiva karakteristi\v cna funkcija, ta da je $\varphi (t) \neq 0$, $\forall t \in \real$.
\end{prop}

\begin{proof}
    Za $n \in \nat$ promatramo $\varphi_n$ "$n$-ti korjen od $\varphi$".
    Zbog $|\varphi_n (t)|^2 = |\varphi (t)|^\frac{2}{n}$, slijedi da postoji
    \begin{equation*}
        g (t) := \lim\limits_{n \to \infty} |\varphi_n (t)|^2
    \end{equation*}
    te vrijedi
    \begin{equation*}
        \begin{gathered}
            g (t) = 0 \quad \iff \quad \varphi (t) = 0\\
            g (t) = 1 \quad \iff \quad \varphi (t) \neq 0. 
        \end{gathered}
    \end{equation*}
    Budu\' ci je $\varphi (0) = 1$ i $\varphi$ neprekidna, znamo da postoji $\varepsilon > 0$, takav da vrijedi
    \begin{equation*}
        |t| < \varepsilon \implies \varphi (t) \neq 0 \implies g (t) = 1,
    \end{equation*}
    odakle slijedi neprekidnost od $g$ u $0$.
    Budu\' ci da je $\varphi_n$ karakteristi\v cna funkcija, slijedi da je i $|\varphi_n|^2$ karakteristi\v cna funkcija.
    Stoga po teoremu neprekidnosti slijedi da je $g$ karakteristi\v cna funkcija.
    Posebno, $g$ je neprekidna na $\real$, stoga slijedi $g \equiv 1$.
\end{proof}

\begin{zad} \label{zad:18.9}
    Doka\v zite da uniformna razdioba na segmentu $\segment{-a}{a}$, $a > 0$, nema beskona\v cno djeljivu karakteristi\v cnu funkciju.
\end{zad}

Uo\v cite da iz propozicije \ref{prop:18.8} slijedi da je "$n$-ti korjen" od beskona\v cno djeljive $\varphi$ zadan sa
\begin{equation}    \label{jed:18.10}
    \varphi_n = e^{\frac{1}{n} \ln \varphi}.
\end{equation}

\begin{prop}    \label{prop: 18.11}
    Ako su $\niz{\psi_m}{m \in \nat}$ beskona\v cno djeljive karakteristi\v cne funkcije i
    \begin{equation*}
        \varphi = \lim\limits_{m \to \infty} \psi_m
    \end{equation*}
    je karakteristi\v cna funkcija, tada je $\varphi$ tako\dj er beskona\v cno djeljiva.
\end{prop}

\begin{proof}
    Prema zadatku \ref{zad:17.26} $|\psi_m|^2$ je beskona\v cno djeljiva karakteristi\v cna funkcija, pa je i $|\psi_m|^\frac{1}{n}$ karakteristi\v cna funkcija za svaki $n \in \nat$.
    Zbog
    \begin{equation*}
        |\varphi|^\frac{2}{n} = \lim\limits_{m \to \infty} |\psi_m|^\frac{2}{n}
    \end{equation*}
    karakteristi\v cna funkcija, za svaki $n \in \nat$.
    Posebno, $|\varphi|^2$ je beskona\v cno djeljiva, pa je po propoziciji \ref{prop:18.8} $\varphi (t) \neq 0$, $\forall t \in \real$.
    Slijedi da je
    \begin{equation*}
        \ln \varphi (t) = \lim\limits_{m \to \infty} \ln \psi_m (t).
    \end{equation*}
    Iz relacije \eqref{jed:18.10} dobijemo
    \begin{equation*}
        \begin{aligned}
            \varphi &:= e^{\frac{1}{n} \ln \varphi} = \lim\limits_{m \to \infty} e^{\frac{1}{n} \ln \psi_m}\\
            &= \lim_{m \to \infty} (\psi_m)_n,    %n-ti korjen \psi_m ?
        \end{aligned}
    \end{equation*}
    pa je po teoremu neprekidnosti $\varphi_n$ karakteristi\v cna funkcija i o\v cito vrijedi
    \begin{equation*}
        (\varphi_n)^n = \varphi.
    \end{equation*}
\end{proof}

\begin{zad} \label{zad:18.12}
    Neka je $\varphi$ beskona\v cno djeljiva karakteristi\v cna funkcija i $\varphi_n = e^{\frac{1}{n} \ln \varphi}$.
    Doka\v zi da tada za svaki $t \in \real$, te uniformno po ograni\v cenim intervalima vrijedi:
    \begin{enumerate}[label=(\alph*)]
        \item    \label{zad:18.12.1}
        \begin{equation*}
            \lim\limits_{n \to \infty} n (\varphi_n (t) - 1) = \ln \varphi (t),
        \end{equation*}
        \item   \label{zad:18.12.2}
        \begin{equation*}
            \lim\limits_{n \to \infty} \varphi_n (t) = 1.
        \end{equation*}
    \end{enumerate}
\end{zad}