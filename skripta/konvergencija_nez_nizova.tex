% poglavlje 11 - konvergencija nezavisnih nizova

\chapter{Konvergencija nezavisnih nizova}

U ovom poglavlju koristit \' cemo stalne oznake: $\vjerojatnosniProstor$, \' ce ozna\v cavati vjerojatnosni prostor, a $\niz{X_n}{n \in \nat}$, niz slu\v cajnih varijabli na $\Omega$.

Uo\v cimo da je skup
\begin{equation*}
    \begin{aligned}
        \conv{(X_n)} &:= \Big\{ \lim\limits_{n} X_n \in \real \Big\} = \skup{\omega \in \Omega}{\lim\limits_{n \to \infty} X_n (\omega) \in \real}\\
        &= \presjek{k = 1}{\infty} \unija{n = 1}{\infty} \presjek{i = 1}{\infty} \bigSkup{\omega \in \Omega}{|X_{n + i} (\omega) - X_n (\omega)| < \frac{1}{k}}
    \end{aligned}
\end{equation*}
uvijek doga\dj aj, te da je
\begin{equation*}
    X:= \limsup\limits_{n \to \infty} X_n
\end{equation*}
pro\v sirena slu\v cajna varijabla.
Stoga
\begin{equation}    \label{jed:11.1}
    \vjeroj{\conv{(X_n)}} = 1 \implies \masP \Big( X = \lim\limits_n X_n \in \real \Big) = 1
\end{equation}
i u tom slu\v caju ka\v zemo da $(X_n)$ \emph{konvergira gotovo sigurno}.
Ako je $Y$ slu\v cajna varijabla i postoji $D \in \famF$ takav da je $\vjeroj{D} = 0$ i za svaki $\omega \in D^c$ je
\begin{equation*}
    Y(\omega) = \lim\limits_{n \to \infty} X_n (\omega)
\end{equation*}
tada ka\v zemo da $(X_n)$ \emph{konvergira gotovo sigurno prema} $Y$ i pi\v semo
\begin{equation*}
    X \xrightarrow[n \to \infty]{g.s.} Y
\end{equation*}
ili
\begin{equation*}
    Y = (g.s.) \; \lim\limits_{n \to \infty} X_n.
\end{equation*}
O\v cito vrijedi
\begin{equation}    \label{jed:11.2}
    (\exists Y) \; X_n \xrightarrow[n \to \infty]{g.s.} Y \iff \vjeroj{\conv{(X_n)} = 1}
\end{equation}
i u tom slu\v caju $X_n \xrightarrow[n \to \infty]{g.s.} X$ i $X = Y \; (g.s.)$.
Nadalje o\v cito vrijedi:
\begin{equation}    \label{jed:11.3}
    X_n \xrightarrow[n \to \infty]{g.s.} Y \iff X_n - Y \xrightarrow[n \to \infty]{g.s.} 0,
\end{equation}
te vrijedi i
\begin{equation}    \label{jed:11.4}
    (X_n) \textnormal{ konvergira } g.s.
    \iff
    \begin{matrix}
        \niz{X_n(\omega)}{n \in \nat} \textnormal{ Cauchyjev}\\
        \textnormal{za gotovo sve } \omega \in \Omega.
    \end{matrix}
\end{equation}

Prisjetimo se napomene \ref{nap:9.1-1}

Va\v znu ulogu u opisu ovog pojma imaju skupovi
\begin{equation*}
    \begin{aligned}
        A_k^n &:= \bigSkup{\omega \in \Omega}{|X_n(\omega) - X(\omega)| > \frac{1}{k}},\\
        A_k &:= \limsup\limits_{n \to \infty} A_k^n,\\
        D &:= \unija{k = 1}{\infty} A_k.
    \end{aligned}
\end{equation*}
Uo\v cimo $l \leq k \implies A_l \subseteq A_k$ i dobivamo
\begin{equation*}
    \begin{gathered}
        D = \bigSkup{\omega \in \Omega}{X_n(\omega) \cancel{\xrightarrow[n \to \infty]{}}X(\omega)}\\
        \implies  \vjeroj{D} = \lim\limits_{k \to \infty} \vjeroj{A_k}\\
        \vjeroj{D} = 0 \iff \vjeroj{A_k} = 0, \; \forall k \in \nat.
    \end{gathered}
\end{equation*}
Sada vrijedi
\begin{equation}    \label{jed:11.5}
    \begin{gathered}
        X_n \xrightarrow[n \to \infty]{g.s.} X\\
        \begin{aligned}
            \iff &\vjeroj{D} = 0\\
            \iff &\lim\limits_{m \to \infty} \masP \Bigg( \unija{l = m}{\infty} A_k^l \Bigg) = 0, \quad \forall k \in \nat\\
            \iff &\lim\limits_{m \to \infty} \masP \Bigg( \unija{l = m}{\infty} \{ |X_l - X| > \varepsilon \} \Bigg) = 0, \quad \forall \varepsilon > 0.
        \end{aligned}
    \end{gathered}
\end{equation}

Iz \eqref{jed:11.4} i \eqref{jed:11.5} direktno dobijemo:

\begin{lm}  \label{lm:11.6}
    Niz $\niz{X_n}{n \in \nat}$ je konvergentan $g.s.$ ako i samo ako vrijedi
    \begin{equation*}
        \lim\limits \masP \Bigg( \unija{l = m}{\infty} \{ |X_l - X| > \varepsilon \} \Bigg) = 0, \quad \forall \varepsilon > 0
    \end{equation*}
\end{lm}

Pretpostavimo sada da su $(X_n)$ nezavisne slu\v cajne varijable.
Uo\v cimo da je $\conv{(X_n)}$ repni doga\dj aj, pa prema Kolmogorovljevom zakonu 0-1 mora biti
\begin{equation}    \label{jed:11.7}
    \vjeroj{\conv{(X_n)}} \in \{ 0, \; 1 \}.
\end{equation}
Nadalje, $\liminf\limits_{n \to \infty} X_n$ i $\limsup\limits_{n \to \infty} X_n$ su repne funkcije, pa postoje $- \infty \leq a \leq A \leq +\infty$ takvi da vrijedi:
\begin{equation}    \label{jed:11.8}
    \begin{aligned}
        \liminf\limits_{n \to \infty} X_n &= a \quad (g.s.)\\
        \limsup\limits_{n \to \infty} X_n &= A \quad (g.s.)
    \end{aligned}
\end{equation}

\begin{tm}  \label{tm:11.9}
    Neka je $\niz{X_n}{n \in \nat}$ niz nezavisnih slu\v cajnih varijabli.
    Sljede\' ce tvrdnje su ekvivalentne:
    \begin{enumerate}[label=(\roman*)]
        \item   \label{tm:11.9.1}
        $(X_n)$ je konvergentan $g.s.$;
        \item   \label{tm:11.9.2}
        $-\infty < a = A < +\infty$;
        \item   \label{tm:11.9.3}
        $a \in \real$ i $X_n \xrightarrow[n \to \infty]{g.s.} a$;
        \item   \label{tm:11.9.4}
        $\suma{n = 1}{\infty} \vjeroj{|X_n - a| > \varepsilon} < +\infty, \quad \forall \varepsilon > 0$.
    \end{enumerate}
\end{tm}

\begin{proof}
    Iz \eqref{jed:11.7} i \eqref{jed:11.8} slijedi \ref{tm:11.9.1} $\iff$ \ref{tm:11.9.2}, dok je \ref{tm:11.9.2} $\iff$ \ref{tm:11.9.3} o\v cito.

    Doka\v zimo \ref{tm:11.9.3} $\implies$ \ref{tm:11.9.4}.
    Ozna\v cimo sa $C_n := \{ |X_n - a| > \varepsilon\}$, za zadani $\varepsilon > 0$.
    Po pretpostavci $\niz{C_n}{n \in \nat}$ su nezavisni doga\dj aji, a zbog $X_n \xrightarrow[n \to \infty]{g.s.} a$ dobivamo $\vjeroj{\limsup\limits_{n \to \infty} C_n} = 0$.
    Po Borelovom zakonu 0-1 $\suma{n = 1}{\infty} \vjeroj{C_n} < +\infty$.

    Doka\v zimo \ref{tm:11.9.4} $\implies$ \ref{tm:11.9.3}.
    Primjetimo
    \begin{equation*}
        \suma{n = 1}{\infty} \vjeroj{C_n} < +\infty \overset{\textnormal{Borel 0-1}}{\implies} \masP \Big( \limsup\limits_{n \to \infty} C_n \Big) = 0.
    \end{equation*}
    Uo\v cimo, ako $X_n \cancel{\to} a$, tada je $\omega \in \limsup\limits_{n \to \infty} C_n$, za neki $\varepsilon = \frac{1}{k}$.
    Dokaz slijedi direktno.
\end{proof}

Dakle, vrlo su rjetki slu\v cajevi, kad niz nezavisnih slu\v cajnih varijabli kongergira gotovo sigurno.
Slu\v caj $\iid$ je jo\v s posebniji i nema potrebe za daljnjim prou\v cavanjem.

\begin{kor} \label{kor:11.10}
    Neka je $\niz{X_n}{n \in \nat} \; \iid$ niz.
    Tada niz $(X_n)$ konverigra gotovo sigurno ako i samo ako je $X_n = a \; (g.s.)$, za svaki $n \in \nat$.
\end{kor}

\begin{proof}
    Dovoljnost je o\v cita, a nu\v znost slijedi iz \ref{tm:11.9} \ref{tm:11.9.4} zbog
    \begin{equation*}
        \begin{gathered}
            +\infty > \suma{n = 1}{\infty} \vjeroj{|X_n - a|>\varepsilon} = \iid = \suma{n = 1}{\infty} \vjeroj{|X_1 - a|> \varepsilon}\\
            \implies \vjeroj{|X_1 - a| > \varepsilon} = 0, \quad \varepsilon > 0.
        \end{gathered}
    \end{equation*}
\end{proof}

\begin{zad} \label{zad:11.11}
    Neka je $\niz{X_n}{n \in \nat}$ $\iid$ niz.
    Doka\v zite:
    \begin{equation*}
        \masP \Big( \limsup\limits_{n \to \infty} X_n  = +\infty \Big) = 1 \iff \vjeroj{X_1 < c} < 1,
    \end{equation*}
    za svaki $0 < c < +\infty$.
\end{zad}

\begin{nap} \label{nap:11.12}
    Time smo uglavnom opisali \v sto se mo\v ze desti s $n \mapsto X_n(\omega)$ za velike $n$.
    Intuitivno mo\v zemo re\' ci da \' cemo za nezavisne $(X_n)$ "rjetko" imati konvergenciju.
    U $\iid$ slu\v caju samo degenerirana razdioba daje konvergenciju.
    Na primjer ako je
    \begin{equation*}
        X_1 \sim
        \begin{pmatrix}
            0 & 1\\
            q & p
        \end{pmatrix}
    \end{equation*}
    i $0 < p < 1$, trajektorija
    \begin{equation*}
        n \mapsto X_n(\omega)
    \end{equation*}
    \' ce za gotovo sve $\omega$ oscilirati "bez pravila" od nule do jedinice $a = 0$, $A = 1$.

    Dakle o samom nizu $\niz{X_n (\omega)}{n \in \nat}$ u nezavisnom slu\v caju nemamo ni\v sta osobito za dodati.
    \v Sto se mo\v ze re\' ci o $\suma{n = 1}{\infty} X_n (\omega)$?
    Ili duga\v cije pitanje, postoji li nizovi konstanati $(a_n)$, $(b_n)$, takvi da
    \begin{equation*}
        \frac{S_n(\omega) - b_n}{a_n}
    \end{equation*}
    konvergira?
    Tipi\v cno $a_n \nearrow +\infty$, a $S_n := X_1 + \ldots + X_n$.
    U ovom slu\v caju teorija je bogatija i takve rezultate nazivamo jakim zakonima velikih brojeva.
\end{nap}

Osim "konvergencije po to\v ckama" $(g.s.)$ teorija vjerojatnosti bavi se i drugim tipovima konvergencije.

\begin{pr}  \label{pr:11.13}
    Neka je $\vjerojatnosniProstor = (\segment{0}{1}, \; \borel{\segment{0}{1}}, \; \lambda)$ i
    \begin{equation*}
        Z_{n, m} (\omega) :=
        \begin{cases}
            1, &\frac{m-1}{n} < \omega \leq \frac{m}{n}\\
            0, &\textnormal{ina\v ce}
        \end{cases}
        \quad
        \begin{matrix}
            n \in \nat,\\
            m \in \{ 1, \ldots, n \}.
        \end{matrix}
    \end{equation*}
    Poredamo $(Z_{n, m})$ u niz $\niz{X_k}{k \in \nat}$ kao $(Z_{1,1}, \; Z_{2,1}, \; Z_{2, 2}, \; Z_{3,1}, \; Z_{3,2}, \; Z_{3,3}, \ldots)$.

    Uo\v cimo da za $\omega \in \lijInt{0}{1}$ $(X_k (\omega))$ ima beskona\v cno nula i beskona\v cno jedinica, pa niz $\niz{X_k (\omega)}{k \in \nat}$ ne konvergira.
    Posebno $X_k \cancel{\xrightarrow[k \to \infty]{g.s.}} 0$.
    S druge strane niz je "po vjerojatnosti" sve bli\v ze nuli, jer za svaki $\varepsilon > 0$ je
    \begin{equation*}
        \vjeroj{|Z_{n, m}| > \varepsilon} = \frac{1}{n} \xrightarrow[n \to \infty]{} 0.
    \end{equation*} 
\end{pr}

\begin{defn}    \label{defn:11.13-1}
    Re\' ci \' cemo da niz $\niz{X_n}{n \in \nat}$ \emph{konvergira po vjerojatnosti} prema slu\v cajnoj varijabli $Y$, ako za svaki $\varepsilon > 0$ vrijedi
    \begin{equation*}
        \lim\limits_{n \to \infty} \vjeroj{|X_n - y| > \varepsilon} = 0.
    \end{equation*}
    To ozna\v cavamo sa
    \begin{equation*}
        X_n \xrightarrow[n \to \infty]{\masP} Y,
    \end{equation*}
    ili
    \begin{equation*}
        (\masP) \lim\limits_{n \to \infty} X_n = Y.
    \end{equation*}
\end{defn}

\begin{zad} \label{zad:11.14}
    Ako je $X_n \xrightarrow[n \to \infty]{\masP} Y$ i $X_n \xrightarrow[n \to \infty]{\masP} Z$, tada $Y =  Z \; (g.s.)$
\end{zad}

Uo\v cimo da je
\begin{equation}    \label{jed:11.15}
    X_n \xrightarrow[n \to \infty]{\masP} Y \iff \masP \Big( |X_n - Y| > \frac{1}{k} \Big) \to 0, \quad \forall k \in \nat.
\end{equation}
Koriste\' ci \eqref{jed:11.5} direktno dobivamo:

\begin{lm}  \label{lm:11.16}
    Sljede\' ce tvrdnje su ekvivalentne:
    \begin{enumerate}[label=(\roman*)]
        \item $X_n \xrightarrow[n \to \infty]{g.s.} Y$;
        \item $(\forall \varepsilon > 0)(\exists \delta > 0) (\exists N \urePar{\varepsilon}{\delta} \in \nat)\\
        (n \in \nat, \; n \geq N \urePar{\varepsilon}{\delta} \implies \masP \big( \presjek{j = n}{\infty} \{ |X_j - Y| \leq \varepsilon \} \big) \geq 1 - \delta)$;
        \item   \label{lm:11.16.3}
        $\sup\limits_{j \geq n} |X_j - Y| \xrightarrow[n  \to \infty]{\masP} 0$.
    \end{enumerate}
\end{lm}
 Iz leme \ref{lm:11.16} \ref{lm:11.16.3} direktno slijedi:

\begin{kor}    \label{kor:11.17}
    Konvergencija gotovo sigurno povla\v ci konvergenciju po vjerojatnosti k istom limesu, odnosno:
    \begin{equation*}
        X_n \xrightarrow[n \to \infty]{g.s.} Y \implies X_n \xrightarrow[n \to \infty]{\masP} Y.
    \end{equation*}
\end{kor}

Obrat op\' cenito ne vrijedi (vidi primjer \ref{pr:11.13}).
Ipak, veza izme\dj u ovih konvergencija je direktna.
 
\begin{lm} \label{lm:11.18}
    Niz $\niz{X_n}{n \in \nat}$ konvergira po vjerojatnosti prema nekoj slu\v cajnoj varijabli ako i samo ako za svaki $\varepsilon > 0$ vrijedi
    \begin{equation*}
        \lim\limits_{n \to \infty} \Bigg( \sup\limits_{
            \begin{smallmatrix}
               m \in \nat\\
               m > n
            \end{smallmatrix}
        }  \vjeroj{ |X_m - X_n| > \varepsilon }\Bigg) = 0.
    \end{equation*}
\end{lm}

\begin{proof}
    Nu\v znost se lako vidi, poka\v zimo dovoljnost.
    \begin{equation*}
        (\forall k \in \nat)(\exists m_k \in \nat)\Big(n > m \geq m_k \implies \masP \Big( |X_n - X_m|>\frac{1}{2^k} \Big) < \frac{1}{2^k} \Big).
    \end{equation*}
    Formirajmo podniz: $n_1 := m_1$, $n_{i + 1} := \max \urePar{n_i + 1}{m_{i + 1}}$, te $X_k' := X_{n_k}$.
    Tada je
    \begin{equation*}
        \suma{k = 1}{\infty} \masP \Big( |X_{k + 1}' - X_k'| > \frac{1}{2^k} \big) \leq \suma{k = 1}{\infty} \frac{1}{2^k},    
    \end{equation*}
    pa prema Borel-Cantellijevoj lemi $(\exists A \in \famF) (\vjeroj{A} = 0)$ za $\omega \in A^c$
    \begin{equation*}
        (\exists k_0 (\omega) \in \nat) \Big(k \geq k_0 \implies |X_{k + 1}'(\omega) - X_k' (\omega)| \leq \frac{1}{2^k} \Big),
    \end{equation*}
    to jest za svaki $\omega \in A^c$ vrijedi
    \begin{equation*}
        \sup\limits_{m > n} |X_m' - X_n'| \leq \suma{k = n}{\infty} |X_{k + 1}' - X_k'| \leq \suma{k = n}{\infty} \frac{1}{2^k} = \frac{1}{2^{n - 1}} \xrightarrow[n \to \infty]{} 0.
    \end{equation*}
    Posebno za $\omega \in A^c$ je $(X_n' (\omega))$ Cauchyjev niz.
    Po jednad\v zbi \eqref{jed:11.4} postoji slu\v cajna varijabla $Y$ takva da $X_k' \xrightarrow[n \to \infty]{g.s.} Y$.
    Prema korolaru \ref{kor:11.17} vrijedi $X_{n_k} \xrightarrow[n \to \infty]{\masP} Y$.
    Budu\' ci je
    \begin{equation*}
        \vjeroj{|X_n - Y| > \varepsilon} \leq \masP \Big(  |X_n - X_{n_k}| > \frac{\varepsilon}{2} \Big) + \masP \Big( |X_{n_k} - Y| > \frac{\varepsilon}{2} \Big),    
    \end{equation*}
    tvrdnja slijedi. 
\end{proof}

\begin{tm} \label{tm:11.19}
    Niz $(X_n)$ konvergira po vjerojatnosti prema $Y$ ako i samo ako za svaki podniz od $(X_n)$ postoji podniz podniza koji konvergira gotovo sigurno prema $Y$.
\end{tm}

\begin{proof}
    Vrijedi
    \begin{equation*}
        X_n \xrightarrow[n \to \infty]{\masP} Y \implies Y_k = X_{n_k} \xrightarrow[k \to \infty]{\masP} Y.
    \end{equation*}
    U dokazu leme \ref{lm:11.18} konstruiran je podniz $Y_k'$ niza $(Y_k)$ koji konvergira gotovo sigurno prema nekoj $Z$.
    Zbog $Y_k' \xrightarrow[k \to \infty]{\masP} Y$ slijedi $Y=Z \; (g.s.)$.
    Obratno, kada $X_n \cancel{\xrightarrow[n \to \infty]{\masP}} Y$ postojao bi podniz $(X_{n_k})$ te $\varepsilon > 0$ i $\delta > 0$ takvi da je $\vjeroj{|X_{n_k} - Y| > \varepsilon} > \delta > 0$.
    Tada niti jedan podniz od $(X_{n_k})$ ne mo\v ze konvergirati prema $Y$ po vjerojatnosti, pa onda niti gotovo sigurno.
\end{proof}

\begin{zad} \label{zad:11.20}
    Niz $(X_n)$ je gotovo sigurno konvergentan ako i samo ako vrijedi
    \begin{equation*}
        \sup\limits_{
            \begin{smallmatrix}
                m \in \nat\\
                m > n
            \end{smallmatrix}
        } |X_m - X_n| \xrightarrow[n \to \infty]{\masP} 0.
    \end{equation*}
\end{zad}

\v Sto mo\v zemo re\' ci o nezavisnim nizovima?

\begin{kor} \label{kor:11.21}
    Neka je $(X_n)$ niz nezavisnih slu\v cajnih varijabli.
    Ako
    \begin{equation*}
        X_n \xrightarrow[n \to \infty]{\masP} Y,    
    \end{equation*}
    tada je $Y$ degenerirana.
\end{kor}

\begin{proof}
    Ako je $X_n \xrightarrow[n \to \infty]{\masP} Y$, tada po toeremu \ref{tm:11.19} postoji podniz $(X_{n_k})$ takav da $X_{n_k} \xrightarrow[n \to \infty]{g.s.} Y$.
    Uo\v cimo da je i $(X_{n_k})$ niz nezavisnih slu\v cajnih varijabli, pa je po teoremu \ref{tm:11.9} $Y$ degenerirana.
\end{proof}

\begin{nap} \label{nap:11.22}
    Zna\v ce li korolar \ref{kor:11.21} i teorem \ref{tm:11.9} da za niz nezavisnih slu\v cajnih varijabli $(X_n)$ vrijedi
    \begin{equation*}
        X_n \xrightarrow[n \to \infty]{g.s.} Y \iff X_n \xrightarrow[n \to \infty]{\masP} Y?
    \end{equation*}
    Odgovor je ne, kao \v sto pokazuje sljede\' ci primjer.
    Vrijedi li barem u slu\v caju
    \begin{equation*}
        X_n \xrightarrow[n \to \infty]{\masP} b \in \real,
    \end{equation*}
    mora li biti $a = b$ ili $A = b$?
    Ponovo ne. U sljede\' cem primjeru $b = 0$, dok je $a = -\infty$, $A = +\infty$.
\end{nap}

\begin{pr}  \label{pr:11.23}
    Neka je $Z \sim N \urePar{0}{1}$. Za svaki $0<t<1$ postoji to\v cno jedan $\eta_t > 0$ takav da je
    \begin{equation*}
        \vjeroj{|Z| > \eta_t} = t.
    \end{equation*}
    Uo\v cimo $t \searrow 0 \implies \eta_t \nearrow +\infty$.
    Za $n \in \nat$ definiramo
    \begin{equation*}
        c_n := \frac{1}{\eta_\frac{1}{n}}\implies c_n > c_{n + 1} \searrow 0.
    \end{equation*}
    Neka je $(Z_n)$ $\iid$ niz takav da je $Z_n \distJed Z$ (u poglavlju \ref{zakoni_01} smo vidjeli da takav uvijek mo\v zemo konstruirati).
    Tada je $(X_n)$, pri \v cemu je
    \begin{equation*}
        X_n := c_n \cdot Z_n,
    \end{equation*}
    niz slu\v cajnih varijabli.
    Budu\' ci da je
    \begin{equation*}
        \suma{n = 1}{\infty} \vjeroj{|X_n| > 1} = \suma{n = 1}{\infty} \masP \Big( |Z| > \frac{1}{\eta_\frac{1}{n}} \Big) = \suma{n = 1}{\infty} \frac{1}{n} = +\infty,
    \end{equation*}
    po teoremu \ref{tm:11.9} \ref{tm:11.9.4} slijedi $X_n \cancel{\xrightarrow[n \to \infty]{g.s.}} 0$.
    S druge strane, za $\varepsilon > 0$,
    \begin{equation*}
        \vjeroj{|X_n| > \varepsilon} = \masP \Big( |Z| > \varepsilon \: \eta_\frac{1}{n} \Big) \xrightarrow[n \to \infty]{} 0, \quad \eta_\frac{1}{n} \nearrow +\infty.
    \end{equation*}
    Stoga $X_n \xrightarrow[n \to \infty]{\masP} 0$.
\end{pr}

U $\iid$ slu\v caju pona\v sanje je isto kao i kod gotovo sigurne konvergencije.

\begin{kor} \label{kor:11.24}
    Ako je $\niz{X_n}{n \in \nat} \; \iid$ niz tada $(X_n)$ konvergira po vjerojatnosti prema nekoj slu\v cajnoj varijabli ako i samo ako postoji $a \in \real$ takav da je
    \begin{equation*}
        X_n = a \; (g.s.) \quad \forall n \in \nat.
    \end{equation*}
\end{kor}

\begin{proof}
    Dovoljnost je o\v cita, a nu\v znost slijedi iz teorema \ref{tm:11.19} i korolar \ref{kor:11.10}
\end{proof}

Pitanja iz napomene \ref{nap:11.12} postavljamo i za $(\masP)$-konvergenciju.
Teoremi o $(\masP)$-konvergenciji
\begin{equation*}
    \frac{S_n - b_n}{a_n}
\end{equation*}
zovu se \emph{slabi zakoni velikih brojeva}.