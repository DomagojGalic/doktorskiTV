% poglavlje 11 - konvergencija nezavisnih nizova

\chapter{Konvergencija nezavisnih nizova}

U ovom poglavlju koristit \' cemo stalne oznake: $\vjerojatnosniProstor$, \' ce ozna\v cavati vjerojatnosni prostor, a $\niz{X_n}{n \in \nat}$, niz slu\v cajnih varijabli na $\Omega$.

Uo\v cimo da je skup $\conv{(X_n)} := \{ \lim\limits_{n} X_n \in \real \} = \skup{\omega \in \Omega}{\lim\limits_{n \in \infty} X_n (\omega) \in \real}$\\
$ = \presjek{k = 1}{\infty} \unija{n = 1}{\infty} \presjek{i = 1}{\infty} \skup{\omega}{|X_{n + i} (\omega) - X_n (\omega)| < \frac{1}{k}}$ uvijek doga\dj aj, te da je $X:= \limsup\limits_{n \to \infty} X_n$ pro\v sirena slu\v cajna varijabla.
Stoga
\begin{equation}    \label{jed:11.1}
    \vjeroj{\conv{(X_n)}} = 1 \implies \vjeroj{X = \lim\limits_n X_n \in \real} = 1
\end{equation}
i u tom slu\v caju ka\v zemo da $(X_n)$ \emph{konvergira gotovo sigurno}.
Ako je $Y$ slu\v cajna varijabla i postoji $D \in \famF$ takav da je $\vjeroj{D} = 0$ i za svaki $\omega \in D^c$ je $Y(\omega) = \lim\limits_{n \to \infty} X_n (\omega)$, tada ka\v zemo da $(X_n)$ \emph{konvergira gotovo sigurno prema} $Y$ i pi\v semo $X \xrightarrow[n \to \infty]{g.s.} Y$ ili $Y = (g.s.) \; \lim\limits_{n \to \infty} X_n$.
O\v cito vrijedi
\begin{equation}    \label{jed:11.2}
    (\exists Y) \; X_n \xrightarrow[n \to \infty]{g.s.} Y \iff \vjeroj{\conv{(X_n)} = 1}
\end{equation}
i u tom slu\v caju $X_n \xrightarrow[n \to \infty]{g.s.} X$ i $X = Y \; (g.s.)$.
Nadalje o\v cito vrijedi:
\begin{equation}    \label{jed:11.3}
    X_n \xrightarrow[n \to \infty]{g.s.} Y \iff X_n - Y \xrightarrow[n \to \infty]{g.s.} 0,
\end{equation}
i
\begin{equation}    \label{jed:11.4}
    (X_n) \textnormal{ konvergira } g.s.
    \iff
    \begin{matrix}
        \niz{X_n(\omega)}{n \in \nat} \textnormal{ Cauchyjev}\\
        \textnormal{za gotovo sve } \omega \in \Omega.
    \end{matrix}
\end{equation}

\begin{nap} \label{nap:11.4-1}
    Prisjetimo se, neka je $\niz{A_n}{n \in \nat}$ niz skupova, tada su limes superior i limes inferior definirani sa:
    \begin{enumerate}[label=(\roman*)]
        \item $\liminf\limits_{n \to \infty} A_n := \unija{n = 1}{\infty} \Bigg( \presjek{k = n}{\infty} A_k \Bigg)$
        \item $\limsup\limits_{n \to \infty} A_n := \presjek{n = 1}{\infty} \Bigg( \unija{k = n}{\infty} A_k \Bigg)$
    \end{enumerate}
\end{nap}

Va\v znu ulogu u opisu ovog pojma imaju skupovi $A_k^n := \skup{\omega \in \Omega}{|X_n(\omega) - X(\omega)| > \frac{1}{k}}$, $A_k := \limsup\limits_{n \to \infty} A_k^n$ i $D := \unija{k = 1}{\infty} A_k$.
Uo\v cimo $l \leq k \implies A_l \subseteq A_k$ i dobivamo
\begin{equation*}
    \begin{aligned}
        D &= \skup{\omega \in \Omega}{X_n(\omega) \cancel{\xrightarrow{}}X(\omega)}\\
        \implies  \vjeroj{D} &= \lim\limits_{k \to \infty} \vjeroj{A_k}\\
        \vjeroj{D} = 0 &\iff \vjeroj{A_k} = 0, \; \forall k \in \nat.
    \end{aligned}
\end{equation*}
Sada vrijedi
\begin{equation}    \label{jed:11.5}
    \begin{aligned}
        &X_n \xrightarrow[n \to \infty]{g.s.} X\\
        \iff &\vjeroj{D} = 0\\
        \iff &\lim\limits_{m \to \infty} \masP \Bigg( \unija{l = m}{\infty} A_k^l \Bigg) = 0, \quad \forall k \in \nat\\
        \iff &\lim\limits_{m \to \infty} \masP \Bigg( \unija{l = m}{\infty} \{ |X_l - X| > \varepsilon \} \Bigg) = 0, \quad \forall \varepsilon > 0.
    \end{aligned}
\end{equation}

Iz \eqref{jed:11.4} i \eqref{jed:11.5} direktno dobijemo:

\begin{lm}  \label{lm:11.6}
    Niz $\niz{X_n}{n \in \nat}$ je konvergentan $g.s.$ ako i samo ako vrijedi
    \begin{equation*}
        \lim\limits \masP \Bigg( \unija{l = m}{\infty} \{ |X_l - X| > \varepsilon \} \Bigg) = 0, \quad \forall \varepsilon > 0
    \end{equation*}
\end{lm}

Pretpostavimo sada da su $(X_n)$ nezavisne slu\v cajne varijable.
Uo\v cimo da je $\conv{(X_n)}$ repni doga\dj aj, pa prema Kolmogorovljevom zakonu 0-1 mora biti
\begin{equation}    \label{jed:11.7}
    \vjeroj{\conv{(X_n)}} \in \{ 0, \; 1 \}.
\end{equation}
Nadalje, $\liminf\limits_{n \to \infty} X_n$ i $\limsup\limits_{n \to \infty} X_n$ su repne funkcije, pa postoje $- \infty \leq a \leq A \leq +\infty$ takvi da vrijedi:
\begin{equation}    \label{jed:11.8}
    \begin{aligned}
        \liminf\limits_{n \to \infty} X_n &= a \quad (g.s.)\\
        \limsup\limits_{n \to \infty} X_n &= A \quad (g.s.)
    \end{aligned}
\end{equation}

\begin{tm}  \label{tm:11.9}
    Neka je $\niz{X_n}{n \in \nat}$ niz nezavisnih slu\v cajnih varijabli.
    Sljede\' ce tvrdnje su ekvivalentne:
    \begin{enumerate}[label=(\roman*)]
        \item   \label{tm:11.9.1}
        $(X_n)$ je konvergentan $g.s.$;
        \item   \label{tm:11.9.2}
        $-\infty < a = A < +\infty$;
        \item   \label{tm:11.9.3}
        $a \in \real$ i $X_n \xrightarrow[n \to \infty]{g.s.} a$;
        \item   \label{tm:11.9.4}
        $\suma{n = 1}{\infty} \vjeroj{|X_n - a| > \varepsilon} < +\infty, \quad \forall \varepsilon > 0$.
    \end{enumerate}
\end{tm}

\begin{proof}
    Iz \eqref{jed:11.7} i \eqref{jed:11.8} slijedi \ref{tm:11.9.1} $\iff$ \ref{tm:11.9.2}, dok je \ref{tm:11.9.2} $\iff$ \ref{tm:11.9.3} o\v cito.
    Doka\v zimo \ref{tm:11.9.3} $\implies$ \ref{tm:11.9.4}.
    Ozna\v cimo sa $C_n := \{ |X_n - a| > \varepsilon\}$, za zadani $\varepsilon > 0$.
    Po pretpostavci $\niz{C_n}{n \in \nat}$ su nezavisni doga\dj aji, a zbog $X_n \xrightarrow[n \to \infty]{g.s.} a$ dobivamo $\vjeroj{\limsup\limits_{n \to \infty} C_n} = 0$.
    Po Borelovom zakonu 0-1 $\suma{n = 1}{\infty} \vjeroj{C_n} < +\infty$.
    Doka\v zimo \ref{tm:11.9.4} $\implies$ \ref{tm:11.9.3}.
    Zbog $\suma{n = 1}{\infty} \vjeroj{C_n}$, Borel 0-1 daje $\vjeroj{\limsup\limits{n \to \infty} C_n} = 0$.
    Uo\v cimo, ako $X_n \cancel{\to} a$, tada je $\omega \in \limsup\limits_{n \to \infty} C_n$, za neki $\varnothing = \frac{1}{k}$.
    Dokaz slijedi direktno.
\end{proof}

Dakle, vrlo su rjetki slu\v cajevi, kak niz nezavisnih slu\v cajnih varijabli kongergira gotovo sigurno.
Slu\v caj $\iid$ je jo\v s posebniji i nema potrebe za daljnjim prou\v cavanjem.

\begin{kor} \label{kor:11.10}
    Neka je $\niz{X_n}{n \in \nat} \; \iid$ niz.
    Tada niz $(X_n)$ konverigra gotovo sigurno ako i samo ako je $X_n = a \; (g.s.)$, za svaki $n \in \nat$.
\end{kor}

\begin{proof}
    Dovoljnost je o\v cita, a nu\v znost slijedi iz \ref{tm:11.9} \ref{tm:11.9.4} zbog
    \begin{equation*}
        \begin{gathered}
            +\infty > \suma{n = 1}{\infty} \vjeroj{|X_n - a|>\varnothing} = \iid = \suma{n = 1}{\infty} \vjeroj{|X_1 - a|> \varepsilon}\\
            \implies \vjeroj{|X_1 - a| > \varepsilon} = 0, \quad \varepsilon > 0.
        \end{gathered}
    \end{equation*}
\end{proof}

\begin{zad} \label{zad:11.11}
    Neka je $\niz{X_n}{n \in \nat}$ $\iid$ niz.
    Doka\v zite:
    \begin{equation*}
        \vjeroj{\limsup\limits_{n \to \infty} X_n  = +\infty} = 1 \iff \vjeroj{X_1 < c} < 1,
    \end{equation*}
    za svaki $0 < c < +\infty$.
\end{zad}