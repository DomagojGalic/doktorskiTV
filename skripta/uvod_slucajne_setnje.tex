% poglavlje 3.5 -> predavanje 15 - uvod u slučajne šetnje

\chapter{Uvod u slu\v cajne \v setnje}

U ovom poglavlju uzimamo da je $\niz{X_n}{n \in \nat}$ je niz nezavisnih jednako distribuiranih slu\v cajnih varijabli na vjerojatnosnom prostoru $\vjerojatnosniProstor$.
Neka je
\begin{equation*}
    \begin{aligned}
        S_0 &:= 0,\\
        S_n &:= X_1 + \ldots + X_n, \quad n \in \nat.
    \end{aligned}
\end{equation*}

Uo\v cimo da je $\niz{S_n}{n \in \nat}$ mo\v zemo promatrati i kao stohasti\v cki proces, koji nazivamo \emph{slu\v cajnom \v setnjom} (s vrijednostima) u $\real$.
Parametar $n \in \nat_0$ obi\v cno interpretiramo kao vrijeme.
Uz tu interpretaciju, zakoni velikih brojeva govore o asimptotskom pona\v sanju \v slu\v cajne \v setnje ("u beskona\v cnosti").
Iz rezultata u poglavlju \ref{pog:2.5} (posebno primjer \ref{pr:10.3} i zadatak \ref{zad:10.4}) \v citamo da su mogu\' ca \v cetiri osnovna tipa pona\v sanja (trivijalni niz, konvergencija u jednu od beskona\v cnosti i "sve ve\' ce oscilacije").
Preciznije, dokazali smo teorem.

\begin{tm}  \label{tm:15.1}
    Ako je $\niz{S_n}{n \in \nat}$ slu\v cajna \v setnja u $\real$, tada vrijedi to\v cno jedna od sljede\' cih mogu\v cnosti (i to je ispunjeno s vjerojatno\v scu 1):
    \begin{enumerate}[label=(\roman*)]
        \item \label{tm:15.1.1}
        $S_n \xrightarrow[n \to \infty]{g.s.} 0,$
        \item \label{tm:15.1.2}
        $S_n \xrightarrow[n \to \infty]{g.s.} +\infty,$
        \item \label{tm:15.1.3}
        $S_n \xrightarrow[n \to \infty]{g.s.} -\infty,$
        \item \label{tm:15.1.4}
        $\liminf\limits_{n \to \infty} S_n = - \infty \; (g.s.), \quad \limsup\limits_{n \to \infty} S_n = + \infty \; (g.s.)$
    \end{enumerate}
\end{tm}

\begin{nap} \label{nap:15.2}
    Poku\v sajmo re\' ci ne\v sto vi\v se o trenu kada \' ce se desiti neke od gornjih mogu\' cnosti.
    \begin{enumerate}[label=(\alph*)]
        \item Slu\v caj \ref{tm:15.1.1} je ekvivalentan tome da $(S_n)$ uop\' ce konvergira kona\v cnome limesu, \v sto je ekvivalentno tome da konvergira $X_n \xrightarrow[n \to \infty]{g.s.} 0$, a \v sto je ekvivalentno trivijanom slu\v caju, to jest da je $X_n = 0 \; (g.s.)$, $\forall n \in \nat$ (vidi korolar \ref{kor:11.10}).
        Dakle slu\v caj \ref{tm:15.1.1} se mo\v ze desiti samo na potpuno trivijan na\v cin.
        \item Slu\v caj \ref{tm:15.1.4} mo\v ze se desiti kada su $X_n$ simetri\v cne, to jest kada je $X_1 \distJed - X_1$ (vidi zadatak \ref{zad:10.4}).
        Ali to nije jedini na\v cin kada se slu\v caj \ref{tm:15.1.4} mo\v ze desiti.
        \item \label{nap:15.2.3}
        Pretpostavimo da $X_n$ imaju prve momente, to jest da je $X_1 \in L^1(\masP)$.
        Tada je $\masE X_1 \in \real$ i po Kolmogorovljevom jakom zakonu
        \begin{equation*}
            \frac{S_n}{n} \xrightarrow[n \to \infty]{g.s.} \masE X_1.
        \end{equation*}
        Posebno, ako je $\masE X_1 > 0$, tada
        \begin{equation*}
            S_n \xrightarrow[n \to \infty]{g.s.} +\infty,
        \end{equation*}
        a ako je $\masE X_1 < 0$, tada je
        \begin{equation*}
            S_n \xrightarrow[n \to \infty]{g.s.} - \infty.
        \end{equation*}
        \item Iz \ref{nap:15.2.3} slijedi da ako je $X_1 \in L^1 (\masP)$ i vrijedi slu\v caj \ref{tm:15.1.4} iz teorema \ref{tm:15.1}, tada mora biti $\masE X_1 = 0$.
        Vrijedi li obrat?
        Odgovor na ovo pitanje zahtjeva malo detaljniju analizu procesa $(S_n)$.
        Va\v zan postaje koncept slu\v cajnog vremena.
    \end{enumerate}
\end{nap}

Mo\v zemo re\' ci da je "sva informacija do trenutka $n$" koju mo\v zemo dobiti iz slu\v cajne \v setnje sadr\v zana u $\sigma$-algebri
\begin{equation*}
    \famF_n = \sigAlg{X_1, \ldots, X_n}.
\end{equation*}

\begin{defn}    \label{defn:15.2-1}
    Re\' ci \' cemo da je slu\v cajna varijabla
    \begin{equation*}
        N : \Omega \to  \nat_0 \cup \{ +\infty \}
    \end{equation*}
    \emph{vrijeme zaustavljanje} ili \emph{opcionalno vrijeme} (u odnosu na slu\v cajnu \v setnju $(S_n)$) ako je za svaki $n \in \nat_0$
    \begin{equation*}
        \{ N = n \} \in \famF_n.
    \end{equation*}
\end{defn}

\begin{pr}  \label{pr:15.3}
    Neka je $A \subseteq \real$ Borelov skup.
    Definiramo
    \begin{equation*}
        N := \inf \indFamilija{n \in \nat_0}{S_n \in A},
    \end{equation*}
    uz dogovor
    \begin{equation*}
        \inf \varnothing = +\infty
    \end{equation*}
    O\v cito je $\inf = \max$, pa je
    \begin{equation*}
        N: \Omega \to \nat_0 \cup \{+\infty\}.
    \end{equation*}
    Za $n \in \nat_0$, vrijedi
    \begin{equation*}
        \{ N = n \} = \{ S_0 \in A^c, \ldots, S_{n - 1} \in A^c, S_n \in A \} \; \in \; \famF_n,
    \end{equation*}
    pa je $N$ (tipi\v can) primjer zaustavljanja.
\end{pr}

\begin{zad} \label{zad:15.4}
    Ako su $N_1$ i $N_2$ vremena zaustavljanja, tada su i
    \begin{itemize}
        \item[] $N_1 \land N_2$
        \item[] $N_1 \lor N_2$
        \item[] $N_1 + N_2$  
    \end{itemize}
    vremena zaustavljanja.
\end{zad}

\begin{defn}    \label{defn:15.4-1}
    Neka je $N$ vrijeme zaustavljanja, definiramo
    \begin{equation}    \label{jed:15.5}
        \famF_N := \bigIndFamilija{A \in \famF}{A \cap \{N = n\} \in \famF, \; \forall n \in \nat_0 }.
    \end{equation}
\end{defn}

Lako se vidi da je $\famF_N \subseteq \famF$ $\sigma$-algebra, te ako su $N_1 \leq N_2$ vremena zaustavljanja, tada je $\famF_{N_1} \subseteq \famF_{N_2}$.
Posebno, budu\' ci da je $N \equiv n$ tako\dj er vrijeme zaustavljanja, mo\v ze se promatrati $\famF_N$ i $\famF_n$, ali su one u tom slu\v caju jednaka i pi\v semo samo $\famF_n$.

\begin{defn}    \label{defn:15.5-1}
    Ka\v zemo da je $\niz{Y_n}{n \in \nat}$ niz slu\v cajnih varijabli na $\vjerojatnosniProstor$, \emph{adaptiran u odnosu na slu\v cajnu \v setnju} ako je $Y_n$ $\famF_n$-izmjeriva, za svaki $n \in \nat$.
\end{defn}

\begin{defn}    \label{defn:15.5-2}
    Neka je $\niz{Y_n}{n \in \nat}$ niz slu\v cajnih varijabli adaptiran u odnosu na slu\v cajnu \v setnju. Za kona\v cno vrijeme zaustavljanja $N$ definiramo funkciju
    \begin{equation}    \label{jed:15.6}
        Y_N (\omega) := Y_{N (\omega)} (\omega), \quad \omega \in \Omega.
    \end{equation}
\end{defn}

Tvrdimo da je $Y_N$ slu\v cajna varijabla, koje je $\famF_N$-izmjeriva.
Neka je $B \subseteq \real$ Borelov skup i neka je $n \in \nat_0$, imamo
\begin{equation*}
    \{ Y_N \in B \} = \unija{n \in \nat}{} \{ Y_n \in B \} \cap \{ N = n \},
\end{equation*}
a budu\' ci da je $\{ Y_n \in B \} \in \famF_n$, dobivamo
\begin{equation*}
    \{ Y_N \in B \} \cap \{ N = n \} = \underbrace{\{Y_n \in B\}}_{\in \famF_n} \cap \underbrace{\{N = n\}}_{\in \famF_n} \in \famF_n.
\end{equation*}

\begin{defn}    \label{defn:15.5-3}
    Neka je sada $N$ kona\v cno vrijeme zaustavljanje i $h: \real \to \real$ Borelova funkcija.
    Definiramo
    \begin{equation*}
        \begin{aligned}
            S_{h, 0} &:= 0\\
            S_{h, n} &:= \suma{k = 1}{n} h(X_k), \quad n \in \nat.
        \end{aligned}
    \end{equation*}
    Tako\dj er za $n \in \nat_0$ definiramo
    \begin{equation*}
        M_{h, n} := \max\limits_{0 \leq k \leq n} S_{h, k}.
    \end{equation*}
\end{defn}

Za slu\v caj beskona\v cnosti imamo
\begin{equation*}
    M_{h, \infty} := \max\limits_{k \in \nat_0} S_{h, k} = \sup\limits_{n \in \nat_0} M_{h, n}.
\end{equation*}
Ako je $h(x) = x$, onda u ovim oznakama ispu\v stamo $h$, te uo\v cimo da je to u skladu s prethodnim oznakama.
Po pokazanom slijedi da su
\begin{equation}    \label{jed:15.7}
    S_{h, N} \; \textnormal{ i } \; M_{h, N} \quad \famF_N \textnormal{-izmjeriva slu\v cajna varijabla},
\end{equation}
dok je
\begin{equation}    \label{jed:15.8}
    M_{h, \infty} \quad \textnormal{pro\v sirena slu\v cajna varijabla}.
\end{equation}
Posebno, $S_N$ je $\famF_N$-izmjeriva slu\v cajna varijabla.

\begin{tm}[Waldova jednakost]  \label{tm:15.9}
    Ako je $X_1 \in L^1(\masP)$ i $N$ je vrijeme zaustavljanja takvo da je $\masE N < +\infty$, tada je
    \begin{equation*}
        \masE S_N = \masE X_1 \cdot \masE N.
    \end{equation*}
\end{tm}

\begin{proof}
    Pretpostavimo prvo da je $X_i \geq 0$.
    $\masE S_N \implies N < +\infty \; (g.s.) \implies$ postoji $S_N$ i vrijedi $S_N \geq 0$.
    Dakle vrijedi:
    \begin{equation*}
        \begin{aligned}
            \masE S_N & \overset{\textnormal{neneg.}}{=} \suma{n = 0}{\infty} \int_{\{ N = n \}} S_n \; d \masP \overset{S_0 = 0}{=} \suma{n = 1}{\infty} \suma{m = n}{\infty} \int_{\{ N = n \}} X_m \; d \masP\\
            &= (\textnormal{Fubini}) = \suma{m = 1}{\infty} \suma{n = m}{\infty} \int_{\{ N = n \}} X_m \; d \masP = \suma{m = 1}{\infty} \int_{\{ N \geq m \}} X_m \; d \masP = (\textnormal{nezav.})\\
            &= \suma{m = 1}{\infty} \masE X_m \cdot \masP (N \geq m) = \masE X_1 \cdot \suma{m = 1}{\infty} \masP (N \geq m) = \masE X_1 \cdot \masE N.
        \end{aligned}
    \end{equation*}
    Za op\' ci slu\v caj imamo
    \begin{equation*}
        + \infty > \suma{m = 1}{\infty} \masE (| X_m |) \cdot \masP (N \geq m) = \suma{m = 1}{\infty} \suma{n = m}{\infty} \masE (|X_m|) \cdot \masP (N = n)
    \end{equation*}
    pa su zadovoljene pretpostavke Fubinijevog teorema i onda isti izvod kao i gore daje rezulatat.
\end{proof}

Pro\v sirimo malo ideju do $S_N$ i sli\v cno, tako da za proizvoljno vrijeme zaustavljanja definiramo odgovaraju\' ce varijable samo na skupu $\{ N < + \infty \}$ (recimo na skupu $\{ N = +\infty \}$ je sve po definicij $0$).

\begin{tm} \label{tm:15.10}
    Ako je $N$ vrijeme zaustavljanja, tada je na skupu $\{ N < +\infty \}$ niz $\niz{X_{N + n}}{n \in \nat}$ nezavisno od $\famF_N$ i jednako distribuiran kao polazni niz $\niz{X_n}{n \in \nat}$.
\end{tm}

\begin{proof}
    Treba pokazati da za $A \in \famF_N$ i za $k \in \nat$ i $B_1, \ldots, B_k$ Borelove podskupove od $\real$ vrijedi
    \begin{equation}    \label{jed:15.11}
        \begin{aligned}
            \masP (A, X_{N + 1} \in B_1, \ldots, X_{N + k} \in B_k, N < +\infty) &= \masP (A, N < +\infty) \cdot \produkt{j = 1}{k} \masP (X_j \in B_j)\\
            &= \masP (A, N < +\infty) \cdot \produkt{j = 1}{k} \masP (X_1 \in B_j)
        \end{aligned}
    \end{equation}
    Uo\v cimo da je $A \cap \{ N = n \} \in \famF_n$ i $\famF_n$ je nezavisna od $\sigAlg{X_{n + 1}, X_{n + 2}, \ldots}$.
    Dobivamo
    \begin{equation*}
        \begin{aligned}
            \masP (A, X_{N + 1} \in B_1, \ldots, X_{N + k} \in B_k, N < +\infty) &= \suma{n = 0}{\infty} \masP (A, N = n, X_{n + 1} \in B_1, \ldots, X_{n + k} \in B_k)\\
            &= \suma{n = 0}{\infty} \masP (A, N = n) \cdot \produkt{j = 1}{k} \masP (X_1 \in B_j)\\
            &= \masP (A, N < +\infty) \cdot \produkt{j = 1}{k} \masP (X_1 \in B_j).
        \end{aligned}
    \end{equation*}
\end{proof}

\begin{kor} \label{kor:15.12}
    Neka je
    \begin{equation*}
        N := \inf \indFamilija{n \in \nat_0}{S_n > 0}.
    \end{equation*}
    Sljede\' ce tvrdnje su ekvivalentne:
    \begin{enumerate}[label=(\roman*)]
        \item \label{kor:15.12.1}
        $\masP (N < + \infty) = 1;$
        \item \label{kor:15.12.2}
        $\limsup\limits_{n \to \infty} S_n = +\infty \; (g.s.);$
        \item \label{kor:15.12.3}
        $\masP (M_\infty = +\infty) = 1.$
    \end{enumerate}
\end{kor}

\begin{proof}
    \ref{kor:15.12.1} $\implies$ \ref{kor:15.12.2} $N < +\infty \; (g.s.)$ daje $S_N > 0 \; (g.s.)$ (i definirana je).
    Kada bi $\masE S_N = 0$ imali bismo $S_N = 0 \; (g.s.)$ \v sto je kontradikcija.
    Dakle $\masE S_N > 0$.
    Po teoremu \ref{tm:15.10}, gledamo iteracije od $N$ i dobijemo niz nezavisnih jednako distribuiranih slu\v cajnih varijabli $\niz{S_{N_k} - S_{N_{k - 1}}}{k \in \nat}$.
    Po Kolmogorovljevom jakom zakonu velikih brojeva (korolar \ref{kor:14.3}) vrijedi
    \begin{equation*}
        \frac{1}{k} S_{N_k} = \suma{j = 1}{k} \frac{S_{N_j} - S_{N_{j-1}}}{k} \xrightarrow[k \to \infty]{g.s.} \masE S_N > 0,
    \end{equation*}
    odakle imamo
    \begin{equation*}
        S_{N_k} \xrightarrow[k \to \infty]{g.s.} +\infty \implies \limsup\limits_{n \to \infty} S_n = +\infty \; (g.s.).
    \end{equation*}

    \ref{kor:15.12.2} $\implies$ \ref{kor:15.12.3} o\v cito, jer $\limsup\limits_{n \to \infty} S_n \leq M_{\infty}$.

    \ref{kor:15.12.3} $\implies$ \ref{kor:15.12.1} o\v cito, jer ako je $M_{\infty} (\omega) < +\infty \implies N (\omega) < +\infty$.
\end{proof}

\begin{tm}  \label{tm:15.13}
    Ako je $X_1 \in L^1 (\masP)$, $X_1$ nije degenerirana i $\masE X_1 = 0$, tada je
    \begin{equation*}
        \liminf\limits_{n \to \infty} S_n = -\infty \; (g.s.),
    \end{equation*}
    te je
    \begin{equation*}
        \limsup\limits_{n \to \infty} S_n = +\infty \; (g.s.).
    \end{equation*}
\end{tm}

\begin{proof}
    Bez smanjenja op\' cenitosti mo\v zemo promatrati $\limsup\limits_{n \to \infty} S_n$.
    Pretpostavimo suprotno, to jest $\limsup\limits_{n \to \infty} S_n = -\infty$, odnosno $S_n \xrightarrow[n \to \infty]{g.s.} -\infty$.
    To bi zna\v cilo da je $\masP (N = +\infty) = q > 0$ i $M_\infty < +\infty \; (g.s.)$.
    Neka je
    \begin{equation*}
        V := \inf \indFamilija{n \in \nat_0}{S_n = M_\infty}.
    \end{equation*}
    Tada je
    \begin{equation*}
        \begin{aligned}
            1 \geq \suma{n = 0}{\infty} \masP (V = n) &= \suma{n = 0}{\infty} \masP (S_j < S_n, 0 \leq j < n, S_k \leq S_n, k > n)\\
            &= \suma{n = 0}{\infty} \masP \Big[ \Big\{ \suma{i = j + 1}{n} X_i > 0, 0 \leq j < n \Big\} \cap \Big\{  \suma{i = n + 1}{k} X_i \leq 0, k > n \Big\} \Big]\\
            &= \textnormal{produkt}\\
            &= \suma{n = 0}{\infty} \masP (X_n > 0, X_n + X_{n-1} > 0, \ldots, X_n + \ldots + X_1 > 0) \cdot \underbrace{\masP (S_j \leq 0, j \geq 0)}_{\masP (N = \infty)}.
        \end{aligned}
    \end{equation*}
    Budu\' ci vrijedi
    \begin{equation*}
        (X_n, X_{n - 1}, \ldots, X_1) \distJed (X_1, \ldots, X_n).
    \end{equation*}
    Ako definiramo
    \begin{equation*}
        \widetilde{N} := \inf \indFamilija{n \in \nat}{S_n \leq 0}.
    \end{equation*}
    Sada vrijedi
    \begin{equation*}
        \suma{n = 0}{\infty} \masP (V = n) = \suma{n = 0}{\infty} \masP (\widetilde{N} > n) \cdot q \quad (q > 0) \infty \masE \widetilde{N} < +\infty.
    \end{equation*}
    Sada po teoremu \ref{tm:15.9}
    \begin{equation*}
        \masE S_{\widetilde{N}} = \masE \widetilde{N} \cdot \masE X_1 = 0,
    \end{equation*}
    a kako je $S_{\widetilde{N}} \leq 0$, slijdi
    \begin{equation*}
        S_{\widetilde{N}} = 0 \; (g.s.).
    \end{equation*}
    Sada vrijedi
    \begin{equation*}
        \masP (S_{\widetilde{N}} < 0) \geq \masP (X_1 < 0), 
    \end{equation*}
    \v sto je ve\' ce od nula jer je $\masE X_1 = 0$ i $X_1$ nije degenerirana, a to nas navodi na kontradikciju.
\end{proof}