% poglavlje 3.5 -> predavanje 15 - uvod u slučajne šetnje

\chapter{Uvod u slu\v cajne \v setnje}

U ovom poglavlju uzimamo da je $\niz{X_n}{n \in \nat}$ je niz nezavisnih jednako distribuiranih slu\v cajnih varijabli na vjerojatnosnom prostoru $\vjerojatnosniProstor$.
Neka je
\begin{equation*}
    \begin{aligned}
        S_0 &:= 0,\\
        S_n &:= X_1 + \ldots + X_n, \quad n \in \nat.
    \end{aligned}
\end{equation*}

Uo\v cimo da je $\niz{S_n}{n \in \nat}$ mo\v zemo promatrati i kao stohasti\v cki proces, koji nazivamo \emph{slu\v cajnom \v setnjom} (s vrijednostima) u $\real$.
Parametar $n \in \nat_0$ obi\v cno interpretiramo kao vrijeme.
Uz tu interpretaciju, zakoni velikih brojeva govore o asimptotskom pona\v sanju \v slu\v cajne \v setnje ("u beskona\v cnosti").
Iz rezultata u poglavlju \ref{pog:2.5} (posebno primjer \ref{pr:10.3} i zadatak \ref{zad:10.4}) \v citamo da su mogu\' ca \v cetiri osnovna tipa pona\v sanja (trivijalni niz, konvergencija u jednu od beskona\v cnosti i "sve ve\' ce oscilacije").
Preciznije, dokazali smo teorem.

\begin{tm}  \label{tm:15.1}
    Ako je $\niz{S_n}{n \in \nat}$ slu\v cajna \v setnja u $\real$, tada vrijedi to\v cno jedna od sljede\' cih mogu\v cnosti (i to je ispunjeno s vjerojatno\v scu 1):
    \begin{enumerate}[label=(\roman*)]
        \item \label{tm:15.1.1}
        $S_n \xrightarrow[n \to \infty]{g.s.} 0,$
        \item \label{tm:15.1.2}
        $S_n \xrightarrow[n \to \infty]{g.s.} +\infty,$
        \item \label{tm:15.1.3}
        $S_n \xrightarrow[n \to \infty]{g.s.} -\infty,$
        \item \label{tm:15.1.4}
        $\liminf\limits_{n \to \infty} S_n = - \infty \; (g.s.), \quad \limsup\limits_{n \to \infty} S_n = + \infty \; (g.s.)$
    \end{enumerate}
\end{tm}

\begin{nap} \label{nap:15.2}
    Poku\v sajmo re\' ci ne\v sto vi\v se o trenu kada \' ce se desiti neke od gornjih mogu\' cnosti.
    \begin{enumerate}[label=(\alph*)]
        \item Slu\v caj \ref{tm:15.1.1} je ekvivalentan tome da $(S_n)$ uop\' ce konvergira kona\v cnome limesu, \v sto je ekvivalentno tome da konvergira $X_n \xrightarrow[n \to \infty]{g.s.} 0$, a \v sto je ekvivalentno trivijanom slu\v caju, to jest da je $X_n = 0 \; (g.s.)$, $\forall n \in \nat$ (vidi korolar \ref{kor:11.10}).
        Dakle slu\v caj \ref{tm:15.1.1} se mo\v ze desiti samo na potpuno trivijan na\v cin.
        \item Slu\v caj \ref{tm:15.1.4} mo\v ze se desiti kada su $X_n$ simetri\v cne, to jest kada je $X_1 \distJed - X_1$ (vidi zadatak \ref{zad:10.4}).
        Ali to nije jedini na\v cin kada se slu\v caj \ref{tm:15.1.4} mo\v ze desiti.
        \item \label{nap:15.2.3}
        Pretpostavimo da $X_n$ imaju prve momente, to jest da je $X_1 \in L^1(\masP)$.
        Tada je $\masE X_1 \in \real$ i po Kolmogorovljevom jakom zakonu
        \begin{equation*}
            \frac{S_n}{n} \xrightarrow[n \to \infty]{g.s.} \masE X_1.
        \end{equation*}
        Posebno, ako je $\masE X_1 > 0$, tada
        \begin{equation*}
            S_n \xrightarrow[n \to \infty]{g.s.} +\infty,
        \end{equation*}
        a ako je $\masE X_1 < 0$, tada je
        \begin{equation*}
            S_n \xrightarrow[n \to \infty]{g.s.} - \infty.
        \end{equation*}
        \item Iz \ref{nap:15.2.3} slijedi da ako je $X_1 \in L^1 (\masP)$ i vrijedi slu\v caj \ref{tm:15.1.4} iz teorema \ref{tm:15.1}, tada mora biti $\masE X_1 = 0$.
        Vrijedi li obrat?
        Odgovor na ovo pitanje zahtjeva malo detaljniju analizu procesa $(S_n)$.
        Va\v zan postaje koncept slu\v cajnog vremena.
    \end{enumerate}
\end{nap}

Mo\v zemo re\' ci da je "sva informacija do trenutka $n$" koju mo\v zemo dobiti iz slu\v cajne \v setnje sadr\v zana u $\sigma$-algebri
\begin{equation*}
    \famF_n = \sigAlg{X_1, \ldots, X_n}.
\end{equation*}

\begin{defn}    \label{defn:15.2-1}
    Re\' ci \' cemo da je slu\v cajna varijabla
    \begin{equation*}
        N : \Omega \to  \nat_0 \cup \{ +\infty \}
    \end{equation*}
    \emph{vrijeme zaustavljanje} ili \emph{opcionalno vrijeme} (u odnosu na slu\v cajnu \v setnju $(S_n)$) ako je za svaki $n \in \nat_0$
    \begin{equation*}
        \{ N = n \} \in \famF_n.
    \end{equation*}
\end{defn}

\begin{pr}  \label{pr:15.3}
    Neka je $A \subseteq \real$ Borelov skup.
    Definiramo
    \begin{equation*}
        N := \inf \indFamilija{n \in \nat_0}{S_n \in A},
    \end{equation*}
    uz dogovor
    \begin{equation*}
        \inf \varnothing = +\infty
    \end{equation*}
    O\v cito je $\inf = \max$, pa je
    \begin{equation*}
        N: \Omega \to \nat_0 \cup \{+\infty\}.
    \end{equation*}
    Za $n \in \nat_0$, vrijedi
    \begin{equation*}
        \{ N = n \} = \{ S_0 \in A^c, \ldots, S_{n - 1} \in A^c, S_n \in A \} \; \in \; \famF_n,
    \end{equation*}
    pa je $N$ (tipi\v can) primjer zaustavljanja.
\end{pr}

\begin{zad} \label{zad:15.4}
    Ako su $N_1$ i $N_2$ vremena zaustavljanja, tada su i
    \begin{itemize}
        \item[] $N_1 \land N_2$
        \item[] $N_1 \lor N_2$
        \item[] $N_1 + N_2$  
    \end{itemize}
    vremena zaustavljanja.
\end{zad}

\begin{defn}    \label{defn:15.4-1}
    Neka je $N$ vrijeme zaustavljanja, definiramo
    \begin{equation}    \label{jed:15.5}
        \famF_N := \bigIndFamilija{A \in \famF}{A \cap \{N = n\} \in \famF, \; \forall n \in \nat_0 }.
    \end{equation}
\end{defn}

Lako se vidi da je $\famF_N \subseteq \famF$ $\sigma$-algebra, te ako su $N_1 \leq N_2$ vremena zaustavljanja, tada je $\famF_{N_1} \subseteq \famF_{N_2}$.
Posebno, budu\' ci da je $N \equiv n$ tako\dj er vrijeme zaustavljanja, mo\v ze se promatrati $\famF_N$ i $\famF_n$, ali su one u tom slu\v caju jednaka i pi\v semo samo $\famF_n$.