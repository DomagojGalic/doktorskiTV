% beskonačne familije slučajnih elemenata

\chapter{Beskona\v cne familije slu\v cajnih elemenata}

Podsjetimo se na oznake u napomeni \ref{nap:4.10}.

\begin{defn}    \label{defn:8.1}
    Neka je $\vjerojatnosniProstor$ vjerojatnosni prostor, $\urePar{E}{\famE}$ izmjeriv prostor, $T \neq \varnothing$ skup indeksa.
    \emph{$E$-zna\v cni stohasti\v cki} (ili \emph{slu\v cajni}) \emph{proces} je svako preslikavanje
    \begin{equation*}
        X : \Omega \to E^T
    \end{equation*}
    koje je $\urePar{\famF}{\famE^T}$-izmjerivo.
    Ako je $T$ beskona\v can i prebrojiv, obi\v cno ka\v zemo da je $X$ \emph{$E$-zna\v cni slu\v cajni niz}, a ako je $T$ kona\v can, ka\v zemo da je $X$ \emph{$E$-zna\v cni slu\v cajni vektor}.
    Ako je $E = \extReal$, $\famE = \borel{\extReal}$, umjesto rije\v ci "$E$-zna\v cni" koristimo rije\v c "pro\v sireni", a ako jo\v s postoji $A \in \famF, \; \vjeroj{A} = 0$, takav da je
    \begin{equation*}
        \unija{t \in T}{} \{ |\pi_t \circ X| = \infty \} \subseteq A,
    \end{equation*}
    tada se rije\v c "$E$-zna\v cni" u gornjoj definicij izbacuje.
\end{defn}

\begin{nap} \label{nap:8.2}
    \begin{enumerate}[label=(\alph*)]
        \item Umjesto $\pi_t \circ X$ naj\v ce\v s\' ce koristimo oznaku $X_t$ i stohasti\v cki proces promatramo kao familiju $E$-zna\v cnih funkcija $\niz{X_t}{t \in T}$.
        Po propoziciji \ref{prop:4.9}, $X$ je $E$-zna\v cni stohasti\v cki proces ako i samo ako su sve $X_t$ $E$-zna\v cni slu\v cajni elementi.
        \item Mo\v ze se dogoditi da $\urePar{T}{\famT}$ ima strukturu izmjerivog prostora.
        Tada je mogu\' ca jo\v s jedna interpretacija stohasti\v ckog procesa $\niz{X_t}{t \in T}$.
        Ponovo se (neprecizno!) koristi slovo $X$ za preslikavanje
        \begin{equation*}
            X : T \times \Omega \to E,
        \end{equation*}
        definirano sa
        \begin{equation*}
            X \urePar{t}{\omega} := X_t (\omega).
        \end{equation*}
        Re\' ci \' cemo da je stohasti\v cki proces $\niz{X_t}{t \in T}$ \emph{izmjeriv} ako je ovako definirani $X$ $\urePar{\famT \otimes \famF}{\famE}$-izmjeriv.
        Uo\v cimo (po \eqref{jed:4.7}) da za svako $\urePar{\famT \otimes \famF}{\famE}$-preslikavanje $X$ vrijedi da su sve $X_t$ slu\v cajni elementi u $E$ to jest $\niz{X_t}{t \in T}$ je $E$-zna\v cni stohasti\v cki proces.
        Obrat op\' cenito ne vrijedi.
        \item Ako je $T \subseteq \overline{\Z}$ ili $T \subseteq \extReal$, obi\v cno $t \in T$ interpretiramo kao vrijeme.
        Za svaki $\omega \in \Omega$ promatramo preslikavanje
        \begin{equation*}
            T \ni t \mapsto X_t (\omega) \in E,
        \end{equation*}
        koje nazivamo \emph{trajektorijom} ("sample path") procesa $X$.
        Ako $E$ ima i topolo\v sku strukturu obi\v cno se promatraju i neka "analiti\v cka" svojstva trajektorije (neprekidnost, neprekidnost slijeva, neprekidnost zdesna, limesi i sli\v cno).
        Naj\v ces\v s\' ci su slu\v cajevi
        \begin{itemize}
            \item $T = \nat_0$ - \emph{diskretni vremenski parametar} (trajektorije su nizovi u $E$);
            \item $T = \desInt{0}{+\infty}$ - \emph{neprekidni vremenski parametar} (trajektorije su funkcije realne varijable).
        \end{itemize}
        U oba slu\v caja $0$ ozna\v cava trenutak po\v cetka  (promatranja) slu\v cajnog procesa.
        \item Razni su nivoi jednakosti kod slu\v cajnih procesa.
        Ako su $\niz{X_t}{t \in T}$, $\niz{Y_t}{t \in T}$ $E$-zna\v cni stohasti\v cki procesi na $\vjerojatnosniProstor$, imamo dvije vrste $(g.s.)$ jednakosti.
        Re\' ci \' cemo da su $X$ i $Y$ \emph{nerazlu\v civi} ako postoje $A \in \famF, \; \vjeroj{A} = 0$, takvi da je
        \begin{equation*}
            \skup{\omega \in \Omega}{(\exists t \in T) \; X_t (\omega) \neq Y_t (\omega)} \subseteq A.
        \end{equation*}
        Re\' ci \' cemo da su $X$ i $Y$ \emph{modifikacije} (jedan drugoga) ako je za svaki $t \in T$ $X_t = Y_t \; (g.s.)$.
        Nerazlu\v civost uvijek povla\' ci modificiranost, ali obrat op\' cenito ne vrijedi.
        Ako je $T$ prebrojiv, pojmovi su ekvivalentni.
    \end{enumerate}
\end{nap}

U skladu s poglavljem \ref{dist_sl_elem}, $E$-zna\v cni stohasti\v cki procesi $X$ i $Y$ (ne nu\v zno definirani na istom vjerojatnosnom prostoru) imaju istu distribuciju ($X \distJed Y$) ako je $\masP_X = \masP_Y$.
Op\' cenito vrijedi
\begin{equation*}
    \textnormal{nerazlu\v civost}
    \begin{smallmatrix}
        \implies\\
        \notimpliedby
    \end{smallmatrix}
    \textnormal{modifikacija}
    \begin{smallmatrix}
        \implies\\
        \notimpliedby
    \end{smallmatrix}
    \textnormal{jednaka distribuiranost}
\end{equation*}
Za\v sto vrijedi ova zadnja implikacija?
Kada su $\masP_X$ i $\masP_Y$ jednaki?
Uo\v cimo da su $\masP_X$ i $\masP_Y$ definirane na $\urePar{E^T}{\famE^T}$ i da je $\prsten{E^T}$ generiraju\' ci poluprsten za $\famE^T$.
Slijedi da je
\begin{equation}    \label{jed:8.3}
    \masP_X = \masP_Y
    \iff
    \restr{\masP_X}{\prsten{E^T}} = \restr{\masP_Y}{\prsten{E^T}}.
\end{equation}

Relacija \eqref{jed:8.3} sugerira da se mo\v zemo ograni\v citi na kona\v cne koordinate kako bismo postigli jedakost po distribuciji.
Preciznije, neka je $J \subseteq T$, $J$ kona\v can i neprazan.

\begin{defn}
    Ozna\v cimo sa $\famE^{T, \: \pi_J} = \indSigAlg{\pi_t}{t \in J}$ $\sigma$-algebru (na $E^T$) i sa $\masP_{X, \: J} = \restr{\masP_X}{\famE^{T, \: \pi_J}}$ restrikciju od $\masP_X$ na $\famE^{T, \: \pi_J}$.
\end{defn}

Uo\v cimo da je jednostavno direktno poistovjetiti mjeru $\masP_{X, \: J}$ sa mjerom na kona\v cnom produktu $\urePar{E^J}{\famE^J}$.
Zato ka\v zemo da je
\begin{equation}    \label{jed:8.4}
    \skup{\masP_{X, \: J}}{J \subseteq T, \; J \neq \varnothing, \; \card{J} < \infty},
\end{equation}
familija \emph{kona\v cno dimenzionalnih distribucija stohasti\v ckog procesa $X$}.
Iz \eqref{jed:8.3} direktno slijedi:

\begin{tm}  \label{tm:8.5}
    $E$-zna\v cni stohasti\v cki procesi $X$ i $Y$ su jednaki po distribuciji ako i samo ako su im sve kona\v cno-dimenzionalne distribucije jednake.
\end{tm}

Promatramo li $\masP_{X, \: J}$ kao mjeru na $\urePar{E^J}{\famE^J}$ mo\v zemo iskoristiti teorem \ref{tm:7.4} i teorem \ref{tm:8.5} kako bismo opisali jednakost po distribuciji u nezavisnom slu\v caju (za jedno\v clane $J = \{j\}$ uvodimo oznaku $\masP_{X, \: J} = \masP_{X, \: j}$ i uo\v cimo da se $\masP_{X, \: j}$ mo\v ze poistovjetiti sa $\masP_{X_j}$).

\begin{kor} \label{kor:8.6}
    Neka je $\niz{X_t}{t \in T}$ $E$-zna\v cni stohasti\v cki proces i nezavisna familija slu\v cajnih elemenata.
    Neka je $\niz{Y_t}{t \in T}$ $E$-zna\v cni stohasti\v cki proces i nezavisna familija slu\v cajnih elemenata.
    Tada je $X \distJed Y$ ako i samo ako je $X_t \distJed Y_t$, za svaki $t \in T$.
\end{kor}

\v Sto re\' ci o egzistenciji procesa uz zadane distribucije?
Ako su za svaki $\varnothing \neq J \subseteq T$, $J$ kona\v can, zadane vjerojatnosti $\masP_J$ na $\famE^{T, \: \pi_J}$, potrebna nam je barem konzistentnost.

\begin{defn}    \label{defn:8.6-1}
    Neka je $\niz{\masP_J}{J \subseteq T, \; J \neq \varnothing, \; \card{J} < \infty}$ familija vjerojatnosti, ka\v zemo da je ona \emph{konzistentna} ako za svake $I$, $J$ vrijedi $\varnothing \neq J \subseteq  I \subseteq T$, te su $J$, $I$ kona\v cni,
    \begin{equation}    \label{jed:8.7}
        \restr{\masP_I}{\famE^{T, \: \pi_J}} = \masP_J.
    \end{equation}
\end{defn}
 
Zaista, ako vrijedi \eqref{jed:8.7} nije te\v sko vidjeti da je sa
\begin{equation*}
    \nu (A) := \masP_J (A), \quad A \in \prsten{E^T} \cap \famE^{T, \: \pi_j},
\end{equation*}
zadane, (to jest dobro definirana) kona\v cno aditivna funkcija
\begin{equation*}
    \nu : \prsten{E^T} \to \segment{0}{1}.
\end{equation*}
Uspijemo li pokazati da je $\nu$ $\sigma$-aditivna, onda Caratheodoryjeva konstrukcija daje mjeru na $\famE^T$.
Pokazuje se da za $\sigma$-aditivnost trebamo neku vrstu kompaktnosti i regularnosti.
Na primjer:

\begin{lm}  \label{lm:8.6}
    Ako je $E$ potpun, separabilan metri\v cki prostor i $\famE = \borel{E}$, tada je $\nu$ $\sigma$-aditivna.
\end{lm}

\begin{proof}{(Skica)}
    Zbog kona\v cne aditivnosti dovoljno je dokazati da za svaki niz $\niz{A_n}{n \in \nat} \subseteq \prsten{E^T}$, takav da je
    $A_1 \supseteq A_2 \supseteq \ldots$ i $\presjek{n}{} A_n = \varnothing$ vrijedi $\lim\limits_{n \to +\infty} \nu (A_n) = 0$.
    U suprotnom postojao bi $\varepsilon \geq 0$ i $\nu (A_n) \geq \varepsilon, \; \forall n \in \nat$.
    Tada se $A_n$ mogu dovoljno dobro aproksimirati odozdo kompaktima $C_n$ (po $\masP_J$, a time i po $\nu$) i dijagonalni postupak (kompaktnost je va\v zna!) po koordinatama daje to\v cku "u presjeku" $C_n$-ova, a time u $\presjek{n}{} A_n$.
\end{proof}

Iz leme \ref{lm:8.6} i prethodne diskusije slijedi:

\begin{tm}[Kolmogorov]    \label{tm:8.9}
    Za svaku konzistentnu familiju vjerojatnosti
    \begin{equation*}
        \bigNiz{\masP_J}{J \subseteq T, \; J \neq \varnothing, \; \card{J} < \infty}
    \end{equation*}
    na $\urePar{E^T}{\famE^T}$ pri \v cemu je $E$ potpun separabilan metri\v cki prostor i $\famE = \borel{E}$, postoji i jedinstvena je vjerojatnost $\masP$ na $\urePar{E^T}{\famE^T}$ takva da je za svaki $\varnothing \neq J \subseteq T$, $J$ kona\v can, $\restr{\masP}{\famE^{T, \: \pi_J}} = \masP_J$.
\end{tm}

\begin{nap} \label{nap:8.10}
    \begin{enumerate}[label=(\alph*)]
        \item Teorem \ref{tm:8.9} garantira da za svaku zadanu distribuciju stohasti\v ckog procesa na $E$ postoji realizacija tog procesa na kanonskom prostru $\urePar{E^T}{\famE^T}$ uz $X_t = \pi_t$.
        \item U nezavisnom slu\v caju teorem \ref{tm:8.9} garantira da za svaku familiju jednodimenzionalnih distribucija na $\urePar{E}{\borel{E}}$ postojih stohasti\v cki proces $(X_t)$ s takvim distribucijama i familija $(X_t)$ je nezavisna.
        Naime, produktne mjere \' ce sigurno biti konzistentne.
        Ovaj zadnji rezultat se mo\v ze u slu\v caju prebrojivog $T$ dobiti i direktnom vjerojatnosnom konstrukcijom.
    \end{enumerate}
\end{nap}

\begin{tm}  \label{tm:8.11}
    Neka je $\niz{\masP_n}{n \in \nat}$ niz vjerojatnosnih mjera na $\urePar{E}{\borel{E}}$, pri \v cemu je $E$ potpun, separabilan metri\v cki prostor.
    Tada na vjerojatnosnom prostoru $(\segment{0}{1}, \; \borel{\segment{0}{1}}, \; \restr{\lambda}{\borel{\segment{0}{1}}})$ postoji $E$-zna\v cni stohasti\v cki proces $\niz{X_n}{n \in \nat}$, takav da je familija $\indFamilija{X_n}{n \in \nat}$ nezavisna i vrijedi $\masP_{X_n} = \masP_n$, za svaki $n \in \nat$.
\end{tm}

\begin{proof}
    Podsjetimo se da za $E$ postoji Borelov skup $A \subseteq \segment{0}{1}$ i bijekcija $f : E \to A$ takav da su $f$ i $f^{-1}$ izmjerive u paru $\sigma$-algebri $\famE =  \borel{E}$ i $\borel{A}$.
    Dakle, bez smanjenja op\' cenitosti uzmemo $E \subseteq \segment{0}{1}$. Neka je $h: \segment{0}{1} \to \segment{0}{1}$, $h(x) = x$, to je uniformno distribuirana slu\v cajna varijabla i po zadatku \ref{zad:7.20} njene binarne znamenke tvore niz $\niz{h_n}{n \in \nat}$ nezavisnih jednako distribuiranih Bernoullijevih slu\v cajnih varijabli.
    Preuredimo niz $(h_n)$ u beskona\v cnu matricu $\niz{h_{ij}}{i, \; j \in \nat}$ (kao na primjer u bijekciji izme\dj u $\nat$ i $\nat \times \nat$).
    Za svaki $i \in \nat$ je $\niz{h_{ij}}{n \in \nat}$ niz nezavisnih Bernoulijevih slu\v cajnih varijabli.
    Definiramo
    \begin{equation*}
        g_i := \suma{j \in \nat}{} 2^{-j} h_{ij}.
    \end{equation*}
    Po korolaru \ref{kor:8.6} slijedi da je $g_i \distJed h$, to jest svaka $g_i$ je uniformno distribuirana na $\segment{0}{1}$.
    Po korolaru \ref{kor:7.9} $\indFamilija{g_i}{i \in \nat}$ \v cine nezavisnu familiju slu\v cajnih varijabli.
    Za svaki $i \in \nat$, $\masP$ odre\dj uje p.d.F. $F_i$, pa kao u propoziciji \ref{prop:7.18} stavimo
    \begin{equation*}
        f_i (x) := \sup \skup{r \in \real}{F_i (r) < x}
    \end{equation*}
    i definiramo $X_i = f_i \circ g_i$.
    Opet po korolaru \ref{kor:7.9} $\niz{X_i}{i \in \nat}$ su nezavisne, a po propozicij \ref{prop:7.18} $F_{X_i} = F_i$, to jest $\masP_{X_i} = \masP_i$.
\end{proof}